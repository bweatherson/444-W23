% Options for packages loaded elsewhere
\PassOptionsToPackage{unicode}{hyperref}
\PassOptionsToPackage{hyphens}{url}
%
\documentclass[
  11pt,
]{article}
\usepackage{amsmath,amssymb}
\usepackage{lmodern}
\usepackage{iftex}
\ifPDFTeX
  \usepackage[T1]{fontenc}
  \usepackage[utf8]{inputenc}
  \usepackage{textcomp} % provide euro and other symbols
\else % if luatex or xetex
  \usepackage{unicode-math}
  \defaultfontfeatures{Scale=MatchLowercase}
  \defaultfontfeatures[\rmfamily]{Ligatures=TeX,Scale=1}
  \setmainfont[Scale=MatchLowercase]{SF Pro Text Light}
\fi
% Use upquote if available, for straight quotes in verbatim environments
\IfFileExists{upquote.sty}{\usepackage{upquote}}{}
\IfFileExists{microtype.sty}{% use microtype if available
  \usepackage[]{microtype}
  \UseMicrotypeSet[protrusion]{basicmath} % disable protrusion for tt fonts
}{}
\makeatletter
\@ifundefined{KOMAClassName}{% if non-KOMA class
  \IfFileExists{parskip.sty}{%
    \usepackage{parskip}
  }{% else
    \setlength{\parindent}{0pt}
    \setlength{\parskip}{6pt plus 2pt minus 1pt}}
}{% if KOMA class
  \KOMAoptions{parskip=half}}
\makeatother
\usepackage{xcolor}
\usepackage[margin=1.5in]{geometry}
\usepackage{graphicx}
\makeatletter
\def\maxwidth{\ifdim\Gin@nat@width>\linewidth\linewidth\else\Gin@nat@width\fi}
\def\maxheight{\ifdim\Gin@nat@height>\textheight\textheight\else\Gin@nat@height\fi}
\makeatother
% Scale images if necessary, so that they will not overflow the page
% margins by default, and it is still possible to overwrite the defaults
% using explicit options in \includegraphics[width, height, ...]{}
\setkeys{Gin}{width=\maxwidth,height=\maxheight,keepaspectratio}
% Set default figure placement to htbp
\makeatletter
\def\fps@figure{htbp}
\makeatother
\setlength{\emergencystretch}{3em} % prevent overfull lines
\providecommand{\tightlist}{%
  \setlength{\itemsep}{0pt}\setlength{\parskip}{0pt}}
\setcounter{secnumdepth}{-\maxdimen} % remove section numbering
% https://github.com/rstudio/rmarkdown/issues/337
\let\rmarkdownfootnote\footnote%
\def\footnote{\protect\rmarkdownfootnote}

% https://github.com/rstudio/rmarkdown/pull/252
\usepackage{titling}
\setlength{\droptitle}{-2em}

\pretitle{\vspace{\droptitle}\centering\huge}
\posttitle{\par}

\preauthor{\centering\large\emph}
\postauthor{\par}

\predate{\centering\large\emph}
\postdate{\par}

\usepackage[normalem]{ulem}
\usepackage{gensymb}
\usepackage{nicefrac}
\usepackage{caption}
\usepackage{istgame}
\usepackage{array}
\usepackage{graphicx}
\usepackage{siunitx}
\usepackage[normalem]{ulem}
\usepackage{colortbl}
\usepackage{multirow}
\usepackage{hhline}
\usepackage{calc}
\usepackage{tabularx}
\usepackage{threeparttable}
\usepackage{wrapfig}
\usepackage{adjustbox}
\usepackage{hyperref}
\ifLuaTeX
  \usepackage{selnolig}  % disable illegal ligatures
\fi
\IfFileExists{bookmark.sty}{\usepackage{bookmark}}{\usepackage{hyperref}}
\IfFileExists{xurl.sty}{\usepackage{xurl}}{} % add URL line breaks if available
\urlstyle{same} % disable monospaced font for URLs
\hypersetup{
  pdftitle={Third Weekly Assignment},
  pdfauthor={Philosophy 444},
  hidelinks,
  pdfcreator={LaTeX via pandoc}}

\title{Third Weekly Assignment}
\author{Philosophy 444}
\date{Due 3 February, 2023}

\begin{document}
\maketitle

\hypertarget{question-1}{%
\subsection{Question 1}\label{question-1}}

Ankita and Beatrice would like to go on a date. They have two options:
burgers at Fleetwood, or cocktails at Gandy Dancer. Ankita first chooses
where to go, and knowing where Ankita went Beatrice also decides where
to go. Ankita prefers Fleetwood, and Beatrice prefers Gandy Dancer. A
player gets 3 if they end up with their preferred date, 1 if they end up
with their unpreferred date, and 0 if they end up at different places.
All these are common knowledge.

\begin{enumerate}
\def\labelenumi{\alph{enumi}.}
\tightlist
\item
  Find a subgame perfect equilibrium to this game.
\item
  Find a Nash equilibrium that is not subgame perfect.
\end{enumerate}

Modify the game a little bit: Beatrice does not automatically know where
Ankita went, but she can learn without any cost. That is, now, without
knowing where Ankita went, Beatrice first chooses between Learn and
Not-Learn; if she chooses Learn, then she knows where Ankita went and
then decides where to go; otherwise she chooses where to go without
learning where Ankita went. The payoffs depend only on where each player
goes, as before. (Again, it is common knowledge that Beatrice has the
ability to learn what Ankita knows before deciding.)

Now find a subgame perfect equilibrium of this new game in which the
outcome (i.e., which place the two of them end up at) is the same as the
outcome of the Nash equilibrium that is not subgame perfect in the
original game.

HINT: In the latter game, Beatrice has sixteen strategies, since there
are four points that she could make a binary choice. First, she has to
decide Learn or Not-Learn. Second, she has to decide what to do if
Not-Learn; that's a single choice since she is in the same position
whether Ankita goes to Fleetwood or Gandy Dancer. Third, what to do if
she chooses Learn and Ankita goes to Fleetwood. Fourth, what to do if
she chooses Learn and Ankita goes to Gandy Dancer.

\textbf{Continued on other side}

\newpage

\hypertarget{question-two}{%
\subsection{Question Two}\label{question-two}}

The players are Coke and Pepsi. Coke is deciding whether to enter a
market, Pepsi is already in the market. (Apparently this happened a lot
in the 1990s, because there were plenty of countries where Pepsi had a
dominant market position and Coke was a newcomer.) Coke has to make two
decisions in the game - whether to enter the market or not, and if they
enter, whether to play tough or not. Pepsi has to make one decision,
whether to play tough or not. If Coke enters, here is the payoff table
for the players, with T for Tough, A for Accommodate.

 
  \providecommand{\huxb}[2]{\arrayrulecolor[RGB]{#1}\global\arrayrulewidth=#2pt}
  \providecommand{\huxvb}[2]{\color[RGB]{#1}\vrule width #2pt}
  \providecommand{\huxtpad}[1]{\rule{0pt}{#1}}
  \providecommand{\huxbpad}[1]{\rule[-#1]{0pt}{#1}}

\begin{table}[h!]
\begin{centerbox}
\begin{threeparttable}
 \label{tab:unnamed-chunk-2}
\setlength{\tabcolsep}{0pt}
\begin{tabular}{l l l}


\hhline{>{\huxb{0, 0, 0}{0.4}}|}
\arrayrulecolor{black}

\multicolumn{1}{!{\huxvb{0, 0, 0}{0}}r!{\huxvb{0, 0, 0}{0.4}}}{\huxtpad{0pt + 1em}\raggedleft \hspace{2pt} 
 \hspace{2pt}\huxbpad{6pt}} &
\multicolumn{1}{c!{\huxvb{0, 0, 0}{0}}}{\huxtpad{0pt + 1em}\centering \hspace{2pt} Pepsi Tough \hspace{2pt}\huxbpad{6pt}} &
\multicolumn{1}{c!{\huxvb{0, 0, 0}{0}}}{\huxtpad{0pt + 1em}\centering \hspace{2pt} Pepsi Accommodate \hspace{2pt}\huxbpad{6pt}} \tabularnewline[-0.5pt]


\hhline{>{\huxb{0, 0, 0}{0.4}}->{\huxb{0, 0, 0}{0.4}}->{\huxb{0, 0, 0}{0.4}}-}
\arrayrulecolor{black}

\multicolumn{1}{!{\huxvb{0, 0, 0}{0}}r!{\huxvb{0, 0, 0}{0.4}}}{\huxtpad{2pt + 1em}\raggedleft \hspace{2pt} Coke Tough \hspace{2pt}\huxbpad{2pt}} &
\multicolumn{1}{c!{\huxvb{0, 0, 0}{0}}}{\huxtpad{2pt + 1em}\centering \hspace{2pt} -2, 1 \hspace{2pt}\huxbpad{2pt}} &
\multicolumn{1}{c!{\huxvb{0, 0, 0}{0}}}{\huxtpad{2pt + 1em}\centering \hspace{2pt} 0, -3 \hspace{2pt}\huxbpad{2pt}} \tabularnewline[-0.5pt]


\hhline{>{\huxb{0, 0, 0}{0.4}}|}
\arrayrulecolor{black}

\multicolumn{1}{!{\huxvb{0, 0, 0}{0}}r!{\huxvb{0, 0, 0}{0.4}}}{\huxtpad{2pt + 1em}\raggedleft \hspace{2pt} Coke Accomodate \hspace{2pt}\huxbpad{2pt}} &
\multicolumn{1}{c!{\huxvb{0, 0, 0}{0}}}{\huxtpad{2pt + 1em}\centering \hspace{2pt} -3, 1 \hspace{2pt}\huxbpad{2pt}} &
\multicolumn{1}{c!{\huxvb{0, 0, 0}{0}}}{\huxtpad{2pt + 1em}\centering \hspace{2pt} 1, 2 \hspace{2pt}\huxbpad{2pt}} \tabularnewline[-0.5pt]


\hhline{>{\huxb{0, 0, 0}{0.4}}|}
\arrayrulecolor{black}
\end{tabular}
\end{threeparttable}\par\end{centerbox}

\end{table}
 

Find all subgame perfect equilibria to the game with the following three
constraints:

\begin{enumerate}
\def\labelenumi{\arabic{enumi}.}
\tightlist
\item
  If Coke doesn't enter, Coke gets 0, Pepsi gets 5. Coke first decides
  whether to enter or not, and then both Coke and Pepsi find out whether
  Coke decided to enter, then each company simultaneously decides on
  Tough or Accommodate.
\item
  Coke decides whether to enter at the same time Pepsi decides whether
  to be Tough or Accommodate. If Coke enters, it then decides whether to
  be Tough or Accommodating, knowing what Pepsi has decided. If Coke
  stays out, it gets 0, and Pepsi gets 0 if Accommodating, -1 if Tough.
\item
  Just like the previous case, but if Coke stays out, and Pepsi is
  Tough, Coke gets +1.
\end{enumerate}

\hypertarget{due-friday-february-3rd-at-5pm}{%
\subsection{Due Friday February 3rd, at
5pm}\label{due-friday-february-3rd-at-5pm}}

\end{document}
