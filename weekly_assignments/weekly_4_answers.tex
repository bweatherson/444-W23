% Options for packages loaded elsewhere
\PassOptionsToPackage{unicode}{hyperref}
\PassOptionsToPackage{hyphens}{url}
%
\documentclass[
  10pt,
]{article}
\usepackage{amsmath,amssymb}
\usepackage{lmodern}
\usepackage{iftex}
\ifPDFTeX
  \usepackage[T1]{fontenc}
  \usepackage[utf8]{inputenc}
  \usepackage{textcomp} % provide euro and other symbols
\else % if luatex or xetex
  \usepackage{unicode-math}
  \defaultfontfeatures{Scale=MatchLowercase}
  \defaultfontfeatures[\rmfamily]{Ligatures=TeX,Scale=1}
  \setmainfont[Scale=MatchLowercase]{SF Pro Text Light}
\fi
% Use upquote if available, for straight quotes in verbatim environments
\IfFileExists{upquote.sty}{\usepackage{upquote}}{}
\IfFileExists{microtype.sty}{% use microtype if available
  \usepackage[]{microtype}
  \UseMicrotypeSet[protrusion]{basicmath} % disable protrusion for tt fonts
}{}
\makeatletter
\@ifundefined{KOMAClassName}{% if non-KOMA class
  \IfFileExists{parskip.sty}{%
    \usepackage{parskip}
  }{% else
    \setlength{\parindent}{0pt}
    \setlength{\parskip}{6pt plus 2pt minus 1pt}}
}{% if KOMA class
  \KOMAoptions{parskip=half}}
\makeatother
\usepackage{xcolor}
\usepackage[margin=1in]{geometry}
\usepackage{graphicx}
\makeatletter
\def\maxwidth{\ifdim\Gin@nat@width>\linewidth\linewidth\else\Gin@nat@width\fi}
\def\maxheight{\ifdim\Gin@nat@height>\textheight\textheight\else\Gin@nat@height\fi}
\makeatother
% Scale images if necessary, so that they will not overflow the page
% margins by default, and it is still possible to overwrite the defaults
% using explicit options in \includegraphics[width, height, ...]{}
\setkeys{Gin}{width=\maxwidth,height=\maxheight,keepaspectratio}
% Set default figure placement to htbp
\makeatletter
\def\fps@figure{htbp}
\makeatother
\setlength{\emergencystretch}{3em} % prevent overfull lines
\providecommand{\tightlist}{%
  \setlength{\itemsep}{0pt}\setlength{\parskip}{0pt}}
\setcounter{secnumdepth}{-\maxdimen} % remove section numbering
% https://github.com/rstudio/rmarkdown/issues/337
\let\rmarkdownfootnote\footnote%
\def\footnote{\protect\rmarkdownfootnote}

% https://github.com/rstudio/rmarkdown/pull/252
\usepackage{titling}
\setlength{\droptitle}{-2em}

\pretitle{\vspace{\droptitle}\centering\huge}
\posttitle{\par}

\preauthor{\centering\large\emph}
\postauthor{\par}

\predate{\centering\large\emph}
\postdate{\par}

\usepackage[normalem]{ulem}
\usepackage{nicefrac}
\usepackage{caption}
\ifLuaTeX
  \usepackage{selnolig}  % disable illegal ligatures
\fi
\IfFileExists{bookmark.sty}{\usepackage{bookmark}}{\usepackage{hyperref}}
\IfFileExists{xurl.sty}{\usepackage{xurl}}{} % add URL line breaks if available
\urlstyle{same} % disable monospaced font for URLs
\hypersetup{
  pdftitle={4th Weekly Assignment Answers},
  pdfauthor={Philosophy 444},
  hidelinks,
  pdfcreator={LaTeX via pandoc}}

\title{4th Weekly Assignment Answers}
\author{Philosophy 444}
\date{Due 12.01pm, 11 February, 2023}

\begin{document}
\maketitle

\hypertarget{answers-to-one-six}{%
\section{Answers to One-Six}\label{answers-to-one-six}}

There is no pure strategy Nash equilibrium, so there must be a mixed
strategy equilibrium. Let \(x\) be the probability that thief plays S,
so \(1-x\) is the probability of \(N\), and \(y\) be the probability
that firm plays Basic, so \(1-y\) is the probability that they play
Enhanced. So we compute \(x\) and \(y\) by setting the other players'
payouts from the two options to be equal.

\begin{align*}
E(S) &= E(N) \\
1y - 5(1-y) &= 0y + 0(1-y) \\
6y - 5 &= 0 \\
y &= \frac{5}{6} \approx 0.83
\end{align*}

\begin{enumerate}
\def\labelenumi{\arabic{enumi}.}
\tightlist
\item
  Probability of Basic is about 0.83.
\item
  Probability of Enhanced is about 0.17.
\end{enumerate}

\begin{align*}
E(B) &= E(E) \\
0x + 1(1-x) &= 2x + 0(1-x) \\
1-x &= 2x \\
1 &= 3x \\
x &= \frac{1}{3} \approx 0.33
\end{align*}

\begin{enumerate}
\def\labelenumi{\arabic{enumi}.}
\setcounter{enumi}{2}
\tightlist
\item
  Probability of Steal is about 0.33.
\item
  Probability of Not Steal is about 0.67.
\end{enumerate}

To work out expected return for a player, we just need to work it out
for one of the strategies, since the two strategies have the same
expected return.

\begin{align*}
E(B) &= 0x + 1(1-x) \\
 &= \frac{2}{3} \approx 0.67.
\end{align*}

\begin{enumerate}
\def\labelenumi{\arabic{enumi}.}
\setcounter{enumi}{4}
\tightlist
\item
  Firm's expected return is about 0.67.
\end{enumerate}

\begin{align*}
E(N) &= 0y + 0(1-y) \\
 &= 0.
\end{align*}

\begin{enumerate}
\def\labelenumi{\arabic{enumi}.}
\setcounter{enumi}{5}
\tightlist
\item
  Thief's expected return is 0.
\end{enumerate}

\hypertarget{answers-to-seven-nine}{%
\section{Answers to Seven-Nine}\label{answers-to-seven-nine}}

The only answers that could change are those that had the \(-5\) as an
input. Since it wasn't part of the input to calculating \(x\), that is
unchanged. So firm's expected payout, which is just a function of \(x\),
is \textbf{unchanged}. And since Thief's expected return from \(N\) is
still 0, and in equilibrium their payouts are the same from \(S\) and
\(N\), Thief's payout is \textbf{unchanged}. But to work out the
probability that the diamond is stolen, which is \(xy\), we need to
recalculate \(y\). (Why \(xy\)? The diamond gets stolen only if Thief
plays \(S\), which has probability \(x\), and firm plays \(B\), which
has probability \(y\). And these are independent events, so the
probability of them both happening is the product of the probabilities
of each of them happening individually.)

\begin{align*}
E(S) &= E(N) \\
1y - 10(1-y) &= 0y + 0(1-y) \\
11y - 10 &= 0 \\
y &= \frac{10}{11} \approx 0.91
\end{align*}

Since the probability of \(x\) was unchanged, and the probability of
\(y\) went up, the probability that the diamond gets stolen has gone
\textbf{up}. Putting all this together.

\begin{enumerate}
\def\labelenumi{\arabic{enumi}.}
\setcounter{enumi}{6}
\tightlist
\item
  Unchanged.
\item
  Unchanged.
\item
  Up.
\end{enumerate}

\hypertarget{answers-to-ten-eleven}{%
\section{Answers to Ten-Eleven}\label{answers-to-ten-eleven}}

In equilibrium, firm will choose the option that maximises its expected
return from the game. We know that if Firm chooses option one or option
two, their expected payout is \(\frac{2}{3}\). We worked that out for
option one in the first set of questions, and in the second set of
questions we worked out that their payout is unchanged. So what about
option 3. Now the calculation for \(x\) changes, as follows.

\begin{align*}
E(B) &= E(E) \\
0x + 1(1-x) &= 3x + 0(1-x) \\
1-x &= 3x \\
1 &= 4x \\
x &= \frac{1}{4} = 0.25
\end{align*}

\begin{enumerate}
\def\labelenumi{\arabic{enumi}.}
\setcounter{enumi}{9}
\tightlist
\item
  Since firm's expected payout is \(1-x\), and \(x = 0.25\), firm's
  expected payout is 0.75.
\end{enumerate}

The probability that the diamond gets stolen in this situation is
\(xy\), where \(x = 0.25\), and \(y = \frac{5}{6}\). So
\(xy = \frac{5}{24} \approx 0.21\).

\begin{enumerate}
\def\labelenumi{\arabic{enumi}.}
\setcounter{enumi}{10}
\tightlist
\item
  The probability that the diamond gets stolen is about 0.21.
\end{enumerate}

\end{document}
