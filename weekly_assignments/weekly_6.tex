% Options for packages loaded elsewhere
\PassOptionsToPackage{unicode}{hyperref}
\PassOptionsToPackage{hyphens}{url}
%
\documentclass[
  11pt,
]{article}
\usepackage{amsmath,amssymb}
\usepackage{lmodern}
\usepackage{iftex}
\ifPDFTeX
  \usepackage[T1]{fontenc}
  \usepackage[utf8]{inputenc}
  \usepackage{textcomp} % provide euro and other symbols
\else % if luatex or xetex
  \usepackage{unicode-math}
  \defaultfontfeatures{Scale=MatchLowercase}
  \defaultfontfeatures[\rmfamily]{Ligatures=TeX,Scale=1}
  \setmainfont[Scale=MatchLowercase]{SF Pro Text Light}
\fi
% Use upquote if available, for straight quotes in verbatim environments
\IfFileExists{upquote.sty}{\usepackage{upquote}}{}
\IfFileExists{microtype.sty}{% use microtype if available
  \usepackage[]{microtype}
  \UseMicrotypeSet[protrusion]{basicmath} % disable protrusion for tt fonts
}{}
\makeatletter
\@ifundefined{KOMAClassName}{% if non-KOMA class
  \IfFileExists{parskip.sty}{%
    \usepackage{parskip}
  }{% else
    \setlength{\parindent}{0pt}
    \setlength{\parskip}{6pt plus 2pt minus 1pt}}
}{% if KOMA class
  \KOMAoptions{parskip=half}}
\makeatother
\usepackage{xcolor}
\usepackage[margin=1.3in]{geometry}
\usepackage{longtable,booktabs,array}
\usepackage{calc} % for calculating minipage widths
% Correct order of tables after \paragraph or \subparagraph
\usepackage{etoolbox}
\makeatletter
\patchcmd\longtable{\par}{\if@noskipsec\mbox{}\fi\par}{}{}
\makeatother
% Allow footnotes in longtable head/foot
\IfFileExists{footnotehyper.sty}{\usepackage{footnotehyper}}{\usepackage{footnote}}
\makesavenoteenv{longtable}
\usepackage{graphicx}
\makeatletter
\def\maxwidth{\ifdim\Gin@nat@width>\linewidth\linewidth\else\Gin@nat@width\fi}
\def\maxheight{\ifdim\Gin@nat@height>\textheight\textheight\else\Gin@nat@height\fi}
\makeatother
% Scale images if necessary, so that they will not overflow the page
% margins by default, and it is still possible to overwrite the defaults
% using explicit options in \includegraphics[width, height, ...]{}
\setkeys{Gin}{width=\maxwidth,height=\maxheight,keepaspectratio}
% Set default figure placement to htbp
\makeatletter
\def\fps@figure{htbp}
\makeatother
\setlength{\emergencystretch}{3em} % prevent overfull lines
\providecommand{\tightlist}{%
  \setlength{\itemsep}{0pt}\setlength{\parskip}{0pt}}
\setcounter{secnumdepth}{-\maxdimen} % remove section numbering
% https://github.com/rstudio/rmarkdown/issues/337
\let\rmarkdownfootnote\footnote%
\def\footnote{\protect\rmarkdownfootnote}

% https://github.com/rstudio/rmarkdown/pull/252
\usepackage{titling}
\setlength{\droptitle}{-2em}

\pretitle{\vspace{\droptitle}\centering\huge}
\posttitle{\par}

\preauthor{\centering\large\emph}
\postauthor{\par}

\predate{\centering\large\emph}
\postdate{\par}

\usepackage[normalem]{ulem}
\usepackage{nicefrac}
\usepackage{caption}
\usepackage{array}
\usepackage{graphicx}
\usepackage{siunitx}
\usepackage[normalem]{ulem}
\usepackage{colortbl}
\usepackage{multirow}
\usepackage{hhline}
\usepackage{calc}
\usepackage{tabularx}
\usepackage{threeparttable}
\usepackage{wrapfig}
\usepackage{adjustbox}
\usepackage{hyperref}
\ifLuaTeX
  \usepackage{selnolig}  % disable illegal ligatures
\fi
\IfFileExists{bookmark.sty}{\usepackage{bookmark}}{\usepackage{hyperref}}
\IfFileExists{xurl.sty}{\usepackage{xurl}}{} % add URL line breaks if available
\urlstyle{same} % disable monospaced font for URLs
\hypersetup{
  pdftitle={Assignment Week 8},
  pdfauthor={Philosophy 444},
  hidelinks,
  pdfcreator={LaTeX via pandoc}}

\title{Assignment Week 8}
\author{Philosophy 444}
\date{Due 12.01pm, 25 February, 2023}

\begin{document}
\maketitle

\hypertarget{iterated-prisoners-dilemma}{%
\section{Iterated Prisoners' Dilemma}\label{iterated-prisoners-dilemma}}

We are going to run a tournament pitting different strategies against
each other in Iterated Prisoners' Dilemma. Your task is to write a
strategy, and write a short document (about 300 words) explaining why
you picked that strategy. Both of these are due \textbf{Friday, March
12} at \textbf{5pm}. You can use the Canvas site to collaborate with
others on ideas, but you must submit your own work. You will be graded
in part on how well you do in the tournament, and in part on how well
you defend your strategy choice.

In the tournament, your strategy will be matched up in turn with each of
the strategies submitted (including your own), plus five pre-defined
strategies. You play `against' each of these others five times. Each
play will consist of roughly 200 rounds of Prisoners' Dilemma (PD). Each
round of PD consists of a game with the following payouts, in points.
(You are Row, the other player is Column.)

 
  \providecommand{\huxb}[2]{\arrayrulecolor[RGB]{#1}\global\arrayrulewidth=#2pt}
  \providecommand{\huxvb}[2]{\color[RGB]{#1}\vrule width #2pt}
  \providecommand{\huxtpad}[1]{\rule{0pt}{#1}}
  \providecommand{\huxbpad}[1]{\rule[-#1]{0pt}{#1}}

\begin{table}[h!]
\begin{centerbox}
\begin{threeparttable}
 \label{tab:unnamed-chunk-2}
\setlength{\tabcolsep}{0pt}
\begin{tabular}{l l l}


\hhline{>{\huxb{0, 0, 0}{0.4}}|}
\arrayrulecolor{black}

\multicolumn{1}{!{\huxvb{0, 0, 0}{0}}r!{\huxvb{0, 0, 0}{0.4}}}{\huxtpad{0pt + 1em}\raggedleft \hspace{2pt} 
 \hspace{2pt}\huxbpad{6pt}} &
\multicolumn{1}{c!{\huxvb{0, 0, 0}{0}}}{\huxtpad{0pt + 1em}\centering \hspace{2pt} C \hspace{2pt}\huxbpad{6pt}} &
\multicolumn{1}{c!{\huxvb{0, 0, 0}{0}}}{\huxtpad{0pt + 1em}\centering \hspace{2pt} D \hspace{2pt}\huxbpad{6pt}} \tabularnewline[-0.5pt]


\hhline{>{\huxb{0, 0, 0}{0.4}}->{\huxb{0, 0, 0}{0.4}}->{\huxb{0, 0, 0}{0.4}}-}
\arrayrulecolor{black}

\multicolumn{1}{!{\huxvb{0, 0, 0}{0}}r!{\huxvb{0, 0, 0}{0.4}}}{\huxtpad{2pt + 1em}\raggedleft \hspace{2pt} C \hspace{2pt}\huxbpad{2pt}} &
\multicolumn{1}{c!{\huxvb{0, 0, 0}{0}}}{\huxtpad{2pt + 1em}\centering \hspace{2pt} 3, 3 \hspace{2pt}\huxbpad{2pt}} &
\multicolumn{1}{c!{\huxvb{0, 0, 0}{0}}}{\huxtpad{2pt + 1em}\centering \hspace{2pt} 0, 5 \hspace{2pt}\huxbpad{2pt}} \tabularnewline[-0.5pt]


\hhline{>{\huxb{0, 0, 0}{0.4}}|}
\arrayrulecolor{black}

\multicolumn{1}{!{\huxvb{0, 0, 0}{0}}r!{\huxvb{0, 0, 0}{0.4}}}{\huxtpad{2pt + 1em}\raggedleft \hspace{2pt} D \hspace{2pt}\huxbpad{2pt}} &
\multicolumn{1}{c!{\huxvb{0, 0, 0}{0}}}{\huxtpad{2pt + 1em}\centering \hspace{2pt} 5, 0 \hspace{2pt}\huxbpad{2pt}} &
\multicolumn{1}{c!{\huxvb{0, 0, 0}{0}}}{\huxtpad{2pt + 1em}\centering \hspace{2pt} 1, 1 \hspace{2pt}\huxbpad{2pt}} \tabularnewline[-0.5pt]


\hhline{>{\huxb{0, 0, 0}{0.4}}|}
\arrayrulecolor{black}
\end{tabular}
\end{threeparttable}\par\end{centerbox}

\end{table}
 

o you will play each `opponent' roughly 1000 times. The number of points
you each get over those 1000 games will be added to your individual
scores, and then you'll go on to play another player. At the end of the
tournament, when everyone has played everyone, whoever has accumulated
the most points wins.

Note that there is a bit of zero-sum-ness to the game at the end. While
you are playing any individual player, it is not a zero-sum situation;
there are ways for both of you to improve or detract from your overall
position. But your ultimate aim is not just to get a lot of points, it
is to get more points than anyone else in the class.

\newpage

\hypertarget{writing-a-strategy}{%
\subsection{Writing a Strategy}\label{writing-a-strategy}}

To explain to the computer how you want to play, you need to write a
short bit of code. The language we'll use for this is fairly basic -
essentially you will describe a \textbf{Finite State Machine}.

Your machine will at any time be in one of a number of \emph{states}.
There will be finitely many of these - hence the name. A state has three
key characteristics.

\begin{enumerate}
\def\labelenumi{\arabic{enumi}.}
\tightlist
\item
  Whether the machine plays C or D in that state.
\item
  What state the machine moves into if the opponent plays C.
\item
  What state the machine moves into if the opponent plays D.
\end{enumerate}

One of these states, state 0, will be the initial state of the machine.
So here is a very simple machine.

\begin{longtable}[]{@{}lllll@{}}
\toprule()
Number & Play & Move if C & Move if D & Notes \\
\midrule()
\endhead
0 & C & 0 & 1 & Cooperate state \\
1 & D & 0 & 1 & Defect state \\
\bottomrule()
\end{longtable}

This machine will play C at first, since that's the play in state 0.
After that, it mimics what the previous player has done. No matter what
state it is in, it will move into cooperate state if the other player
cooperates, and move into defect state if the other player defects. This
strategy is known as \textbf{Tit-for-Tat}, and is one of the two
pre-defined strategies I mentioned above.

Let's look at a slightly more complicated strategy.

\begin{longtable}[]{@{}lllll@{}}
\toprule()
Number & Play & Move if C & Move if D & Notes \\
\midrule()
\endhead
0 & C & 0 & 1 & Cooperate state \\
1 & C & 0 & 1 & After single defection \\
2 & D & 0 & 2 & After multiple defections \\
\bottomrule()
\end{longtable}

This strategy will start cooperating, then will defect any time the
opponent has defected on the previous two turns. Here's how it does
that. State 0 is cooperate, so it starts off cooperating. It is the base
state; think of it as being happy. State 1 is when the opponent has
defected once. It's what happens when (and only when) the machine is
happy, then sees its opponent defect. In state 1, the machine still
cooperates, but it is vigilant. From now on, any more defections will
send it to state 2, where it defects. But at any stage, a single
cooperation from the opponent will send it back to happy.

Finally, a strategy that keeps count of things. The idea is to play
Tit-for-Tat, but with a fuse. Once the opponent defects four times, we
defect forever. So we need to keep count of defections. We do that by
having more states. In the notes, I'll list the count of defections.

\newpage

\begin{longtable}[]{@{}lllll@{}}
\toprule()
Number & Play & Move if C & Move if D & Notes \\
\midrule()
\endhead
0 & C & 0 & 1 & Base; count=0 \\
1 & D & 2 & 3 & Defect; count=1 \\
2 & C & 2 & 3 & Coop; count=1 \\
3 & D & 4 & 5 & Defect; count=2 \\
4 & C & 4 & 5 & Coop; count=2 \\
5 & D & 6 & 7 & Defect; count=3 \\
6 & C & 6 & 7 & Coop; count=3 \\
7 & D & 7 & 7 & Defect; count=4 \\
\bottomrule()
\end{longtable}

The machine really cares about two things; how many times the opponent
has defected, and what the opponent did on the previous play. At each
time (until it loses patience at the end), it responds to cooperating
with cooperating and staying in place, and it responds to defecting by
defecting and moving closer to the lose patience state. Once it gets to
state 7, it stays there; no matter what the opponent does it will still
defect.

For this assignment, you should submit two things. Your program should
be a plain text file (saved as .txt on your computer) with the following
structure.

\begin{itemize}
\tightlist
\item
  The first three lines are your name, your strategy's name, and the
  number of states in your machine.
\item
  Each line after that is a state. It should say what move to make in
  that state, then a comma, then what state to move to if the opponent
  cooperates, then another comma, then what state to move to if the
  opponent defects.
\end{itemize}

Here, for instance, is how I would submit a file if I was playing
Tit-for-Tat.

Brian Weatherson\\
Very Original Strategy\\
2\\
C, 0, 1\\
D, 0, 1

You should also write a note - roughly 300 words long - about what your
strategy aims to do, and why you chose it. And make sure that your
program has the right syntax; we can't debug programs. If you want to
check for sure, install the Oyun software
\url{https://oyun.charlespence.net} and run the program yourself to see.

We'll compile all the strategies over the weekend, and next week we'll
discuss how they did. Note that the tournament will include five
pre-defined strategies:

\begin{enumerate}
\def\labelenumi{\arabic{enumi}.}
\tightlist
\item
  Tit-for-Tat
\item
  Random, who plays randomly.
\item
  All C, who always cooperates.
\item
  All D, who always defects.
\item
  No forgive, who cooperates until you defect, then always defects.
\end{enumerate}

\end{document}
