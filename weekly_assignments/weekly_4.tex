% Options for packages loaded elsewhere
\PassOptionsToPackage{unicode}{hyperref}
\PassOptionsToPackage{hyphens}{url}
%
\documentclass[
  11pt,
]{article}
\usepackage{amsmath,amssymb}
\usepackage{lmodern}
\usepackage{iftex}
\ifPDFTeX
  \usepackage[T1]{fontenc}
  \usepackage[utf8]{inputenc}
  \usepackage{textcomp} % provide euro and other symbols
\else % if luatex or xetex
  \usepackage{unicode-math}
  \defaultfontfeatures{Scale=MatchLowercase}
  \defaultfontfeatures[\rmfamily]{Ligatures=TeX,Scale=1}
  \setmainfont[Scale=MatchLowercase]{SF Pro Text Light}
\fi
% Use upquote if available, for straight quotes in verbatim environments
\IfFileExists{upquote.sty}{\usepackage{upquote}}{}
\IfFileExists{microtype.sty}{% use microtype if available
  \usepackage[]{microtype}
  \UseMicrotypeSet[protrusion]{basicmath} % disable protrusion for tt fonts
}{}
\makeatletter
\@ifundefined{KOMAClassName}{% if non-KOMA class
  \IfFileExists{parskip.sty}{%
    \usepackage{parskip}
  }{% else
    \setlength{\parindent}{0pt}
    \setlength{\parskip}{6pt plus 2pt minus 1pt}}
}{% if KOMA class
  \KOMAoptions{parskip=half}}
\makeatother
\usepackage{xcolor}
\usepackage[margin=1.5in]{geometry}
\usepackage{graphicx}
\makeatletter
\def\maxwidth{\ifdim\Gin@nat@width>\linewidth\linewidth\else\Gin@nat@width\fi}
\def\maxheight{\ifdim\Gin@nat@height>\textheight\textheight\else\Gin@nat@height\fi}
\makeatother
% Scale images if necessary, so that they will not overflow the page
% margins by default, and it is still possible to overwrite the defaults
% using explicit options in \includegraphics[width, height, ...]{}
\setkeys{Gin}{width=\maxwidth,height=\maxheight,keepaspectratio}
% Set default figure placement to htbp
\makeatletter
\def\fps@figure{htbp}
\makeatother
\setlength{\emergencystretch}{3em} % prevent overfull lines
\providecommand{\tightlist}{%
  \setlength{\itemsep}{0pt}\setlength{\parskip}{0pt}}
\setcounter{secnumdepth}{-\maxdimen} % remove section numbering
% https://github.com/rstudio/rmarkdown/issues/337
\let\rmarkdownfootnote\footnote%
\def\footnote{\protect\rmarkdownfootnote}

% https://github.com/rstudio/rmarkdown/pull/252
\usepackage{titling}
\setlength{\droptitle}{-2em}

\pretitle{\vspace{\droptitle}\centering\huge}
\posttitle{\par}

\preauthor{\centering\large\emph}
\postauthor{\par}

\predate{\centering\large\emph}
\postdate{\par}

\usepackage[normalem]{ulem}
\usepackage{nicefrac}
\usepackage{caption}
\usepackage{array}
\usepackage{graphicx}
\usepackage{siunitx}
\usepackage[normalem]{ulem}
\usepackage{colortbl}
\usepackage{multirow}
\usepackage{hhline}
\usepackage{calc}
\usepackage{tabularx}
\usepackage{threeparttable}
\usepackage{wrapfig}
\usepackage{adjustbox}
\usepackage{hyperref}
\ifLuaTeX
  \usepackage{selnolig}  % disable illegal ligatures
\fi
\IfFileExists{bookmark.sty}{\usepackage{bookmark}}{\usepackage{hyperref}}
\IfFileExists{xurl.sty}{\usepackage{xurl}}{} % add URL line breaks if available
\urlstyle{same} % disable monospaced font for URLs
\hypersetup{
  pdftitle={4th Weekly Assignment},
  pdfauthor={Philosophy 444},
  hidelinks,
  pdfcreator={LaTeX via pandoc}}

\title{4th Weekly Assignment}
\author{Philosophy 444}
\date{Due 12.01pm, 11 February, 2023}

\begin{document}
\maketitle

\hypertarget{questions-one-six}{%
\section{Questions One-Six}\label{questions-one-six}}

A thief wants to steal a particular diamond owned by the city council.
The security firm guarding the diamond knows that the thief might strike
tonight. They have to decide whether to put in Basic security (B) or
Enhanced security (E). The thief has to decide whether to Steal (S) or
Not steal (N) the diamond. The thief can evade basic security, but will
be caught by enhanced security. Here is the payoff table for the two
parties (with thief payouts first, since she's row).

 
  \providecommand{\huxb}[2]{\arrayrulecolor[RGB]{#1}\global\arrayrulewidth=#2pt}
  \providecommand{\huxvb}[2]{\color[RGB]{#1}\vrule width #2pt}
  \providecommand{\huxtpad}[1]{\rule{0pt}{#1}}
  \providecommand{\huxbpad}[1]{\rule[-#1]{0pt}{#1}}

\begin{table}[h!]
\begin{centerbox}
\begin{threeparttable}
 \label{tab:unnamed-chunk-2}
\setlength{\tabcolsep}{0pt}
\begin{tabular}{l l l}


\hhline{>{\huxb{0, 0, 0}{0.4}}|}
\arrayrulecolor{black}

\multicolumn{1}{!{\huxvb{0, 0, 0}{0}}r!{\huxvb{0, 0, 0}{0.4}}}{\huxtpad{0pt + 1em}\raggedleft \hspace{2pt} 
 \hspace{2pt}\huxbpad{6pt}} &
\multicolumn{1}{c!{\huxvb{0, 0, 0}{0}}}{\huxtpad{0pt + 1em}\centering \hspace{2pt} B \hspace{2pt}\huxbpad{6pt}} &
\multicolumn{1}{c!{\huxvb{0, 0, 0}{0}}}{\huxtpad{0pt + 1em}\centering \hspace{2pt} E \hspace{2pt}\huxbpad{6pt}} \tabularnewline[-0.5pt]


\hhline{>{\huxb{0, 0, 0}{0.4}}->{\huxb{0, 0, 0}{0.4}}->{\huxb{0, 0, 0}{0.4}}-}
\arrayrulecolor{black}

\multicolumn{1}{!{\huxvb{0, 0, 0}{0}}r!{\huxvb{0, 0, 0}{0.4}}}{\huxtpad{2pt + 1em}\raggedleft \hspace{2pt} S \hspace{2pt}\huxbpad{2pt}} &
\multicolumn{1}{c!{\huxvb{0, 0, 0}{0}}}{\huxtpad{2pt + 1em}\centering \hspace{2pt} 1, 0 \hspace{2pt}\huxbpad{2pt}} &
\multicolumn{1}{c!{\huxvb{0, 0, 0}{0}}}{\huxtpad{2pt + 1em}\centering \hspace{2pt} -5, 2 \hspace{2pt}\huxbpad{2pt}} \tabularnewline[-0.5pt]


\hhline{>{\huxb{0, 0, 0}{0.4}}|}
\arrayrulecolor{black}

\multicolumn{1}{!{\huxvb{0, 0, 0}{0}}r!{\huxvb{0, 0, 0}{0.4}}}{\huxtpad{2pt + 1em}\raggedleft \hspace{2pt} N \hspace{2pt}\huxbpad{2pt}} &
\multicolumn{1}{c!{\huxvb{0, 0, 0}{0}}}{\huxtpad{2pt + 1em}\centering \hspace{2pt} 0, 1 \hspace{2pt}\huxbpad{2pt}} &
\multicolumn{1}{c!{\huxvb{0, 0, 0}{0}}}{\huxtpad{2pt + 1em}\centering \hspace{2pt} 0, 0 \hspace{2pt}\huxbpad{2pt}} \tabularnewline[-0.5pt]


\hhline{>{\huxb{0, 0, 0}{0.4}}|}
\arrayrulecolor{black}
\end{tabular}
\end{threeparttable}\par\end{centerbox}

\end{table}
 

Intuitively, the thief wants the diamond, doesn't want to get caught,
and cares way more about getting caught that the diamond. The security
firm doesn't want the bad reputation of the diamond being stolen on
their watch, or the cost of running Enhanced security, but they would
like the reward for catching the thief.

The following questions are all about the mixed strategy Nash
equilibrium of this game. (Answers should be in decimal form, not
fractions, and accurate to two decimal places.)

\begin{enumerate}
\def\labelenumi{\arabic{enumi}.}
\tightlist
\item
  What is the probability of B?
\item
  What is the probability of E?
\item
  What is the probability of S?
\item
  What is the probability of N?
\item
  What is firm's expected return?
\item
  What is thief's expected return?
\end{enumerate}

\hypertarget{questions-seven-nine}{%
\section{Questions Seven-Nine}\label{questions-seven-nine}}

A tough-on-crime faction on the city council wants to prevent theft by
increasing the punishment for being caught stealing. If they had their
way, the payout in the upper-right cell (i.e., SE), would be -10, 2. If
this change was made, how would the new equilibrium compare to the old
equilibrium?

\begin{enumerate}
\def\labelenumi{\arabic{enumi}.}
\setcounter{enumi}{6}
\tightlist
\item
  Would firm's expected payout go up, go down, or be unchanged?
\item
  Would thief's expected payout go up, go down, or be unchanged?
\item
  Would the probability that the diamond gets stolen go up, go down, or
  be unchanged?
\end{enumerate}

\hypertarget{questions-ten---eleven}{%
\section{Questions Ten - Eleven}\label{questions-ten---eleven}}

The tough-on-crime faction is defeated, but there is a new dispute over
exactly what we should do to the payout in the top right cell. There are
three options

\begin{itemize}
\tightlist
\item
  Option one: leave it unchanged.
\item
  Option two: change it to -10, 2, as the tough-on-crime faction wanted.
\item
  Option three: change it to -5, 3, i.e., increase the reward for
  catching the thief.
\end{itemize}

Happily for the security firm, a nice bit of regulatory capture means
that they control the deciding vote. So this is now a dynamic game. Firm
will choose option one, two or three, and then choose Basic or Enhanced.
Thief will find out whether one, two or three was chosen, then choose
Steal or Not. (Thief will not know whether Basic or Enhanced was chosen,
since that's a corporate decision by the firm. But they can read the
city council minutes like anyone else.)

The following questions concern the subgame perfect equilibrium of this
new game.

\begin{enumerate}
\def\labelenumi{\arabic{enumi}.}
\setcounter{enumi}{9}
\tightlist
\item
  What is firm's expected payout?
\item
  What is the probability that the diamond gets stolen?
\end{enumerate}

\end{document}
