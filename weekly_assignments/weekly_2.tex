% Options for packages loaded elsewhere
\PassOptionsToPackage{unicode}{hyperref}
\PassOptionsToPackage{hyphens}{url}
%
\documentclass[
  11pt,
]{article}
\usepackage{amsmath,amssymb}
\usepackage{lmodern}
\usepackage{iftex}
\ifPDFTeX
  \usepackage[T1]{fontenc}
  \usepackage[utf8]{inputenc}
  \usepackage{textcomp} % provide euro and other symbols
\else % if luatex or xetex
  \usepackage{unicode-math}
  \defaultfontfeatures{Scale=MatchLowercase}
  \defaultfontfeatures[\rmfamily]{Ligatures=TeX,Scale=1}
  \setmainfont[Scale=MatchLowercase]{SF Pro Text Light}
\fi
% Use upquote if available, for straight quotes in verbatim environments
\IfFileExists{upquote.sty}{\usepackage{upquote}}{}
\IfFileExists{microtype.sty}{% use microtype if available
  \usepackage[]{microtype}
  \UseMicrotypeSet[protrusion]{basicmath} % disable protrusion for tt fonts
}{}
\makeatletter
\@ifundefined{KOMAClassName}{% if non-KOMA class
  \IfFileExists{parskip.sty}{%
    \usepackage{parskip}
  }{% else
    \setlength{\parindent}{0pt}
    \setlength{\parskip}{6pt plus 2pt minus 1pt}}
}{% if KOMA class
  \KOMAoptions{parskip=half}}
\makeatother
\usepackage{xcolor}
\usepackage[margin=1in]{geometry}
\usepackage{longtable,booktabs,array}
\usepackage{calc} % for calculating minipage widths
% Correct order of tables after \paragraph or \subparagraph
\usepackage{etoolbox}
\makeatletter
\patchcmd\longtable{\par}{\if@noskipsec\mbox{}\fi\par}{}{}
\makeatother
% Allow footnotes in longtable head/foot
\IfFileExists{footnotehyper.sty}{\usepackage{footnotehyper}}{\usepackage{footnote}}
\makesavenoteenv{longtable}
\usepackage{graphicx}
\makeatletter
\def\maxwidth{\ifdim\Gin@nat@width>\linewidth\linewidth\else\Gin@nat@width\fi}
\def\maxheight{\ifdim\Gin@nat@height>\textheight\textheight\else\Gin@nat@height\fi}
\makeatother
% Scale images if necessary, so that they will not overflow the page
% margins by default, and it is still possible to overwrite the defaults
% using explicit options in \includegraphics[width, height, ...]{}
\setkeys{Gin}{width=\maxwidth,height=\maxheight,keepaspectratio}
% Set default figure placement to htbp
\makeatletter
\def\fps@figure{htbp}
\makeatother
\setlength{\emergencystretch}{3em} % prevent overfull lines
\providecommand{\tightlist}{%
  \setlength{\itemsep}{0pt}\setlength{\parskip}{0pt}}
\setcounter{secnumdepth}{-\maxdimen} % remove section numbering
% https://github.com/rstudio/rmarkdown/issues/337
\let\rmarkdownfootnote\footnote%
\def\footnote{\protect\rmarkdownfootnote}

% https://github.com/rstudio/rmarkdown/pull/252
\usepackage{titling}
\setlength{\droptitle}{-2em}

\pretitle{\vspace{\droptitle}\centering\huge}
\posttitle{\par}

\preauthor{\centering\large\emph}
\postauthor{\par}

\predate{\centering\large\emph}
\postdate{\par}

\usepackage[normalem]{ulem}
\usepackage{gensymb}
\usepackage{nicefrac}
\usepackage{caption}
\usepackage{istgame}
\ifLuaTeX
  \usepackage{selnolig}  % disable illegal ligatures
\fi
\IfFileExists{bookmark.sty}{\usepackage{bookmark}}{\usepackage{hyperref}}
\IfFileExists{xurl.sty}{\usepackage{xurl}}{} % add URL line breaks if available
\urlstyle{same} % disable monospaced font for URLs
\hypersetup{
  pdftitle={Assignment Week 3},
  pdfauthor={Philosophy 444},
  hidelinks,
  pdfcreator={LaTeX via pandoc}}

\title{Assignment Week 3}
\author{Philosophy 444}
\date{Due 12.01pm, 21 January, 2023}

\begin{document}
\maketitle

Most of the questions here ask you to solve a game using backwards
induction. That is, we want you to say what strategies each player will
play, if they use backwards induction reasoning. Remember that a
\emph{strategy} in this sense has to say what a player would do at
\emph{every} node, including nodes that are ruled out by their earlier
moves.

\hypertarget{questions-1-2}{%
\subsection{Questions 1-2}\label{questions-1-2}}

Here is the game tree for (one version of) the centipede game. The game
works as follows. Before each turn, the experimenter puts \$2 into a pot
in the middle of the table. This pot starts empty, and grows with every
turn. At each turn, players choose whether to \emph{Take} or
\emph{Pass}. If they take, the money in the pot is split and the game
ends. But it isn't split evenly; the player who takes the money gets \$2
more. So if there is \$68 in the pot, the player who takes gets \$35,
and the other player gets \$33. The game ends when either one player
takes the money, or if player 2 passes with \$200 in the pot. At this
point, both players get \$100. Assume that both players only care about
getting as much money for themselves as they can.

\begin{istgame}[scale=1.5]
   \setistgrowdirection{south east}
   \xtdistance{10mm}{20mm}
   \istroot(0)[initial node]{1}
     \istb{Take}[r]{(2,0)}[b]  \istb{Pass}[a]  \endist
   \istroot(1)(0-2){2}
     \istb{Take}[r]{(1,3)}[b]  \istb{Pass}[a]  \endist
   \istroot(2)(1-2){1}
     \istb{Take}[r]{(4,2)}[b]  \istb{Pass}[a]  \endist
   \xtInfoset(2-2)([xshift=5mm]2-2)
   %-------------
   \istroot(3)([xshift=5mm]2-2){2}
       \istb{Take}[r]{(97,99)}[b]  \istb{Pass}[a]  \endist
   \istroot(4)(3-2){1}
       \istb{Take}[r]{(100,98)}[b]  \istb{Pass}[a]  \endist
    \istroot(5)(4-2){2}
        \istb{Take}[r]{(99,101)}[b]  \istb{Pass}[a]{(100,100)}[r]  \endist
\end{istgame}

\begin{enumerate}
\def\labelenumi{\arabic{enumi}.}
\tightlist
\item
  Use backwards induction to find the solution to the game.
\item
  Make two changes to the game. First, assume that there is a wealth tax
  in place - anything that a player collects over \$50 has to be handed
  over to the government. Second, assume that players are
  \emph{minimally} altruistic. Given a choice between two options where
  they get the same amount, they prefer the other player to get more
  rather than less. (But they would still prefer to get more money than
  less, even if this costs the other player a lot.) Use backwards
  induction to solve this revised game.
\end{enumerate}

Continues on next page.

\newpage

\hypertarget{questions-3-5}{%
\subsection{Questions 3-5}\label{questions-3-5}}

There are two distinct proposals, A and B, being debated in Washington.
The Congress likes proposal A, and the president likes proposal B. The
proposals are not mutually exclusive; either or both or neither may
become law. Thus there are four possible outcomes, and the rankings of
the two sides are as follows (where as always a larger number means they
prefer it).

\begin{longtable}[]{@{}lcc@{}}
\caption{Schoolhouse Rock Game}\tabularnewline
\toprule()
& Congress & President \\
\midrule()
\endfirsthead
\toprule()
& Congress & President \\
\midrule()
\endhead
A becomes law & 4 & 1 \\
B becomes law & 1 & 4 \\
Both A and B become law & 3 & 3 \\
Neither A nor B becomes law & 2 & 2 \\
\bottomrule()
\end{longtable}

The way legislation works (in this game) is as follows. First, Congress
decides whether to pass a bill, and if so whether it will contain A
only, B only, or both. Then the president decides whether to veto the
bill or not. And that's it. (For simplicity, assume that there is no
veto-override mechanism, and that there is no chance that the issue will
get revisited.)

\begin{enumerate}
\def\labelenumi{\arabic{enumi}.}
\setcounter{enumi}{2}
\tightlist
\item
  Use backwards induction to solve this game.
\item
  Change the rules, so the President gets a `line-item veto'. That is,
  if Congress passes a law with both A and B in it, the President can
  veto just one of them, and let the other pass into law. Now, use
  backwards induction to solve \emph{this} game.
\item
  Explain intuitively why these games are different.
\end{enumerate}

\hypertarget{due-friday-january-20-at-5pm}{%
\subsection{Due Friday January 20, at
5pm}\label{due-friday-january-20-at-5pm}}

\end{document}
