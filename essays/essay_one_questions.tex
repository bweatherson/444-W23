% Options for packages loaded elsewhere
\PassOptionsToPackage{unicode}{hyperref}
\PassOptionsToPackage{hyphens}{url}
%
\documentclass[
  11pt,
]{article}
\usepackage{amsmath,amssymb}
\usepackage{lmodern}
\usepackage{setspace}
\usepackage{iftex}
\ifPDFTeX
  \usepackage[T1]{fontenc}
  \usepackage[utf8]{inputenc}
  \usepackage{textcomp} % provide euro and other symbols
\else % if luatex or xetex
  \usepackage{unicode-math}
  \defaultfontfeatures{Scale=MatchLowercase}
  \defaultfontfeatures[\rmfamily]{Ligatures=TeX,Scale=1}
  \setmainfont[Scale=MatchLowercase]{Merriweather}
\fi
% Use upquote if available, for straight quotes in verbatim environments
\IfFileExists{upquote.sty}{\usepackage{upquote}}{}
\IfFileExists{microtype.sty}{% use microtype if available
  \usepackage[]{microtype}
  \UseMicrotypeSet[protrusion]{basicmath} % disable protrusion for tt fonts
}{}
\makeatletter
\@ifundefined{KOMAClassName}{% if non-KOMA class
  \IfFileExists{parskip.sty}{%
    \usepackage{parskip}
  }{% else
    \setlength{\parindent}{0pt}
    \setlength{\parskip}{6pt plus 2pt minus 1pt}}
}{% if KOMA class
  \KOMAoptions{parskip=half}}
\makeatother
\usepackage{xcolor}
\usepackage[margin=1.5in]{geometry}
\usepackage{graphicx}
\makeatletter
\def\maxwidth{\ifdim\Gin@nat@width>\linewidth\linewidth\else\Gin@nat@width\fi}
\def\maxheight{\ifdim\Gin@nat@height>\textheight\textheight\else\Gin@nat@height\fi}
\makeatother
% Scale images if necessary, so that they will not overflow the page
% margins by default, and it is still possible to overwrite the defaults
% using explicit options in \includegraphics[width, height, ...]{}
\setkeys{Gin}{width=\maxwidth,height=\maxheight,keepaspectratio}
% Set default figure placement to htbp
\makeatletter
\def\fps@figure{htbp}
\makeatother
\setlength{\emergencystretch}{3em} % prevent overfull lines
\providecommand{\tightlist}{%
  \setlength{\itemsep}{0pt}\setlength{\parskip}{0pt}}
\setcounter{secnumdepth}{5}
\usepackage[italic]{mathastext}
\ifLuaTeX
  \usepackage{selnolig}  % disable illegal ligatures
\fi
\IfFileExists{bookmark.sty}{\usepackage{bookmark}}{\usepackage{hyperref}}
\IfFileExists{xurl.sty}{\usepackage{xurl}}{} % add URL line breaks if available
\urlstyle{same} % disable monospaced font for URLs
\hypersetup{
  pdftitle={Essay Questions},
  pdfauthor={Brian Weatherson},
  hidelinks,
  pdfcreator={LaTeX via pandoc}}

\title{Essay Questions}
\author{Brian Weatherson}
\date{Due March 31}

\begin{document}
\maketitle

\setstretch{1.15}
You can write on any particular part of the book that you like, but I
know that it is hard to come up with an essay question. So here are some
that you can choose from. (Or, again, you can do an essay on any part of
the book that you would like, as long as you're clear what you are
writing about.)

\begin{enumerate}
\def\labelenumi{\arabic{enumi}.}
\tightlist
\item
  Describe some ways in which the evolutionary game theory that O'Connor
  uses differs from the classical, rational choice, game theory that we
  studied in Bonanno's textbook. How does the use of evolutionary game
  theory make her models more plausible? How does it make them less
  plausible.
\item
  Compare the coordination games that are central to part I with the
  bargaining games that are central to part II. Which real world
  situations are best modelled by one or other of these games? Which
  real world situations relevant to the growth or persistence of gender
  inequality are not well modelled by either?
\item
  Which parts of O'Connor's argument require an assumption that society
  will be organised around male-female pairs? Which parts do not require
  this assumption?
\item
  O'Connor's explanation for the development of gender inequality relies
  on the idea that there are advantages to using \emph{types} to resolve
  certain kinds of games, especially coordination games and bargaining
  games. If this is why gender inequality develops, should we expect
  gender norms and gender inequality to be everywhere, or should we
  expect that some communities will use different types to play the role
  that gender plays in societies we are familiar with?
\end{enumerate}

\end{document}
