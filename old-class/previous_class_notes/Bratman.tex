\documentclass[11pt,]{article}
\usepackage{lmodern}
\usepackage{amssymb,amsmath}
\usepackage{ifxetex,ifluatex}
\usepackage{fixltx2e} % provides \textsubscript
\ifnum 0\ifxetex 1\fi\ifluatex 1\fi=0 % if pdftex
  \usepackage[T1]{fontenc}
  \usepackage[utf8]{inputenc}
\else % if luatex or xelatex
  \ifxetex
    \usepackage{mathspec}
  \else
    \usepackage{fontspec}
  \fi
  \defaultfontfeatures{Ligatures=TeX,Scale=MatchLowercase}
    \setmainfont[]{SF Pro Text Light}
\fi
% use upquote if available, for straight quotes in verbatim environments
\IfFileExists{upquote.sty}{\usepackage{upquote}}{}
% use microtype if available
\IfFileExists{microtype.sty}{%
\usepackage{microtype}
\UseMicrotypeSet[protrusion]{basicmath} % disable protrusion for tt fonts
}{}
\usepackage[margin=1in]{geometry}
\usepackage{hyperref}
\hypersetup{unicode=true,
            pdftitle={Bratman on Group Action},
            pdfauthor={Philosophy 444},
            pdfborder={0 0 0},
            breaklinks=true}
\urlstyle{same}  % don't use monospace font for urls
\usepackage{graphicx,grffile}
\makeatletter
\def\maxwidth{\ifdim\Gin@nat@width>\linewidth\linewidth\else\Gin@nat@width\fi}
\def\maxheight{\ifdim\Gin@nat@height>\textheight\textheight\else\Gin@nat@height\fi}
\makeatother
% Scale images if necessary, so that they will not overflow the page
% margins by default, and it is still possible to overwrite the defaults
% using explicit options in \includegraphics[width, height, ...]{}
\setkeys{Gin}{width=\maxwidth,height=\maxheight,keepaspectratio}
\IfFileExists{parskip.sty}{%
\usepackage{parskip}
}{% else
\setlength{\parindent}{0pt}
\setlength{\parskip}{6pt plus 2pt minus 1pt}
}
\setlength{\emergencystretch}{3em}  % prevent overfull lines
\providecommand{\tightlist}{%
  \setlength{\itemsep}{0pt}\setlength{\parskip}{0pt}}
\setcounter{secnumdepth}{0}
% Redefines (sub)paragraphs to behave more like sections
\ifx\paragraph\undefined\else
\let\oldparagraph\paragraph
\renewcommand{\paragraph}[1]{\oldparagraph{#1}\mbox{}}
\fi
\ifx\subparagraph\undefined\else
\let\oldsubparagraph\subparagraph
\renewcommand{\subparagraph}[1]{\oldsubparagraph{#1}\mbox{}}
\fi

%%% Use protect on footnotes to avoid problems with footnotes in titles
\let\rmarkdownfootnote\footnote%
\def\footnote{\protect\rmarkdownfootnote}

%%% Change title format to be more compact
\usepackage{titling}

% Create subtitle command for use in maketitle
\providecommand{\subtitle}[1]{
  \posttitle{
    \begin{center}\large#1\end{center}
    }
}

\setlength{\droptitle}{-2em}

  \title{Bratman on Group Action}
    \pretitle{\vspace{\droptitle}\centering\huge}
  \posttitle{\par}
    \author{Philosophy 444}
    \preauthor{\centering\large\emph}
  \postauthor{\par}
      \predate{\centering\large\emph}
  \postdate{\par}
    \date{18 November, 2019}

\usepackage{mathastext}
\usepackage{nicefrac}

\begin{document}
\maketitle

\hypertarget{two-initial-points}{%
\section{Two Initial Points}\label{two-initial-points}}

\begin{enumerate}
\def\labelenumi{\arabic{enumi}.}
\tightlist
\item
  Gilbert on authority. How much real-life authority comes from being
  the person who has been making the decisions and has others going
  along with them? (See, for example, what's going on in Bolivia right
  now.)
\item
  Bratman's picture. Start with a simple theory, and add complications
  to deal with puzzle cases. The history of philosophy is that this
  doesn't end well. Work through the details, but be cautious.
\end{enumerate}

\hypertarget{three-conditions}{%
\section{Three Conditions}\label{three-conditions}}

\begin{itemize}
\tightlist
\item
  Mutual Responsiveness
\item
  Commitment to joint activity; i.e., we both intend to do this very
  activity, under something like this description.
\item
  Commitment to mutual support; i.e., we both intend to help the other
  should they falter, and not claim all the glory.
\end{itemize}

The last condition is a strengthening of the idea that cooperative
activity is not side-by-side activity.

\hypertarget{can-i-intend-that-we-f}{%
\section{Can I intend that we F?}\label{can-i-intend-that-we-f}}

\begin{itemize}
\tightlist
\item
  Sure - I can intend to spend a sunny day at the beach, without
  intending the sunshine
\item
  I can even, I think, do it without being 100\% sure of the sunshine
\item
  Don't need complete control
\item
  Another example: I can intend to holiday in Paris, although I can't
  control all the aspects of my getting to Paris.
\end{itemize}

\hypertarget{mesh}{%
\section{Mesh}\label{mesh}}

\begin{itemize}
\tightlist
\item
  As stated this feels too strong.
\item
  Imagine that your job is to get the paint. I have views about where to
  get the paint from (as in Bratman's example), but also how to drive
  there. This feels like it shouldn't matter; it's your job to get the
  paint.
\item
  How much counterfactual resiliency of mesh is interesting here.
  Bratman's pun about `beyond the pale' drives an interesting point.
\item
  In practice, it can feel almost coercive to include a strong
  restriction on sub-plans.
\end{itemize}

\hypertarget{reflexivity}{%
\section{Reflexivity}\label{reflexivity}}

\begin{itemize}
\tightlist
\item
  Long tradition of thinking about the point of intention is that action
  is brought about as a result of this very intention.
\item
  Bratman is extending this to a group setting.
\end{itemize}

\hypertarget{what-exactly-counts-as-coercion}{%
\section{What exactly counts as
coercion?}\label{what-exactly-counts-as-coercion}}

If I dictate all the terms, that's coercive. But where we draw the line
between power imbalance and coercion is tricky. (Famously!)

\hypertarget{what-counts-as-support}{%
\section{What Counts as Support}\label{what-counts-as-support}}

\begin{itemize}
\tightlist
\item
  The single possible kind of support feels really weak.
\item
  What if there is a kind of thing I can't stand seeing anyone suffer
  through.
\item
  Feels like we need a generic here not an existential
\end{itemize}

\hypertarget{explicit-disavowal-of-commitment}{%
\section{Explicit disavowal of
commitment}\label{explicit-disavowal-of-commitment}}

\begin{itemize}
\tightlist
\item
  Discuss these for a bit
\item
  Do they defeat shared intention; shared cooperative activity?
\item
  Do they
\end{itemize}

\hypertarget{can-we-get-commitment-from-elsewhere}{%
\section{Can we get commitment from
elsewhere}\label{can-we-get-commitment-from-elsewhere}}

\begin{itemize}
\tightlist
\item
  Typical case - assurances and responsibility to live up to assurances
\item
  Big picture - let people know what game they are playing
\item
  Let people rule out options
\item
  This is a good thing to do
\end{itemize}

\hypertarget{are-competitive-games-scas-or-group-actions}{%
\section{Are Competitive Games SCAs, or Group
Actions}\label{are-competitive-games-scas-or-group-actions}}

\begin{itemize}
\tightlist
\item
  I mean sort of yes, sort of no.
\item
  What turns on this question?
\end{itemize}


\end{document}
