\documentclass[11pt]{article}
\usepackage{fullpage}
\usepackage{tabulary}
\usepackage{booktabs}
\usepackage{multirow}

\usepackage[no-math]{fontspec}
%\setmainfont[Ligatures=TeX, BoldFont = Garamond Bold]{Adobe Garamond Pro}
%\setmainfont[Mapping=tex-text, BoldFont = Baskerville SemiBold]{Baskerville}
%\setmainfont[Mapping=tex-text, BoldFont = ACaslonPro-Semibold.otf]{Adobe Caslon Pro}
\setmainfont[Ligatures=TeX, Mapping=tex-text, BoldFont = SF Pro Text Semibold]{SF Pro Text Light}
%\setmainfont{Old Standard TT}
%\setmainfont{Gentium Basic}
\usepackage[defaultmathsizes,italic]{mathastext}

\begin{document}

\begin{center}
\begin{large}
\textbf{Groups and Choices, September 4}
\end{large}
\end{center}

\noindent Your task is to figure out a strategy to play in each of the three games below. On an index card write down your name, and your strategy choice for each of the following three games.

This is not actually for credit, though I want you to pretend that it is. That is, I want you to pretend that the payoffs for the games are grades in an assignment that will count for 5\% of the course credit.

Each of the three games has roles for two players, you and 'Other'. What I will do is select two cards at random, select a game at random, and play out how that game would go if the two of you played that game. That will determine your grade for the assignment.

In each game, there are two choices for you, and two choices for Other, making for four (i.e., two times two) outcomes. Correspondingly, each of the games has four cells in its outcome table. You will choose which row is played, and Other will determine which column is played. The first grade listed is your grade, the second grade listed is Other’s grade. (Note that the games are all symmetric, which makes this kind of game play possible. From someone else’s perspective, you are Other.)

You might think it is unfair that (a) who you are paired with is random, and (b) which game you play is random. Both of these are true; some games will definitely yield higher averages than others, and some players will play the same way but get different outcomes solely due to the luck of the draw. I'll simply note that (a) life is often unfair, and (b) this isn't actually for credit. If you’d like, in the spirit of pretending to play this for credit, you should pretend to get upset about the unfairness of it all.

\bigskip
\begin{large}
\textbf{Game 1}
\end{large}

\begin{center}
\begin{tabular}{l r | c c}
 & & \multicolumn{2}{c}{\textbf{Other}} \\
& & X & Y \\ \hline
\multirow{2}{*}{\textbf{You}} & X & A, A & D, B+ \\ & Y & B+, D & B, B \\
\end{tabular}
\end{center}

\bigskip
\begin{large}
\textbf{Game 2}
\end{large}

\begin{center}
\begin{tabular}{l r | c c}
 & & \multicolumn{2}{c}{\textbf{Other}} \\
& & X & Y \\ \hline
\multirow{2}{*}{\textbf{You}} & X & B-, B- & A, B+ \\ & Y & B+, A & B-, B- \\
\end{tabular}
\end{center}

\bigskip
\begin{large}
\textbf{Game 3}
\end{large}

\begin{center}
\begin{tabular}{l r | c c}
 & & \multicolumn{2}{c}{\textbf{Other}} \\
& & X & Y \\ \hline
\multirow{2}{*}{\textbf{You}} & X & D, D & A, C \\ & Y & C, A & B+, B+
\end{tabular}
\end{center}

\end{document}