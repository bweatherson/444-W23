\documentclass[11pt]{article}
\usepackage{fullpage}
\usepackage{tabulary}
\usepackage{mdwlist}
\usepackage{hyperref}
\usepackage{istgame}
\linespread{1.2}

\begin{document}

\begin{center}
{\Large \textbf{Backwards Induction}}
\end{center}

\noindent Most of the questions here ask you to solve a game using backwards induction. That is, we want you to say what strategies each player will play, if they use backwards induction reasoning. Remember that a \textit{strategy} in this sense has to say what a player would do at \textbf{every} node, including nodes that are ruled out by their earlier moves.

\section*{Questions One-Two}

Here is the game tree for (one version of) the centipede game. The game works as follows. Before each turn, the experimenter puts \$2 into a pot in the middle of the table. This pot starts empty, and grows with every turn. At each turn, players choose whether to \textit{Take} or \textit{Pass}. If they take, the money in the pot is split and the game ends. But it isn't split evenly; the player who takes the money gets \$2 more. So if there is \$68 in the pot, the player who takes gets \$35, and the other player gets \$33. The game ends when either one player takes the money, or if player 2 passes with \$200 in the pot. At this point, both players get \$100. Assume that both players only care about getting as much money for themselves as they can.

\bigskip

\begin{istgame}[scale=1.5]
   \setistgrowdirection{south east}
   \xtdistance{10mm}{20mm}
   \istroot(0)[initial node]{1}
     \istb{Take}[r]{(2,0)}[b]  \istb{Pass}[a]  \endist
   \istroot(1)(0-2){2}
     \istb{Take}[r]{(1,3)}[b]  \istb{Pass}[a]  \endist
   \istroot(2)(1-2){1}
     \istb{Take}[r]{(4,2)}[b]  \istb{Pass}[a]  \endist
   \xtInfoset(2-2)([xshift=5mm]2-2)
   %-------------
   \istroot(3)([xshift=5mm]2-2){2}
       \istb{Take}[r]{(97,99)}[b]  \istb{Pass}[a]  \endist
   \istroot(4)(3-2){1}
       \istb{Take}[r]{(100,98)}[b]  \istb{Pass}[a]  \endist
    \istroot(5)(4-2){2}
        \istb{Take}[r]{(99,101)}[b]  \istb{Pass}[a]{(100,100)}[r]  \endist
\end{istgame}

\bigskip

\begin{enumerate*}
\item Use backwards induction to find the solution to the game.
\item Change the rules so that player 1 is limited to getting at most \$$m$, and player 2 is limited to getting at most \$$n$ (for some $m, n \in \{1, 2, \dots, 101\}$). Use backwards induction to find the solution (or solutions) to such a game. (Note that the solution(s) will be different for different values of $m, n$ - you should say what the answer is in terms of $m$ and $n$.)
\end{enumerate*}

\newpage

\section*{Questions Three-Five}

There are two distinct proposals, A and B, being debated in Washington. The Congress likes proposal A, and the president likes proposal B. The proposals are not mutually exclusive; either or both or neither may become law. Thus there are four possible outcomes, and the rankings of the two sides are as follows (where as always a larger number means they prefer it).

\bigskip

\begin{center}
\begin{tabular}{l c c}
\textbf{Outcome} & \textbf{Congress} & \textbf{President} \\ \hline
A becomes law & 4 & 1 \\
B becomes law & 1 & 4 \\
A and B both become law & 3 & 3 \\
Neither A nor B become law & 3 & 3 \\
\end{tabular}
\end{center}

\bigskip
\noindent The way legislation works is like this. First, Congress decides whether to pass a bill, and if so whether it will contain A only, B only, or both. Then the president decides whether to veto the bill or not. And that's it. (For simplicity, assume that there is no veto mechanism, and that there is no chance that the issue will get revisited.)

\begin{enumerate*}
\setcounter{enumi}{2}
\item Use backwards induction to solve this game.
\item Change the rules, so the President gets a `line-item veto'. That is, if Congress passes a law with both A and B in it, the President can veto just one of them, and let the other pass into law. Now, use backwards induction to solve \textit{this} game.
\item Explain intuitively why these games are different.
\end{enumerate*}

\subsection*{Due Friday Jan 19th, at 5pm}

\end{document}

