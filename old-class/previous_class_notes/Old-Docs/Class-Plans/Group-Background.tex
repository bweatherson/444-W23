\input{../styles/handout-leader}
\def\mytitle{Group Belief, Background Notes}
\def\myauthor{Brian Weatherson}
\def\mydate{March 15, 2018}
\input{../styles/handout-begin}

\footnotesize

\section{Metaphorical vs Literal}
\label{metaphoricalvsliteral}

We attribute, or at least appear to attribute, mental states to things all the time.

\begin{itemize}
\item{} The dog wants his owner to come home.

\item{} The computer thinks there is a Bluetooth keyboard attached to it.

\end{itemize}
Big question: Are these to be taken literally, or just as a kind of metaphor? We can ask the same question about groups.

\section{Belief vs Acceptance}
\label{beliefvsacceptance}

Belief paradigmatically has three characteristics.

\begin{itemize}
\item{} It is grounded in evidence (or at least we think it is)

\item{} It can serve as a premise in reasoning

\item{} It can ground action

\end{itemize}
To \emph{accept} something is to be in a state that has the last two features - you use \emph{p} as a premise in theoretical and practical reasoning, but you don't think this is because of the good evidence for \emph{p}. It might be for practical reasons (e.g., the truth is too complicated to reason with). This brings up two questions:

\begin{enumerate}
\item{} Do groups ever believe things, or just accept them?

\item{} Are group beliefs based in beliefs of the members, or acceptances?

\end{enumerate}
For one particular instance of this, see Tuomela's discussion of cultural practices.

\section{Implicit vs Explicit}
\label{implicitvsexplicit}

Here's another foundational question about the nature of belief. How explicitly do you have to represent that \emph{p} in order to believe that \emph{p}? For a specific example, before I asked you this, did you believe or not believe the following is true:

\begin{itemize}
\item{} Cape Town, South Africa is south of Cape Cod, Massachussetts

\end{itemize}
There are two big pictures here. On one, beliefs are (as Frank Ramsey said) ``a map by which we steer''. To believe \emph{p} is for your mental map to represent the world as making \emph{p} true. And my mental map includes all kinds of geographic facts that I hadn't made explicit - like this one. On another, beliefs are something like sentences in the language of thought (this idea is best developed by Jerry Fodor). To believe something is to have it `written' in your `belief box'. And I hadn't written anything like that in my belief box, so I didn't believe it. I believed things that readily entailed it, but not this.

There are two distinct notions here: map-beliefs and sentence-beliefs. When we ask whether groups have beliefs, it might be worth thinking through which of these we mean. I think a lot of disputes about cases trace back to this.

\section{Holism vs Atomism}
\label{holismvsatomism}

An even bigger picture question, and one that will be relevant later on, is whether the metaphysics of belief is holistic or atomistic. On the holistic picture, whether I believe that \emph{p} is a function of various macro-level properties of my mental state - including several non-doxastic properties. On an atomistic picture, it is a much more isolatable feature.

This correlates (strongly, but imperfectly) with the map vs sentence idea. Take a standard map of the world. Does it represent the world as being mostly covered with water? Well yes, of course it does. Where on the map does it represent this? Everywhere! Compare this to a book of facts about the world. Most of its representation will be local. If you can ask where the book says that most of the world is covered with water, you can give a precise answer, like page 247.

Again, disputes about cases might often turn on disputes about whether we see belief as a holistic property of a system (like features of a map) or an isolatable property (like features of a book).

\section{The Organizational Spectrum}
\label{theorganizationalspectrum}

Some groups (like the Marine Corps) are very organised. Some groups (like the mob storming a tyrant's palace) are very disorganised. And some groups (like the scientific community) are in between. This suggests a question:

\begin{itemize}
\item{} How much should our theory of group belief be sensitive to the organisational structure of the group?

\end{itemize}
One theory is that it should be very sensitive. In an organised group, there might be a small number of \emph{operative members}. For example, consider what it would take for this to be (intuitively) true

\begin{itemize}
\item{} The CIA believes that the Ruritanian Prime Minister has an illegitimate child.

\end{itemize}
Assume Ruritania is a small country is Eastern\slash Central Europe. Presumably if the child was common knowledge among the CIA's experts on that part of the world, and they'd tell anyone else in the CIA who asked, that is enough to make it true. But that might happen even though only 5\% or less of the members of the CIA believe it. So in an organised group, you can have group belief with a small fraction of the group thinking something. That isn't true for mobs.

\section{Metaphysics of Groups}
\label{metaphysicsofgroups}

Here are a few questions about groups that we could try to answer. I'm mostly going to fly by intuitions here, since this is a social epistemology course not a social metaphysics course, but they are relevant.

\begin{itemize}
\item{} When do some individuals form a group?

\end{itemize}
The United States Marine Corps is a group in the sense we're interested in. So were the crowds at the recent gun control rallies. But the people currently sitting in the cafeteria across the building are not a group in this sense. Nor is the collection of Barack Obama, Teresa May and me. Just what is required here is a delicate question, but we'll mostly keep away from the borderline cases.

\begin{itemize}
\item{} When do we have the same group over time?

\end{itemize}
Bands change their members. Committees, courts, university departments etc have people come and go all the time. Just when you have the same group as you had before is hard to say in tough cases. My preferred view is throughly conventionalist, but I'm really not going to argue for that here. But even that doesn't solve the hard cases. Is the 113th Congress the same group as the 112th? Is it the same group as the 1st? I don't really know.

\begin{itemize}
\item{} When do we have distinct groups at a single time?

\end{itemize}
This is relevant to Gilbert's examples. She thinks we can have distinct groups with common membership at a time, if, for example, they have different organisational structures, or different memberships at distinct times. If this is possible then some interesting things follow for group belief.

\section{Group Action}
\label{groupaction}

Here's one way to think about group belief:

\begin{itemize}
\item{} Look at what actions groups take (either actually or rationally)

\item{} Plug those actions into a standard theory of belief, and see what happens.

\end{itemize}
But here the holism comes back into play. What even is it for a group to take an action? Sometimes we know this - the US military invades some countries, the mob storms a prison etc. But sometimes it is merely individuals who act, not members of a group. Not every misdeed of a police officer is a misdeed by a police department (though some of them are). This is useful, but not entirely easy, way to attack the problem.

\section{Group Belief without Individual Belief}
\label{groupbeliefwithoutindividualbelief}

Three cases to think through.

\begin{enumerate}
\item{} Conjunctive beliefs - especially if you think beliefs are map-like. Different specialists might contribute different parts of the map, but the whole map implies something that no one (yet) believes.

\item{} Dead specialists - if you think that the group `memory' can be stored in physical things (like books or journals), then the group can retain a belief while no member of the group retains it.

\item{} Official doctrines - Tuomela argues that a group doctrine can be a group belief even if everyone in the group has abandoned it.

\end{enumerate}

\end{document}
