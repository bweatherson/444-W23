\input{../styles/handout-leader}
\def\mytitle{Group Beliefs}
\def\myauthor{Notes for self}
\def\mydate{March 20, 2018}
\input{../styles/handout-begin}

\section{Group Beliefs}
\label{groupbeliefs}

\begin{enumerate}
\item{} Literally, a group belief, e.g., a group report.

\item{} Deferral to group of experts who don't all agree

\item{} Deferral to peers

\end{enumerate}
\section{The Puzzle}
\label{thepuzzle}

\begin{itemize}
\item{} Majority opinions aren't closed under logical entailment.

\item{} Indeed, can have $p$, $q$ and $\neg (p \wedge q)$.

\item{} If you have three propositions, you can have unanimity of negated conjunction, and majority for each conjunct.

\end{itemize}
\section{Solution One - Talk it Out}
\label{solutionone-talkitout}

\begin{itemize}
\item{} In reality, you should always talk through these puzzles.

\item{} Why this is so is a bit interesting, and maybe will come back to.

\end{itemize}
\section{Solution Two - Set an Agenda}
\label{solutiontwo-setanagenda}

\begin{itemize}
\item{} Fix some things to decide upon, and take questions to be settled once they are entailed by things that are decided.

\item{} But then the order of operations will matter.

\item{} And very big question - should the order be set externally or internally.

\item{} An external setting is where we decide before seeing the questions what we'll do

\item{} An internal setting is where we take the opinions to determine the order

\item{} So settle the unanimous ones first is a kind of internal setting.

\end{itemize}
\section{How we do this in Practice}
\label{howwedothisinpractice}

\begin{itemize}
\item{} Really totally varies.

\item{} Not even clear how we should do it.

\item{} Worth thinking through some examples

\end{itemize}
\section{Experts}
\label{experts}

\begin{itemize}
\item{} In practice, we don't want to defer equally to each person.

\item{} We want to defer to the physics experts on physics, the hockey experts on hockey, etc.

\item{} The problem is, this turns out to be hard when there are propositions about both.

\item{} This will relate to another problem, but here's the basic idea

\item{} We have to solve various problems about conjunction.

\item{} If they agree the problems are independent, then easy.

\item{} If they agree the problems are dependent, it is sort of like a peer question.

\item{} Imagine one of them thinks the questions are linked, the other does not.

\item{} How do we determine what to do?

\end{itemize}
\section{Probability}
\label{probability}

\begin{itemize}
\item{} Linear averaging solves a lot of problems

\item{} The linear average of a bunch of probability functions is a probability function.

\item{} Hooray, but wait a minute

\item{} Problem about experts

\item{} Problem about independence

\end{itemize}
\section{Expert Problem}
\label{expertproblem}

\begin{table}[htbp]
\begin{minipage}{\linewidth}
\setlength{\tymax}{0.5\linewidth}
\centering
\small
\begin{tabulary}{\textwidth}{@{}lll@{}} \toprule
 Prop & A & B \\
\midrule

 p\&q & 0 & 0 \\
 p\&\textsubscript{q} & 0.6 & 0.2 \\
 \textsubscript{p}\&q & 0.2 & 0.6 \\
 \textsubscript{p\&}q& 0.2 & 0.2 \\
\bottomrule

\end{tabulary}
\end{minipage}
\end{table}

\begin{itemize}
\item{} If you settle the unanimous questions first, you end up with a weird view on $p, q$

\item{} Make A expert on $p$, B expert on $q$; both think the prop they are expert on is more likely than not, but hard to see rule that gets you to that result.

\end{itemize}
\section{Independence Problem}
\label{independenceproblem}

We know it. Talk through basics and get to GroupThink

\end{document}
