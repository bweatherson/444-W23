\documentclass[11pt,]{article}
\usepackage{lmodern}
\usepackage{amssymb,amsmath}
\usepackage{ifxetex,ifluatex}
\usepackage{fixltx2e} % provides \textsubscript
\ifnum 0\ifxetex 1\fi\ifluatex 1\fi=0 % if pdftex
  \usepackage[T1]{fontenc}
  \usepackage[utf8]{inputenc}
\else % if luatex or xelatex
  \ifxetex
    \usepackage{mathspec}
  \else
    \usepackage{fontspec}
  \fi
  \defaultfontfeatures{Ligatures=TeX,Scale=MatchLowercase}
    \setmainfont[]{SF Pro Text Light}
\fi
% use upquote if available, for straight quotes in verbatim environments
\IfFileExists{upquote.sty}{\usepackage{upquote}}{}
% use microtype if available
\IfFileExists{microtype.sty}{%
\usepackage{microtype}
\UseMicrotypeSet[protrusion]{basicmath} % disable protrusion for tt fonts
}{}
\usepackage[margin=0.7in]{geometry}
\usepackage{hyperref}
\hypersetup{unicode=true,
            pdftitle={Mixed Strategies},
            pdfauthor={Philosophy 444},
            pdfborder={0 0 0},
            breaklinks=true}
\urlstyle{same}  % don't use monospace font for urls
\usepackage{longtable,booktabs}
\usepackage{graphicx,grffile}
\makeatletter
\def\maxwidth{\ifdim\Gin@nat@width>\linewidth\linewidth\else\Gin@nat@width\fi}
\def\maxheight{\ifdim\Gin@nat@height>\textheight\textheight\else\Gin@nat@height\fi}
\makeatother
% Scale images if necessary, so that they will not overflow the page
% margins by default, and it is still possible to overwrite the defaults
% using explicit options in \includegraphics[width, height, ...]{}
\setkeys{Gin}{width=\maxwidth,height=\maxheight,keepaspectratio}
\IfFileExists{parskip.sty}{%
\usepackage{parskip}
}{% else
\setlength{\parindent}{0pt}
\setlength{\parskip}{6pt plus 2pt minus 1pt}
}
\setlength{\emergencystretch}{3em}  % prevent overfull lines
\providecommand{\tightlist}{%
  \setlength{\itemsep}{0pt}\setlength{\parskip}{0pt}}
\setcounter{secnumdepth}{0}
% Redefines (sub)paragraphs to behave more like sections
\ifx\paragraph\undefined\else
\let\oldparagraph\paragraph
\renewcommand{\paragraph}[1]{\oldparagraph{#1}\mbox{}}
\fi
\ifx\subparagraph\undefined\else
\let\oldsubparagraph\subparagraph
\renewcommand{\subparagraph}[1]{\oldsubparagraph{#1}\mbox{}}
\fi

%%% Use protect on footnotes to avoid problems with footnotes in titles
\let\rmarkdownfootnote\footnote%
\def\footnote{\protect\rmarkdownfootnote}

%%% Change title format to be more compact
\usepackage{titling}

% Create subtitle command for use in maketitle
\providecommand{\subtitle}[1]{
  \posttitle{
    \begin{center}\large#1\end{center}
    }
}

\setlength{\droptitle}{-2em}

  \title{Mixed Strategies}
    \pretitle{\vspace{\droptitle}\centering\huge}
  \posttitle{\par}
    \author{Philosophy 444}
    \preauthor{\centering\large\emph}
  \postauthor{\par}
      \predate{\centering\large\emph}
  \postdate{\par}
    \date{30 September, 2019}

\usepackage{gensymb}
\usepackage{nicefrac}
\usepackage{mathastext}
\usepackage{multicol}

\begin{document}
\maketitle

\hypertarget{best-responses}{%
\section{Best Responses}\label{best-responses}}

We have done as much as we can for now by merely thinking about ordinal
utility. We now need to look at games where cardinal utility matters.
And we assume at every stage that agents are trying to maximise expected
utility. But that requires a probability function, and where do the
probabilities come from. We will start with the notion of a \textbf{best
response}, which I'll define then illustrate.

\begin{description}
\tightlist
\item[Best Response]
A strategy \(s_i\) is a best response for player \(i\) iff there is some
probability distribution \(\Pr\) over the possible strategies of other
players such that playing \(s_i\) maximises \(i\)'s \emph{expected}
payoff, given \(\Pr\). (Note that we're using `maximise' in such a way
that it allows that other strategies do just as well; it just rules out
other strategies doing better.)
\end{description}

\bigskip

\begin{longtable}[]{@{}lcc@{}}
\toprule
& \(l\) & \(r\)\tabularnewline
\midrule
\endhead
\(U\) & \(3,0\) & \(0,0\)\tabularnewline
\(M\) & \(2,0\) & \(2,0\)\tabularnewline
\(D\) & \(0,0\) & \(3,0\)\tabularnewline
\bottomrule
\end{longtable}

Consider that game from the perspective of \(R\), the player who chooses
the row. What we want to show is that \emph{all three} of the possible
moves here are best responses.

It is clear that \(U\) is a best response. Set
\(\Pr(l) = 1, \Pr(r) = 0\). Then \(E(U) = 3, E(M) = 2, E(D) = 0\). It is
also clear that \(D\) is a best response. Set
\(\Pr(l) = 0, \Pr(r) = 1\). Then \(E(U) = 0, E(M) = 2, E(D) = 3\).

The striking thing is that \(M\) can also be a best response. Set
\(\Pr(l) = \Pr(r) = \nicefrac{1}{2}\). Then
\(E(U) = E(D) = \nicefrac{3}{2}\). But \(E(M) = 2\), which is greater
than \(\nicefrac{3}{2}\). So if \(R\) thinks it is equally likely that
\(C\) will play either \(l\) or \(r\), then \(R\) maximises expected
utility by playing \(M\). Of course, she doesn't maximise actual
utility. Maximising actual utility requires making a gamble on which
choice \(C\) will make. That isn't always wise; it might be best to take
the safe option.

But note that if we change the game just a little, changing the cardinal
utilities but not the ordinal utilities, the analysis of the game
changes.

\begin{longtable}[]{@{}lcc@{}}
\toprule
& \(l\) & \(r\)\tabularnewline
\midrule
\endhead
\(U\) & \(3,0\) & \(0,0\)\tabularnewline
\(M\) & \(1,0\) & \(1,0\)\tabularnewline
\(D\) & \(0,0\) & \(3,0\)\tabularnewline
\bottomrule
\end{longtable}

Now \(M\) cannot be a best response. No matter what \(\Pr(l)\) is, it
won't maximise expected utility to play \(M\). Indeed, there is a good
sense in which \(M\) is a dominated option. Even though no option is
guaranteed to do better than it, you can't come up with a good reason to
play it. And by reason here, we mean probability distribution over what
the other person will do. No matter what strategy Column plays,
\emph{pure or mixed}, it isn't best to play \(M\).

Now I won't say much today about what it \textbf{means} to play a mixed
strategy. That's for Wednesday. For now, I'll just assume that players
can, instead of choosing a strategy, choose to assign a probability to
each possible strategy. And if they do that, strategies for the other
players have expected returns, rather than guaranteed returns.

\hypertarget{nash-equilibrium}{%
\section{Nash Equilibrium}\label{nash-equilibrium}}

We previously defined a Nash Equilibrium as a set of moves that no
player can improve their position on by unilaterally defecting from the
equilibrium. We can equivalently define it the following way.

\begin{itemize}
\tightlist
\item
  A Nash Equilibrium is a set of strategies such that every move is a
  best response to the strategies the other players actually play.
\end{itemize}

And remember that a strategy might now be a probability over options.
You might think that many games do not have a Nash Equilibrium, but in
fact all (finite) games do. (I am not going to try proving this; hi
proving it was what lead to Nash equilibrium being named after Nash.)
Here is one that you might think does not, a game you may be familiar
with.

\begin{longtable}[]{@{}lccc@{}}
\toprule
& \(r\) & \(p\) & \(s\)\tabularnewline
\midrule
\endhead
\(R\) & \(1,1\) & \(0,2\) & \(2,0\)\tabularnewline
\(P\) & \(2,0\) & \(1,1\) & \(0,2\)\tabularnewline
\(S\) & \(0,2\) & \(2,0\) & \(1,1\)\tabularnewline
\bottomrule
\end{longtable}

For each player, the equilibrium strategy is to play each option with
probability \(\frac{1}{3}\). (Exercise: Prove this is an equilibrium.)

\hypertarget{finding-mixed-strategy-equilibria}{%
\section{Finding Mixed Strategy
Equilibria}\label{finding-mixed-strategy-equilibria}}

Consider this asymmetric version of Death in Damascus. (I'll go over in
class why it gets this name.)

\begin{longtable}[]{@{}lcc@{}}
\toprule
& Damascus & Aleppo\tabularnewline
\midrule
\endhead
Damascus & \(1, -1\) & \(-1,0.5\)\tabularnewline
Aleppo & \(-1,1\) & \(1,-1.5\)\tabularnewline
\bottomrule
\end{longtable}

I've set up the game with Death is the Row player, and the Man is the
Column player. Death wants to catch Man, Man wants to avoid Death. But
we've added a 0.5 penalty for Man choosing Aleppo. It's an unpleasant
journey from Damascus to Aleppo, particularly if you fear Death is at
the other end.

There is still no pure strategy equilibrium in this game. Whatever Death
plays, Man would prefer to play the other. And whatever Man plays, Death
wants to play it. So there couldn't be a set of pure choices that they
would both be happy with given that they know the other's play.

But the `choose each option with equal probability' strategy isn't a
Nash equilibrium either. We'll write \(\langle x, y \rangle\) for the
mixed strategy of going to Damascus with probability \(x\), and going to
Aleppo with probability \(y\). Clearly we should have \(x + y = 1\), but
it will make the representation easier to use two variables here, rather
than just writing \(\langle x, 1-x \rangle\) for the mixed strategies.

Given that representation, we can ask whether the state where each
player plays \(\langle \nicefrac{1}{2}, \nicefrac{1}{2} \rangle\) is a
Nash equilibrium. And, as you might guess, it is not. You might have
guessed this because the game is not symmetric, so it would be odd if
the equilibrium solution to the game is symmetric. But let's prove that
it isn't an equilibrium. Assume that Death plays
\(\langle \nicefrac{1}{2}, \nicefrac{1}{2} \rangle\). Then Man's
expected return from staying in Damascus is:

\[
\nicefrac{1}{2} \times -1 + \nicefrac{1}{2} \times 1 = 0
\]

while his return from going to Aleppo is

\[
\nicefrac{1}{2} \times 0.5 + \nicefrac{1}{2} \times -1.5 = -0.5
\]

So if Death plays \(\langle \nicefrac{1}{2}, \nicefrac{1}{2} \rangle\),
Man is better off staying in Damascus than going to Aleppo. And if he's
better off staying in Damascus that going to Aleppo, he's also better
off staying in Damascus than playing some mixed strategy that gives some
probability of going to Aleppo. In fact, the strategy
\(\langle x, y \rangle\) will have expected return \(\nicefrac{-y}{2}\),
which is clearly worse than 0 when \(y > 0\).

There's a general point here. The expected return of a mixed strategy is
the weighted average of the returns of the pure strategies that make up
the mixed strategy. In this example, for instance, if the expected value
of staying in Damascus is \(d\), and the expected value of going to
Aleppo is \(a\), the mixed strategy \(\langle x, y \rangle\) will have
expected value \(xd + ya\). And since \(x + y = 1\), the value of that
will be strictly between \(a\) and \(d\) if \(a \neq d\). On the other
hand, if \(a = d\), then \(x + y = 1\) entails that \(xd + ya = a = d\).
So if \(a = d\), then any mixed strategy will be just as good as any
other, or indeed as either of the pure strategies. That implies that
mixed strategies are candidates to be equilibrium~points, since there is
nothing to be gained by moving away from them.

This leads to an immediate, though somewhat counterintuitive,
conclusion. Let's say we want to find strategies
\(\langle x_D, y_D \rangle\) for Death and \(\langle x_M, y_M \rangle\)
for Man that are in equilibrium. If the strategies are in equilibrium,
then neither party can gain by moving away from them. And we just showed
that that means that the expected return of Damascus must equal the
expected return of Aleppo. So to find \(\langle x_D, y_D \rangle\), we
need to find values for \(x_D\) and \(y_D\) such that, given Man's
values, staying in Damascus and leaving for Aleppo are equally valued.
Note, and this is the slightly counterintuitive part, we don't need to
look at \textit{Death's} values. All that matters is that Death's
strategy and Man's values together entail that the two options open to
Man are equally valuable.

Given that Death is playing \(\langle x_D, y_D \rangle\), we can work
out the expected utility of Man's options fairly easily. (We'll
occasionally appeal to the fact that \(x_D + y_D = 1\).)

\begin{align*}
U(\text{Damascus}) &= x_D \times -1 + y_D \times 1 \\
&= y_D - x_D \\
&= 1 - 2x_D \\
U(\text{Aleppo}) &= x_D \times 0.5 + y_D \times -1.5 \\
&= 0.5x_D - 1.5(1 - x_D) \\
&= 2x_D - 1.5 
\end{align*}

So there is equilibrium when \(1 - 2x_D = 2x_D - 1.5\), i.e., when
\(x_D = \nicefrac{5}{8}\). So any mixed strategy equilibrium~will have
to have Death playing
\(\langle \nicefrac{5}{8}, \nicefrac{3}{8} \rangle\).

Now let's do the same calculation for Man's strategy. Given that Man is
playing \(\langle x_D, y_D \rangle\), we can work out the expected
utility of Death's options. (Again, we'll occasionally appeal to the
fact that \(x_M + y_M = 1\).)

\begin{align*}
U(\text{Damascus}) &= x_M \times 1 + y_M \times -1 \\
&= x_M - y_M \\
&= 2x_M - 1 \\
U(\text{Aleppo}) &= x_M \times -1 + y_M \times 1 \\
&= y_M - x_M \\
&= 1 - 2x_M 
\end{align*}

So there is equilibrium when \(2x_M - 1 = 1 - 2x_M\), i.e., when
\(x_M = \nicefrac{1}{2}\). So any mixed strategy equilibrium will have
to have Man playing
\(\langle \nicefrac{1}{2}, \nicefrac{1}{2} \rangle\). Indeed, we can
work out that if Death plays
\(\langle \nicefrac{5}{8}, \nicefrac{3}{8} \rangle\), and Man plays
\(\langle \nicefrac{1}{2}, \nicefrac{1}{2} \rangle\), then any strategy
for Death will have expected return 0, and any strategy for Man will
have expected return of \(\nicefrac{-1}{4}\). So this pair is an
equilibrium.

But note something very odd about what we just concluded. When we
chang-ed the payoffs for the two cities, we made it worse for Man, not
Death, to go to Aleppo. I would have guessed that should make Man more
likely to stay in Damascus. But it turns out this isn't right, at least
if the players play equilibrium strategies. The change to Man's payoffs
doesn't change Man's strategy at all; he still plays
\(\langle \nicefrac{1}{2}, \nicefrac{1}{2} \rangle\). What it does is
change Death's strategy from
\(\langle \nicefrac{1}{2}, \nicefrac{1}{2} \rangle\) to
\(\langle \nicefrac{5}{8}, \nicefrac{3}{8} \rangle\).

Let's generalise this to a general recipe for finding equilibrium
strategies in two player games with conflicting incentives. Assume we
have the following very abstract form of a game:

\begin{longtable}[]{@{}lcc@{}}
\toprule
& \(l\) & \(r\)\tabularnewline
\midrule
\endhead
\(U\) & \(a_1, a_2\) & \(b_1, b_2\)\tabularnewline
\(D\) & \(c_1, c_2\) & \(d_1, d_2\)\tabularnewline
\bottomrule
\end{longtable}

As usual, \(R\)ow chooses between \(U\)p and \(D\)own, while \(C\)olumn
chooses between \(l\)eft and \(r\)ight. We will assume that \(R\)
prefers the outcome to be on the north-west-southeast diagonal; that is,
\(a_1 > c_1\), and \(d_1 > b_1\). And we'll assume that \(C\) prefers
the other diagonal; that is, \(c_2 > a_2\), and \(b_2 > d_2\). We then
have to find a pair of mixed strategies \(\langle x_U, x_D \rangle\) and
\(\langle x_l, x_r \rangle\) that are in equilibrium. (We'll use \(x_A\)
for the probability of playing \(A\).)

What's crucial is that for each player, the expected value of each
option is equal given what the other person plays. Let's compute them
the expected value of playing \(U\) and \(D\), given that \(C\) is
playing \(\langle x_l, x_r \rangle\).

\begin{align*}
U(U) &= x_la_1 + x_rb_1 \\
U(D) &= x_lc_1 + x_rd_1
\end{align*}

We get equilibrium~when these two values are equal, and
\(x_l + x_r = 1\). So we can solve for \(x_l\) the following way:

\begin{minipage}[t]{0.5\textwidth}
\begin{align*}
&x_la_1 +x_rb_1 = x_lc_1 + x_rd_1 \\
\Leftrightarrow \hspace{6pt} &x_la_1 -x_lc_1 = x_rd_1 - x_rb_1 \\
\Leftrightarrow \hspace{6pt} &x_l(a_1 - c_1) = x_r(d_1 - b_1) \\
\Leftrightarrow \hspace{6pt} &x_l\frac{a_1 - c_1}{d_1 - b_1} = x_r \\
\Leftrightarrow \hspace{6pt} &x_l\frac{a_1 - c_1}{d_1 - b_1} = 1 - x_l \\
\Leftrightarrow \hspace{6pt} &x_l\frac{a_1 - c_1}{d_1 - b_1} + x_l= 1 \\
\Leftrightarrow \hspace{6pt} &x_l(\frac{a_1 - c_1}{d_1 - b_1} + 1)= 1 \\
\Leftrightarrow \hspace{6pt} &x_l= \frac{1}{\frac{a_1 - c_1}{d_1 - b_1} + 1} \\
\end{align*}
\end{minipage}
\begin{minipage}[t]{0.5\textwidth}
\bigskip
I won't go through all the same steps, but a similar argument shows that

$$
x_U = \frac{1}{\frac{b_2 - a_2}{c_2 - d_2}+1}
$$

I'll leave it as an exercise to confirm these answers are correct by working out the expected return of $U, D, l$ and $r$ if these strategies are played.
\bigskip
The crucial take-away lesson from this discussion is that to find a mixed strategy equilibrium, we look at the interaction between one player's mixture and the other player's payoffs. The idea is to set the probability for each move in such a way that even if the other player knew this, they wouldn't be able to improve their position, since any move would be just as good for them as any other.
\end{minipage}


\end{document}
