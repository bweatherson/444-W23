\documentclass[11pt,]{article}
\usepackage{lmodern}
\usepackage{amssymb,amsmath}
\usepackage{ifxetex,ifluatex}
\usepackage{fixltx2e} % provides \textsubscript
\ifnum 0\ifxetex 1\fi\ifluatex 1\fi=0 % if pdftex
  \usepackage[T1]{fontenc}
  \usepackage[utf8]{inputenc}
\else % if luatex or xelatex
  \ifxetex
    \usepackage{mathspec}
  \else
    \usepackage{fontspec}
  \fi
  \defaultfontfeatures{Ligatures=TeX,Scale=MatchLowercase}
    \setmainfont[]{SF Pro Text Light}
\fi
% use upquote if available, for straight quotes in verbatim environments
\IfFileExists{upquote.sty}{\usepackage{upquote}}{}
% use microtype if available
\IfFileExists{microtype.sty}{%
\usepackage{microtype}
\UseMicrotypeSet[protrusion]{basicmath} % disable protrusion for tt fonts
}{}
\usepackage[margin=1in]{geometry}
\usepackage{hyperref}
\hypersetup{unicode=true,
            pdftitle={Levy and Alfano},
            pdfauthor={Philosophy 444},
            pdfborder={0 0 0},
            breaklinks=true}
\urlstyle{same}  % don't use monospace font for urls
\usepackage{graphicx,grffile}
\makeatletter
\def\maxwidth{\ifdim\Gin@nat@width>\linewidth\linewidth\else\Gin@nat@width\fi}
\def\maxheight{\ifdim\Gin@nat@height>\textheight\textheight\else\Gin@nat@height\fi}
\makeatother
% Scale images if necessary, so that they will not overflow the page
% margins by default, and it is still possible to overwrite the defaults
% using explicit options in \includegraphics[width, height, ...]{}
\setkeys{Gin}{width=\maxwidth,height=\maxheight,keepaspectratio}
\IfFileExists{parskip.sty}{%
\usepackage{parskip}
}{% else
\setlength{\parindent}{0pt}
\setlength{\parskip}{6pt plus 2pt minus 1pt}
}
\setlength{\emergencystretch}{3em}  % prevent overfull lines
\providecommand{\tightlist}{%
  \setlength{\itemsep}{0pt}\setlength{\parskip}{0pt}}
\setcounter{secnumdepth}{0}
% Redefines (sub)paragraphs to behave more like sections
\ifx\paragraph\undefined\else
\let\oldparagraph\paragraph
\renewcommand{\paragraph}[1]{\oldparagraph{#1}\mbox{}}
\fi
\ifx\subparagraph\undefined\else
\let\oldsubparagraph\subparagraph
\renewcommand{\subparagraph}[1]{\oldsubparagraph{#1}\mbox{}}
\fi

%%% Use protect on footnotes to avoid problems with footnotes in titles
\let\rmarkdownfootnote\footnote%
\def\footnote{\protect\rmarkdownfootnote}

%%% Change title format to be more compact
\usepackage{titling}

% Create subtitle command for use in maketitle
\providecommand{\subtitle}[1]{
  \posttitle{
    \begin{center}\large#1\end{center}
    }
}

\setlength{\droptitle}{-2em}

  \title{Levy and Alfano}
    \pretitle{\vspace{\droptitle}\centering\huge}
  \posttitle{\par}
    \author{Philosophy 444}
    \preauthor{\centering\large\emph}
  \postauthor{\par}
      \predate{\centering\large\emph}
  \postdate{\par}
    \date{2 December, 2019}

\usepackage{mathastext}
\usepackage{nicefrac}

\begin{document}
\maketitle

Virtues and vices Aristotle Character not outcome or even action
Doctrine of the mean E.g., courage as mean between cowardice and
foolishness

Virtue epistemology Introduced to solve Dharmottara cases But generally
more interesting than that Open-mindedness, inquisitiveness, modesty,
curiosity, etc Which of these are Aristotelian means?

Private vices as public goods Mandeville (1705/1729) - Saving is a
private virtue but public vice (but did he really) Explorers, innovators
etc Need a bunch of excessively confident people for things to move
forward

Epistemic version Two main cases One they spend most time on is cultural
evolution The other, perhaps as important, is argumentation

Defences Obviously two - these aren't public goods, or they aren't
private vices Maybe they are not public \emph{epistemic} goods? But I'm
more interested in them being private virtues How discriminating does
imitation have to be for it to be a virtue Compare the headbutting the
light switch example - maybe this is common What do we mean by
`over-imitating'? If it's just not actual utility maximising on this
occasion, that's too strong

Evidence of evidence General picture - often we find evidence that
someone has evidence that p That's defeasibly good evidence that p Maybe
that's what all these folks are doing And maybe that's rational

Vigilence This is a really important notion, and one that's worth
spelling out Example of walking through a crowded corridor Maybe that's
the right attitude to take towards people telling you stuff And maybe
that's enough to make these virtues - copying with vigilance is kinda
good?


\end{document}
