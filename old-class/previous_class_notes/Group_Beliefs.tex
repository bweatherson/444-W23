\documentclass[11pt,]{article}
\usepackage{lmodern}
\usepackage{amssymb,amsmath}
\usepackage{ifxetex,ifluatex}
\usepackage{fixltx2e} % provides \textsubscript
\ifnum 0\ifxetex 1\fi\ifluatex 1\fi=0 % if pdftex
  \usepackage[T1]{fontenc}
  \usepackage[utf8]{inputenc}
\else % if luatex or xelatex
  \ifxetex
    \usepackage{mathspec}
  \else
    \usepackage{fontspec}
  \fi
  \defaultfontfeatures{Ligatures=TeX,Scale=MatchLowercase}
    \setmainfont[]{SF Pro Text Light}
\fi
% use upquote if available, for straight quotes in verbatim environments
\IfFileExists{upquote.sty}{\usepackage{upquote}}{}
% use microtype if available
\IfFileExists{microtype.sty}{%
\usepackage{microtype}
\UseMicrotypeSet[protrusion]{basicmath} % disable protrusion for tt fonts
}{}
\usepackage[margin=1in]{geometry}
\usepackage{hyperref}
\hypersetup{unicode=true,
            pdftitle={Group Beliefs},
            pdfauthor={Philosophy 444},
            pdfborder={0 0 0},
            breaklinks=true}
\urlstyle{same}  % don't use monospace font for urls
\usepackage{longtable,booktabs}
\usepackage{graphicx,grffile}
\makeatletter
\def\maxwidth{\ifdim\Gin@nat@width>\linewidth\linewidth\else\Gin@nat@width\fi}
\def\maxheight{\ifdim\Gin@nat@height>\textheight\textheight\else\Gin@nat@height\fi}
\makeatother
% Scale images if necessary, so that they will not overflow the page
% margins by default, and it is still possible to overwrite the defaults
% using explicit options in \includegraphics[width, height, ...]{}
\setkeys{Gin}{width=\maxwidth,height=\maxheight,keepaspectratio}
\IfFileExists{parskip.sty}{%
\usepackage{parskip}
}{% else
\setlength{\parindent}{0pt}
\setlength{\parskip}{6pt plus 2pt minus 1pt}
}
\setlength{\emergencystretch}{3em}  % prevent overfull lines
\providecommand{\tightlist}{%
  \setlength{\itemsep}{0pt}\setlength{\parskip}{0pt}}
\setcounter{secnumdepth}{0}
% Redefines (sub)paragraphs to behave more like sections
\ifx\paragraph\undefined\else
\let\oldparagraph\paragraph
\renewcommand{\paragraph}[1]{\oldparagraph{#1}\mbox{}}
\fi
\ifx\subparagraph\undefined\else
\let\oldsubparagraph\subparagraph
\renewcommand{\subparagraph}[1]{\oldsubparagraph{#1}\mbox{}}
\fi

%%% Use protect on footnotes to avoid problems with footnotes in titles
\let\rmarkdownfootnote\footnote%
\def\footnote{\protect\rmarkdownfootnote}

%%% Change title format to be more compact
\usepackage{titling}

% Create subtitle command for use in maketitle
\providecommand{\subtitle}[1]{
  \posttitle{
    \begin{center}\large#1\end{center}
    }
}

\setlength{\droptitle}{-2em}

  \title{Group Beliefs}
    \pretitle{\vspace{\droptitle}\centering\huge}
  \posttitle{\par}
    \author{Philosophy 444}
    \preauthor{\centering\large\emph}
  \postauthor{\par}
      \predate{\centering\large\emph}
  \postdate{\par}
    \date{11 November, 2019}

\usepackage{mathastext}
\usepackage{nicefrac}

\begin{document}
\maketitle

\hypertarget{group-beliefs}{%
\section{Group Beliefs}\label{group-beliefs}}

Why are we interested in group beliefs? Three possible circumstances.

\begin{enumerate}
\def\labelenumi{\arabic{enumi}.}
\tightlist
\item
  We are tasked with writing a group report, and we have to aggregate
  our views.
\item
  We want to learn about an area, and there are several people to whom
  we could plausibly defer.
\item
  We want to say, as theorists, what it is that such-and-such group
  thinks.
\end{enumerate}

The last question has a theoretical background question. Should we think
of reports of group beliefs (e.g., ``The CIA thinks\ldots{}'') as being
literal or metaphorical? Are groups the kind of things that can really
have beliefs? This will hover in the background for a bit, but it's
something that we should keep in mind.

\hypertarget{core-puzzle}{%
\section{Core Puzzle}\label{core-puzzle}}

The rule

\begin{quote}
Group \(G\) believes that \(A\) if and only if a majority of members of
\(G\) believe that \(A\)
\end{quote}

is incoherent. Or, at least, it commits some ordinary looking groups to
incoherence. The circumstances where the incoherence arises are fairly
banal.

The CIA has three analysts working on corruption in Western Europe. We
ask them what they think, and they give these reports.

\begin{itemize}
\tightlist
\item
  A thinks that the French and German leaders are both corrupt.
\item
  B thinks that the French leader is corrupt, but the German leader is
  not.
\item
  C thinks that the German leader is corrupt, but the French leader is
  not.
\end{itemize}

And we have to construct a report on behalf of the CIA, with this expert
testimony. The report has to answer these questions.

\begin{enumerate}
\def\labelenumi{\arabic{enumi}.}
\tightlist
\item
  Is the French leader corrupt?
\item
  Is the German leader corrupt?
\item
  Are both the French and German leaders corrupt?
\end{enumerate}

And if we go by majority rule, the answers are, \emph{Yes}, \emph{Yes}
and \emph{No}. And that would not make for a particularly satisfying
Presidential Daily Briefing.

\hypertarget{three-big-options}{%
\section{Three Big Options}\label{three-big-options}}

\begin{enumerate}
\def\labelenumi{\arabic{enumi}.}
\tightlist
\item
  Try to aggregate inputs not outputs.
\item
  Set an agenda of logically independent items, and answer questions as
  they come up, then set other questions not on the agenda by logic.
\item
  Find experts on each individual question and defer to them.
\end{enumerate}

\hypertarget{option-one---inputs-not-outputs}{%
\section{Option One - Inputs Not
Outputs}\label{option-one---inputs-not-outputs}}

\begin{itemize}
\tightlist
\item
  This is kind of what we do when we talk over a problem.
\item
  It's a good strategy!
\item
  In general, the group shouldn't aggregate its
  beliefs/desires/preferences, but it's evidence/values/reasons.
\item
  \textbf{Signature weakness}: When some people disagree about the
  function from inputs -\textgreater{} outputs.
\item
  Then all the other problems recur.
\item
  But let's not get too fancy - this is the right solution.
\end{itemize}

\hypertarget{option-two---agenda}{%
\section{Option Two - Agenda}\label{option-two---agenda}}

\begin{itemize}
\tightlist
\item
  This is what List and Pettit spend a lot of time on.
\item
  We don't use majority rule for everything, just for things on the
  agenda.
\item
  So we don't have a meeting, or a vote, about whether both leaders are
  corrupt.
\item
  We, perhaps, vote on whether the first is corrupt, then we vote on
  whetehr the second is corrupt, then we are done. Logic then settles
  the answers to the conjunction question.
\item
  \textbf{Signature weakness}: Sensitive to the agenda. Could just as
  easily set agenda as ``Is French leader corrupt?'', and ``Are both of
  them corrupt?''.
\item
  Practical question: What is the right `agenda' for a legal body like a
  jury or a court panel.
\item
  IANAL, but I think this one is really interesting and hard (and
  possibly does not get a stable answer)
\item
  Choice point: Do you set the agenda externally, or do you let features
  of the views of the participants set the agenda order?
\item
  For example, do you say that the first thing to do is find any
  proposition that \textbf{everyone} agrees on, and take that as given?
\item
  Either answer here leads to some odd results. Giving up on unanimity
  is weird, but so is fixing odd conjunctions/disjunctions before the
  start of play.
\item
  And do you treat factual and evaluative questions separately, or let
  them interact as well?
\end{itemize}

\hypertarget{option-three---designate-an-expert-for-each-question}{%
\section{Option Three - Designate an Expert for Each
question}\label{option-three---designate-an-expert-for-each-question}}

\begin{itemize}
\tightlist
\item
  In practice, we don't want to defer equally to each person.
\item
  We want to defer to the physics experts on physics, the hockey experts
  on hockey, etc.
\item
  Maybe we need to identify the person who knows France best, and they
  settle the French question etc.
\item
  The problem is, this turns out to be hard when there are propositions
  about multiple subject matters that we care about, and we care about
  conjunctions of such propositions.
\item
  A particularly vivid version of this problem is when one expert thinks
  two debates are linked, and another does not.
\end{itemize}

\hypertarget{probability}{%
\section{Probability}\label{probability}}

\begin{itemize}
\tightlist
\item
  Linear averaging solves a lot of problems
\item
  The linear average of a bunch of probability functions is a
  probability function.
\item
  And it is often a very natural function to use as the group judgment.
\item
  Hooray, but wait a minute. There are two problems.
\item
  First is a problem about overlapping expertise
\item
  Second is a problem about independence.
\end{itemize}

\hypertarget{expert-problem}{%
\section{Expert Problem}\label{expert-problem}}

A and B are experts on different subjects, and here are their
probabilities.

\begin{longtable}[]{@{}cll@{}}
\toprule
& A & B\tabularnewline
\midrule
\endhead
\(p \wedge q\) & 0 & 0\tabularnewline
\(p \wedge \neg q\) & 0.6 & 0.2\tabularnewline
\(\neg p \wedge q\) & 0.2 & 0.6\tabularnewline
\(\neg p \wedge \neg q\) & 0.2 & 0.2\tabularnewline
\bottomrule
\end{longtable}

Assume A is the expert on \(p\), and B is the expert on \(q\). Then you
can't do the following three things

\begin{enumerate}
\def\labelenumi{\arabic{enumi}.}
\tightlist
\item
  Have the probability in \(p\) equal expert A's probability that \(p\).
\item
  Have the probability in \(q\) equal expert B's probability that \(q\).
\item
  Assign probability 0 to the one thing your two experts agree is not
  going to happen.
\end{enumerate}

\hypertarget{independence-problem}{%
\section{Independence Problem}\label{independence-problem}}

A and B both know a lot about \(p, q\), and they have long excellent
track records of making probabilistic forecasts. Here is what they think
about the four possibilities.

\begin{longtable}[]{@{}ccc@{}}
\toprule
& A & B\tabularnewline
\midrule
\endhead
\(p \wedge q\) & 0.81 & 0.01\tabularnewline
\(p \wedge \neg q\) & 0.09 & 0.09\tabularnewline
\(\neg p \wedge q\) & 0.09 & 0.09\tabularnewline
\(\neg p \wedge \neg q\) & 0.01 & 0.81\tabularnewline
\bottomrule
\end{longtable}

Both think \(p, q\) independent and equiprobable, but they have wildly
different views about them. We can't do all the following

\begin{itemize}
\tightlist
\item
  Agree with the experts that \(p, q\) independent.
\item
  Agree with the experts about the two propositions that they agree on
  the probability.
\item
  Agree with the experts that \(p, q\) are equiprobable.
\end{itemize}

So which do we give up. And, frankly, what credences should we have? I
have no idea on the answer to this last one.


\end{document}
