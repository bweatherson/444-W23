\documentclass[11pt,]{article}
\usepackage{lmodern}
\usepackage{amssymb,amsmath}
\usepackage{ifxetex,ifluatex}
\usepackage{fixltx2e} % provides \textsubscript
\ifnum 0\ifxetex 1\fi\ifluatex 1\fi=0 % if pdftex
  \usepackage[T1]{fontenc}
  \usepackage[utf8]{inputenc}
\else % if luatex or xelatex
  \ifxetex
    \usepackage{mathspec}
  \else
    \usepackage{fontspec}
  \fi
  \defaultfontfeatures{Ligatures=TeX,Scale=MatchLowercase}
    \setmainfont[]{SF Pro Text Light}
\fi
% use upquote if available, for straight quotes in verbatim environments
\IfFileExists{upquote.sty}{\usepackage{upquote}}{}
% use microtype if available
\IfFileExists{microtype.sty}{%
\usepackage{microtype}
\UseMicrotypeSet[protrusion]{basicmath} % disable protrusion for tt fonts
}{}
\usepackage[margin=1in]{geometry}
\usepackage{hyperref}
\hypersetup{unicode=true,
            pdftitle={Gilbert on Group Action},
            pdfauthor={Philosophy 444},
            pdfborder={0 0 0},
            breaklinks=true}
\urlstyle{same}  % don't use monospace font for urls
\usepackage{graphicx,grffile}
\makeatletter
\def\maxwidth{\ifdim\Gin@nat@width>\linewidth\linewidth\else\Gin@nat@width\fi}
\def\maxheight{\ifdim\Gin@nat@height>\textheight\textheight\else\Gin@nat@height\fi}
\makeatother
% Scale images if necessary, so that they will not overflow the page
% margins by default, and it is still possible to overwrite the defaults
% using explicit options in \includegraphics[width, height, ...]{}
\setkeys{Gin}{width=\maxwidth,height=\maxheight,keepaspectratio}
\IfFileExists{parskip.sty}{%
\usepackage{parskip}
}{% else
\setlength{\parindent}{0pt}
\setlength{\parskip}{6pt plus 2pt minus 1pt}
}
\setlength{\emergencystretch}{3em}  % prevent overfull lines
\providecommand{\tightlist}{%
  \setlength{\itemsep}{0pt}\setlength{\parskip}{0pt}}
\setcounter{secnumdepth}{0}
% Redefines (sub)paragraphs to behave more like sections
\ifx\paragraph\undefined\else
\let\oldparagraph\paragraph
\renewcommand{\paragraph}[1]{\oldparagraph{#1}\mbox{}}
\fi
\ifx\subparagraph\undefined\else
\let\oldsubparagraph\subparagraph
\renewcommand{\subparagraph}[1]{\oldsubparagraph{#1}\mbox{}}
\fi

%%% Use protect on footnotes to avoid problems with footnotes in titles
\let\rmarkdownfootnote\footnote%
\def\footnote{\protect\rmarkdownfootnote}

%%% Change title format to be more compact
\usepackage{titling}

% Create subtitle command for use in maketitle
\providecommand{\subtitle}[1]{
  \posttitle{
    \begin{center}\large#1\end{center}
    }
}

\setlength{\droptitle}{-2em}

  \title{Gilbert on Group Action}
    \pretitle{\vspace{\droptitle}\centering\huge}
  \posttitle{\par}
    \author{Philosophy 444}
    \preauthor{\centering\large\emph}
  \postauthor{\par}
      \predate{\centering\large\emph}
  \postdate{\par}
    \date{18 November, 2019}

\usepackage{mathastext}
\usepackage{nicefrac}

\begin{document}
\maketitle

\hypertarget{two-big-questions}{%
\subsection{Two Big Questions}\label{two-big-questions}}

\begin{enumerate}
\def\labelenumi{\arabic{enumi}.}
\tightlist
\item
  Does Gilbert have the right analysis of ``walking together'', or other
  small group activities?
\item
  Is it the right model for larger group activities?
\end{enumerate}

\hypertarget{a-traditional-way-of-thinking-about-problem}{%
\subsection{A Traditional Way of Thinking About
Problem}\label{a-traditional-way-of-thinking-about-problem}}

\begin{enumerate}
\def\labelenumi{\arabic{enumi}.}
\tightlist
\item
  What makes some people a \emph{group}, as opposed to merely some
  people?
\item
  What makes it the case that that group is engaged in a group action,
  shares a group intention, and so on?
\end{enumerate}

Gilbert's view is that this is the wrong way to look at things. Rather,
these two questions should be answered simultaneously.

\hypertarget{two-theories-of-group-action}{%
\subsection{Two Theories of Group
Action}\label{two-theories-of-group-action}}

\begin{description}
\tightlist
\item[Weak Shared Plan]
All the people in the group have the same plan.
\item[Strong Shared Plan]
All the people in the group have the same plan, and this is common
knowledge.
\end{description}

The argument against the first of these is reasonably simple.

\begin{enumerate}
\def\labelenumi{\arabic{enumi}.}
\tightlist
\item
  If each person is trying to do X, and thinks they are the only one
  trying to do X, then there is no group action of trying to do X.
\item
  If \textbf{Weak Shared Plan} is true, then in such a situation there
  is a group action of trying to do X.
\item
  So \textbf{Weak Shared Plan} is false.
\end{enumerate}

The argument against the second is more controversial.

\begin{enumerate}
\def\labelenumi{\arabic{enumi}.}
\tightlist
\item
  If \textbf{Strong Shared Plan} is true, then the members of the group
  have no obligation to the others to continue with the plan if they
  lose interest in it.
\item
  In cases of group action, members of the group do have an obligation
  to the others to continue with the plan even if they lose interest in
  it.
\item
  So \textbf{Strong Shared Plan} is false.
\end{enumerate}

Both parts of this are controversial. Gilbert spends time on each, first
defending 1, then clarifying 2.

\hypertarget{why-no-obligation}{%
\subsection{Why No Obligation}\label{why-no-obligation}}

\begin{itemize}
\tightlist
\item
  Because Strong Shared Plan gives you at most mutual reliance, not
  trust.
\item
  Set out Baier's example of the difference between trust and reliance.
\item
  The kind of criticism you can make of someone who doesn't follow
  through on the plan, according to Gilbert, is kind of like a breach of
  trust.
\item
  What SSP would give you is kind of like known reliance.
\item
  \textbf{Objection}: Are we really sure that known reliance is not
  enough?
\end{itemize}

\hypertarget{what-is-the-obligation}{%
\subsection{What is the Obligation}\label{what-is-the-obligation}}

It's not a moral obligation. Here is Gilbert's argument.

\begin{enumerate}
\def\labelenumi{\arabic{enumi}.}
\tightlist
\item
  You can have shared plan between people with no concept of moral
  obligation.
\item
  If the obligation is moral obligation, that's impossible.
\item
  So the obligation is not moral obligation.
\end{enumerate}

This is, I think, a bad argument. 1 is only true for psychopaths, and
not clear they can engage in group action. A better argument is

\begin{enumerate}
\def\labelenumi{\arabic{enumi}.}
\tightlist
\item
  You can have a shared plan to do an immoral thing.
\item
  You don't have moral obligations to do immoral things.
\item
  So the obligation is not a moral obligation.
\end{enumerate}

It's also not a prudential obligation. This should be clear, but Gilbert
spends a bit of time on it.

So what kind of weird sui generis obligation is it? This is a big
question for her to answer.

\hypertarget{positive-view}{%
\subsection{Positive View}\label{positive-view}}

That there is a group action when (and only when) the people form a
plural subject. What's that?

\begin{itemize}
\tightlist
\item
  I do not understand the difference between the positive view and the
  things it is supposed to be distinguished from at the bottom of page
  7.
\item
  The connection to philosophy of language is not good.
\item
  Sometimes I can use `we' to describe things that are not a group.
\item
  E.g., I can use it to pick out the people on a bus, or even an
  internally hostile group. (``We are about to start killing each other
  for food.'')
\item
  There is an important distinction between distributive and collective
  readings of plural sentences, but this is not the same thing.
\item
  The good point around here is authority by doing. This is a real and
  important and (to my mind) interesting phenomenon. Big question how
  much actual political authority starts this way.
\item
  And note the point at the bottom of 11 top of 12 about the very
  special nature of 2 person groups, namely that for now they have their
  membership essentially. This I think is a huge problem for Gilbert's
  view.
\end{itemize}

How accurate is it to take the obligations to just be the definition of
the plural subject? That is how I end up reading it, though the textual
evidence is not the strongest.

\hypertarget{puzzle-cases}{%
\subsection{Puzzle Cases}\label{puzzle-cases}}

\begin{enumerate}
\def\labelenumi{\arabic{enumi}.}
\tightlist
\item
  Large groups
\item
  Immoral group activities
\item
  Explicit disavowal of long term commitment
\end{enumerate}


\end{document}
