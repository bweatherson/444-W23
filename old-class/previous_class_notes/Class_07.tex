\documentclass[11pt,]{article}
\usepackage{lmodern}
\usepackage{amssymb,amsmath}
\usepackage{ifxetex,ifluatex}
\usepackage{fixltx2e} % provides \textsubscript
\ifnum 0\ifxetex 1\fi\ifluatex 1\fi=0 % if pdftex
  \usepackage[T1]{fontenc}
  \usepackage[utf8]{inputenc}
\else % if luatex or xelatex
  \ifxetex
    \usepackage{mathspec}
  \else
    \usepackage{fontspec}
  \fi
  \defaultfontfeatures{Ligatures=TeX,Scale=MatchLowercase}
    \setmainfont[]{SF Pro Text Light}
\fi
% use upquote if available, for straight quotes in verbatim environments
\IfFileExists{upquote.sty}{\usepackage{upquote}}{}
% use microtype if available
\IfFileExists{microtype.sty}{%
\usepackage{microtype}
\UseMicrotypeSet[protrusion]{basicmath} % disable protrusion for tt fonts
}{}
\usepackage[margin=1in]{geometry}
\usepackage{hyperref}
\hypersetup{unicode=true,
            pdftitle={Probability and Expected Value},
            pdfauthor={Philosophy 444},
            pdfborder={0 0 0},
            breaklinks=true}
\urlstyle{same}  % don't use monospace font for urls
\usepackage{longtable,booktabs}
\usepackage{graphicx,grffile}
\makeatletter
\def\maxwidth{\ifdim\Gin@nat@width>\linewidth\linewidth\else\Gin@nat@width\fi}
\def\maxheight{\ifdim\Gin@nat@height>\textheight\textheight\else\Gin@nat@height\fi}
\makeatother
% Scale images if necessary, so that they will not overflow the page
% margins by default, and it is still possible to overwrite the defaults
% using explicit options in \includegraphics[width, height, ...]{}
\setkeys{Gin}{width=\maxwidth,height=\maxheight,keepaspectratio}
\IfFileExists{parskip.sty}{%
\usepackage{parskip}
}{% else
\setlength{\parindent}{0pt}
\setlength{\parskip}{6pt plus 2pt minus 1pt}
}
\setlength{\emergencystretch}{3em}  % prevent overfull lines
\providecommand{\tightlist}{%
  \setlength{\itemsep}{0pt}\setlength{\parskip}{0pt}}
\setcounter{secnumdepth}{0}
% Redefines (sub)paragraphs to behave more like sections
\ifx\paragraph\undefined\else
\let\oldparagraph\paragraph
\renewcommand{\paragraph}[1]{\oldparagraph{#1}\mbox{}}
\fi
\ifx\subparagraph\undefined\else
\let\oldsubparagraph\subparagraph
\renewcommand{\subparagraph}[1]{\oldsubparagraph{#1}\mbox{}}
\fi

%%% Use protect on footnotes to avoid problems with footnotes in titles
\let\rmarkdownfootnote\footnote%
\def\footnote{\protect\rmarkdownfootnote}

%%% Change title format to be more compact
\usepackage{titling}

% Create subtitle command for use in maketitle
\providecommand{\subtitle}[1]{
  \posttitle{
    \begin{center}\large#1\end{center}
    }
}

\setlength{\droptitle}{-2em}

  \title{Probability and Expected Value}
    \pretitle{\vspace{\droptitle}\centering\huge}
  \posttitle{\par}
    \author{Philosophy 444}
    \preauthor{\centering\large\emph}
  \postauthor{\par}
      \predate{\centering\large\emph}
  \postdate{\par}
    \date{25 September, 2019}

\usepackage{mathastext}

\begin{document}
\maketitle

\hypertarget{inverting-conditional-probability}{%
\section{Inverting Conditional
Probability}\label{inverting-conditional-probability}}

Here is a common enough situation. We have two variables, \(X, Y\) that
we initially don't know the values of. We know, however, that they are
(in some informal sense) correlated. We know the prior probability for
each value of \(X\), usually from frequency data. We also know the
probability for each value of \(Y\) given a value of \(X\). We want to
solve for the conditional probability of a value of \(X\) given a value
of \(Y\). Or, equivalently, we want to know what the probability of a
value of \(X\) should be once we learn a value of \(Y\).

Let's say the possible values of \(X\) are \(x_1, \dots, x_m\) and the
possible values of \(Y\) are \(y_1, \dots, y_n\). For each \(x_i\) the
prior probability that \(X = x_i\) is \(c_i\). And for each pair
\(x_i, y_j\) the probability that \(Y = y_j\) given \(X = x_i\) is
\(p_{ij}\). Then we can make a table up with the values of \(X\) on the
rows, and the values of \(Y\) on the columns, and each cell representing
the (prior) probability of a pair of \(X\) value and \(Y\) value. (When
I say a table, I mean this somewhat literally; I usually do this kind of
computation in Excel or some other spreadsheet program.)

\begin{longtable}[]{@{}rcccccc@{}}
\toprule
& \(y_1\) & \(y_2\) & \(\dots\) & \(y_j\) & \(\dots\) &
\(y_n\)\tabularnewline
\midrule
\endhead
\(x_1\) & \(c_1p_{11}\) & \(c_1p_{12}\) & \(\dots\) & \(c_1p_{1j}\) &
\(\dots\) & \(c_1p_{1n}\)\tabularnewline
\(x_2\) & \(c_2p_{21}\) & \(c_2p_{22}\) & \(\dots\) & \(c_2p_{2j}\) &
\(\dots\) & \(c_2p_{2n}\)\tabularnewline
\(\dots\) & \(\dots\) & \(\dots\) & \(\dots\) & \(\dots\) & \(\dots\) &
\(\dots\)\tabularnewline
\(x_i\) & \(c_ip_{i1}\) & \(c_ip_{i2}\) & \(\dots\) & \(c_ip_{ij}\) &
\(\dots\) & \(c_ip_{in}\)\tabularnewline
\(\dots\) & \(\dots\) & \(\dots\) & \(\dots\) & \(\dots\) & \(\dots\) &
\(\dots\)\tabularnewline
\(x_m\) & \(c_mp_{m1}\) & \(c_mp_{m2}\) & \(\dots\) & \(c_mp_{mj}\) &
\(\dots\) & \(c_mp_{mn}\)\tabularnewline
\bottomrule
\end{longtable}

The probability of each \(Y\) value is then given by adding up the
numbers in each column. So the probability of \(Y = y_j\) is given by
this formula:

\[
c_1p_{1j} + c_2p_{2j} + \dots + c_ip_{ij} + \dots + c_mp_{mj}
\]

And then the probability of \(X = x_i\) conditional on \(Y = y_j\) is
the probability of \(X = x_i \wedge Y = y_j\) divided by the probability
of \(Y = y_j\). That is, it is

\[
\frac{c_ip_{ij}}{c_1p_{1j} + c_2p_{2j} + \dots + c_ip_{ij} + \dots + c_mp_{mj}}
\]

That's the full theory, though it might be good to work through an
example or two. Here's one that's slightly more complicated than on
Tuesday, or in the book.

There are only two status we care about: having disease D or not having
it. (Those are the X's.) The test, however, has three outcomes:
positive, unclear, or negative. Here are the relevant probabilities:

\begin{itemize}
\tightlist
\item
  Prior probability of having the disease is 7\%, i.e., 0.07.
\item
  Probability of positive test given the disease is 80\%, and
  probability of an unclear test is 20\%.
\item
  Probability of negative test given not having the disease is 85\%, and
  probability of an unclear test is 10\%.
\end{itemize}

So we can deduce two more relevant acts

\begin{itemize}
\tightlist
\item
  Prior probability of not having the disease is 0.93, i.e.,
  \(1 - 0.07\).
\item
  Probability of positive test given not having the disease is 0.05,
  i.e., \(1 - 0.85 - 0.1\).
\end{itemize}

So now we can make the full table.

\begin{longtable}[]{@{}rlll@{}}
\toprule
\begin{minipage}[b]{0.06\columnwidth}\raggedleft
\strut
\end{minipage} & \begin{minipage}[b]{0.28\columnwidth}\raggedright
Positive Test\strut
\end{minipage} & \begin{minipage}[b]{0.26\columnwidth}\raggedright
Unclear Test\strut
\end{minipage} & \begin{minipage}[b]{0.28\columnwidth}\raggedright
Negative Test\strut
\end{minipage}\tabularnewline
\midrule
\endhead
\begin{minipage}[t]{0.06\columnwidth}\raggedleft
Has Disease\strut
\end{minipage} & \begin{minipage}[t]{0.28\columnwidth}\raggedright
\(0.07 \times 0.8 = 0.056\)\strut
\end{minipage} & \begin{minipage}[t]{0.26\columnwidth}\raggedright
\(0.07 \times 0.2 = 0.014\)\strut
\end{minipage} & \begin{minipage}[t]{0.28\columnwidth}\raggedright
\(0.07 \times 0 = 0\)\strut
\end{minipage}\tabularnewline
\begin{minipage}[t]{0.06\columnwidth}\raggedleft
No Disease\strut
\end{minipage} & \begin{minipage}[t]{0.28\columnwidth}\raggedright
\(0.93 \times 0.05 = 0.0465\)\strut
\end{minipage} & \begin{minipage}[t]{0.26\columnwidth}\raggedright
\(0.93 \times 0.1 = 0.093\)\strut
\end{minipage} & \begin{minipage}[t]{0.28\columnwidth}\raggedright
\(0.93 \times 0.85 = 0.7905\)\strut
\end{minipage}\tabularnewline
\bottomrule
\end{longtable}

The probability of having the disease given a positive test is

\[
\frac{0.056}{0.056 + 0.0465} \approx 0.546
\]

The probability of having the disease given an unclear test is

\[
\frac{0.014}{0.014 + 0.093} \approx 0.13
\]

And obviously the probability of having the disease given a negative
test is 0.

\hypertarget{random-variables}{%
\section{Random Variables}\label{random-variables}}

A \textbf{random variable} is simply a variable that takes different
numerical values in different states. In other words, it is a function
from possibilities to numbers. It need not be `random' in any familiar
sense. The function from possible situations to the value of 2 + 2 in
that situation is a random variable, albeit a constant one. It's just a
slightly confusing term for any variable that takes different,
numerical, values in different situations.

Typically, random variables are denoted by capital letters. So we might
have a random variable \(X\) whose value is the age of the next
President of the United States, at his or her inauguration. Or we might
have a random variable \(Y\) that is the number of children you will
have in your lifetime. Basically any mapping from possibilities to
numbers can be a random variable.

Here's one topical example. You've asked each of your friends who will
win the big football game this weekend. 9 said the home team will win.
(I don't know who is the home team in the Superbowl; I think it's the
AFC team.) 5 said the away team will win. Then we can let \(X\) be a
random variable measuring the number of your friends who correctly
predicted the result of the game.

\begin{equation*}
X = 
    \begin{cases}
        9,& \text{if the home team wins} ,\\ 
        5,& \text{if the away team wins} .
    \end{cases}
\end{equation*}

\hypertarget{expected-value}{%
\section{Expected Value}\label{expected-value}}

Given a random variable \(X\) and a probability function \(\Pr\), we can
work out the \textbf{expected value} of that random variable with
respect to that probability function. Intuitively, the expected value of
\(X\) is a weighted average of the possible values of \(X\), where the
weights are given by the probability (according to \(\Pr\)) of each
value coming about.

More formally, we work out the expected value of \(X\) this way.

\begin{itemize}
\tightlist
\item
  For each possibility, we multiply the value of \(X\) in that case by
  the probability of the possibility obtaining.
\item
  Then we sum the numbers we've produced, and the result is the expected
  value of \(X\).
\item
  We'll write the expected value of \(X\) as \(Exp(X)\).
\end{itemize}

So in the earlier example, if the probability that the home wins is 0.8,
and the probability that the away team wins is 0.2, then

\begin{align*}
Exp(X) &= 9 \times 0.8 + 5 \times 0.2 \\
 &= 7.2 + 1 \\
 &= 8.2
\end{align*}

The expected value of \(X\) isn't in any sense the value that we expect
\(X\) to take. It's more like an average. If this kind of situation
recurs a lot, you would expect the long run average value \(X\) takes to
be roundabout the expected value. That's a better way of conceptualising
what expected values are.

\hypertarget{maximise-expected-value}{%
\section{Maximise Expected Value}\label{maximise-expected-value}}

Here's the core principle of most modern theories of decision. If you
want to promote some value, you should \textbf{maximise its expected
value}. So if you're a business, and you want to make money, you should
maximise the expected profit from ventures. If you're a sport team, and
you want to score a lot of points, you should maximise the expected
points from each play.

The reason is that in the long run, is what's going to produce the best
results. But what if there is no long run? Well, the same reasoning
still goes through as long as the thing we are focussing on is the only
thing we care about.

And it might not be. Sports teams don't just try to score points, they
try to prevent the other team scoring, and ultimately they try to win
games. Businesses don't just try to maximise profits; they might have
reasons for earning profits at a particular time, or they might have
other non-financial goals that they are trying to promote.

\hypertarget{utility}{%
\section{Utility}\label{utility}}

The orthodox view is that (for rational, coherent) people we can
collapse all the things that they care about into a single measure,
utility. A rational choice is one that maximises expected utility. But
note this is a little misleading; it suggests that what makes a choice
rational is that it maximises expected utility. Some theorists think
that is true, but others think the order of explanation goes the other
way. (I'm sort of on that latter team.)

Anyway, the big picture is that there is some random variable \(U\) that
measures how well off you are in each possibility. And the key feature
of \(U\) is that it obeys the following two constraints, which we sort
of take to amount to the same thing.

\begin{quote}
If \(U(w_2) - U(w_1) = U(w_3) - U(w_2)\) then\\
- The amount that \(w_3\) is better than \(w_2\) equals the amount that
\(w_2\) is better than \(w_1\). - You are indifferent between having
\(w_2\) as a sure outcome, and a 50/50 chance of ending up in either
\(w_1\) or \(w_3\).
\end{quote}

It's worth thinking through this in some more practical cases. Imagine
you are offered a choice of two envelopes, one Red, and one Green. The
Red envelope has a lottery ticket in it that will lead to you either
winning \$R, or winning nothing, each of them equally likely. The Green
envelope has cash in it, specifically \$G. For each of the following
values for \(R\), find a value for \(G\) such that you personally would
be equally happy to be given the Red envelope or the Green envelope.

\begin{longtable}[]{@{}cc@{}}
\toprule
Red & Green\tabularnewline
\midrule
\endhead
\$10 & ??\tabularnewline
\$100 & ??\tabularnewline
\$1,000 & ??\tabularnewline
\$10,000 & ??\tabularnewline
\$1,000,000 & ??\tabularnewline
\$1,000,000,000 & ??\tabularnewline
\bottomrule
\end{longtable}

Theory is silent on how you should answer this; it depends on how much
you value near versus far increases in wealth.


\end{document}
