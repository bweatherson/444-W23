\documentclass[11pt,]{article}
\usepackage{lmodern}
\usepackage{amssymb,amsmath}
\usepackage{ifxetex,ifluatex}
\usepackage{fixltx2e} % provides \textsubscript
\ifnum 0\ifxetex 1\fi\ifluatex 1\fi=0 % if pdftex
  \usepackage[T1]{fontenc}
  \usepackage[utf8]{inputenc}
\else % if luatex or xelatex
  \ifxetex
    \usepackage{mathspec}
  \else
    \usepackage{fontspec}
  \fi
  \defaultfontfeatures{Ligatures=TeX,Scale=MatchLowercase}
    \setmainfont[]{SF Pro Text Light}
\fi
% use upquote if available, for straight quotes in verbatim environments
\IfFileExists{upquote.sty}{\usepackage{upquote}}{}
% use microtype if available
\IfFileExists{microtype.sty}{%
\usepackage{microtype}
\UseMicrotypeSet[protrusion]{basicmath} % disable protrusion for tt fonts
}{}
\usepackage[margin=1in]{geometry}
\usepackage{hyperref}
\hypersetup{unicode=true,
            pdftitle={Arrow's Theorem},
            pdfauthor={Philosophy 444},
            pdfborder={0 0 0},
            breaklinks=true}
\urlstyle{same}  % don't use monospace font for urls
\usepackage{longtable,booktabs}
\usepackage{graphicx,grffile}
\makeatletter
\def\maxwidth{\ifdim\Gin@nat@width>\linewidth\linewidth\else\Gin@nat@width\fi}
\def\maxheight{\ifdim\Gin@nat@height>\textheight\textheight\else\Gin@nat@height\fi}
\makeatother
% Scale images if necessary, so that they will not overflow the page
% margins by default, and it is still possible to overwrite the defaults
% using explicit options in \includegraphics[width, height, ...]{}
\setkeys{Gin}{width=\maxwidth,height=\maxheight,keepaspectratio}
\IfFileExists{parskip.sty}{%
\usepackage{parskip}
}{% else
\setlength{\parindent}{0pt}
\setlength{\parskip}{6pt plus 2pt minus 1pt}
}
\setlength{\emergencystretch}{3em}  % prevent overfull lines
\providecommand{\tightlist}{%
  \setlength{\itemsep}{0pt}\setlength{\parskip}{0pt}}
\setcounter{secnumdepth}{0}
% Redefines (sub)paragraphs to behave more like sections
\ifx\paragraph\undefined\else
\let\oldparagraph\paragraph
\renewcommand{\paragraph}[1]{\oldparagraph{#1}\mbox{}}
\fi
\ifx\subparagraph\undefined\else
\let\oldsubparagraph\subparagraph
\renewcommand{\subparagraph}[1]{\oldsubparagraph{#1}\mbox{}}
\fi

%%% Use protect on footnotes to avoid problems with footnotes in titles
\let\rmarkdownfootnote\footnote%
\def\footnote{\protect\rmarkdownfootnote}

%%% Change title format to be more compact
\usepackage{titling}

% Create subtitle command for use in maketitle
\providecommand{\subtitle}[1]{
  \posttitle{
    \begin{center}\large#1\end{center}
    }
}

\setlength{\droptitle}{-2em}

  \title{Arrow's Theorem}
    \pretitle{\vspace{\droptitle}\centering\huge}
  \posttitle{\par}
    \author{Philosophy 444}
    \preauthor{\centering\large\emph}
  \postauthor{\par}
      \predate{\centering\large\emph}
  \postdate{\par}
    \date{4 November, 2019}

\usepackage{mathastext}
\usepackage{nicefrac}

\begin{document}
\maketitle

In the other notes I was a bit informal about stating Arrow's Theorem,
and this led to some confusion in class I think. So I thought I'd write
down the formal version of it carefully here, though we won't be
referring to it a lot.

Assume we are trying to construct a \textbf{social choice function}
(scf) with the following features. There are \(n\) citizens and \(k\)
options. Each citizen has a preference function that is a \textbf{total
preorder} over the \(k\) options. The scf takes these \(n\) total
preorders as inputs, and produces a total preorder representing the
social preferences as output.

A total preorder is a function \(\geq\) over the options satisfying:

\begin{description}
\tightlist
\item[Completeness]
For any \(x, y\), either \(x \geq y\) or \(y \geq x\) (or both).
\item[Transitivity]
For any \(x, y, z\) if \(x \geq y\) and \(y \geq z\) then \(x \geq z\).
\end{description}

Intuitively, \(x \geq y\) means that the relevant citizen (or group)
thinks that \(x\) is at least as good as \(y\). So we are assuming, in
effect, that each citizen's preference ranking over the options is a
total preorder.

For each citizen, there is a corresponding dictator function. The
dictator function for citizen \(i\) is the scf that takes as inputs all
the preference rankings, and outputs the rankings of citizen \(i\),
whatever the other \(n-1\) citizens say.

Arrow's Theorem is that these \(n\) dictator functions are the only
functions that satisfy

\begin{description}
\tightlist
\item[Universal Domain]
For any possible input, the scf deterministically produces an output.
\item[Independence of Irrelevant Alternatives (IIA)]
The social rank of \(x\) and \(y\) is a function of just how the \(n\)
citizens rank \(x\) and \(y\), not of how they rank these two options
compared to other options.
\item[Pareto]
If all citzens prefer \(x\) to \(y\), then so does the social ranking
\end{description}

Universal domain does not rule out any kind of randomness. Since a total
preorder allows for ties, when both \(x \geq y\) and \(y \geq x\), it
could be that the scf outputs a ranking where \(x\) and \(y\) are tied
at the top. And it could be that we use some kind of random process to
choose between them in that circumstance. But it doesn't allow for any
output where you would naturally use weighted probabilities - e.g.,
choose \(x\) with probability 0.8 and \(y\) with probability 0.2.

IIA is the really strong condition, the one that is violated by pretty
much all formal systems that we use.

In the first version of his impossibility theorem, Arrow imposed a
stronger condition than Pareto, namely monotonicity

\newpage

\begin{description}
\tightlist
\item[Monotonicity]
Improving the position of \(x\) in one citizen's rankings, and making no
other changes, cannot reduce \(x\)'s position in the scf.
\end{description}

This condition is satisfied by plurality voting, but it is violated by
runoff voting. (And, as a consequence, by instant runoff voting.) To see
this, imagine a contest with three candidates, Left, Center and Right.
(This example is similar to, but simpler than, the example of
monotonicity violation on the wikipedia page about monotonicity.) There
are 100 voters, and they are divided up as follows.

\begin{longtable}[]{@{}ccc@{}}
\toprule
1st Choice & 2nd Choice & Voters\tabularnewline
\midrule
\endhead
Left & Center & 45\tabularnewline
Center & Left & 12\tabularnewline
Center & Right & 15\tabularnewline
Right & Center & 28\tabularnewline
\bottomrule
\end{longtable}

Left gets 45, Center gets 27 and Right gets 28. So in the runoff it's
Left vs Right, and Left wins 57-43.

Now imagine that two citizens are radicalised by a dramatic experience.
They go from being Right-wing voters to Left-wing voters. The table now
looks like

\begin{longtable}[]{@{}ccc@{}}
\toprule
1st Choice & 2nd Choice & Voters\tabularnewline
\midrule
\endhead
Left & Center & 47\tabularnewline
Center & Left & 12\tabularnewline
Center & Right & 15\tabularnewline
Right & Center & 26\tabularnewline
\bottomrule
\end{longtable}

Now Left gets 47, Center gets 27 and Right gets 26. So in the runoff
it's Left vs Center, and Center wins 53-47.

Something seems off here, though it's not immediately obvious where the
bad thing happened. But having these two voters start voting for Left
doesn't seem like it should cause Left to lose.

Some critics of runoff voting (and IRV) think this is a serious reason
to abandon it. I am a little suspicious of how widespread an issue this
is. IRV is used for hundreds (if not thousands) of elections in
Australia each year, and there are a handful of cases in all of
electoral history where this might have been an issue. So I suspect it
is more of a theoretical problem than a practical problem, but it is a
surprising theoretical problem.


\end{document}
