\documentclass[11pt,]{article}
\usepackage{lmodern}
\usepackage{amssymb,amsmath}
\usepackage{ifxetex,ifluatex}
\usepackage{fixltx2e} % provides \textsubscript
\ifnum 0\ifxetex 1\fi\ifluatex 1\fi=0 % if pdftex
  \usepackage[T1]{fontenc}
  \usepackage[utf8]{inputenc}
\else % if luatex or xelatex
  \ifxetex
    \usepackage{mathspec}
  \else
    \usepackage{fontspec}
  \fi
  \defaultfontfeatures{Ligatures=TeX,Scale=MatchLowercase}
    \setmainfont[]{SF Pro Text Light}
\fi
% use upquote if available, for straight quotes in verbatim environments
\IfFileExists{upquote.sty}{\usepackage{upquote}}{}
% use microtype if available
\IfFileExists{microtype.sty}{%
\usepackage{microtype}
\UseMicrotypeSet[protrusion]{basicmath} % disable protrusion for tt fonts
}{}
\usepackage[margin=0.7in]{geometry}
\usepackage{hyperref}
\hypersetup{unicode=true,
            pdftitle={Plan for Second SPE Class},
            pdfauthor={Philosophy 444},
            pdfborder={0 0 0},
            breaklinks=true}
\urlstyle{same}  % don't use monospace font for urls
\usepackage{graphicx,grffile}
\makeatletter
\def\maxwidth{\ifdim\Gin@nat@width>\linewidth\linewidth\else\Gin@nat@width\fi}
\def\maxheight{\ifdim\Gin@nat@height>\textheight\textheight\else\Gin@nat@height\fi}
\makeatother
% Scale images if necessary, so that they will not overflow the page
% margins by default, and it is still possible to overwrite the defaults
% using explicit options in \includegraphics[width, height, ...]{}
\setkeys{Gin}{width=\maxwidth,height=\maxheight,keepaspectratio}
\IfFileExists{parskip.sty}{%
\usepackage{parskip}
}{% else
\setlength{\parindent}{0pt}
\setlength{\parskip}{6pt plus 2pt minus 1pt}
}
\setlength{\emergencystretch}{3em}  % prevent overfull lines
\providecommand{\tightlist}{%
  \setlength{\itemsep}{0pt}\setlength{\parskip}{0pt}}
\setcounter{secnumdepth}{0}
% Redefines (sub)paragraphs to behave more like sections
\ifx\paragraph\undefined\else
\let\oldparagraph\paragraph
\renewcommand{\paragraph}[1]{\oldparagraph{#1}\mbox{}}
\fi
\ifx\subparagraph\undefined\else
\let\oldsubparagraph\subparagraph
\renewcommand{\subparagraph}[1]{\oldsubparagraph{#1}\mbox{}}
\fi

%%% Use protect on footnotes to avoid problems with footnotes in titles
\let\rmarkdownfootnote\footnote%
\def\footnote{\protect\rmarkdownfootnote}

%%% Change title format to be more compact
\usepackage{titling}

% Create subtitle command for use in maketitle
\providecommand{\subtitle}[1]{
  \posttitle{
    \begin{center}\large#1\end{center}
    }
}

\setlength{\droptitle}{-2em}

  \title{Plan for Second SPE Class}
    \pretitle{\vspace{\droptitle}\centering\huge}
  \posttitle{\par}
    \author{Philosophy 444}
    \preauthor{\centering\large\emph}
  \postauthor{\par}
      \predate{\centering\large\emph}
  \postdate{\par}
    \date{9 October, 2019}

\usepackage{gensymb}
\usepackage{nicefrac}
\usepackage{mathastext}
\usepackage{multicol}

\begin{document}
\maketitle

\hypertarget{lemons}{%
\section{Lemons}\label{lemons}}

\begin{itemize}
\tightlist
\item
  Assume 80\% good, 20\% bad
\item
  Assume good are worth 15, bad are worth 5.
\item
  Assume seller will take a 10\% discount because need to sell
\item
  If no one knows about quality, all cars sell
\item
  If everyone knows about quality, all cars sell
\item
  If buyer knows and seller does not, all cars sell
\item
  If seller knows and buyer does not, no cars sell
\end{itemize}

\hypertarget{exit-or-play}{%
\section{Exit or Play}\label{exit-or-play}}

\begin{itemize}
\tightlist
\item
  This is a big reason to use SPE
\item
  Choice between playing PD and getting 2, you take the 2
\item
  Same with these various mixed strategy NEs
\item
  Really works best when there is a single NE
\end{itemize}

\hypertarget{steering-towards-ne}{%
\section{Steering towards NE}\label{steering-towards-ne}}

\begin{itemize}
\tightlist
\item
  Choice of getting 2 or playing Battle of Sexes
\item
  If you play, obviously intend to play to get 4
\item
  So that's what you'll get
\item
  Huge advantage here
\item
  Note that Battle of Sexes has three equilibria - but the other two are
  ruled out by playing
\end{itemize}

\hypertarget{first-puzzle-about-spe---rules-out-too-much}{%
\section{First Puzzle about SPE - rules out too
much}\label{first-puzzle-about-spe---rules-out-too-much}}

\begin{itemize}
\tightlist
\item
  Chain store game
\item
  Go over the basic structure
\item
  First competitor chooses exit or entry
\item
  If exit, payoff is 5,1 (incumbent first)
\item
  If compete, then tough or conciliate
\item
  Tough = 0,0; Conciliate = 2,2
\item
  Nash says do either, SPE says Enter-Conciliate
\item
  And it says this if there are 100 successive possible entrants
\item
  But really?
\end{itemize}

\hypertarget{what-is-a-strategy}{%
\section{What is a Strategy}\label{what-is-a-strategy}}

\begin{itemize}
\tightlist
\item
  Key point: strategies include moves to be made at ruled out nodes
\item
  How to interpret that?
\item
  Metaphysical - it's what to do if you screw up - but why include that
  - and why assume you'll be able to carry it out
\item
  Epistemological - it's what the other person should believe
\item
  But now there are two problems
\item
  First, intuitively you choose a strategy, but you don't choose what
  the other person believes
\item
  Second, why should they believe that you'll go back to SPE after not
  doing it
\item
  This is particularly pressing in Chain Store Game
\item
  If I compete the first five times, the sixth competitor should get the
  message
\item
  But the unique SPE has me cooperating at that very node
\item
  For Nash, we avoided this by saying it's all about the long run
\item
  But in SPE settings, the long run is just another game that SPE should
  apply to
\item
  This is something of a mess, and I don't think anyone has a good way
  out
\item
  At this point, some folks get off the bus and say we should just do
  experimental work
\end{itemize}

\hypertarget{second-puzzle-about-spe---rules-out-too-litte}{%
\section{Second Puzzle about SPE - rules out too
litte}\label{second-puzzle-about-spe---rules-out-too-litte}}

\begin{itemize}
\tightlist
\item
  Money burning game
\item
  See other handout
\end{itemize}


\end{document}
