\documentclass[11pt,]{article}
\usepackage{lmodern}
\usepackage{amssymb,amsmath}
\usepackage{ifxetex,ifluatex}
\usepackage{fixltx2e} % provides \textsubscript
\ifnum 0\ifxetex 1\fi\ifluatex 1\fi=0 % if pdftex
  \usepackage[T1]{fontenc}
  \usepackage[utf8]{inputenc}
\else % if luatex or xelatex
  \ifxetex
    \usepackage{mathspec}
  \else
    \usepackage{fontspec}
  \fi
  \defaultfontfeatures{Ligatures=TeX,Scale=MatchLowercase}
    \setmainfont[]{SF Pro Text Light}
\fi
% use upquote if available, for straight quotes in verbatim environments
\IfFileExists{upquote.sty}{\usepackage{upquote}}{}
% use microtype if available
\IfFileExists{microtype.sty}{%
\usepackage{microtype}
\UseMicrotypeSet[protrusion]{basicmath} % disable protrusion for tt fonts
}{}
\usepackage[margin=1in]{geometry}
\usepackage{hyperref}
\hypersetup{unicode=true,
            pdftitle={Honest Signalling},
            pdfauthor={Philosophy 444},
            pdfborder={0 0 0},
            breaklinks=true}
\urlstyle{same}  % don't use monospace font for urls
\usepackage{graphicx,grffile}
\makeatletter
\def\maxwidth{\ifdim\Gin@nat@width>\linewidth\linewidth\else\Gin@nat@width\fi}
\def\maxheight{\ifdim\Gin@nat@height>\textheight\textheight\else\Gin@nat@height\fi}
\makeatother
% Scale images if necessary, so that they will not overflow the page
% margins by default, and it is still possible to overwrite the defaults
% using explicit options in \includegraphics[width, height, ...]{}
\setkeys{Gin}{width=\maxwidth,height=\maxheight,keepaspectratio}
\IfFileExists{parskip.sty}{%
\usepackage{parskip}
}{% else
\setlength{\parindent}{0pt}
\setlength{\parskip}{6pt plus 2pt minus 1pt}
}
\setlength{\emergencystretch}{3em}  % prevent overfull lines
\providecommand{\tightlist}{%
  \setlength{\itemsep}{0pt}\setlength{\parskip}{0pt}}
\setcounter{secnumdepth}{0}
% Redefines (sub)paragraphs to behave more like sections
\ifx\paragraph\undefined\else
\let\oldparagraph\paragraph
\renewcommand{\paragraph}[1]{\oldparagraph{#1}\mbox{}}
\fi
\ifx\subparagraph\undefined\else
\let\oldsubparagraph\subparagraph
\renewcommand{\subparagraph}[1]{\oldsubparagraph{#1}\mbox{}}
\fi

%%% Use protect on footnotes to avoid problems with footnotes in titles
\let\rmarkdownfootnote\footnote%
\def\footnote{\protect\rmarkdownfootnote}

%%% Change title format to be more compact
\usepackage{titling}

% Create subtitle command for use in maketitle
\providecommand{\subtitle}[1]{
  \posttitle{
    \begin{center}\large#1\end{center}
    }
}

\setlength{\droptitle}{-2em}

  \title{Honest Signalling}
    \pretitle{\vspace{\droptitle}\centering\huge}
  \posttitle{\par}
    \author{Philosophy 444}
    \preauthor{\centering\large\emph}
  \postauthor{\par}
      \predate{\centering\large\emph}
  \postdate{\par}
    \date{30 October, 2019}

\usepackage{mathastext}
\usepackage{nicefrac}

\begin{document}
\maketitle

\hypertarget{big-themes}{%
\subsection{Big Themes}\label{big-themes}}

\begin{itemize}
\tightlist
\item
  There are other signalling models than the Spence model. You can think
  that tail feathers or stotting or college attendance is signalling
  without thinking this is the right model.
\item
  The Spence model says that non-signallers \textbf{can} signal but
  \textbf{won't}. We should also think about models where non-signallers
  \textbf{would} signal but \textbf{can't}.
\end{itemize}

\hypertarget{three-big-questions}{%
\subsection{Three Big Questions}\label{three-big-questions}}

\begin{enumerate}
\def\labelenumi{\arabic{enumi}.}
\tightlist
\item
  Why does receiver take the signal seriously?
\item
  Why does signaller send signal? In particular, why do they send this
  signal? Why don't signallers collectively organise a cheaper signal?
\item
  Why don't non-signallers send signal?
\end{enumerate}

\hypertarget{four-questions-from-last-time}{%
\subsection{Four Questions from Last
Time}\label{four-questions-from-last-time}}

\begin{enumerate}
\def\labelenumi{\arabic{enumi}.}
\tightlist
\item
  What aspects of this model seem to resemble the world you (as in
  literally you personally) find yourself in?
\item
  What aspects of it seem to differ from the world in significant ways?
\item
  What empirical data would make you think this model was right in some
  important way?
\item
  What empirical data would make you think this model was wrong in some
  important way?
\end{enumerate}

\hypertarget{my-answers}{%
\subsection{My Answers}\label{my-answers}}

\begin{enumerate}
\def\labelenumi{\arabic{enumi}.}
\tightlist
\item
  The basic structure seems plausible. It isn't obvious how what we do
  here makes you a more valuable employee. It might make you better
  citizens, but employers don't care about that. And calculus class
  really is less pleasant for people who won't be as valuable.
\item
  The wage premium is so high that it's hard to believe \(c_2 > 1\). The
  same is true for the tail feathers and stotting examples. The payouts
  to Difficult are really really high, and calculus class isn't that
  differentially unpleasant.
\item
  Restricting things to data I know exist - one thing that supports the
  model is the `sheepskin effect'. If you divide the `some college'
  group by how much college they got, the skills hypotheis would predict
  that the more college you did, the higher your wage premium. That
  might be approximately true. But it would also predict that if you're
  one course from graduation, you would get 95\% or more of the wage
  premium. And that's wildly false.
\item
  One thing that's trouble for the model is that the wage premium is
  really high among older workers. To make that work, you need one of a
  few implausible things. One possibility is that you need employers who
  are so unobservant that they can't tell the valuable workers from the
  not valuable workers after years and years of work, so they still have
  to rely on degrees as a signalling device.
\end{enumerate}

\newpage

\hypertarget{another-hypothesis}{%
\subsection{Another Hypothesis}\label{another-hypothesis}}

So far we've assumed everyone can do Difficult or Easy, but it is more
costly for Low than High. Maybe we should drop that assumption. Here is
another kind of signalling model that we could consider.

\begin{itemize}
\tightlist
\item
  The basic structure of sender, receiver, signal and receiver actions
  are the same.
\item
  Now the cost of Difficult is the same for High and Low.
\item
  But now Sender isn't in full control of their actions.
\item
  If they choose Easy, then Easy happens.
\item
  But if they choose Difficult, then they have to pay the cost \(c\),
  but what happens, and what Receiver sees, is Difficult with
  probability \(p\), and Easy with probability \(1-p\).
\item
  And the probability is high for High and low for Low.
\item
  Again, we get a separating equilibrium.
\item
  Receiver does Risky if they see High, and Safe if they see Low. (Or
  the other way around in the stotting game.)
\item
  And the expected payouts are such that High should take the chance and
  do Difficult, while Low should not.
\item
  At the extreme, the probability of success is 1 for High and 0 for
  Low, but the model works even with much more balanced probabilities.
\end{itemize}

This is sometimes called the \textbf{Honest Signalling} model, or the
\textbf{Indexical Signalling} model, as opposed to the \textbf{Handicap
Principle} model we started with.

\hypertarget{mixing-the-hypotheses}{%
\subsection{Mixing the Hypotheses}\label{mixing-the-hypotheses}}

Maybe for some Senders, they have a choice about what costs to incur,
with the more costs they incur increasing their probability of success.

\begin{itemize}
\tightlist
\item
  This is easier to see in the college case. Imagine a person who has
  the skills to finish a college degree, but doesn't have the skills to
  finish while holding down a 40 hour job, and it would be really costly
  to give up the 40 hour job.
\item
  For that person, going to college and keeping the job might not be
  worth it - it would be like doing the low-probability Difficult
  signal, which usually just results in paying a cost and getting no
  return.
\item
  But going to college and giving up the job might not be worth it
  either. If the cost \(c\) isn't just the tuition cost, but the money
  foresaken from the job, and perhaps the interest on the loans taken
  out to cover that money, then perhaps the cost is higher than the
  premium.
\end{itemize}

In general, we might want to be a little sceptical that there is clean
line between cases where a player chooses not to send a costly signal,
and cases where that player doesn't have the ability to (reliably) send
that signal. Maybe the signal success probability is a function of the
costs incurred, and there is no level of costs they can justify
spending.

\hypertarget{for-more-information}{%
\subsection{For More information}\label{for-more-information}}

For stats on the college wage premium over time, see

\begin{itemize}
\tightlist
\item
  \url{https://fredblog.stlouisfed.org/2018/07/is-college-still-worth-it/}
\end{itemize}

For information on the college wealth premium, plus stats on the
demographics of both the wage premium and the wealth premium, see

\begin{itemize}
\tightlist
\item
  \url{https://www.stlouisfed.org/~/media/files/pdfs/hfs/is-college-worth-it/emmons_symposium.pdf?la=en}
\end{itemize}


\end{document}
