\documentclass[11pt,]{article}
\usepackage{lmodern}
\usepackage{amssymb,amsmath}
\usepackage{ifxetex,ifluatex}
\usepackage{fixltx2e} % provides \textsubscript
\ifnum 0\ifxetex 1\fi\ifluatex 1\fi=0 % if pdftex
  \usepackage[T1]{fontenc}
  \usepackage[utf8]{inputenc}
\else % if luatex or xelatex
  \ifxetex
    \usepackage{mathspec}
  \else
    \usepackage{fontspec}
  \fi
  \defaultfontfeatures{Ligatures=TeX,Scale=MatchLowercase}
    \setmainfont[]{SF Pro Text Light}
\fi
% use upquote if available, for straight quotes in verbatim environments
\IfFileExists{upquote.sty}{\usepackage{upquote}}{}
% use microtype if available
\IfFileExists{microtype.sty}{%
\usepackage{microtype}
\UseMicrotypeSet[protrusion]{basicmath} % disable protrusion for tt fonts
}{}
\usepackage[margin=0.7in]{geometry}
\usepackage{hyperref}
\hypersetup{unicode=true,
            pdftitle={Plan for Dynamic Class},
            pdfauthor={Philosophy 444},
            pdfborder={0 0 0},
            breaklinks=true}
\urlstyle{same}  % don't use monospace font for urls
\usepackage{graphicx,grffile}
\makeatletter
\def\maxwidth{\ifdim\Gin@nat@width>\linewidth\linewidth\else\Gin@nat@width\fi}
\def\maxheight{\ifdim\Gin@nat@height>\textheight\textheight\else\Gin@nat@height\fi}
\makeatother
% Scale images if necessary, so that they will not overflow the page
% margins by default, and it is still possible to overwrite the defaults
% using explicit options in \includegraphics[width, height, ...]{}
\setkeys{Gin}{width=\maxwidth,height=\maxheight,keepaspectratio}
\IfFileExists{parskip.sty}{%
\usepackage{parskip}
}{% else
\setlength{\parindent}{0pt}
\setlength{\parskip}{6pt plus 2pt minus 1pt}
}
\setlength{\emergencystretch}{3em}  % prevent overfull lines
\providecommand{\tightlist}{%
  \setlength{\itemsep}{0pt}\setlength{\parskip}{0pt}}
\setcounter{secnumdepth}{0}
% Redefines (sub)paragraphs to behave more like sections
\ifx\paragraph\undefined\else
\let\oldparagraph\paragraph
\renewcommand{\paragraph}[1]{\oldparagraph{#1}\mbox{}}
\fi
\ifx\subparagraph\undefined\else
\let\oldsubparagraph\subparagraph
\renewcommand{\subparagraph}[1]{\oldsubparagraph{#1}\mbox{}}
\fi

%%% Use protect on footnotes to avoid problems with footnotes in titles
\let\rmarkdownfootnote\footnote%
\def\footnote{\protect\rmarkdownfootnote}

%%% Change title format to be more compact
\usepackage{titling}

% Create subtitle command for use in maketitle
\providecommand{\subtitle}[1]{
  \posttitle{
    \begin{center}\large#1\end{center}
    }
}

\setlength{\droptitle}{-2em}

  \title{Plan for Dynamic Class}
    \pretitle{\vspace{\droptitle}\centering\huge}
  \posttitle{\par}
    \author{Philosophy 444}
    \preauthor{\centering\large\emph}
  \postauthor{\par}
      \predate{\centering\large\emph}
  \postdate{\par}
    \date{7 October, 2019}

\usepackage{gensymb}
\usepackage{nicefrac}
\usepackage{mathastext}
\usepackage{multicol}

\begin{document}
\maketitle

\hypertarget{information-sets}{%
\section{Information sets}\label{information-sets}}

\begin{itemize}
\tightlist
\item
  Sets of nodes that player doesn't know which they are at
\item
  Gotta have same choices available
\item
  None is predecessor of the other
\item
  Perfect recall: her prior moves are same at each node
\item
  See violations of this on page 120
\end{itemize}

\hypertarget{how-to-represent-choice-followed-by-strategic-game}{%
\section{How to represent choice followed by strategic
game}\label{how-to-represent-choice-followed-by-strategic-game}}

\begin{itemize}
\tightlist
\item
  Would like to draw grids in the game, but we can't do that.
\item
  Instead we use a hack - pretend the moves are sequential, but
  unrevealed
\item
  Feels like a hack, and sort of is
\end{itemize}

\hypertarget{strategies}{%
\section{Strategies}\label{strategies}}

\begin{itemize}
\tightlist
\item
  A choice for each information set
\item
  This is a generalisation of what we previously had
\end{itemize}

\hypertarget{subgame}{%
\section{Subgame}\label{subgame}}

\begin{itemize}
\tightlist
\item
  Only singleton nodes launch subgames
\item
  Every successor of the launch node must not `cut' info set
\item
  That is, no successor is in info set with non-successor
\item
  See example on page 127
\item
  This matters for definition of SPE
\end{itemize}

\hypertarget{spe}{%
\section{SPE}\label{spe}}

\begin{itemize}
\tightlist
\item
  Strategy set is Nash in whole game and every subgame
\item
  Incredible threats are not SPEs
\end{itemize}

\hypertarget{spe-and-sequential-rationality}{%
\section{SPE and sequential
rationality}\label{spe-and-sequential-rationality}}

\begin{itemize}
\tightlist
\item
  Consider a game where A moves, then B, but B isn't told A's move
\item
  And what B would do is dominated (draw example of this b4 class)
\item
  Well, B is irrational
\item
  But that could be SPE (if A never goes where B dominated)
\item
  Feels like we need more
\item
  We'll get to this
\item
  The basic way to generate these games is to take an incredible threat
  game, and then make the move that would be responded irrationally to
  into a pair of moves.
\item
  So simple game is that I choose attack/defend, and then if attack you
  choose counter/retreat. And one of the Nash equilibria is
  Defend/Counter, but Counter is not backwards induction reliable.
\item
  Now complicate the game so I have three choices -
  Attack+Coffee/Attack+Tea/Defend, and you're told whether Attack or
  Not, but not whether Coffee or Tea
\item
  Now Defend/Counter could be a subgame perfect equilibrium, because
  there isn't a subgame.
\item
  But this is obviously silly - so we need yet more detail
\item
  See also page 241 for another example
\end{itemize}

\hypertarget{chance-moves}{%
\section{Chance moves}\label{chance-moves}}

\begin{itemize}
\tightlist
\item
  Need a basic story about probability
\item
  Additive to 1
\item
  Maybe talk through lemon example
\end{itemize}

\hypertarget{lemon-example-outcomes}{%
\section{Lemon example outcomes}\label{lemon-example-outcomes}}

\begin{itemize}
\tightlist
\item
  Get and keep good car: 10, 0
\item
  Get and keep lemon: -5, 0
\item
  Get high price for good car: 5, 5
\item
  Get high price for lemon: 5, -5
\item
  Get low price for good car: 0, 10
\item
  Get low price for lemon: 0, 5
\item
  In equilibrium, accept all offers for lemon, reject for good car
\item
  So (roughly) would only ever make low offers
\item
  So would only offer to sell lemons
\item
  How does this extend to 3rd party buyer?
\end{itemize}


\end{document}
