\documentclass[11pt,]{article}
\usepackage{lmodern}
\usepackage{amssymb,amsmath}
\usepackage{ifxetex,ifluatex}
\usepackage{fixltx2e} % provides \textsubscript
\ifnum 0\ifxetex 1\fi\ifluatex 1\fi=0 % if pdftex
  \usepackage[T1]{fontenc}
  \usepackage[utf8]{inputenc}
\else % if luatex or xelatex
  \ifxetex
    \usepackage{mathspec}
  \else
    \usepackage{fontspec}
  \fi
  \defaultfontfeatures{Ligatures=TeX,Scale=MatchLowercase}
    \setmainfont[]{SF Pro Text Light}
\fi
% use upquote if available, for straight quotes in verbatim environments
\IfFileExists{upquote.sty}{\usepackage{upquote}}{}
% use microtype if available
\IfFileExists{microtype.sty}{%
\usepackage{microtype}
\UseMicrotypeSet[protrusion]{basicmath} % disable protrusion for tt fonts
}{}
\usepackage[margin=1in]{geometry}
\usepackage{hyperref}
\hypersetup{unicode=true,
            pdftitle={Monty Hall},
            pdfauthor={Philosophy 444},
            pdfborder={0 0 0},
            breaklinks=true}
\urlstyle{same}  % don't use monospace font for urls
\usepackage{graphicx,grffile}
\makeatletter
\def\maxwidth{\ifdim\Gin@nat@width>\linewidth\linewidth\else\Gin@nat@width\fi}
\def\maxheight{\ifdim\Gin@nat@height>\textheight\textheight\else\Gin@nat@height\fi}
\makeatother
% Scale images if necessary, so that they will not overflow the page
% margins by default, and it is still possible to overwrite the defaults
% using explicit options in \includegraphics[width, height, ...]{}
\setkeys{Gin}{width=\maxwidth,height=\maxheight,keepaspectratio}
\IfFileExists{parskip.sty}{%
\usepackage{parskip}
}{% else
\setlength{\parindent}{0pt}
\setlength{\parskip}{6pt plus 2pt minus 1pt}
}
\setlength{\emergencystretch}{3em}  % prevent overfull lines
\providecommand{\tightlist}{%
  \setlength{\itemsep}{0pt}\setlength{\parskip}{0pt}}
\setcounter{secnumdepth}{0}
% Redefines (sub)paragraphs to behave more like sections
\ifx\paragraph\undefined\else
\let\oldparagraph\paragraph
\renewcommand{\paragraph}[1]{\oldparagraph{#1}\mbox{}}
\fi
\ifx\subparagraph\undefined\else
\let\oldsubparagraph\subparagraph
\renewcommand{\subparagraph}[1]{\oldsubparagraph{#1}\mbox{}}
\fi

%%% Use protect on footnotes to avoid problems with footnotes in titles
\let\rmarkdownfootnote\footnote%
\def\footnote{\protect\rmarkdownfootnote}

%%% Change title format to be more compact
\usepackage{titling}

% Create subtitle command for use in maketitle
\providecommand{\subtitle}[1]{
  \posttitle{
    \begin{center}\large#1\end{center}
    }
}

\setlength{\droptitle}{-2em}

  \title{Monty Hall}
    \pretitle{\vspace{\droptitle}\centering\huge}
  \posttitle{\par}
    \author{Philosophy 444}
    \preauthor{\centering\large\emph}
  \postauthor{\par}
      \predate{\centering\large\emph}
  \postdate{\par}
    \date{25 September, 2019}

\usepackage{mathastext}

\begin{document}
\maketitle

I messed up the explanation of the Monty Hall Problem last time, so this
is to go over a little more clear.

\hypertarget{rules-of-the-game}{%
\subsection{Rules of the Game}\label{rules-of-the-game}}

\begin{itemize}
\tightlist
\item
  There are three doors.
\item
  There is a prize behind one, and nothing behind the other two. The
  winning door has been chosen at random, with each door chosen with
  probability \(\frac{1}{3}\).
\item
  Once you select a door, the host will show you a door.
\item
  This door is chosen to have two characteristics - it is not the door
  you chose, and it is not the door that has the prize.
\item
  If you chose incorrectly the first time, the host has no choice, there
  is only one possibility available to show you. (You should confirm for
  yourself this is true.)
\item
  If you chose correctly the first time, the host has a choice. In that
  case, we will assume that the probability that he chooses either door
  is \(\frac{1}{2}\).
\end{itemize}

\hypertarget{terminology}{%
\subsection{Terminology}\label{terminology}}

\begin{itemize}
\tightlist
\item
  I'll use \(P_i\) for the proposition that the prize is behind door
  \(i\).
\item
  And I'll use \(S_i\) for the proposition that the host shows you door
  \(i\).
\item
  One of the consequences of the rules of the game is that for each
  \(i\), we know \(S_i \rightarrow \neg P_i\); the prize is not behind
  the shown door.
\end{itemize}

\hypertarget{prior-probabilities}{%
\subsection{Prior Probabilities}\label{prior-probabilities}}

Let's assume, as in class, that we choose door 3. Then there are four
possibilities available.

\begin{itemize}
\tightlist
\item
  \(P_1 \wedge S_2\)
\item
  \(P_2 \wedge S_1\)
\item
  \(P_3 \wedge S_1\)
\item
  \(P_3 \wedge S_2\)
\end{itemize}

What are the probabilities of each? We can use the following formula.

\begin{quote}
\(\Pr(X \wedge Y) = \Pr(X | Y)\Pr(Y)\)
\end{quote}

Note that this is just the conditional probability formula with each
side multiplied by \(\Pr(Y)\).

By the rules of the game \(\Pr(S_2 | P_1) = 1\). Given \(P_1\), it is
forced that you are shown door 2. And \(\Pr(P_1) = \frac{1}{3}\). So
\(\Pr(S_2 \wedge P_1) = \Pr(S_2 | P_1) \times \Pr(P_1) = 1 \times \frac{1}{3} = \frac{1}{3}\).

The same reasoning shows that \(\Pr(S_1 \wedge P_2) = \frac{1}{3}\).

What about \(P_3 \wedge S_1\). Well, we can use the same approach.
\(\Pr(S_3 \wedge P_1) = \Pr(S_3 | P_1) \times \Pr(P_1)\). But now we
need to know \(\Pr(S_3 | P_1)\). The rules, however, say this is
\(\frac{1}{2}\) - this is the part that says the host will choose each
door with equal probability if he has a free choice about what to do. So
the calculation is
\(\Pr(S_3 \wedge P_1) = \Pr(S_3 | P_1) \times \Pr(P_1) = \frac{1}{2} \times \frac{1}{3} = \frac{1}{6}\).

And the same kind of calculation shows that
\(\Pr(P_3 \wedge S_2) = \frac{1}{6}\).

Summing all that up

\begin{itemize}
\tightlist
\item
  \(\Pr(P_1 \wedge S_2) = \frac{1}{3}\)
\item
  \(\Pr(P_2 \wedge S_1) = \frac{1}{3}\)
\item
  \(\Pr(P_3 \wedge S_1) = \frac{1}{6}\)
\item
  \(\Pr(P_3 \wedge S_2) = \frac{1}{6}\)
\end{itemize}

Let's assume door 2 is shown. (The calculations will be the same if door
1 is shown.) We need to flip the conditional probabilities around. We
want \(\Pr(P_1 | S_2)\). Here's the calculation for that.

\begin{align*}
\Pr(P_1 | S_2) &= \frac{\Pr(P_1 \wedge S_2)}{\Pr(S_2)} \\
 &= \frac{\Pr(P_1 \wedge S_2)}{\Pr((S_2 \wedge P_1) \vee (S_2 \wedge P_3))} \\
 &= \frac{\Pr(P_1 \wedge S_2)}{\Pr(S_2 \wedge P_1) + \Pr(S_2 \wedge P_3)} \\
 &= \frac{\frac{1}{3}}{\frac{1}{3} + \frac{1}{6}} \\
 &= \frac{2}{3}
\end{align*}

So you should switch. If you switch, you have a \(\frac{2}{3}\) chance
of getting the prize. Why then does it seem that you should have a 50/50
shot of getting the prize? Because the following is true.

\[
\Pr(P_1 | \neg P_2) = \frac{1}{2}
\]

On learning \(\neg P_2\), and nothing else, you should think each
remaining door is equally likely to have the prize behind it. And then
you may as well stay.

But the crucial philosophical point, and this is why it is the same kind
of puzzle as the planes and bullets story, is that we have to
distinguish what we've learned from the fact that we've learned it.
Sometimes that latter fact tells us something really crucial. At a very
general level of abstraction, it tells us that a certain process
produced an output. That process might be a game show host choosing to
show us a door, or planes making it back across the English Channel in,
if not one piece, enough state they could get back. In all these cases
you should think not just about what you're being told, but why you're
being told it.


\end{document}
