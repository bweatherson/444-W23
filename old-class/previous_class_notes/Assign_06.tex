\PassOptionsToPackage{unicode=true}{hyperref} % options for packages loaded elsewhere
\PassOptionsToPackage{hyphens}{url}
%
\documentclass[11pt,]{article}
\usepackage{lmodern}
\usepackage{amssymb,amsmath}
\usepackage{ifxetex,ifluatex}
\usepackage{fixltx2e} % provides \textsubscript
\ifnum 0\ifxetex 1\fi\ifluatex 1\fi=0 % if pdftex
  \usepackage[T1]{fontenc}
  \usepackage[utf8]{inputenc}
  \usepackage{textcomp} % provides euro and other symbols
\else % if luatex or xelatex
  \usepackage{unicode-math}
  \defaultfontfeatures{Ligatures=TeX,Scale=MatchLowercase}
    \setmainfont[]{SF Pro Text Light}
\fi
% use upquote if available, for straight quotes in verbatim environments
\IfFileExists{upquote.sty}{\usepackage{upquote}}{}
% use microtype if available
\IfFileExists{microtype.sty}{%
\usepackage[]{microtype}
\UseMicrotypeSet[protrusion]{basicmath} % disable protrusion for tt fonts
}{}
\IfFileExists{parskip.sty}{%
\usepackage{parskip}
}{% else
\setlength{\parindent}{0pt}
\setlength{\parskip}{6pt plus 2pt minus 1pt}
}
\usepackage{hyperref}
\hypersetup{
            pdftitle={Assigment 6},
            pdfauthor={Philosophy 444},
            pdfborder={0 0 0},
            breaklinks=true}
\urlstyle{same}  % don't use monospace font for urls
\usepackage[margin=1in]{geometry}
\usepackage{longtable,booktabs}
% Fix footnotes in tables (requires footnote package)
\IfFileExists{footnote.sty}{\usepackage{footnote}\makesavenoteenv{longtable}}{}
\usepackage{graphicx,grffile}
\makeatletter
\def\maxwidth{\ifdim\Gin@nat@width>\linewidth\linewidth\else\Gin@nat@width\fi}
\def\maxheight{\ifdim\Gin@nat@height>\textheight\textheight\else\Gin@nat@height\fi}
\makeatother
% Scale images if necessary, so that they will not overflow the page
% margins by default, and it is still possible to overwrite the defaults
% using explicit options in \includegraphics[width, height, ...]{}
\setkeys{Gin}{width=\maxwidth,height=\maxheight,keepaspectratio}
\setlength{\emergencystretch}{3em}  % prevent overfull lines
\providecommand{\tightlist}{%
  \setlength{\itemsep}{0pt}\setlength{\parskip}{0pt}}
\setcounter{secnumdepth}{0}
% Redefines (sub)paragraphs to behave more like sections
\ifx\paragraph\undefined\else
\let\oldparagraph\paragraph
\renewcommand{\paragraph}[1]{\oldparagraph{#1}\mbox{}}
\fi
\ifx\subparagraph\undefined\else
\let\oldsubparagraph\subparagraph
\renewcommand{\subparagraph}[1]{\oldsubparagraph{#1}\mbox{}}
\fi

% set default figure placement to htbp
\makeatletter
\def\fps@figure{htbp}
\makeatother

\usepackage{gensymb}
\usepackage{nicefrac}
\usepackage{caption}
\usepackage{istgame}
\usepackage{mathastext}

\title{Assigment 6}
\author{Philosophy 444}
\date{Due November 8, 2019}

\begin{document}
\maketitle

In each of the following sets of questions, there is an election between
four candidates A, B, C and D. The question sets start with a table
showing how many voters have each of the 24 possible preference
orderings over the four candidates. Your job is to figure out how the
election would go under different voting systems. You should assume
(unrealistically) that every voter votes sincerely, and not
strategically. So if a candidate prefers A to B to C to D, they will
vote A in a plurality election, vote B if it is two candidate run off
between B and C, and write down the order ABCD if they are asked to rank
the candidates.

In the tables that follow, higher ranked candidates are to the left. So
the first row of the table shows you the number of voters (339 as it
turns out) who have A as their top choice, then B as their second
choice, then C, and finally D as the last choice.

\newpage

\hypertarget{questions-1-to-2}{%
\section{Questions 1 to 2}\label{questions-1-to-2}}

Here is the table of voters.

\begin{longtable}[]{@{}ll@{}}
\toprule
Ranking & Voters\tabularnewline
\midrule
\endhead
ABCD & 339\tabularnewline
ABDC & 710\tabularnewline
ACBD & 312\tabularnewline
ACDB & 798\tabularnewline
ADBC & 151\tabularnewline
ADCB & 592\tabularnewline
BACD & 956\tabularnewline
BADC & 243\tabularnewline
BCAD & 939\tabularnewline
BCDA & 114\tabularnewline
BDAC & 984\tabularnewline
BDCA & 632\tabularnewline
CABD & 598\tabularnewline
CADB & 902\tabularnewline
CBAD & 574\tabularnewline
CBDA & 245\tabularnewline
CDAB & 860\tabularnewline
CDBA & 74\tabularnewline
DABC & 438\tabularnewline
DACB & 759\tabularnewline
DBAC & 245\tabularnewline
DBCA & 378\tabularnewline
DCAB & 397\tabularnewline
DCBA & 526\tabularnewline
\bottomrule
\end{longtable}

\begin{enumerate}
\def\labelenumi{\arabic{enumi}.}
\tightlist
\item
  Who would win the election if was run as a plurality election?
\item
  Who would win the election if there was a runoff, with the top 2
  candidates advancing to the runoff?
\end{enumerate}

\newpage

\hypertarget{questions-3-to-5}{%
\section{Questions 3 to 5}\label{questions-3-to-5}}

Here is the table of voters

\begin{longtable}[]{@{}ll@{}}
\toprule
Ranking & Voters\tabularnewline
\midrule
\endhead
ABCD & 704\tabularnewline
ABDC & 169\tabularnewline
ACBD & 773\tabularnewline
ACDB & 401\tabularnewline
ADBC & 879\tabularnewline
ADCB & 299\tabularnewline
BACD & 667\tabularnewline
BADC & 121\tabularnewline
BCAD & 769\tabularnewline
BCDA & 725\tabularnewline
BDAC & 346\tabularnewline
BDCA & 456\tabularnewline
CABD & 900\tabularnewline
CADB & 184\tabularnewline
CBAD & 668\tabularnewline
CBDA & 647\tabularnewline
CDAB & 103\tabularnewline
CDBA & 493\tabularnewline
DABC & 240\tabularnewline
DACB & 687\tabularnewline
DBAC & 968\tabularnewline
DBCA & 194\tabularnewline
DCAB & 498\tabularnewline
DCBA & 79\tabularnewline
\bottomrule
\end{longtable}

\begin{enumerate}
\def\labelenumi{\arabic{enumi}.}
\setcounter{enumi}{2}
\tightlist
\item
  Who would win the election if was run as a plurality election?
\item
  Who would win the election if there was a runoff, with the top 2
  candidates advancing to the runoff?
\item
  Imagine that it was still being run as a runoff, but some number of
  A's voters changed their mind and voted for some other candidate. Find
  the smallest number of such voters that are sufficient to make A
  \emph{win} the election.
\end{enumerate}

\newpage

\hypertarget{questions-6-to-10}{%
\section{Questions 6 to 10}\label{questions-6-to-10}}

Here is the table of voters

\begin{longtable}[]{@{}ll@{}}
\toprule
Ranking & Voters\tabularnewline
\midrule
\endhead
ABCD & 10\tabularnewline
ABDC & 61\tabularnewline
ACBD & 46\tabularnewline
ACDB & 25\tabularnewline
ADBC & 54\tabularnewline
ADCB & 62\tabularnewline
BACD & 74\tabularnewline
BADC & 73\tabularnewline
BCAD & 45\tabularnewline
BCDA & 14\tabularnewline
BDAC & 66\tabularnewline
BDCA & 73\tabularnewline
CABD & 34\tabularnewline
CADB & 98\tabularnewline
CBAD & 87\tabularnewline
CBDA & 15\tabularnewline
CDAB & 14\tabularnewline
CDBA & 65\tabularnewline
DABC & 90\tabularnewline
DACB & 23\tabularnewline
DBAC & 67\tabularnewline
DBCA & 1\tabularnewline
DCAB & 22\tabularnewline
DCBA & 36\tabularnewline
\bottomrule
\end{longtable}

\begin{enumerate}
\def\labelenumi{\arabic{enumi}.}
\setcounter{enumi}{5}
\tightlist
\item
  Imagine that the election is being run using the Borda score. The
  candidate with the highest Borda score will win. Who wins the
  election?
\item
  Imagine that candidate A drops out, so there are just three candidates
  remaining, but no voter changes their view about the relative ranking
  of the other candidates. Who wins the election now?
\item
  Imagine that candidate B drops out, so there are just three candidates
  remaining, but no voter changes their view about the relative ranking
  of the other candidates. Who wins the election now?
\item
  Imagine that candidate C drops out, so there are just three candidates
  remaining, but no voter changes their view about the relative ranking
  of the other candidates. Who wins the election now?
\item
  Imagine that candidate D drops out, so there are just three candidates
  remaining, but no voter changes their view about the relative ranking
  of the other candidates. Who wins the election now?
\end{enumerate}

\end{document}
