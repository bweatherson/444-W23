\documentclass[11pt,]{article}
\usepackage{lmodern}
\usepackage{amssymb,amsmath}
\usepackage{ifxetex,ifluatex}
\usepackage{fixltx2e} % provides \textsubscript
\ifnum 0\ifxetex 1\fi\ifluatex 1\fi=0 % if pdftex
  \usepackage[T1]{fontenc}
  \usepackage[utf8]{inputenc}
\else % if luatex or xelatex
  \ifxetex
    \usepackage{mathspec}
  \else
    \usepackage{fontspec}
  \fi
  \defaultfontfeatures{Ligatures=TeX,Scale=MatchLowercase}
    \setmainfont[]{SF Pro Text Light}
\fi
% use upquote if available, for straight quotes in verbatim environments
\IfFileExists{upquote.sty}{\usepackage{upquote}}{}
% use microtype if available
\IfFileExists{microtype.sty}{%
\usepackage{microtype}
\UseMicrotypeSet[protrusion]{basicmath} % disable protrusion for tt fonts
}{}
\usepackage[margin=1in]{geometry}
\usepackage{hyperref}
\hypersetup{unicode=true,
            pdftitle={Assigment 7},
            pdfauthor={Philosophy 444},
            pdfborder={0 0 0},
            breaklinks=true}
\urlstyle{same}  % don't use monospace font for urls
\usepackage{longtable,booktabs}
\usepackage{graphicx,grffile}
\makeatletter
\def\maxwidth{\ifdim\Gin@nat@width>\linewidth\linewidth\else\Gin@nat@width\fi}
\def\maxheight{\ifdim\Gin@nat@height>\textheight\textheight\else\Gin@nat@height\fi}
\makeatother
% Scale images if necessary, so that they will not overflow the page
% margins by default, and it is still possible to overwrite the defaults
% using explicit options in \includegraphics[width, height, ...]{}
\setkeys{Gin}{width=\maxwidth,height=\maxheight,keepaspectratio}
\IfFileExists{parskip.sty}{%
\usepackage{parskip}
}{% else
\setlength{\parindent}{0pt}
\setlength{\parskip}{6pt plus 2pt minus 1pt}
}
\setlength{\emergencystretch}{3em}  % prevent overfull lines
\providecommand{\tightlist}{%
  \setlength{\itemsep}{0pt}\setlength{\parskip}{0pt}}
\setcounter{secnumdepth}{0}
% Redefines (sub)paragraphs to behave more like sections
\ifx\paragraph\undefined\else
\let\oldparagraph\paragraph
\renewcommand{\paragraph}[1]{\oldparagraph{#1}\mbox{}}
\fi
\ifx\subparagraph\undefined\else
\let\oldsubparagraph\subparagraph
\renewcommand{\subparagraph}[1]{\oldsubparagraph{#1}\mbox{}}
\fi

%%% Use protect on footnotes to avoid problems with footnotes in titles
\let\rmarkdownfootnote\footnote%
\def\footnote{\protect\rmarkdownfootnote}

%%% Change title format to be more compact
\usepackage{titling}

% Create subtitle command for use in maketitle
\providecommand{\subtitle}[1]{
  \posttitle{
    \begin{center}\large#1\end{center}
    }
}

\setlength{\droptitle}{-2em}

  \title{Assigment 7}
    \pretitle{\vspace{\droptitle}\centering\huge}
  \posttitle{\par}
    \author{Philosophy 444}
    \preauthor{\centering\large\emph}
  \postauthor{\par}
      \predate{\centering\large\emph}
  \postdate{\par}
    \date{Due November 8, 2019}

\usepackage{gensymb}
\usepackage{nicefrac}
\usepackage{caption}
\usepackage{istgame}
\usepackage{mathastext}

\begin{document}
\maketitle

Answer \textbf{one} of the two following questions, in about 400-500
words. You should answer the question at the end, defend your answer,
and say a bit about the best argument for the opposing answer, and why
you don't think that argument works.

\hypertarget{question-one}{%
\section{Question One}\label{question-one}}

Alice, Betty, Carla, Daria and Elise are deciding where to go for
brunch. They have two choices, Fred's and George's. They have talked
about each of the following propositions, and have settled views about
each of them.

\begin{enumerate}
\def\labelenumi{\arabic{enumi}.}
\tightlist
\item
  Which place has tastier food?
\item
  Which places serve only free-range eggs?
\item
  Should we boycott places that serve eggs that aren't free range?
\item
  Where should we go?
\end{enumerate}

They each agree that they want the tastiest food at a place they aren't
boycotting, but beyond that they don't agree on much. Here are their
answers.

\begin{longtable}[]{@{}ccccc@{}}
\toprule
& Q1 & Q2 & Q3 & Verdict\tabularnewline
\midrule
\endhead
Alice & Fred's & Only George's & Yes & George's\tabularnewline
Betty & Fred's & Only George's & No & Fred's\tabularnewline
Carla & Fred's & Both & Yes & Fred's\tabularnewline
Daria & George's & Only George's & Yes & George's\tabularnewline
Elise & Fred's & Both & No & Fred's\tabularnewline
\bottomrule
\end{longtable}

Where should the friends eat?

(Question Two is on the next page)

\newpage

\hypertarget{question-two}{%
\section{Question Two}\label{question-two}}

Alice is offered a choice of two bets that concern a pair of basketball
games. Neither of them can lose money, so this is just a gift, but how
much they win, and how likely they are to win, varies between the bets.
The Ducks are playing in the first game, and the Emus are playing in the
second game.

\begin{itemize}
\tightlist
\item
  Option A wins \$100 if the Ducks and Emus both win, and nothing
  otherwise.
\item
  Option B wins \$50 if either the Ducks or the Emus lose (i.e., if it
  is not the case that Option A wins), and nothing otherwise.
\end{itemize}

Alice must chose one, and just one, of these options.

Alice knows almost nothing about basketball, so she asks her two friends
Betty and Carla. They both have lots of knowledge about basketball, and
have extremely good track records at forecasting basketball games.

Betty says that the two games are probabilistically independent, and
each team (the Ducks and the Emus) has an 80\% chance to win their game.
So the chance that Option A will win is 64\%, and the chance that Option
B will win is 36\%.

Carla says that the two games are probabilistically independent, and
each team (the Ducks and the Emus) has an 30\% chance to win their game.
So the chance that Option A will win is 9\%, and the chance that Option
B will win is 91\%.

Alice has no other information about the games. Which option should she
take?


\end{document}
