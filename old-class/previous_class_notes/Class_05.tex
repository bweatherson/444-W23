\documentclass[11pt,]{article}
\usepackage{lmodern}
\usepackage{amssymb,amsmath}
\usepackage{ifxetex,ifluatex}
\usepackage{fixltx2e} % provides \textsubscript
\ifnum 0\ifxetex 1\fi\ifluatex 1\fi=0 % if pdftex
  \usepackage[T1]{fontenc}
  \usepackage[utf8]{inputenc}
\else % if luatex or xelatex
  \ifxetex
    \usepackage{mathspec}
  \else
    \usepackage{fontspec}
  \fi
  \defaultfontfeatures{Ligatures=TeX,Scale=MatchLowercase}
    \setmainfont[]{SF Pro Text Light}
\fi
% use upquote if available, for straight quotes in verbatim environments
\IfFileExists{upquote.sty}{\usepackage{upquote}}{}
% use microtype if available
\IfFileExists{microtype.sty}{%
\usepackage{microtype}
\UseMicrotypeSet[protrusion]{basicmath} % disable protrusion for tt fonts
}{}
\usepackage[margin=1in]{geometry}
\usepackage{hyperref}
\hypersetup{unicode=true,
            pdftitle={Backwards Induction},
            pdfauthor={Philosophy 444},
            pdfborder={0 0 0},
            breaklinks=true}
\urlstyle{same}  % don't use monospace font for urls
\usepackage{graphicx,grffile}
\makeatletter
\def\maxwidth{\ifdim\Gin@nat@width>\linewidth\linewidth\else\Gin@nat@width\fi}
\def\maxheight{\ifdim\Gin@nat@height>\textheight\textheight\else\Gin@nat@height\fi}
\makeatother
% Scale images if necessary, so that they will not overflow the page
% margins by default, and it is still possible to overwrite the defaults
% using explicit options in \includegraphics[width, height, ...]{}
\setkeys{Gin}{width=\maxwidth,height=\maxheight,keepaspectratio}
\IfFileExists{parskip.sty}{%
\usepackage{parskip}
}{% else
\setlength{\parindent}{0pt}
\setlength{\parskip}{6pt plus 2pt minus 1pt}
}
\setlength{\emergencystretch}{3em}  % prevent overfull lines
\providecommand{\tightlist}{%
  \setlength{\itemsep}{0pt}\setlength{\parskip}{0pt}}
\setcounter{secnumdepth}{0}
% Redefines (sub)paragraphs to behave more like sections
\ifx\paragraph\undefined\else
\let\oldparagraph\paragraph
\renewcommand{\paragraph}[1]{\oldparagraph{#1}\mbox{}}
\fi
\ifx\subparagraph\undefined\else
\let\oldsubparagraph\subparagraph
\renewcommand{\subparagraph}[1]{\oldsubparagraph{#1}\mbox{}}
\fi

%%% Use protect on footnotes to avoid problems with footnotes in titles
\let\rmarkdownfootnote\footnote%
\def\footnote{\protect\rmarkdownfootnote}

%%% Change title format to be more compact
\usepackage{titling}

% Create subtitle command for use in maketitle
\providecommand{\subtitle}[1]{
  \posttitle{
    \begin{center}\large#1\end{center}
    }
}

\setlength{\droptitle}{-2em}

  \title{Backwards Induction}
    \pretitle{\vspace{\droptitle}\centering\huge}
  \posttitle{\par}
    \author{Philosophy 444}
    \preauthor{\centering\large\emph}
  \postauthor{\par}
      \predate{\centering\large\emph}
  \postdate{\par}
    \date{18 September, 2019}

\usepackage{gensymb}
\usepackage{nicefrac}
\usepackage{caption}
\usepackage{istgame}

\begin{document}
\maketitle

I want you to divide into groups of 2-3, and work through these
problems. After a few minutes, we'll talk through the first, then
regroup, then talk about the second, and so on. But if you're done with
the first, move onto the subsequent problems; we'll get time to talk
about them.

All the problems are from William Spaniel's \emph{Game Theory 101}. This
is a not bad introduction to game theory, less technical than the book
we're using, and only a few dollars on Kindle. (And he has a bunch of
videos to go with it.)

Note that in none of the texts describing the games am I specifying
precisely what the payoffs to each player are. I'll usually say enough
to specify that, but in cases of ambiguity, it's worth thinking about,
and discussing, what the payoffs will be.

\hypertarget{burning-bridges}{%
\subsection{Burning Bridges}\label{burning-bridges}}

A small island sits between two countries Each country has only one
bridge that can access it. Although valuable, the island is not worth
fighting over; each side would rather concede the territory to its
opponent than fight a battle over the territory.

\begin{itemize}
\tightlist
\item
  The first country crosses its bridge to occupy the island. Afterward,
  the soldiers decide whether to burn the bridge behind them.
\item
  The second country decides whether to invade.
\item
  If the first country has no bridge to use as an escape route, it must
  fight a battle.
\item
  However, if the bridge still stands, the first country decides whether
  to fight or retreat.
\end{itemize}

Draw the game tree, and find the backwards induction solution to the
game.

\hypertarget{tying-hands}{%
\subsection{Tying Hands}\label{tying-hands}}

A boss notices that one of his unscrupulous employees has been stealing
company materials lately. He values honesty in himself and his
employees, but the stolen property was not valuable. Consequently, the
boss prefers keeping her around rather than having to hire and train a
replacement. Nevertheless, he would ideally like stop her from stealing.

At the company meeting today, he is thinking about issuing a warning:
the next person caught stealing any company property will be immediately
fired.

Should he issue such a warning?

\newpage

\hypertarget{commitment-problem}{%
\subsection{Commitment Problem}\label{commitment-problem}}

Suppose you are a college graduate from San Diego, California, and you
were recently admitted to a PhD program in political science in
Rochester, New York. Naturally, you pack up all of your earthly
belongings into your compact Honda Civic, cover everything with an old
sheet to shield the items from the prying eyes of a potential thief, and
embark on a cross-country adventure to your new home.

But trouble strikes halfway there. As you are driving through a small
town police lights flash behind you. You pull over and roll down your
window.

The officer explains that the town is in the middle of a drug war and
that you appear suspicious, coming from California in a vehicle filled
with unknown objects under a sheet. He politely requests to search your
vehicle.

You tell him you are a graduate student moving from San Diego to
Rochester and object to such a search, noting that he has no legal right
to look through your belongings. Begrudgingly, the officer accepts that
he cannot search your vehicle without permission. However, he notes that
he could call in a K-9 unit to sniff around the vehicle. But the K-9
unit is stationed a half hour away, so it would take a while for it to
arrive.

He suggests a compromise: you allow him to conduct a quick search, and
you can be on your way in a few minutes. He stresses that the quick
search will be better than the K-9 for both parties, as neither of you
will have to wait in the hot summer sun. Should you take the officer's
offer?

Having studied game theory, you mentally draw out the game tree. Your
move is first: you can either demand the K-9 unit or allow the officer
to search. If you allow the search, the officer decides between
conducting a quick search as he originally offered or reneging on that
agreement and conducting an extensive search. You most prefer a quick
search and least prefer an invasive extensive search. Meanwhile, the
police officer would most like to conduct an extensive search to ensure
you are not carrying drugs but least prefers waiting a long time for the
K-9 to arrive.

Find the backwards induction solution to this game, and discuss what
small changes to the game would result in a different solution.

\hypertarget{pirates-hard}{%
\subsection{Pirates! (Hard)}\label{pirates-hard}}

The Dread Pirate Nash captures 10 pieces of gold from the Selten, the
Saltwater Scoundrel. He must decide how to divide the coins among the
four other members of the crew.

According to pirate tradition, the captain of the crew proposes a
division of the coins to his crew. If at least half of the crew (captain
included) accepts the offer, coins are divided according to the
proposal, and bargaining ends. If a majority rejects the proposal,
however, the captain must walk the plank. Afterward, the second in
command takes over as captain and proposes a new division with the same
rules as before. Bargaining continues until all of the pirates are dead
or at least half accept an offer. The Dread Pirate Nash, Pirate 2,
Pirate 3, Pirate 4, and Pirate 5 primarily want to survive. Given their
survival, they then want to maximize their share of the gold coins. And
given a certain allotment of coins, they prefer having that number and
having a higher rank in the chain of command than having that same
number and a lower rank.

Assume that voting is non-strategic; that is, the pirates always vote
according to their preferences. (Although having strategic voting would
not change the game's outcome, it does make an already complicated game
even more convoluted.)

Find the backwards induction solution.


\end{document}
