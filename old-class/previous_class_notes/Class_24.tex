\documentclass[11pt,]{article}
\usepackage{lmodern}
\usepackage{amssymb,amsmath}
\usepackage{ifxetex,ifluatex}
\usepackage{fixltx2e} % provides \textsubscript
\ifnum 0\ifxetex 1\fi\ifluatex 1\fi=0 % if pdftex
  \usepackage[T1]{fontenc}
  \usepackage[utf8]{inputenc}
\else % if luatex or xelatex
  \ifxetex
    \usepackage{mathspec}
  \else
    \usepackage{fontspec}
  \fi
  \defaultfontfeatures{Ligatures=TeX,Scale=MatchLowercase}
    \setmainfont[]{SF Pro Text Light}
\fi
% use upquote if available, for straight quotes in verbatim environments
\IfFileExists{upquote.sty}{\usepackage{upquote}}{}
% use microtype if available
\IfFileExists{microtype.sty}{%
\usepackage{microtype}
\UseMicrotypeSet[protrusion]{basicmath} % disable protrusion for tt fonts
}{}
\usepackage[margin=1in]{geometry}
\usepackage{hyperref}
\hypersetup{unicode=true,
            pdftitle={Group Knowledge},
            pdfauthor={Philosophy 444},
            pdfborder={0 0 0},
            breaklinks=true}
\urlstyle{same}  % don't use monospace font for urls
\usepackage{graphicx,grffile}
\makeatletter
\def\maxwidth{\ifdim\Gin@nat@width>\linewidth\linewidth\else\Gin@nat@width\fi}
\def\maxheight{\ifdim\Gin@nat@height>\textheight\textheight\else\Gin@nat@height\fi}
\makeatother
% Scale images if necessary, so that they will not overflow the page
% margins by default, and it is still possible to overwrite the defaults
% using explicit options in \includegraphics[width, height, ...]{}
\setkeys{Gin}{width=\maxwidth,height=\maxheight,keepaspectratio}
\IfFileExists{parskip.sty}{%
\usepackage{parskip}
}{% else
\setlength{\parindent}{0pt}
\setlength{\parskip}{6pt plus 2pt minus 1pt}
}
\setlength{\emergencystretch}{3em}  % prevent overfull lines
\providecommand{\tightlist}{%
  \setlength{\itemsep}{0pt}\setlength{\parskip}{0pt}}
\setcounter{secnumdepth}{0}
% Redefines (sub)paragraphs to behave more like sections
\ifx\paragraph\undefined\else
\let\oldparagraph\paragraph
\renewcommand{\paragraph}[1]{\oldparagraph{#1}\mbox{}}
\fi
\ifx\subparagraph\undefined\else
\let\oldsubparagraph\subparagraph
\renewcommand{\subparagraph}[1]{\oldsubparagraph{#1}\mbox{}}
\fi

%%% Use protect on footnotes to avoid problems with footnotes in titles
\let\rmarkdownfootnote\footnote%
\def\footnote{\protect\rmarkdownfootnote}

%%% Change title format to be more compact
\usepackage{titling}

% Create subtitle command for use in maketitle
\providecommand{\subtitle}[1]{
  \posttitle{
    \begin{center}\large#1\end{center}
    }
}

\setlength{\droptitle}{-2em}

  \title{Group Knowledge}
    \pretitle{\vspace{\droptitle}\centering\huge}
  \posttitle{\par}
    \author{Philosophy 444}
    \preauthor{\centering\large\emph}
  \postauthor{\par}
      \predate{\centering\large\emph}
  \postdate{\par}
    \date{2 December, 2019}

\usepackage{mathastext}
\usepackage{nicefrac}

\begin{document}
\maketitle

In each of the following examples, I want you to think about whether the
group in question has knowledge. One answer might be, ``It depends on
details you haven't filled in'', but maybe sometimes the answer is
clearly yes or no.

\hypertarget{example-one}{%
\section{Example One}\label{example-one}}

The Ruritanian President is a French spy. The CIA agent tasked with
understanding Ruritania figured this out, and told her superiors. They
incorporated this into their plans, but kept the information
super-secret. So most CIA agents, those who don't have to deal with
Ruritania, have the orthodox view that the Ruritanian President is in
fact a German spy.

Does the CIA know that the Ruritanian President is a French spy?

\hypertarget{example-two}{%
\section{Example Two}\label{example-two}}

Fred and George, the twins, are going on a train trip. Fred has looked
at a timetable, and through that formed the belief that the next train
leaves at 12.07. George doesn't know when the trains are, but knows that
Fred is looking at an old timetable. He also knows that the new
timetable only changes the times of evening trains - trains during the
day have not changed.

\begin{itemize}
\tightlist
\item
  Does Fred know when the next train leaves?
\item
  Does George know when the next train leaves?
\item
  Do the twins know when the next train leaves?
\end{itemize}

\hypertarget{example-three}{%
\section{Example Three}\label{example-three}}

Fred and George, the twins, got worried that there is a ghost in their
house. Their house just has an upstairs and a downstairs, so they
divided it up and started looking. After a search, Fred knows that there
is no ghost upstairs. George knows that there is no ghost downstairs.
They haven't yet had a chance to communicate with each other.

\begin{itemize}
\tightlist
\item
  Does Fred know there is no ghost in the house?
\item
  Does George know there is no ghost in the house?
\item
  Do the twins know there is no ghost in the house.
\end{itemize}

\newpage

\hypertarget{example-four}{%
\section{Example Four}\label{example-four}}

Building a car is hard, and requires a lot of knowledge. The Ford Motor
Company has people who specialise in each of the steps required, and
managers who know how to get these specialists to work together. But
none of the specialists could do the jobs of the others, and none of the
managers could do any of the specialist jobs. Still, the company manages
to output a lot of cars.

\begin{itemize}
\tightlist
\item
  Does the Ford Motor Company know how to build cars?
\item
  Does any person in the Ford Motor Company know how to build cars?
\end{itemize}

\hypertarget{example-five}{%
\section{Example Five}\label{example-five}}

The historical records say that in 1896, the mayor of Ann Arbor was
Warren E. Walker. He's a largely forgotten figure - he doesn't even have
a wikipedia page. In the world of the story (not this world!) he's even
more forgotten than this. There is not a person alive who could tell
you, without looking it up, who was the mayor of Ann Arbor in 1896. But
there are many people who could look it up - it's right there on
Wikipedia for one.

\begin{itemize}
\tightlist
\item
  Which groups, if any, know that in 1896, the mayor of Ann Arbor was
  Warren E. Walker?
\end{itemize}

\hypertarget{lackeys-argument}{%
\section{Lackey's Argument}\label{lackeys-argument}}

I'll talk through this much more in class (perhaps much much more), but
it's useful to have a very brief sketch of the argument on paper.

\begin{enumerate}
\def\labelenumi{\arabic{enumi}.}
\tightlist
\item
  If someone, or some group, knows that \(p\), then that group can
  rationally take it for granted that \(p\) in choosing an action.
\item
  In at least example five, and maybe others, the groups cannot take the
  allegedly known things for granted in rationally choosing an action.
\item
  So in at least example five, and maybe others, the groups do not have
  knowledge.
\end{enumerate}


\end{document}
