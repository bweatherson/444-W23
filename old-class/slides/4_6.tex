% Options for packages loaded elsewhere
\PassOptionsToPackage{unicode}{hyperref}
\PassOptionsToPackage{hyphens}{url}
%
\documentclass[
  ignorenonframetext,
]{beamer}
\usepackage{pgfpages}
\setbeamertemplate{caption}[numbered]
\setbeamertemplate{caption label separator}{: }
\setbeamercolor{caption name}{fg=normal text.fg}
\beamertemplatenavigationsymbolsempty
% Prevent slide breaks in the middle of a paragraph
\widowpenalties 1 10000
\raggedbottom
\setbeamertemplate{part page}{
  \centering
  \begin{beamercolorbox}[sep=16pt,center]{part title}
    \usebeamerfont{part title}\insertpart\par
  \end{beamercolorbox}
}
\setbeamertemplate{section page}{
  \centering
  \begin{beamercolorbox}[sep=12pt,center]{part title}
    \usebeamerfont{section title}\insertsection\par
  \end{beamercolorbox}
}
\setbeamertemplate{subsection page}{
  \centering
  \begin{beamercolorbox}[sep=8pt,center]{part title}
    \usebeamerfont{subsection title}\insertsubsection\par
  \end{beamercolorbox}
}
\AtBeginPart{
  \frame{\partpage}
}
\AtBeginSection{
  \ifbibliography
  \else
    \frame{\sectionpage}
  \fi
}
\AtBeginSubsection{
  \frame{\subsectionpage}
}
\usepackage{amsmath,amssymb}
\usepackage{lmodern}
\usepackage{ifxetex,ifluatex}
\ifnum 0\ifxetex 1\fi\ifluatex 1\fi=0 % if pdftex
  \usepackage[T1]{fontenc}
  \usepackage[utf8]{inputenc}
  \usepackage{textcomp} % provide euro and other symbols
\else % if luatex or xetex
  \usepackage{unicode-math}
  \defaultfontfeatures{Scale=MatchLowercase}
  \defaultfontfeatures[\rmfamily]{Ligatures=TeX,Scale=1}
  \setmainfont[BoldFont = SF Pro Rounded Semibold]{SF Pro Rounded}
  \setmathfont[]{STIX Two Math}
\fi
\usefonttheme{serif} % use mainfont rather than sansfont for slide text
% Use upquote if available, for straight quotes in verbatim environments
\IfFileExists{upquote.sty}{\usepackage{upquote}}{}
\IfFileExists{microtype.sty}{% use microtype if available
  \usepackage[]{microtype}
  \UseMicrotypeSet[protrusion]{basicmath} % disable protrusion for tt fonts
}{}
\makeatletter
\@ifundefined{KOMAClassName}{% if non-KOMA class
  \IfFileExists{parskip.sty}{%
    \usepackage{parskip}
  }{% else
    \setlength{\parindent}{0pt}
    \setlength{\parskip}{6pt plus 2pt minus 1pt}}
}{% if KOMA class
  \KOMAoptions{parskip=half}}
\makeatother
\usepackage{xcolor}
\IfFileExists{xurl.sty}{\usepackage{xurl}}{} % add URL line breaks if available
\IfFileExists{bookmark.sty}{\usepackage{bookmark}}{\usepackage{hyperref}}
\hypersetup{
  pdftitle={444 Lecture 4.6 - Utility},
  pdfauthor={Brian Weatherson},
  hidelinks,
  pdfcreator={LaTeX via pandoc}}
\urlstyle{same} % disable monospaced font for URLs
\newif\ifbibliography
\setlength{\emergencystretch}{3em} % prevent overfull lines
\providecommand{\tightlist}{%
  \setlength{\itemsep}{0pt}\setlength{\parskip}{0pt}}
\setcounter{secnumdepth}{-\maxdimen} % remove section numbering
\let\Tiny=\tiny

 \setbeamertemplate{navigation symbols}{} 

% \usetheme{Madrid}
 \usetheme[numbering=none, progressbar=foot]{metropolis}
 \usecolortheme{wolverine}
 \usepackage{color}
 \usepackage{MnSymbol}
% \usepackage{movie15}

\usepackage{amssymb}% http://ctan.org/pkg/amssymb
\usepackage{pifont}% http://ctan.org/pkg/pifont
\newcommand{\cmark}{\ding{51}}%
\newcommand{\xmark}{\ding{55}}%

\DeclareSymbolFont{symbolsC}{U}{txsyc}{m}{n}
\DeclareMathSymbol{\boxright}{\mathrel}{symbolsC}{128}
\DeclareMathAlphabet{\mathpzc}{OT1}{pzc}{m}{it}

\setlength{\parskip}{1ex plus 0.5ex minus 0.2ex}

\AtBeginSection[]
{
\begin{frame}
	\Huge{\color{darkblue} \insertsection}
\end{frame}
}

\renewenvironment*{quote}	
	{\list{}{\rightmargin   \leftmargin} \item } 	
	{\endlist }

\definecolor{darkgreen}{rgb}{0,0.7,0}
\definecolor{darkblue}{rgb}{0,0,0.8}

\usepackage[italic]{mathastext}
\usepackage{nicefrac}

\setbeamertemplate{caption}{\raggedright\insertcaption}

%\def\toprule{}
%\def\bottomrule{}
%\def\midrule{}
\usepackage{etoolbox}
\AfterEndEnvironment{description}{\vspace{9pt}}
\AfterEndEnvironment{oltableau}{\vspace{9pt}}
\BeforeBeginEnvironment{oltableau}{\vspace{9pt}}
\AfterEndEnvironment{center}{\vspace{9pt}}
\BeforeBeginEnvironment{tabular}{\vspace{9pt}}
\AfterEndEnvironment{longtable}{\vspace{-6pt}}
\usepackage{booktabs}
\usepackage{longtable}
\usepackage{array}
\usepackage{multirow}
\usepackage{wrapfig}
\usepackage{float}
\usepackage{colortbl}
\usepackage{pdflscape}
\usepackage{tabu}
\usepackage{threeparttable} 
\usepackage{threeparttablex} 
\usepackage[normalem]{ulem} 
\usepackage{makecell}
\usepackage{xcolor}
\usepackage{ulem}

\setlength\heavyrulewidth{0ex}
\setlength\lightrulewidth{0.08ex}

\aboverulesep=0ex
\belowrulesep=0ex
\renewcommand{\arraystretch}{1.2}
\ifluatex
  \usepackage{selnolig}  % disable illegal ligatures
\fi

\title{444 Lecture 4.6 - Utility}
\author{Brian Weatherson}
\date{}

\begin{document}
\frame{\titlepage}

\begin{frame}{Plan}
\protect\hypertarget{plan}{}
To introduce the notion of utility.
\end{frame}

\begin{frame}{Reading}
\protect\hypertarget{reading}{}
Bonanno, chapter 5 - though note we aren't following the book precisely
in this chapter.
\end{frame}

\begin{frame}{Ranking}
\protect\hypertarget{ranking}{}
\begin{itemize}
\tightlist
\item
  So far the theories we've looked just solve games using the
  \textbf{rankings} of various options.
\item
  It doesn't look at how much a player prefers one option over another,
  just on what is preferred to what.
\end{itemize}
\end{frame}

\begin{frame}{Ordinal Utility}
\protect\hypertarget{ordinal-utility}{}
\begin{itemize}
\tightlist
\item
  To use the technical language, so far our theories have just used
  \textbf{ordinal utilities}.
\item
  The term \textbf{ordinal} here means that we only look at the
  \textbf{order} of the options.
\end{itemize}
\end{frame}

\begin{frame}{Cardinal Utility}
\protect\hypertarget{cardinal-utility}{}
\begin{itemize}
\tightlist
\item
  The rules that we'll look at from now on use \textbf{cardinal
  utilities}.
\item
  Whenever we're associating outcomes with numbers in a way that the
  magnitudes of the differences between the numbers matters, we're using
  cardinal utilities.
\end{itemize}
\end{frame}

\begin{frame}{Utility}
\protect\hypertarget{utility}{}
\begin{itemize}
\tightlist
\item
  Intuitively, think of utilities as measuring how good an outcome is.
\item
  The theory we're building towards is thoroughly subjectivist, so think
  of `how good' as meaning `how good along all and only dimensions the
  agent making the decision cares about'.
\end{itemize}
\end{frame}

\begin{frame}{Scale}
\protect\hypertarget{scale}{}
\begin{itemize}
\tightlist
\item
  Utilities aren't really measured on any scale.
\item
  Indeed, like temperature measures, or year numberings, they don't even
  have a fixed zero point.
\item
  It is usually convenient to associate 0 utility with the status quo,
  and then have negative numbers for outcomes worse than status quo, and
  positive numbers for outcomes better than status quo.
\item
  But that's just a convention; you can set the 0 wherever you like.
\item
  And you can set the utility 1 point at anything better than 0.
\end{itemize}
\end{frame}

\begin{frame}{Scale (continued)}
\protect\hypertarget{scale-continued}{}
\begin{itemize}
\tightlist
\item
  But that's where the convention stops.
\item
  Once you fix the 0 and 1 points, nothing else is fixed by pure
  convention.
\item
  Temperatures are like this too.
\item
  Though years are not - all the different calendars have years of the
  same length.
\end{itemize}
\end{frame}

\begin{frame}{Meaning of the Scale}
\protect\hypertarget{meaning-of-the-scale}{}
We will come back to this much more in the future, but here is the key
equation.

\[
U(B) = \frac{U(A) + U(C)}{2}
\]

Means that the agent is indifferent between getting \(B\) for sure, and
a coin flip that means they get \(A\) if Heads and \(C\) if Tails.
\end{frame}

\begin{frame}{Meaning of the Scale}
\protect\hypertarget{meaning-of-the-scale-1}{}
It's a little unintuitive to think about this (though it helps if you've
moved between Celsius and Farenheit countries).

\begin{itemize}
\tightlist
\item
  What matters is the ratio of differences.
\item
  If \(U(A) - U(B) = U(B) - U(C)\), that's really meaningful, even if
  none of the individual numbers are meaningful.
\end{itemize}
\end{frame}

\begin{frame}{For Next Time}
\protect\hypertarget{for-next-time}{}
\begin{itemize}
\tightlist
\item
  We will look at how utility relates to money.
\end{itemize}
\end{frame}

\end{document}
