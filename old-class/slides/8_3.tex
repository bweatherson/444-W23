% Options for packages loaded elsewhere
\PassOptionsToPackage{unicode}{hyperref}
\PassOptionsToPackage{hyphens}{url}
%
\documentclass[
  ignorenonframetext,
]{beamer}
\usepackage{pgfpages}
\setbeamertemplate{caption}[numbered]
\setbeamertemplate{caption label separator}{: }
\setbeamercolor{caption name}{fg=normal text.fg}
\beamertemplatenavigationsymbolsempty
% Prevent slide breaks in the middle of a paragraph
\widowpenalties 1 10000
\raggedbottom
\setbeamertemplate{part page}{
  \centering
  \begin{beamercolorbox}[sep=16pt,center]{part title}
    \usebeamerfont{part title}\insertpart\par
  \end{beamercolorbox}
}
\setbeamertemplate{section page}{
  \centering
  \begin{beamercolorbox}[sep=12pt,center]{part title}
    \usebeamerfont{section title}\insertsection\par
  \end{beamercolorbox}
}
\setbeamertemplate{subsection page}{
  \centering
  \begin{beamercolorbox}[sep=8pt,center]{part title}
    \usebeamerfont{subsection title}\insertsubsection\par
  \end{beamercolorbox}
}
\AtBeginPart{
  \frame{\partpage}
}
\AtBeginSection{
  \ifbibliography
  \else
    \frame{\sectionpage}
  \fi
}
\AtBeginSubsection{
  \frame{\subsectionpage}
}
\usepackage{amsmath,amssymb}
\usepackage{lmodern}
\usepackage{ifxetex,ifluatex}
\ifnum 0\ifxetex 1\fi\ifluatex 1\fi=0 % if pdftex
  \usepackage[T1]{fontenc}
  \usepackage[utf8]{inputenc}
  \usepackage{textcomp} % provide euro and other symbols
\else % if luatex or xetex
  \usepackage{unicode-math}
  \defaultfontfeatures{Scale=MatchLowercase}
  \defaultfontfeatures[\rmfamily]{Ligatures=TeX,Scale=1}
  \setmainfont[BoldFont = SF Pro Rounded Semibold]{SF Pro Rounded}
  \setmathfont[]{STIX Two Math}
\fi
\usefonttheme{serif} % use mainfont rather than sansfont for slide text
% Use upquote if available, for straight quotes in verbatim environments
\IfFileExists{upquote.sty}{\usepackage{upquote}}{}
\IfFileExists{microtype.sty}{% use microtype if available
  \usepackage[]{microtype}
  \UseMicrotypeSet[protrusion]{basicmath} % disable protrusion for tt fonts
}{}
\makeatletter
\@ifundefined{KOMAClassName}{% if non-KOMA class
  \IfFileExists{parskip.sty}{%
    \usepackage{parskip}
  }{% else
    \setlength{\parindent}{0pt}
    \setlength{\parskip}{6pt plus 2pt minus 1pt}}
}{% if KOMA class
  \KOMAoptions{parskip=half}}
\makeatother
\usepackage{xcolor}
\IfFileExists{xurl.sty}{\usepackage{xurl}}{} % add URL line breaks if available
\IfFileExists{bookmark.sty}{\usepackage{bookmark}}{\usepackage{hyperref}}
\hypersetup{
  pdftitle={444 Lecture 8.3 - Stag Hunt},
  pdfauthor={Brian Weatherson},
  hidelinks,
  pdfcreator={LaTeX via pandoc}}
\urlstyle{same} % disable monospaced font for URLs
\newif\ifbibliography
\setlength{\emergencystretch}{3em} % prevent overfull lines
\providecommand{\tightlist}{%
  \setlength{\itemsep}{0pt}\setlength{\parskip}{0pt}}
\setcounter{secnumdepth}{-\maxdimen} % remove section numbering
\let\Tiny=\tiny

 \setbeamertemplate{navigation symbols}{} 

% \usetheme{Madrid}
 \usetheme[numbering=none, progressbar=foot]{metropolis}
 \usecolortheme{wolverine}
 \usepackage{color}
 \usepackage{MnSymbol}
% \usepackage{movie15}

\usepackage{amssymb}% http://ctan.org/pkg/amssymb
\usepackage{pifont}% http://ctan.org/pkg/pifont
\newcommand{\cmark}{\ding{51}}%
\newcommand{\xmark}{\ding{55}}%

\DeclareSymbolFont{symbolsC}{U}{txsyc}{m}{n}
\DeclareMathSymbol{\boxright}{\mathrel}{symbolsC}{128}
\DeclareMathAlphabet{\mathpzc}{OT1}{pzc}{m}{it}

\setlength{\parskip}{1ex plus 0.5ex minus 0.2ex}

\AtBeginSection[]
{
\begin{frame}
	\Huge{\color{darkblue} \insertsection}
\end{frame}
}

\renewenvironment*{quote}	
	{\list{}{\rightmargin   \leftmargin} \item } 	
	{\endlist }

\definecolor{darkgreen}{rgb}{0,0.7,0}
\definecolor{darkblue}{rgb}{0,0,0.8}

\usepackage[italic]{mathastext}
\usepackage{nicefrac}
\usepackage{istgame}

\setbeamertemplate{caption}{\raggedright\insertcaption}

%\def\toprule{}
%\def\bottomrule{}
%\def\midrule{}
\usepackage{etoolbox}
\AfterEndEnvironment{description}{\vspace{9pt}}
\AfterEndEnvironment{oltableau}{\vspace{9pt}}
\BeforeBeginEnvironment{oltableau}{\vspace{9pt}}
\AfterEndEnvironment{center}{\vspace{9pt}}
\BeforeBeginEnvironment{tabular}{\vspace{9pt}}
\AfterEndEnvironment{longtable}{\vspace{-6pt}}
\usepackage{booktabs}
\usepackage{longtable}
\usepackage{array}
\usepackage{multirow}
\usepackage{wrapfig}
\usepackage{float}
\usepackage{colortbl}
\usepackage{pdflscape}
\usepackage{tabu}
\usepackage{threeparttable} 
\usepackage{threeparttablex} 
\usepackage[normalem]{ulem} 
\usepackage{makecell}
\usepackage{xcolor}
\usepackage{ulem}

\setlength\heavyrulewidth{0ex}
\setlength\lightrulewidth{0.08ex}

\aboverulesep=0ex
\belowrulesep=0ex
\renewcommand{\arraystretch}{1.2}
\ifluatex
  \usepackage{selnolig}  % disable illegal ligatures
\fi

\title{444 Lecture 8.3 - Stag Hunt}
\author{Brian Weatherson}
\date{}

\begin{document}
\frame{\titlepage}

\begin{frame}{Stag Hunt}
\protect\hypertarget{stag-hunt}{}
\begin{table}[!h]
\centering
\begin{tabular}[t]{>{}r|cc}
\toprule
 & gather & hunt\\
\midrule
Gather & x, x & y, z\\
Hunt & z, y & w, w\\
\bottomrule
\end{tabular}
\end{table}

With the following constraints:

\begin{itemize}
\tightlist
\item
  \(x > z\)
\item
  \(w > y\)
\item
  \(w > x\)
\item
  \(x + y > z + w\)
\end{itemize}
\end{frame}

\begin{frame}{Concrete Example of Stag Hunt}
\protect\hypertarget{concrete-example-of-stag-hunt}{}
\begin{table}[!h]
\centering
\begin{tabular}[t]{>{}r|cc}
\toprule
 & gather & hunt\\
\midrule
Gather & 2, 2 & 4, 0\\
Hunt & 0, 4 & 5, 5\\
\bottomrule
\end{tabular}
\end{table}
\end{frame}

\begin{frame}{Differences with Prisoners' Dilemma}
\protect\hypertarget{differences-with-prisoners-dilemma}{}
\begin{itemize}
\tightlist
\item
  Again, there is a cooperative move (in this case Hunt), which is
  socially better than the individualistic move (Gather).
\item
  But in this case, cooperation is an equilibrium; it isn't dominated.
\item
  The problem is that there are nevertheless reasons to do the
  individualistic thing.
\end{itemize}
\end{frame}

\begin{frame}{Regret Based Reasons}
\protect\hypertarget{regret-based-reasons}{}
\begin{itemize}
\tightlist
\item
  Whatever you do in Stag Hunt, you're hoping/guessing that the other
  player does the same thing.
\item
  If you guess wrong, you'll regret your choice. \pause
\item
  If you Gather when the other player Hunts, you'll get 4 and you could
  have got 5 - a regret of 1. \pause
\item
  If you Hunt when the other player Gathers, you'll get 0 and you could
  have got 2 - a regret of 2. \pause
\item
  Mistakenly Hunting leads to higher regret than mistakenly Gathering.
\item
  Minimising regret, which a lot of people think is important in
  decisions under radical uncertainty, implies Gathering.
\end{itemize}
\end{frame}

\begin{frame}{Random Choice}
\protect\hypertarget{random-choice}{}
\begin{itemize}
\tightlist
\item
  There are two equilibria.
\item
  Maybe it's reasonable, as a first pass, to have equal probabilities in
  each hypothesis about what the other player will pick.
\item
  So in this case, you'd be (as a first pass), 50/50 about whether the
  other person will Gather or Hunt. \pause
\item
  But then it maximises expected utility to Gather.
\item
  That has expected utility 3, while Hunting has expected utility 2.5.
\end{itemize}
\end{frame}

\begin{frame}{Evolutionary Explanations}
\protect\hypertarget{evolutionary-explanations}{}
\begin{itemize}
\tightlist
\item
  Imagine an Axelrod type evolutionary situation, that starts out with
  equal numbers of Gatherers and Hunters.
\item
  Each person interacts with everyone else in the community, and they
  add up their score.
\item
  Then in the next generation, the number of Gatherers and Hunters is
  proportionate to the score that Gatherers and Hunters get in this
  generation. \pause
\item
  Well, in fairly short order, you have a population of more or less all
  Gatherers.
\item
  Indeed, that happens unless you start with at least 2/3 Hunters.
\end{itemize}
\end{frame}

\begin{frame}{Social Challenge}
\protect\hypertarget{social-challenge}{}
\begin{itemize}
\tightlist
\item
  How do we get people to be cooperative, i.e., Hunt?
\item
  Note that we don't have to imagine changing the payouts, i.e.,
  punishing, or taking away options.
\item
  It suffices to get everyone to (truly) believe that others will Hunt.
\item
  This isn't trivial, but it's a very different kind of challenge than
  in PD.
\end{itemize}
\end{frame}

\begin{frame}{Modeling Challenge}
\protect\hypertarget{modeling-challenge}{}
\begin{itemize}
\tightlist
\item
  Which cases are really Stag Hunt not PD?
\item
  I'm going to talk about this a bit more, but it's really worth
  thinking through real life cases.
\item
  Is there a genuine equilibrium where merely by everyone believing that
  everyone else will Hunt, it becomes in their own interest to Hunt?
  \pause
\item
  Note that it is really great if this is so.
\item
  The view from Hobbes on was that getting out of PD required heavy
  handed intervention.
\item
  But getting to the cooperative equilibrium in Stag Hunt might just
  require nudging.
\end{itemize}
\end{frame}

\begin{frame}{Mixing the Issues}
\protect\hypertarget{mixing-the-issues}{}
\begin{itemize}
\tightlist
\item
  The modeling challenge and the social challenge can run together.
\item
  If we want to change behavior, it helps to know what kind of game
  people are, or take themselves, to be playing.
\item
  So the theoretical question of how to conceptualise a practice might
  be related to the social question of how to repair it.
\end{itemize}
\end{frame}

\begin{frame}{For Next Time}
\protect\hypertarget{for-next-time}{}
I'll go over a bit what PD and Stag Hunt look like in \(n\) player
cases.
\end{frame}

\end{document}
