% Options for packages loaded elsewhere
\PassOptionsToPackage{unicode}{hyperref}
\PassOptionsToPackage{hyphens}{url}
%
\documentclass[
  ignorenonframetext,
]{beamer}
\usepackage{pgfpages}
\setbeamertemplate{caption}[numbered]
\setbeamertemplate{caption label separator}{: }
\setbeamercolor{caption name}{fg=normal text.fg}
\beamertemplatenavigationsymbolsempty
% Prevent slide breaks in the middle of a paragraph
\widowpenalties 1 10000
\raggedbottom
\setbeamertemplate{part page}{
  \centering
  \begin{beamercolorbox}[sep=16pt,center]{part title}
    \usebeamerfont{part title}\insertpart\par
  \end{beamercolorbox}
}
\setbeamertemplate{section page}{
  \centering
  \begin{beamercolorbox}[sep=12pt,center]{part title}
    \usebeamerfont{section title}\insertsection\par
  \end{beamercolorbox}
}
\setbeamertemplate{subsection page}{
  \centering
  \begin{beamercolorbox}[sep=8pt,center]{part title}
    \usebeamerfont{subsection title}\insertsubsection\par
  \end{beamercolorbox}
}
\AtBeginPart{
  \frame{\partpage}
}
\AtBeginSection{
  \ifbibliography
  \else
    \frame{\sectionpage}
  \fi
}
\AtBeginSubsection{
  \frame{\subsectionpage}
}
\usepackage{amsmath,amssymb}
\usepackage{lmodern}
\usepackage{ifxetex,ifluatex}
\ifnum 0\ifxetex 1\fi\ifluatex 1\fi=0 % if pdftex
  \usepackage[T1]{fontenc}
  \usepackage[utf8]{inputenc}
  \usepackage{textcomp} % provide euro and other symbols
\else % if luatex or xetex
  \usepackage{unicode-math}
  \defaultfontfeatures{Scale=MatchLowercase}
  \defaultfontfeatures[\rmfamily]{Ligatures=TeX,Scale=1}
  \setmainfont[BoldFont = SF Pro Rounded Semibold]{SF Pro Rounded}
  \setmathfont[]{STIX Two Math}
\fi
\usefonttheme{serif} % use mainfont rather than sansfont for slide text
% Use upquote if available, for straight quotes in verbatim environments
\IfFileExists{upquote.sty}{\usepackage{upquote}}{}
\IfFileExists{microtype.sty}{% use microtype if available
  \usepackage[]{microtype}
  \UseMicrotypeSet[protrusion]{basicmath} % disable protrusion for tt fonts
}{}
\makeatletter
\@ifundefined{KOMAClassName}{% if non-KOMA class
  \IfFileExists{parskip.sty}{%
    \usepackage{parskip}
  }{% else
    \setlength{\parindent}{0pt}
    \setlength{\parskip}{6pt plus 2pt minus 1pt}}
}{% if KOMA class
  \KOMAoptions{parskip=half}}
\makeatother
\usepackage{xcolor}
\IfFileExists{xurl.sty}{\usepackage{xurl}}{} % add URL line breaks if available
\IfFileExists{bookmark.sty}{\usepackage{bookmark}}{\usepackage{hyperref}}
\hypersetup{
  pdftitle={444 Lecture 7.2 - Two Puzzles about Bayesian Equilibrium},
  pdfauthor={Brian Weatherson},
  hidelinks,
  pdfcreator={LaTeX via pandoc}}
\urlstyle{same} % disable monospaced font for URLs
\newif\ifbibliography
\setlength{\emergencystretch}{3em} % prevent overfull lines
\providecommand{\tightlist}{%
  \setlength{\itemsep}{0pt}\setlength{\parskip}{0pt}}
\setcounter{secnumdepth}{-\maxdimen} % remove section numbering
\let\Tiny=\tiny

 \setbeamertemplate{navigation symbols}{} 

% \usetheme{Madrid}
 \usetheme[numbering=none, progressbar=foot]{metropolis}
 \usecolortheme{wolverine}
 \usepackage{color}
 \usepackage{MnSymbol}
% \usepackage{movie15}

\usepackage{amssymb}% http://ctan.org/pkg/amssymb
\usepackage{pifont}% http://ctan.org/pkg/pifont
\newcommand{\cmark}{\ding{51}}%
\newcommand{\xmark}{\ding{55}}%

\DeclareSymbolFont{symbolsC}{U}{txsyc}{m}{n}
\DeclareMathSymbol{\boxright}{\mathrel}{symbolsC}{128}
\DeclareMathAlphabet{\mathpzc}{OT1}{pzc}{m}{it}

\setlength{\parskip}{1ex plus 0.5ex minus 0.2ex}

\AtBeginSection[]
{
\begin{frame}
	\Huge{\color{darkblue} \insertsection}
\end{frame}
}

\renewenvironment*{quote}	
	{\list{}{\rightmargin   \leftmargin} \item } 	
	{\endlist }

\definecolor{darkgreen}{rgb}{0,0.7,0}
\definecolor{darkblue}{rgb}{0,0,0.8}

\usepackage[italic]{mathastext}
\usepackage{nicefrac}
\usepackage{istgame}

\setbeamertemplate{caption}{\raggedright\insertcaption}

%\def\toprule{}
%\def\bottomrule{}
%\def\midrule{}
\usepackage{etoolbox}
\AfterEndEnvironment{description}{\vspace{9pt}}
\AfterEndEnvironment{oltableau}{\vspace{9pt}}
\BeforeBeginEnvironment{oltableau}{\vspace{9pt}}
\AfterEndEnvironment{center}{\vspace{9pt}}
\BeforeBeginEnvironment{tabular}{\vspace{9pt}}
\AfterEndEnvironment{longtable}{\vspace{-6pt}}
\usepackage{booktabs}
\usepackage{longtable}
\usepackage{array}
\usepackage{multirow}
\usepackage{wrapfig}
\usepackage{float}
\usepackage{colortbl}
\usepackage{pdflscape}
\usepackage{tabu}
\usepackage{threeparttable} 
\usepackage{threeparttablex} 
\usepackage[normalem]{ulem} 
\usepackage{makecell}
\usepackage{xcolor}
\usepackage{ulem}

\setlength\heavyrulewidth{0ex}
\setlength\lightrulewidth{0.08ex}

\aboverulesep=0ex
\belowrulesep=0ex
\renewcommand{\arraystretch}{1.2}
\ifluatex
  \usepackage{selnolig}  % disable illegal ligatures
\fi

\title{444 Lecture 7.2 - Two Puzzles about Bayesian Equilibrium}
\author{Brian Weatherson}
\date{}

\begin{document}
\frame{\titlepage}

\begin{frame}{First Puzzle}
\protect\hypertarget{first-puzzle}{}
\begin{center}
\begin{istgame}
%\setistgrowdirection'{east}
\xtdistance{15mm}{30mm}
\istroot(0){Alice}
  \istb{A}[al]{(2,2)}
  \istb{B}[r]
  \istb{C}[ar]
  \endist
\xtdistance{10mm}{20mm}
\istroot(1)(0-2)
  \istb{l}[al]{(4,4)}
  \istb{r}[ar]{(1,0)}
  \endist
\istroot(2)(0-3)
  \istb{l}[al]{(3,0)}
  \istb{r}[ar]{(0,4)}
  \endist
\xtInfoset(1)(2){Billie}
\end{istgame}
\end{center}

\begin{itemize}
\tightlist
\item
  I don't really know what to do here.
\end{itemize}
\end{frame}

\begin{frame}{Strategy Table}
\protect\hypertarget{strategy-table}{}
\begin{table}[!h]
\centering
\begin{tabular}[t]{>{}r|cc}
\toprule
 & l & r\\
\midrule
A & 2, 2 & 2, 2\\
B & 4, 4 & 1, 0\\
C & 3, 0 & 0, 4\\
\bottomrule
\end{tabular}
\end{table}

Note that \(\langle A, r \rangle\) is a Nash equilibrium, and B
dominates C.
\end{frame}

\begin{frame}{Bayesian Equilibrium}
\protect\hypertarget{bayesian-equilibrium}{}
The following is even a Bayesian Equilibrium

\begin{itemize}
\tightlist
\item
  Alice plays A.
\item
  Billie believes with probability 1 that Alice will play A.
\item
  If Billie finds herself at the B/C set, she will believe that Alice
  played C.
\item
  So she will do what's best for her given this belief, i.e., play r.
\item
  Alice knows this.
\end{itemize}

Alice plays A and gets 2, and she believes she would get 1 if she played
B and 0 if she played C. So she's doing the thing that's best by her
lights.
\end{frame}

\begin{frame}{Wait a Minute!}
\protect\hypertarget{wait-a-minute}{}
\begin{itemize}
\tightlist
\item
  Billie is disposed to believe that if Alice does something weird (play
  B or C), she will do something really weird (play a dominated
  strategy).
\item
  Is this an OK thing for Billie to believe?
\item
  It seems a bit weird.
\item
  Maybe if Alice has not given Billie conclusive evidence that she will
  do something bizarre - i.e., play a dominated strategy - it's wrong to
  believe that she's done something bizarre.
\item
  So if we get to B/C, Billie should believe we're at B, so should play
  l, and knowing that, Alice should play B.
\end{itemize}
\end{frame}

\begin{frame}{Which is right?}
\protect\hypertarget{which-is-right}{}
\begin{itemize}
\tightlist
\item
  I don't really know.
\item
  Here we're already into contested territory.
\item
  Let's take one further step into contested territory.
\end{itemize}
\end{frame}

\begin{frame}{Second Puzzle}
\protect\hypertarget{second-puzzle}{}
\begin{center}
\begin{istgame}
%\setistgrowdirection'{east}
\xtdistance{15mm}{30mm}
\istroot(0){Alice}
  \istb{A}[al]{(3,3)}
  \istb{B}[r]
  \istb{C}[ar]
  \endist
\xtdistance{10mm}{20mm}
\istroot(1)(0-2)
  \istb{l}[al]{(4,4)}
  \istb{r}[ar]{(0,0)}
  \endist
\istroot(2)(0-3)
  \istb{l}[al]{(2,0)}
  \istb{r}[ar]{(1,4)}
  \endist
\xtInfoset(1)(2){Billie}
\end{istgame}
\end{center}

\begin{itemize}
\tightlist
\item
  I really don't know what to do here.
\end{itemize}
\end{frame}

\begin{frame}{Why is it Puzzling}
\protect\hypertarget{why-is-it-puzzling}{}
\begin{itemize}
\tightlist
\item
  \(\langle A, r \rangle\) is a Bayesian (and Nash and subgame perfect)
  equilibrium.
\item
  But maybe Alice should choose B. \pause
\item
  Note that A dominates C.
\item
  So if Billie knows that Alice doesn't choose dominated options, and
  this survives learning that Alice has chosen B or C, then Billie will
  know that Alice has chosen B.
\item
  And then Billie will choose l, so Alice should choose B.
\end{itemize}
\end{frame}

\begin{frame}{Two Big Questions}
\protect\hypertarget{two-big-questions}{}
\begin{enumerate}
\tightlist
\item
  Are there philosophical grounds to change our theory of equilibrium
  selection to make Alice choose B in both of these games?
\item
  If we want to do that, what's the best mathematical theory that
  generates the intended result?
\end{enumerate}

\begin{itemize}
\tightlist
\item
  Both of these are hard questions, but not one's I'm going to address
  in this course.
\item
  We have to stop somewhere, and there are always more questions like
  this to ask and answer.
\item
  Instead I'm going to a special class of games: signaling games.
\end{itemize}
\end{frame}

\end{document}
