% Options for packages loaded elsewhere
\PassOptionsToPackage{unicode}{hyperref}
\PassOptionsToPackage{hyphens}{url}
%
\documentclass[
  ignorenonframetext,
]{beamer}
\usepackage{pgfpages}
\setbeamertemplate{caption}[numbered]
\setbeamertemplate{caption label separator}{: }
\setbeamercolor{caption name}{fg=normal text.fg}
\beamertemplatenavigationsymbolsempty
% Prevent slide breaks in the middle of a paragraph
\widowpenalties 1 10000
\raggedbottom
\setbeamertemplate{part page}{
  \centering
  \begin{beamercolorbox}[sep=16pt,center]{part title}
    \usebeamerfont{part title}\insertpart\par
  \end{beamercolorbox}
}
\setbeamertemplate{section page}{
  \centering
  \begin{beamercolorbox}[sep=12pt,center]{part title}
    \usebeamerfont{section title}\insertsection\par
  \end{beamercolorbox}
}
\setbeamertemplate{subsection page}{
  \centering
  \begin{beamercolorbox}[sep=8pt,center]{part title}
    \usebeamerfont{subsection title}\insertsubsection\par
  \end{beamercolorbox}
}
\AtBeginPart{
  \frame{\partpage}
}
\AtBeginSection{
  \ifbibliography
  \else
    \frame{\sectionpage}
  \fi
}
\AtBeginSubsection{
  \frame{\subsectionpage}
}
\usepackage{amsmath,amssymb}
\usepackage{lmodern}
\usepackage{ifxetex,ifluatex}
\ifnum 0\ifxetex 1\fi\ifluatex 1\fi=0 % if pdftex
  \usepackage[T1]{fontenc}
  \usepackage[utf8]{inputenc}
  \usepackage{textcomp} % provide euro and other symbols
\else % if luatex or xetex
  \usepackage{unicode-math}
  \defaultfontfeatures{Scale=MatchLowercase}
  \defaultfontfeatures[\rmfamily]{Ligatures=TeX,Scale=1}
  \setmainfont[BoldFont = SF Pro Rounded Semibold]{SF Pro Rounded}
  \setmathfont[]{STIX Two Math}
\fi
\usefonttheme{serif} % use mainfont rather than sansfont for slide text
% Use upquote if available, for straight quotes in verbatim environments
\IfFileExists{upquote.sty}{\usepackage{upquote}}{}
\IfFileExists{microtype.sty}{% use microtype if available
  \usepackage[]{microtype}
  \UseMicrotypeSet[protrusion]{basicmath} % disable protrusion for tt fonts
}{}
\makeatletter
\@ifundefined{KOMAClassName}{% if non-KOMA class
  \IfFileExists{parskip.sty}{%
    \usepackage{parskip}
  }{% else
    \setlength{\parindent}{0pt}
    \setlength{\parskip}{6pt plus 2pt minus 1pt}}
}{% if KOMA class
  \KOMAoptions{parskip=half}}
\makeatother
\usepackage{xcolor}
\IfFileExists{xurl.sty}{\usepackage{xurl}}{} % add URL line breaks if available
\IfFileExists{bookmark.sty}{\usepackage{bookmark}}{\usepackage{hyperref}}
\hypersetup{
  pdftitle={444 Lecture 7.5 - Going to College},
  pdfauthor={Brian Weatherson},
  hidelinks,
  pdfcreator={LaTeX via pandoc}}
\urlstyle{same} % disable monospaced font for URLs
\newif\ifbibliography
\setlength{\emergencystretch}{3em} % prevent overfull lines
\providecommand{\tightlist}{%
  \setlength{\itemsep}{0pt}\setlength{\parskip}{0pt}}
\setcounter{secnumdepth}{-\maxdimen} % remove section numbering
\let\Tiny=\tiny

 \setbeamertemplate{navigation symbols}{} 

% \usetheme{Madrid}
 \usetheme[numbering=none, progressbar=foot]{metropolis}
 \usecolortheme{wolverine}
 \usepackage{color}
 \usepackage{MnSymbol}
% \usepackage{movie15}

\usepackage{amssymb}% http://ctan.org/pkg/amssymb
\usepackage{pifont}% http://ctan.org/pkg/pifont
\newcommand{\cmark}{\ding{51}}%
\newcommand{\xmark}{\ding{55}}%

\DeclareSymbolFont{symbolsC}{U}{txsyc}{m}{n}
\DeclareMathSymbol{\boxright}{\mathrel}{symbolsC}{128}
\DeclareMathAlphabet{\mathpzc}{OT1}{pzc}{m}{it}

\setlength{\parskip}{1ex plus 0.5ex minus 0.2ex}

\AtBeginSection[]
{
\begin{frame}
	\Huge{\color{darkblue} \insertsection}
\end{frame}
}

\renewenvironment*{quote}	
	{\list{}{\rightmargin   \leftmargin} \item } 	
	{\endlist }

\definecolor{darkgreen}{rgb}{0,0.7,0}
\definecolor{darkblue}{rgb}{0,0,0.8}

\usepackage[italic]{mathastext}
\usepackage{nicefrac}
\usepackage{istgame}

\setbeamertemplate{caption}{\raggedright\insertcaption}

%\def\toprule{}
%\def\bottomrule{}
%\def\midrule{}
\usepackage{etoolbox}
\AfterEndEnvironment{description}{\vspace{9pt}}
\AfterEndEnvironment{oltableau}{\vspace{9pt}}
\BeforeBeginEnvironment{oltableau}{\vspace{9pt}}
\AfterEndEnvironment{center}{\vspace{9pt}}
\BeforeBeginEnvironment{tabular}{\vspace{9pt}}
\AfterEndEnvironment{longtable}{\vspace{-6pt}}
\usepackage{booktabs}
\usepackage{longtable}
\usepackage{array}
\usepackage{multirow}
\usepackage{wrapfig}
\usepackage{float}
\usepackage{colortbl}
\usepackage{pdflscape}
\usepackage{tabu}
\usepackage{threeparttable} 
\usepackage{threeparttablex} 
\usepackage[normalem]{ulem} 
\usepackage{makecell}
\usepackage{xcolor}
\usepackage{ulem}

\setlength\heavyrulewidth{0ex}
\setlength\lightrulewidth{0.08ex}

\aboverulesep=0ex
\belowrulesep=0ex
\renewcommand{\arraystretch}{1.2}
\ifluatex
  \usepackage{selnolig}  % disable illegal ligatures
\fi

\title{444 Lecture 7.5 - Going to College}
\author{Brian Weatherson}
\date{}

\begin{document}
\frame{\titlepage}

\begin{frame}{The College Game}
\protect\hypertarget{the-college-game}{}
\begin{itemize}
\tightlist
\item
  Sender is deciding whether to go to college.
\item
  There are two attributes of Sender that we're going to be interested
  in.
\item
  They are either a High Value or Low Value employee.
\item
  They will either Like or Dislike college.
\item
  Let's assume that these attributes are perfectly correlated: all and
  only the High Value employees Like college.
\end{itemize}
\end{frame}

\begin{frame}{Features of College in this Game}
\protect\hypertarget{features-of-college-in-this-game}{}
\begin{itemize}[<+->]
\tightlist
\item
  College does not change anyone's value to employers - High Value
  employees are high value whether or not they go to college, and Low
  Value employees are low value either way.
\item
  College is fun for people who Like it (i.e., the High Values), but
  it's not so much fun to be actually worth the expense. But it's a
  relatively minor overpay for the people who Like it, and both
  unbearable and exorbitantly expensive for those who Dislike it.
\item
  I am \emph{not} saying either of these are true, though I don't
  entirely disagree with the second.
\end{itemize}
\end{frame}

\begin{frame}{The Hiring Decision}
\protect\hypertarget{the-hiring-decision}{}
\begin{itemize}
\tightlist
\item
  Hearer is an employer who pays high salaries, but gets good value for
  this high salary from High Value employees.
\item
  Unfortunately, they have literally no way of telling who is High Value
  and who is Low Value.
\item
  All they know is that only 40\% of people are High Value.
\end{itemize}
\end{frame}

\begin{frame}{Payouts}
\protect\hypertarget{payouts}{}
\begin{itemize}
\tightlist
\item
  Everyone starts with 0 points, unless one of the conditions below is
  triggered.
\item
  Sender gets 2 points if they get Recruited.
\item
  They lose 1 point if they Like college and go to college.
\item
  They lose 3 points if they Dislike college and go to college.
\item
  Hearer gets 1 point if they Recruit a High Value Sender.
\item
  They lose 1 point if they Recruit a Low Value Sender.
\end{itemize}
\end{frame}

\begin{frame}
\begin{center}
\begin{istgame}[scale=1.3]
   \xtdistance{20mm}{20mm}
   \istroot(0)[chance node]{$c$}
     \istb<grow=left>{0.4}[a]
     \istb<grow=right>{0.6}[a]
     \endist
   \xtdistance{10mm}{20mm}
   \istroot(1)(0-1)<180>{1}
     \istb<grow=north>{College}[l]
     \istb<grow=south>{Beach}[l]
     \endist
   \istroot(2)(0-2)<0>{1}
     \istb<grow=north>{College}[r]
     \istb<grow=south>{Beach}[r]
     \endist
   \istroot'[north](a1)(1-1)
     \istb{R}[bl]{2, 1}
     \istb{P}[br]{-1,0}
     \endist
   \istroot(b1)(1-2)
     \istb{R}[al]{3,1}
     \istb{P}[ar]{0,0}
     \endist
   \istroot(a2)(2-2)
     \istb{R}[al]{3,-1}
     \istb{P}[ar]{0,0}
     \endist
   \istroot'[north](b2)(2-1)
     \istb{R}[bl]{-1,-1}
     \istb{P}[br]{-3,0}
     \endist
   \xtInfoset(a1)(b2){2}
   \xtInfoset(b1)(a2){2}
   \end{istgame}
\end{center}
\end{frame}

\begin{frame}{Some Notes}
\protect\hypertarget{some-notes}{}
\begin{itemize}
\tightlist
\item
  In the original Spence game, Sender gets to choose how much to spend
  on education from a range. They have infinitely many choices, not just
  the binary College/Beach choice. This doesn't really affect the
  analysis.
\item
  What is crucial is that education is more costly for Low Value
  employees.
\item
  There are a lot of equilibria to this game, but the most natural is
  the separating equilibria, where Like/High go to college, and
  Dislike/Low go to the Beach.
\item
  For reasons I don't know (but can guess about), the wikipedia page on
  signaling games is dire. This is odd because most of the game theory
  pages are really very good.
\end{itemize}
\end{frame}

\begin{frame}{Plausibility}
\protect\hypertarget{plausibility}{}
Here are some ways in which the model (or at least the separating
equilibrium of the model) does seem to look a bit like the real world.

\begin{itemize}
\tightlist
\item
  College grads get paid a lot more than non-grads.
\item
  It isn't immediately obvious how what we do here explains the higher
  pay.
\item
  Yet there is a ton of demand for places in college (at least
  pre-pandemic), and obviously a lot of demand for college grads.
\item
  College is more fun, i.e., less costly, for people with certain skills
  (perserverence, curiosity, writing/mathematical aptitude) that are
  independently valuable to employers.
\end{itemize}
\end{frame}

\begin{frame}{Implausibility}
\protect\hypertarget{implausibility}{}
But there are several ways in which the model does not seem particularly
plausible.

\begin{itemize}
\tightlist
\item
  At least after a few weeks/months/years in the job, employers have
  some ability to tell who is High Value, so if education was purely a
  signal, it should wear off after a little while.
\item
  The correlation between High Value and Liking college is a long way
  from perfect. At least in my day, the people who \emph{really} liked
  college were not at all what I'd think of as High Value employees for
  most businesses.
\item
  Even if the people who Dislike college really really hate calculus
  class, it's a little hard to see how they could hate it so much to
  turn down the college wage premium.
\end{itemize}
\end{frame}

\end{document}
