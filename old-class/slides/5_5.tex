% Options for packages loaded elsewhere
\PassOptionsToPackage{unicode}{hyperref}
\PassOptionsToPackage{hyphens}{url}
%
\documentclass[
  ignorenonframetext,
]{beamer}
\usepackage{pgfpages}
\setbeamertemplate{caption}[numbered]
\setbeamertemplate{caption label separator}{: }
\setbeamercolor{caption name}{fg=normal text.fg}
\beamertemplatenavigationsymbolsempty
% Prevent slide breaks in the middle of a paragraph
\widowpenalties 1 10000
\raggedbottom
\setbeamertemplate{part page}{
  \centering
  \begin{beamercolorbox}[sep=16pt,center]{part title}
    \usebeamerfont{part title}\insertpart\par
  \end{beamercolorbox}
}
\setbeamertemplate{section page}{
  \centering
  \begin{beamercolorbox}[sep=12pt,center]{part title}
    \usebeamerfont{section title}\insertsection\par
  \end{beamercolorbox}
}
\setbeamertemplate{subsection page}{
  \centering
  \begin{beamercolorbox}[sep=8pt,center]{part title}
    \usebeamerfont{subsection title}\insertsubsection\par
  \end{beamercolorbox}
}
\AtBeginPart{
  \frame{\partpage}
}
\AtBeginSection{
  \ifbibliography
  \else
    \frame{\sectionpage}
  \fi
}
\AtBeginSubsection{
  \frame{\subsectionpage}
}
\usepackage{amsmath,amssymb}
\usepackage{lmodern}
\usepackage{ifxetex,ifluatex}
\ifnum 0\ifxetex 1\fi\ifluatex 1\fi=0 % if pdftex
  \usepackage[T1]{fontenc}
  \usepackage[utf8]{inputenc}
  \usepackage{textcomp} % provide euro and other symbols
\else % if luatex or xetex
  \usepackage{unicode-math}
  \defaultfontfeatures{Scale=MatchLowercase}
  \defaultfontfeatures[\rmfamily]{Ligatures=TeX,Scale=1}
  \setmainfont[BoldFont = SF Pro Rounded Semibold]{SF Pro Rounded}
  \setmathfont[]{STIX Two Math}
\fi
\usefonttheme{serif} % use mainfont rather than sansfont for slide text
% Use upquote if available, for straight quotes in verbatim environments
\IfFileExists{upquote.sty}{\usepackage{upquote}}{}
\IfFileExists{microtype.sty}{% use microtype if available
  \usepackage[]{microtype}
  \UseMicrotypeSet[protrusion]{basicmath} % disable protrusion for tt fonts
}{}
\makeatletter
\@ifundefined{KOMAClassName}{% if non-KOMA class
  \IfFileExists{parskip.sty}{%
    \usepackage{parskip}
  }{% else
    \setlength{\parindent}{0pt}
    \setlength{\parskip}{6pt plus 2pt minus 1pt}}
}{% if KOMA class
  \KOMAoptions{parskip=half}}
\makeatother
\usepackage{xcolor}
\IfFileExists{xurl.sty}{\usepackage{xurl}}{} % add URL line breaks if available
\IfFileExists{bookmark.sty}{\usepackage{bookmark}}{\usepackage{hyperref}}
\hypersetup{
  pdftitle={444 Lecture 5.5 - Dominance by Mixture},
  pdfauthor={Brian Weatherson},
  hidelinks,
  pdfcreator={LaTeX via pandoc}}
\urlstyle{same} % disable monospaced font for URLs
\newif\ifbibliography
\setlength{\emergencystretch}{3em} % prevent overfull lines
\providecommand{\tightlist}{%
  \setlength{\itemsep}{0pt}\setlength{\parskip}{0pt}}
\setcounter{secnumdepth}{-\maxdimen} % remove section numbering
\let\Tiny=\tiny

 \setbeamertemplate{navigation symbols}{} 

% \usetheme{Madrid}
 \usetheme[numbering=none, progressbar=foot]{metropolis}
 \usecolortheme{wolverine}
 \usepackage{color}
 \usepackage{MnSymbol}
% \usepackage{movie15}

\usepackage{amssymb}% http://ctan.org/pkg/amssymb
\usepackage{pifont}% http://ctan.org/pkg/pifont
\newcommand{\cmark}{\ding{51}}%
\newcommand{\xmark}{\ding{55}}%

\DeclareSymbolFont{symbolsC}{U}{txsyc}{m}{n}
\DeclareMathSymbol{\boxright}{\mathrel}{symbolsC}{128}
\DeclareMathAlphabet{\mathpzc}{OT1}{pzc}{m}{it}

\setlength{\parskip}{1ex plus 0.5ex minus 0.2ex}

\AtBeginSection[]
{
\begin{frame}
	\Huge{\color{darkblue} \insertsection}
\end{frame}
}

\renewenvironment*{quote}	
	{\list{}{\rightmargin   \leftmargin} \item } 	
	{\endlist }

\definecolor{darkgreen}{rgb}{0,0.7,0}
\definecolor{darkblue}{rgb}{0,0,0.8}

\usepackage[italic]{mathastext}
\usepackage{nicefrac}

\setbeamertemplate{caption}{\raggedright\insertcaption}

%\def\toprule{}
%\def\bottomrule{}
%\def\midrule{}
\usepackage{etoolbox}
\AfterEndEnvironment{description}{\vspace{9pt}}
\AfterEndEnvironment{oltableau}{\vspace{9pt}}
\BeforeBeginEnvironment{oltableau}{\vspace{9pt}}
\AfterEndEnvironment{center}{\vspace{9pt}}
\BeforeBeginEnvironment{tabular}{\vspace{9pt}}
\AfterEndEnvironment{longtable}{\vspace{-6pt}}
\usepackage{booktabs}
\usepackage{longtable}
\usepackage{array}
\usepackage{multirow}
\usepackage{wrapfig}
\usepackage{float}
\usepackage{colortbl}
\usepackage{pdflscape}
\usepackage{tabu}
\usepackage{threeparttable} 
\usepackage{threeparttablex} 
\usepackage[normalem]{ulem} 
\usepackage{makecell}
\usepackage{xcolor}
\usepackage{ulem}

\setlength\heavyrulewidth{0ex}
\setlength\lightrulewidth{0.08ex}

\aboverulesep=0ex
\belowrulesep=0ex
\renewcommand{\arraystretch}{1.2}
\ifluatex
  \usepackage{selnolig}  % disable illegal ligatures
\fi

\title{444 Lecture 5.5 - Dominance by Mixture}
\author{Brian Weatherson}
\date{}

\begin{document}
\frame{\titlepage}

\begin{frame}{Plan}
\protect\hypertarget{plan}{}
Discuss a new form of dominance reasoning - mixed strategy dominance.
\end{frame}

\begin{frame}{Reading}
\protect\hypertarget{reading}{}
Bonanno, Section 6.4.
\end{frame}

\begin{frame}{Basic Example}
\protect\hypertarget{basic-example}{}
\begin{table}[!h]
\centering
\begin{tabular}[t]{>{}r|cc}
\toprule
 & Left & Right\\
\midrule
Up & 3, 0 & 0, 0\\
Middle & 1, 0 & 1, 0\\
Down & 0, 0 & 3, 0\\
\bottomrule
\end{tabular}
\end{table}

This is a bit boring for Column, but let's focus on Row for now.
\end{frame}

\begin{frame}{Dominance Reasoning}
\protect\hypertarget{dominance-reasoning}{}
\begin{table}[!h]
\centering
\begin{tabular}[t]{>{}r|cc}
\toprule
 & Left & Right\\
\midrule
Up & 3, 0 & 0, 0\\
Middle & 1, 0 & 1, 0\\
Down & 0, 0 & 3, 0\\
\bottomrule
\end{tabular}
\end{table}

At first it looks like there are no dominated strategies.

\begin{itemize}
\tightlist
\item
  Up does worse than Middle and Down if Column plays Right, so it
  doesn't dominate anything.
\item
  Middle does worse that Up if Column plays Left, and worse than Down if
  Column plays Right.
\item
  Down does worse than both of them if Column plays Left.
\end{itemize}
\end{frame}

\begin{frame}{Dominance Reasoning}
\protect\hypertarget{dominance-reasoning-1}{}
\begin{table}[!h]
\centering
\begin{tabular}[t]{>{}r|cc}
\toprule
 & Left & Right\\
\midrule
Up & 3, 0 & 0, 0\\
Middle & 1, 0 & 1, 0\\
Down & 0, 0 & 3, 0\\
\bottomrule
\end{tabular}
\end{table}

But compare these two strategies.

\begin{itemize}
\tightlist
\item
  Middle
\item
  The mixed strategy of Up with probability 0.5, and Down with
  probability 0.5.
\end{itemize}
\end{frame}

\begin{frame}{Dominance Reasoning}
\protect\hypertarget{dominance-reasoning-2}{}
\begin{table}[!h]
\centering
\begin{tabular}[t]{>{}r|cc}
\toprule
 & Left & Right\\
\midrule
Up & 3, 0 & 0, 0\\
Middle & 1, 0 & 1, 0\\
Down & 0, 0 & 3, 0\\
\bottomrule
\end{tabular}
\end{table}

\begin{itemize}
\tightlist
\item
  Middle gets an actual return of 1 whatever Column does.
\item
  The mixed strategy gets an expected return of 1.5 whatever Column
  does.
\item
  So it has a higher expected return given Left (1.5 \textgreater{} 1),
  and a higher expected return given Right (1.5 \textgreater{} 1).
\end{itemize}
\end{frame}

\begin{frame}{Dominance Reasoning}
\protect\hypertarget{dominance-reasoning-3}{}
\begin{table}[!h]
\centering
\begin{tabular}[t]{>{}r|cc}
\toprule
 & Left & Right\\
\midrule
Up & 3, 0 & 0, 0\\
Middle & 1, 0 & 1, 0\\
Down & 0, 0 & 3, 0\\
\bottomrule
\end{tabular}
\end{table}

\begin{itemize}
\tightlist
\item
  If that happens, then we'll say that Middle is dominated by this
  mixture.
\item
  When we're deleting dominated strategies, we should delete it too.
\end{itemize}
\end{frame}

\begin{frame}{Nash and Dominance}
\protect\hypertarget{nash-and-dominance}{}
\begin{itemize}
\tightlist
\item
  A strategy that is dominated by a mixture like this can never be part
  of a Nash equilibrium.
\item
  After all, the player would be better off playing the mixture than
  playing it, so it fails the test that there is nothing better to do.
\item
  So being able to find these dominating mixtures can be very helpful in
  working out what the Nash equilibrium is.
\end{itemize}
\end{frame}

\begin{frame}{Rational Play and Dominance}
\protect\hypertarget{rational-play-and-dominance}{}
\begin{itemize}
\tightlist
\item
  But even beyond that, it seems wrong to play strategies that are
  dominated in this way.
\item
  If you're thinking about playing Middle (as Row), you increase your
  expected return by simply flipping a coin to choose between Left and
  Right.
\item
  So that's what you should do.
\end{itemize}
\end{frame}

\begin{frame}{For Next Time}
\protect\hypertarget{for-next-time}{}
We'll connect this expanded notion of dominance up to an expanded notion
of best responses.
\end{frame}

\end{document}
