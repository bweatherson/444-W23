% Options for packages loaded elsewhere
\PassOptionsToPackage{unicode}{hyperref}
\PassOptionsToPackage{hyphens}{url}
%
\documentclass[
  ignorenonframetext,
]{beamer}
\usepackage{pgfpages}
\setbeamertemplate{caption}[numbered]
\setbeamertemplate{caption label separator}{: }
\setbeamercolor{caption name}{fg=normal text.fg}
\beamertemplatenavigationsymbolsempty
% Prevent slide breaks in the middle of a paragraph
\widowpenalties 1 10000
\raggedbottom
\setbeamertemplate{part page}{
  \centering
  \begin{beamercolorbox}[sep=16pt,center]{part title}
    \usebeamerfont{part title}\insertpart\par
  \end{beamercolorbox}
}
\setbeamertemplate{section page}{
  \centering
  \begin{beamercolorbox}[sep=12pt,center]{part title}
    \usebeamerfont{section title}\insertsection\par
  \end{beamercolorbox}
}
\setbeamertemplate{subsection page}{
  \centering
  \begin{beamercolorbox}[sep=8pt,center]{part title}
    \usebeamerfont{subsection title}\insertsubsection\par
  \end{beamercolorbox}
}
\AtBeginPart{
  \frame{\partpage}
}
\AtBeginSection{
  \ifbibliography
  \else
    \frame{\sectionpage}
  \fi
}
\AtBeginSubsection{
  \frame{\subsectionpage}
}
\usepackage{amsmath,amssymb}
\usepackage{lmodern}
\usepackage{ifxetex,ifluatex}
\ifnum 0\ifxetex 1\fi\ifluatex 1\fi=0 % if pdftex
  \usepackage[T1]{fontenc}
  \usepackage[utf8]{inputenc}
  \usepackage{textcomp} % provide euro and other symbols
\else % if luatex or xetex
  \usepackage{unicode-math}
  \defaultfontfeatures{Scale=MatchLowercase}
  \defaultfontfeatures[\rmfamily]{Ligatures=TeX,Scale=1}
  \setmainfont[BoldFont = SF Pro Rounded Semibold]{SF Pro Rounded}
  \setmathfont[]{STIX Two Math}
\fi
\usefonttheme{serif} % use mainfont rather than sansfont for slide text
% Use upquote if available, for straight quotes in verbatim environments
\IfFileExists{upquote.sty}{\usepackage{upquote}}{}
\IfFileExists{microtype.sty}{% use microtype if available
  \usepackage[]{microtype}
  \UseMicrotypeSet[protrusion]{basicmath} % disable protrusion for tt fonts
}{}
\makeatletter
\@ifundefined{KOMAClassName}{% if non-KOMA class
  \IfFileExists{parskip.sty}{%
    \usepackage{parskip}
  }{% else
    \setlength{\parindent}{0pt}
    \setlength{\parskip}{6pt plus 2pt minus 1pt}}
}{% if KOMA class
  \KOMAoptions{parskip=half}}
\makeatother
\usepackage{xcolor}
\IfFileExists{xurl.sty}{\usepackage{xurl}}{} % add URL line breaks if available
\IfFileExists{bookmark.sty}{\usepackage{bookmark}}{\usepackage{hyperref}}
\hypersetup{
  pdftitle={444 Lecture 2.6 - Iterated Deletion},
  pdfauthor={Brian Weatherson},
  hidelinks,
  pdfcreator={LaTeX via pandoc}}
\urlstyle{same} % disable monospaced font for URLs
\newif\ifbibliography
\setlength{\emergencystretch}{3em} % prevent overfull lines
\providecommand{\tightlist}{%
  \setlength{\itemsep}{0pt}\setlength{\parskip}{0pt}}
\setcounter{secnumdepth}{-\maxdimen} % remove section numbering
\let\Tiny=\tiny

 \setbeamertemplate{navigation symbols}{} 

% \usetheme{Madrid}
 \usetheme[numbering=none, progressbar=foot]{metropolis}
 \usecolortheme{wolverine}
 \usepackage{color}
 \usepackage{MnSymbol}
% \usepackage{movie15}

\usepackage{amssymb}% http://ctan.org/pkg/amssymb
\usepackage{pifont}% http://ctan.org/pkg/pifont
\newcommand{\cmark}{\ding{51}}%
\newcommand{\xmark}{\ding{55}}%

\DeclareSymbolFont{symbolsC}{U}{txsyc}{m}{n}
\DeclareMathSymbol{\boxright}{\mathrel}{symbolsC}{128}
\DeclareMathAlphabet{\mathpzc}{OT1}{pzc}{m}{it}

\setlength{\parskip}{1ex plus 0.5ex minus 0.2ex}

\AtBeginSection[]
{
\begin{frame}
	\Huge{\color{darkblue} \insertsection}
\end{frame}
}

\renewenvironment*{quote}	
	{\list{}{\rightmargin   \leftmargin} \item } 	
	{\endlist }

\definecolor{darkgreen}{rgb}{0,0.7,0}
\definecolor{darkblue}{rgb}{0,0,0.8}

\usepackage[italic]{mathastext}
\usepackage{nicefrac}

\setbeamertemplate{caption}{\raggedright\insertcaption}

%\def\toprule{}
%\def\bottomrule{}
%\def\midrule{}
\usepackage{etoolbox}
\AfterEndEnvironment{description}{\vspace{9pt}}
\AfterEndEnvironment{oltableau}{\vspace{9pt}}
\BeforeBeginEnvironment{oltableau}{\vspace{9pt}}
\AfterEndEnvironment{center}{\vspace{9pt}}
\BeforeBeginEnvironment{tabular}{\vspace{9pt}}
\AfterEndEnvironment{longtable}{\vspace{-6pt}}
\usepackage{booktabs}
\usepackage{longtable}
\usepackage{array}
\usepackage{multirow}
\usepackage{wrapfig}
\usepackage{float}
\usepackage{colortbl}
\usepackage{pdflscape}
\usepackage{tabu}
\usepackage{threeparttable} 
\usepackage{threeparttablex} 
\usepackage[normalem]{ulem} 
\usepackage{makecell}
\usepackage{xcolor}
\usepackage{ulem}

\setlength\heavyrulewidth{0ex}
\setlength\lightrulewidth{0.08ex}

\aboverulesep=0ex
\belowrulesep=0ex
\renewcommand{\arraystretch}{1.2}
\ifluatex
  \usepackage{selnolig}  % disable illegal ligatures
\fi

\title{444 Lecture 2.6 - Iterated Deletion}
\author{Brian Weatherson}
\date{}

\begin{document}
\frame{\titlepage}

\begin{frame}{Plan}
\protect\hypertarget{plan}{}
To describe the strategy of solving games by iteratively deleting
strategies.
\end{frame}

\begin{frame}{Reading}
\protect\hypertarget{reading}{}
Bonanno, section 2.5
\end{frame}

\begin{frame}{Initial Idea}
\protect\hypertarget{initial-idea}{}
\begin{itemize}[<+->]
\tightlist
\item
  If an option is strongly dominated, it shouldn't be chosen.
\item
  In the simple case, if all options but one are strongly dominated,
  that one should be chosen.
\item
  But we can say more than this.
\item
  If a strategy only makes sense if the other player plays a dominated
  strategy, then it doesn't make sense.
\item
  Let's work through some examples to see how this works in practice.
\end{itemize}
\end{frame}

\begin{frame}{Easy Example}
\protect\hypertarget{easy-example}{}
\begin{table}[!h]
\centering
\begin{tabular}[t]{>{}r|cc}
\toprule
 & Left & Right\\
\midrule
Up & 4, 1 & 2, 2\\
Down & 3, 3 & 1, 4\\
\bottomrule
\end{tabular}
\end{table}

We can solve this using just domination.

\begin{itemize}[<+->]
\tightlist
\item
  Up dominates Down, so Row should play Up.
\item
  Right dominates Left, so Column should play Right.
\item
  So the solution is Up/Right.
\end{itemize}
\end{frame}

\begin{frame}{Only Slightly Harder Example}
\protect\hypertarget{only-slightly-harder-example}{}
\begin{table}[!h]
\centering
\begin{tabular}[t]{>{}r|cc}
\toprule
 & Left & Right\\
\midrule
Up & 4, 0 & 2, 1\\
Down & 3, 1 & 1, 0\\
\bottomrule
\end{tabular}
\end{table}

Now Column doesn't have a dominating option, but that doesn't stop us.

\begin{itemize}[<+->]
\tightlist
\item
  Up dominates Down, so Row should play Up.
\item
  If Row is playing Up, Right is better than Left (1 beats 0).
\item
  So since Row is playing Up, Column should play Right.
\item
  So the solution (again) is Up/Right.
\end{itemize}
\end{frame}

\begin{frame}{Iterated Dominance}
\protect\hypertarget{iterated-dominance}{}
\begin{table}[!h]
\centering
\begin{tabular}[t]{>{}r|ccc}
\toprule
 & Left & Center & Right\\
\midrule
Up & 4, 2 & 3, 1 & 0, 0\\
Middle & 3, 0 & 2, 2 & 1, 1\\
Down & 2, 0 & 1, 0 & 0, 3\\
\bottomrule
\end{tabular}
\end{table}

We can't immediately solve this with dominance, but we can in a few
steps.
\end{frame}

\begin{frame}{Iterated Dominance}
\protect\hypertarget{iterated-dominance-1}{}
\begin{table}[!h]
\centering
\begin{tabular}[t]{>{}r|ccc}
\toprule
 & Left & Center & Right\\
\midrule
Up & 4, 2 & 3, 1 & 0, 0\\
Middle & 3, 0 & 2, 2 & 1, 1\\
Down & 2, 0 & 1, 0 & 0, 3\\
\bottomrule
\end{tabular}
\end{table}

\begin{itemize}
\tightlist
\item
  Note first that Middle dominates Down.
\item
  So Down should not be played.
\item
  In fact, we might even act as if it is not there.
\end{itemize}
\end{frame}

\begin{frame}{Iterated Dominance}
\protect\hypertarget{iterated-dominance-2}{}
\begin{table}[!h]
\centering
\begin{tabular}[t]{>{}r|ccc}
\toprule
 & Left & Center & Right\\
\midrule
Up & 4, 2 & 3, 1 & 0, 0\\
Middle & 3, 0 & 2, 2 & 1, 1\\
\bottomrule
\end{tabular}
\end{table}

\begin{itemize}
\tightlist
\item
  Here's what happens if we \textbf{delete} the dominated option Down.
\item
  Now Center dominates Right.
\item
  It didn't a minute ago - Right is a better response to Down than
  Center is - but Down is deleted.
\item
  So Right is out, and we'll delete it too.
\end{itemize}
\end{frame}

\begin{frame}{Iterated Dominance}
\protect\hypertarget{iterated-dominance-3}{}
\begin{table}[!h]
\centering
\begin{tabular}[t]{>{}r|cc}
\toprule
 & Left & Center\\
\midrule
Up & 4, 2 & 3, 1\\
Middle & 3, 0 & 2, 2\\
\bottomrule
\end{tabular}
\end{table}

\begin{itemize}
\tightlist
\item
  In this game, Up dominates Middle.
\item
  So Middle has to go.
\end{itemize}
\end{frame}

\begin{frame}{Iterated Dominance}
\protect\hypertarget{iterated-dominance-4}{}
\begin{table}[!h]
\centering
\begin{tabular}[t]{>{}r|cc}
\toprule
 & Left & Center\\
\midrule
Up & 4, 2 & 3, 1\\
\bottomrule
\end{tabular}
\end{table}

\begin{itemize}
\tightlist
\item
  And in this game, Left dominates Center.
\item
  So the solution to the game is Up/Left.
\end{itemize}
\end{frame}

\begin{frame}{General Strategy}
\protect\hypertarget{general-strategy}{}
\begin{itemize}
\tightlist
\item
  Start deleting dominated strategies.
\item
  Then see if some strategies are dominated in the new version of the
  game.
\item
  If you're lucky, the result will be that just one option for each
  player is left.
\item
  If so, we'll call that the solution of the game.
\end{itemize}
\end{frame}

\begin{frame}{For Next Time}
\protect\hypertarget{for-next-time}{}
\begin{itemize}
\tightlist
\item
  We'll see some complications that arise when we delete weakly
  dominated strategies.
\end{itemize}
\end{frame}

\end{document}
