% Options for packages loaded elsewhere
\PassOptionsToPackage{unicode}{hyperref}
\PassOptionsToPackage{hyphens}{url}
%
\documentclass[
  ignorenonframetext,
]{beamer}
\usepackage{pgfpages}
\setbeamertemplate{caption}[numbered]
\setbeamertemplate{caption label separator}{: }
\setbeamercolor{caption name}{fg=normal text.fg}
\beamertemplatenavigationsymbolsempty
% Prevent slide breaks in the middle of a paragraph
\widowpenalties 1 10000
\raggedbottom
\setbeamertemplate{part page}{
  \centering
  \begin{beamercolorbox}[sep=16pt,center]{part title}
    \usebeamerfont{part title}\insertpart\par
  \end{beamercolorbox}
}
\setbeamertemplate{section page}{
  \centering
  \begin{beamercolorbox}[sep=12pt,center]{part title}
    \usebeamerfont{section title}\insertsection\par
  \end{beamercolorbox}
}
\setbeamertemplate{subsection page}{
  \centering
  \begin{beamercolorbox}[sep=8pt,center]{part title}
    \usebeamerfont{subsection title}\insertsubsection\par
  \end{beamercolorbox}
}
\AtBeginPart{
  \frame{\partpage}
}
\AtBeginSection{
  \ifbibliography
  \else
    \frame{\sectionpage}
  \fi
}
\AtBeginSubsection{
  \frame{\subsectionpage}
}
\usepackage{amsmath,amssymb}
\usepackage{lmodern}
\usepackage{ifxetex,ifluatex}
\ifnum 0\ifxetex 1\fi\ifluatex 1\fi=0 % if pdftex
  \usepackage[T1]{fontenc}
  \usepackage[utf8]{inputenc}
  \usepackage{textcomp} % provide euro and other symbols
\else % if luatex or xetex
  \usepackage{unicode-math}
  \defaultfontfeatures{Scale=MatchLowercase}
  \defaultfontfeatures[\rmfamily]{Ligatures=TeX,Scale=1}
  \setmainfont[BoldFont = SF Pro Rounded Semibold]{SF Pro Rounded}
  \setmathfont[]{STIX Two Math}
\fi
\usefonttheme{serif} % use mainfont rather than sansfont for slide text
% Use upquote if available, for straight quotes in verbatim environments
\IfFileExists{upquote.sty}{\usepackage{upquote}}{}
\IfFileExists{microtype.sty}{% use microtype if available
  \usepackage[]{microtype}
  \UseMicrotypeSet[protrusion]{basicmath} % disable protrusion for tt fonts
}{}
\makeatletter
\@ifundefined{KOMAClassName}{% if non-KOMA class
  \IfFileExists{parskip.sty}{%
    \usepackage{parskip}
  }{% else
    \setlength{\parindent}{0pt}
    \setlength{\parskip}{6pt plus 2pt minus 1pt}}
}{% if KOMA class
  \KOMAoptions{parskip=half}}
\makeatother
\usepackage{xcolor}
\IfFileExists{xurl.sty}{\usepackage{xurl}}{} % add URL line breaks if available
\IfFileExists{bookmark.sty}{\usepackage{bookmark}}{\usepackage{hyperref}}
\hypersetup{
  pdftitle={444 Lecture 9.2 - O'Connor Chapters 2-3},
  pdfauthor={Brian Weatherson},
  hidelinks,
  pdfcreator={LaTeX via pandoc}}
\urlstyle{same} % disable monospaced font for URLs
\newif\ifbibliography
\setlength{\emergencystretch}{3em} % prevent overfull lines
\providecommand{\tightlist}{%
  \setlength{\itemsep}{0pt}\setlength{\parskip}{0pt}}
\setcounter{secnumdepth}{-\maxdimen} % remove section numbering
\let\Tiny=\tiny

 \setbeamertemplate{navigation symbols}{} 

% \usetheme{Madrid}
 \usetheme[numbering=none, progressbar=foot]{metropolis}
 \usecolortheme{wolverine}
 \usepackage{color}
 \usepackage{MnSymbol}
% \usepackage{movie15}

\usepackage{amssymb}% http://ctan.org/pkg/amssymb
\usepackage{pifont}% http://ctan.org/pkg/pifont
\newcommand{\cmark}{\ding{51}}%
\newcommand{\xmark}{\ding{55}}%

\DeclareSymbolFont{symbolsC}{U}{txsyc}{m}{n}
\DeclareMathSymbol{\boxright}{\mathrel}{symbolsC}{128}
\DeclareMathAlphabet{\mathpzc}{OT1}{pzc}{m}{it}

\setlength{\parskip}{1ex plus 0.5ex minus 0.2ex}

\AtBeginSection[]
{
\begin{frame}
	\Huge{\color{darkblue} \insertsection}
\end{frame}
}

\renewenvironment*{quote}	
	{\list{}{\rightmargin   \leftmargin} \item } 	
	{\endlist }

\definecolor{darkgreen}{rgb}{0,0.7,0}
\definecolor{darkblue}{rgb}{0,0,0.8}

\usepackage[italic]{mathastext}
\usepackage{nicefrac}
\usepackage{istgame}

\setbeamertemplate{caption}{\raggedright\insertcaption}

%\def\toprule{}
%\def\bottomrule{}
%\def\midrule{}
\usepackage{etoolbox}
\AfterEndEnvironment{description}{\vspace{9pt}}
\AfterEndEnvironment{oltableau}{\vspace{9pt}}
\BeforeBeginEnvironment{oltableau}{\vspace{9pt}}
\AfterEndEnvironment{center}{\vspace{9pt}}
\BeforeBeginEnvironment{tabular}{\vspace{9pt}}
\AfterEndEnvironment{longtable}{\vspace{-6pt}}
\usepackage{booktabs}
\usepackage{longtable}
\usepackage{array}
\usepackage{multirow}
\usepackage{wrapfig}
\usepackage{float}
\usepackage{colortbl}
\usepackage{pdflscape}
\usepackage{tabu}
\usepackage{threeparttable} 
\usepackage{threeparttablex} 
\usepackage[normalem]{ulem} 
\usepackage{makecell}
\usepackage{xcolor}
\usepackage{ulem}

\setlength\heavyrulewidth{0ex}
\setlength\lightrulewidth{0.08ex}

\aboverulesep=0ex
\belowrulesep=0ex
\renewcommand{\arraystretch}{1.2}
\AtBeginSection[]
{
    \begin{frame}
        \frametitle{Day Plan}
        \tableofcontents[currentsection]
    \end{frame}
}
\ifluatex
  \usepackage{selnolig}  % disable illegal ligatures
\fi

\title{444 Lecture 9.2 - O'Connor Chapters 2-3}
\author{Brian Weatherson}
\date{}

\begin{document}
\frame{\titlepage}

\begin{frame}{Day Plan}
\protect\hypertarget{day-plan}{}
\tableofcontents
\end{frame}

\hypertarget{types}{%
\section{Types}\label{types}}

\begin{frame}{Types}
\protect\hypertarget{types-1}{}
\begin{itemize}
\tightlist
\item
  Remember that the key thing abut types is that they are visible.
\item
  In any interaction, everyone knows who is of which type.
\item
  And everyone knows everyone knows that.
\item
  So part of the theory is that a method of typing will have to go
  along, socially, with visible markers.
\item
  This is interesting in the context of religious typing - and worth
  thinking about how religious groups have voluntarily or involuntarily
  adopted visible markers.
\end{itemize}
\end{frame}

\hypertarget{resumuxe9-studies}{%
\section{Resumé Studies}\label{resumuxe9-studies}}

\begin{frame}{Resumé Studies}
\protect\hypertarget{resumuxe9-studies-1}{}
\begin{itemize}
\tightlist
\item
  These are really fascinating, and worth looking up.
\item
  You can find some of them at this UM site:
  \url{https://advance.umich.edu/stride-readings/}
\item
  Do be careful about dates.
\item
  Obviously racism/sexism have not gone away in the last 40 years.
\item
  But they have changed some, and results from 40 years ago might not
  replicate now.
\end{itemize}
\end{frame}

\hypertarget{handfield-model}{%
\section{Handfield Model}\label{handfield-model}}

\begin{frame}{Handfield Model}
\protect\hypertarget{handfield-model-1}{}
This is the model discussed on page 60 of the book, and we spent a bit
of time on it last time.

\begin{itemize}[<+->]
\tightlist
\item
  O'Connor thinks it puts too much weight on rational choice. We'll come
  back to her alternative to rational choice models in a bit.
\item
  She also thinks it can't explain the stability of gender roles. I'm
  not really sure why that is true.
\item
  I'm worried that it requires 100\% pairing; even with a 90\%
  likelihood of pairing, you'd expect to see non-trivial investment in
  non-normative skills, as basically insurance. But we often didn't see
  even that.
\end{itemize}
\end{frame}

\hypertarget{basins-of-attraction}{%
\section{Basins of Attraction}\label{basins-of-attraction}}

\begin{frame}{Basins of Attraction}
\protect\hypertarget{basins-of-attraction-1}{}
\begin{itemize}
\tightlist
\item
  These are going to be significant, and I encourage you to ask about
  them if you're not following.
\item
  Here's one thing about them that threw me at first.
\item
  As O'Connor is using them, these are population level models.
\item
  When there is an equilibrium point that is 70\% A/30\% not-A (or
  whatever), that doesn't mean each player adopts the mixed strategy 0.7
  A, 0.3 not A.
\item
  Rather, it means 70\% of the population do A, and 30\% do not-A.
\item
  That doesn't amount to much mathematically, but it matters for how we
  interpret the model.
\end{itemize}
\end{frame}

\hypertarget{two-types-of-game-theory}{%
\section{Two Types of Game Theory}\label{two-types-of-game-theory}}

\begin{frame}{Two Types}
\protect\hypertarget{two-types}{}
\begin{itemize}
\tightlist
\item
  (Rational Choice) Game Theory
\item
  Evolutionary Game Theory
\end{itemize}
\end{frame}

\begin{frame}{Sociological Question}
\protect\hypertarget{sociological-question}{}
A lot of economists believe all of the following things.

\begin{enumerate}
\tightlist
\item
  Game theory is useful in economic modelling.
\item
  Economic actors for the most part (more or less) act rationally.
\item
  Economic actors that don't act rationally tend to become economically
  insignificant.
\item
  Points 2 and 3 complement each other; failures of rationality will
  become less significant because they are made by people/firms who will
  become less significant.
\end{enumerate}

I think that picture (which I'm sympathetic to!) looks much stronger if
you don't distinguish (rational choice) game theory from evolutionary
game theory.
\end{frame}

\begin{frame}{The Differences}
\protect\hypertarget{the-differences}{}
(Rational choice) game theory

\begin{itemize}
\tightlist
\item
  High rationality assumptions
\item
  Comparative statics method
\end{itemize}

Evolutionary game theory

\begin{itemize}
\tightlist
\item
  Low rationality assumptions
\item
  Dynamic method
\end{itemize}
\end{frame}

\begin{frame}{Big Picture Worries}
\protect\hypertarget{big-picture-worries}{}
\begin{itemize}[<+->]
\tightlist
\item
  The dynamism of evolutionary views is just good. Who cares if a
  position is stable if it could never be reached?
\item
  And not requiring full rationality is good too.
\item
  But \ldots{} requiring not full rationality is a bit iffy I think.
\end{itemize}
\end{frame}

\begin{frame}{Two Approaches O'Connor Takes}
\protect\hypertarget{two-approaches-oconnor-takes}{}
\begin{enumerate}[<+->]
\tightlist
\item
  Behavior acquisition is completely arational; it's just copying the
  successful. That makes more sense evolutionarily than behaviorally.
  Sure we copy somewhat, but is that all we do?
\item
  ``Bounded rationality'' approaches, where people do the best they can
  assuming that the population structure they've observed in the
  (immediate) past is the population structure of the present.
\end{enumerate}
\end{frame}

\begin{frame}{Back to Rationality?}
\protect\hypertarget{back-to-rationality}{}
\begin{itemize}
\tightlist
\item
  I'm not sure if there is a fully rational dynamical model we ever get.
\item
  But we do get models where these arational/irrational dynamics get to
  an end state that is rationally stable.
\item
  Is that rationality enough?
\end{itemize}
\end{frame}

\end{document}
