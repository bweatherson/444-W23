% Options for packages loaded elsewhere
\PassOptionsToPackage{unicode}{hyperref}
\PassOptionsToPackage{hyphens}{url}
%
\documentclass[
  ignorenonframetext,
]{beamer}
\usepackage{pgfpages}
\setbeamertemplate{caption}[numbered]
\setbeamertemplate{caption label separator}{: }
\setbeamercolor{caption name}{fg=normal text.fg}
\beamertemplatenavigationsymbolsempty
% Prevent slide breaks in the middle of a paragraph
\widowpenalties 1 10000
\raggedbottom
\setbeamertemplate{part page}{
  \centering
  \begin{beamercolorbox}[sep=16pt,center]{part title}
    \usebeamerfont{part title}\insertpart\par
  \end{beamercolorbox}
}
\setbeamertemplate{section page}{
  \centering
  \begin{beamercolorbox}[sep=12pt,center]{part title}
    \usebeamerfont{section title}\insertsection\par
  \end{beamercolorbox}
}
\setbeamertemplate{subsection page}{
  \centering
  \begin{beamercolorbox}[sep=8pt,center]{part title}
    \usebeamerfont{subsection title}\insertsubsection\par
  \end{beamercolorbox}
}
\AtBeginPart{
  \frame{\partpage}
}
\AtBeginSection{
  \ifbibliography
  \else
    \frame{\sectionpage}
  \fi
}
\AtBeginSubsection{
  \frame{\subsectionpage}
}
\usepackage{amsmath,amssymb}
\usepackage{lmodern}
\usepackage{ifxetex,ifluatex}
\ifnum 0\ifxetex 1\fi\ifluatex 1\fi=0 % if pdftex
  \usepackage[T1]{fontenc}
  \usepackage[utf8]{inputenc}
  \usepackage{textcomp} % provide euro and other symbols
\else % if luatex or xetex
  \usepackage{unicode-math}
  \defaultfontfeatures{Scale=MatchLowercase}
  \defaultfontfeatures[\rmfamily]{Ligatures=TeX,Scale=1}
  \setmainfont[BoldFont = SF Pro Rounded Semibold]{SF Pro Rounded}
  \setmathfont[]{STIX Two Math}
\fi
\usefonttheme{serif} % use mainfont rather than sansfont for slide text
% Use upquote if available, for straight quotes in verbatim environments
\IfFileExists{upquote.sty}{\usepackage{upquote}}{}
\IfFileExists{microtype.sty}{% use microtype if available
  \usepackage[]{microtype}
  \UseMicrotypeSet[protrusion]{basicmath} % disable protrusion for tt fonts
}{}
\makeatletter
\@ifundefined{KOMAClassName}{% if non-KOMA class
  \IfFileExists{parskip.sty}{%
    \usepackage{parskip}
  }{% else
    \setlength{\parindent}{0pt}
    \setlength{\parskip}{6pt plus 2pt minus 1pt}}
}{% if KOMA class
  \KOMAoptions{parskip=half}}
\makeatother
\usepackage{xcolor}
\IfFileExists{xurl.sty}{\usepackage{xurl}}{} % add URL line breaks if available
\IfFileExists{bookmark.sty}{\usepackage{bookmark}}{\usepackage{hyperref}}
\hypersetup{
  pdftitle={444 Lecture 4.2 - Structure of Information Sets},
  pdfauthor={Brian Weatherson},
  hidelinks,
  pdfcreator={LaTeX via pandoc}}
\urlstyle{same} % disable monospaced font for URLs
\newif\ifbibliography
\setlength{\emergencystretch}{3em} % prevent overfull lines
\providecommand{\tightlist}{%
  \setlength{\itemsep}{0pt}\setlength{\parskip}{0pt}}
\setcounter{secnumdepth}{-\maxdimen} % remove section numbering
\let\Tiny=\tiny

 \setbeamertemplate{navigation symbols}{} 

% \usetheme{Madrid}
 \usetheme[numbering=none, progressbar=foot]{metropolis}
 \usecolortheme{wolverine}
 \usepackage{color}
 \usepackage{MnSymbol}
% \usepackage{movie15}

\usepackage{amssymb}% http://ctan.org/pkg/amssymb
\usepackage{pifont}% http://ctan.org/pkg/pifont
\newcommand{\cmark}{\ding{51}}%
\newcommand{\xmark}{\ding{55}}%

\DeclareSymbolFont{symbolsC}{U}{txsyc}{m}{n}
\DeclareMathSymbol{\boxright}{\mathrel}{symbolsC}{128}
\DeclareMathAlphabet{\mathpzc}{OT1}{pzc}{m}{it}

\setlength{\parskip}{1ex plus 0.5ex minus 0.2ex}

\AtBeginSection[]
{
\begin{frame}
	\Huge{\color{darkblue} \insertsection}
\end{frame}
}

\renewenvironment*{quote}	
	{\list{}{\rightmargin   \leftmargin} \item } 	
	{\endlist }

\definecolor{darkgreen}{rgb}{0,0.7,0}
\definecolor{darkblue}{rgb}{0,0,0.8}

\usepackage[italic]{mathastext}
\usepackage{nicefrac}

\setbeamertemplate{caption}{\raggedright\insertcaption}

%\def\toprule{}
%\def\bottomrule{}
%\def\midrule{}
\usepackage{etoolbox}
\AfterEndEnvironment{description}{\vspace{9pt}}
\AfterEndEnvironment{oltableau}{\vspace{9pt}}
\BeforeBeginEnvironment{oltableau}{\vspace{9pt}}
\AfterEndEnvironment{center}{\vspace{9pt}}
\BeforeBeginEnvironment{tabular}{\vspace{9pt}}
\AfterEndEnvironment{longtable}{\vspace{-6pt}}
\usepackage{booktabs}
\usepackage{longtable}
\usepackage{array}
\usepackage{multirow}
\usepackage{wrapfig}
\usepackage{float}
\usepackage{colortbl}
\usepackage{pdflscape}
\usepackage{tabu}
\usepackage{threeparttable} 
\usepackage{threeparttablex} 
\usepackage[normalem]{ulem} 
\usepackage{makecell}
\usepackage{xcolor}
\usepackage{ulem}

\setlength\heavyrulewidth{0ex}
\setlength\lightrulewidth{0.08ex}

\aboverulesep=0ex
\belowrulesep=0ex
\renewcommand{\arraystretch}{1.2}
\ifluatex
  \usepackage{selnolig}  % disable illegal ligatures
\fi

\title{444 Lecture 4.2 - Structure of Information Sets}
\author{Brian Weatherson}
\date{}

\begin{document}
\frame{\titlepage}

\begin{frame}{Plan}
\protect\hypertarget{plan}{}
To discuss some presuppositions of the theory of information sets.
\end{frame}

\begin{frame}{Reading}
\protect\hypertarget{reading}{}
This isn't in the books; it's not something game theorists discuss.
\end{frame}

\begin{frame}{Three Features of Information Sets}
\protect\hypertarget{three-features-of-information-sets}{}
\begin{enumerate}
\tightlist
\item
  Reflexive
\item
  Symmetric
\item
  Transitive
\end{enumerate}
\end{frame}

\begin{frame}{Reflexive}
\protect\hypertarget{reflexive}{}
Each point is in its own information set.

\begin{itemize}
\tightlist
\item
  This seems fair enough; if you're somewhere, then for all you know,
  you are there.
\end{itemize}
\end{frame}

\begin{frame}{Symmetric}
\protect\hypertarget{symmetric}{}
\begin{itemize}
\tightlist
\item
  If when you're at \(a\) you might be at \(b\), then
\item
  When you're at \(b\) you might be at \(a\).
\item
  In information set terms, if \(b\) is in \(a\)'s information set, then
  \(a\) is in \(b\)'s information set.
\end{itemize}
\end{frame}

\begin{frame}{Symmetry}
\protect\hypertarget{symmetry}{}
What this means is that what happens earlier in the game can't affect a
player's powers of discrimination.

\begin{itemize}
\tightlist
\item
  This seems like an inappropriate assumption in, e.g., drinking games.
\item
  At least in some games, one's ability to discriminate between some
  options will be dependent on the path taken to get to those options.
\item
  The standard treatment of partial information doesn't allow us to
  represent this.
\end{itemize}
\end{frame}

\begin{frame}{Transitive}
\protect\hypertarget{transitive}{}
\begin{itemize}
\tightlist
\item
  If when you're at \(a\) you might be at \(b\), and
\item
  When you're at \(b\) you might be at \(c\), then
\item
  When you're at \(a\) you might be at \(c\).
\item
  In information set terms, if \(b\) is in \(a\)'s information set, and
  \(c\) is in \(b\)'s information set, then it must be that \(c\) is in
  \(a\)'s information set.
\end{itemize}
\end{frame}

\begin{frame}{Transitivity}
\protect\hypertarget{transitivity}{}
This rules out games where players can tell that they are in a certain
`neighbourhood'. For example,

\begin{itemize}
\tightlist
\item
  Player 1 puts some jelly beans in a jar and gives it to Player 2.
\item
  It matters to Player 2 how many jelly beans are in the jar, but she
  doesn't have direct access to that.
\item
  Still, she isn't totally ignorant. She can see the jar and guess the
  number to the nearest, say 10.
\item
  So if there are 160 jelly beans in the jar, she knows that there are
  between 150 and 170.
\item
  And in fact, though Player 2 doesn't know this, that's true - there
  are 160 jelly beans.
\end{itemize}
\end{frame}

\begin{frame}{Transitivity}
\protect\hypertarget{transitivity-1}{}
\begin{itemize}
\tightlist
\item
  Let \(a\) be the node where there are 160 jelly beans in the jar.
\item
  And \(b\) be the node where there are 150 jelly beans in the jar.
\item
  And \(c\) be the node where there are 140 jelly beans in the jar.
\item
  Player 2 knows she's not at \(c\); her information set should exclude
  that.
\item
  And her information set should include \(a\) - reflexivity guarantees
  that.
\item
  Should it include \(b\)?
\end{itemize}
\end{frame}

\begin{frame}{A Challenge}
\protect\hypertarget{a-challenge}{}
\begin{itemize}
\tightlist
\item
  On the one hand, it should - since from \(a\) all she knows is that
  there are between 150 and 170 jelly beans in the jar.
\item
  On the other hand, it should not - since she can rule out \(c\), and
  if \(b\) were actual, she could not rule out \(c\).
\end{itemize}
\end{frame}

\begin{frame}{What To Do}
\protect\hypertarget{what-to-do}{}
\begin{itemize}
\tightlist
\item
  In theory, there is an opening here for someone working out a theory
  of games with imperfect information that drops either the symmetry or
  transitivity assumption.
\item
  In practice, no one has actually worked out that theory, and I'm not
  going to try teaching a non-existent theory.
\item
  There is a little work on games involving ``unawareness'', which gets
  close to this, but it's way too novel a field to know where it will
  go.
\end{itemize}
\end{frame}

\begin{frame}{For Next Time}
\protect\hypertarget{for-next-time}{}
\begin{itemize}
\tightlist
\item
  We will return to orthodoxy, and look at the notion of a strategy.
\end{itemize}
\end{frame}

\end{document}
