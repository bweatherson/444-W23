% Options for packages loaded elsewhere
\PassOptionsToPackage{unicode}{hyperref}
\PassOptionsToPackage{hyphens}{url}
%
\documentclass[
  ignorenonframetext,
]{beamer}
\usepackage{pgfpages}
\setbeamertemplate{caption}[numbered]
\setbeamertemplate{caption label separator}{: }
\setbeamercolor{caption name}{fg=normal text.fg}
\beamertemplatenavigationsymbolsempty
% Prevent slide breaks in the middle of a paragraph
\widowpenalties 1 10000
\raggedbottom
\setbeamertemplate{part page}{
  \centering
  \begin{beamercolorbox}[sep=16pt,center]{part title}
    \usebeamerfont{part title}\insertpart\par
  \end{beamercolorbox}
}
\setbeamertemplate{section page}{
  \centering
  \begin{beamercolorbox}[sep=12pt,center]{part title}
    \usebeamerfont{section title}\insertsection\par
  \end{beamercolorbox}
}
\setbeamertemplate{subsection page}{
  \centering
  \begin{beamercolorbox}[sep=8pt,center]{part title}
    \usebeamerfont{subsection title}\insertsubsection\par
  \end{beamercolorbox}
}
\AtBeginPart{
  \frame{\partpage}
}
\AtBeginSection{
  \ifbibliography
  \else
    \frame{\sectionpage}
  \fi
}
\AtBeginSubsection{
  \frame{\subsectionpage}
}
\usepackage{amsmath,amssymb}
\usepackage{lmodern}
\usepackage{ifxetex,ifluatex}
\ifnum 0\ifxetex 1\fi\ifluatex 1\fi=0 % if pdftex
  \usepackage[T1]{fontenc}
  \usepackage[utf8]{inputenc}
  \usepackage{textcomp} % provide euro and other symbols
\else % if luatex or xetex
  \usepackage{unicode-math}
  \defaultfontfeatures{Scale=MatchLowercase}
  \defaultfontfeatures[\rmfamily]{Ligatures=TeX,Scale=1}
  \setmainfont[BoldFont = SF Pro Rounded Semibold]{SF Pro Rounded}
  \setmathfont[]{STIX Two Math}
\fi
\usefonttheme{serif} % use mainfont rather than sansfont for slide text
% Use upquote if available, for straight quotes in verbatim environments
\IfFileExists{upquote.sty}{\usepackage{upquote}}{}
\IfFileExists{microtype.sty}{% use microtype if available
  \usepackage[]{microtype}
  \UseMicrotypeSet[protrusion]{basicmath} % disable protrusion for tt fonts
}{}
\makeatletter
\@ifundefined{KOMAClassName}{% if non-KOMA class
  \IfFileExists{parskip.sty}{%
    \usepackage{parskip}
  }{% else
    \setlength{\parindent}{0pt}
    \setlength{\parskip}{6pt plus 2pt minus 1pt}}
}{% if KOMA class
  \KOMAoptions{parskip=half}}
\makeatother
\usepackage{xcolor}
\IfFileExists{xurl.sty}{\usepackage{xurl}}{} % add URL line breaks if available
\IfFileExists{bookmark.sty}{\usepackage{bookmark}}{\usepackage{hyperref}}
\hypersetup{
  pdftitle={444 Lecture 2.3 - Ordinal and Cardinal Utility},
  pdfauthor={Brian Weatherson},
  hidelinks,
  pdfcreator={LaTeX via pandoc}}
\urlstyle{same} % disable monospaced font for URLs
\newif\ifbibliography
\usepackage{longtable,booktabs,array}
\usepackage{calc} % for calculating minipage widths
\usepackage{caption}
% Make caption package work with longtable
\makeatletter
\def\fnum@table{\tablename~\thetable}
\makeatother
\setlength{\emergencystretch}{3em} % prevent overfull lines
\providecommand{\tightlist}{%
  \setlength{\itemsep}{0pt}\setlength{\parskip}{0pt}}
\setcounter{secnumdepth}{-\maxdimen} % remove section numbering
\let\Tiny=\tiny

 \setbeamertemplate{navigation symbols}{} 

% \usetheme{Madrid}
 \usetheme[numbering=none, progressbar=foot]{metropolis}
 \usecolortheme{wolverine}
 \usepackage{color}
 \usepackage{MnSymbol}
% \usepackage{movie15}

\usepackage{amssymb}% http://ctan.org/pkg/amssymb
\usepackage{pifont}% http://ctan.org/pkg/pifont
\newcommand{\cmark}{\ding{51}}%
\newcommand{\xmark}{\ding{55}}%

\DeclareSymbolFont{symbolsC}{U}{txsyc}{m}{n}
\DeclareMathSymbol{\boxright}{\mathrel}{symbolsC}{128}
\DeclareMathAlphabet{\mathpzc}{OT1}{pzc}{m}{it}

\setlength{\parskip}{1ex plus 0.5ex minus 0.2ex}

\AtBeginSection[]
{
\begin{frame}
	\Huge{\color{darkblue} \insertsection}
\end{frame}
}

\renewenvironment*{quote}	
	{\list{}{\rightmargin   \leftmargin} \item } 	
	{\endlist }

\definecolor{darkgreen}{rgb}{0,0.7,0}
\definecolor{darkblue}{rgb}{0,0,0.8}

\usepackage[italic]{mathastext}
\usepackage{nicefrac}

\setbeamertemplate{caption}{\raggedright\insertcaption}

%\def\toprule{}
%\def\bottomrule{}
%\def\midrule{}
\usepackage{etoolbox}
\AfterEndEnvironment{description}{\vspace{9pt}}
\AfterEndEnvironment{oltableau}{\vspace{9pt}}
\BeforeBeginEnvironment{oltableau}{\vspace{9pt}}
\AfterEndEnvironment{center}{\vspace{9pt}}
\BeforeBeginEnvironment{tabular}{\vspace{9pt}}
\AfterEndEnvironment{longtable}{\vspace{-6pt}}
\ifluatex
  \usepackage{selnolig}  % disable illegal ligatures
\fi

\title{444 Lecture 2.3 - Ordinal and Cardinal Utility}
\author{Brian Weatherson}
\date{}

\begin{document}
\frame{\titlepage}

\begin{frame}{Plan}
\protect\hypertarget{plan}{}
\begin{itemize}
\tightlist
\item
  Explain the difference between ordinal and cardinal utility.
\end{itemize}
\end{frame}

\begin{frame}{Associated Reading}
\protect\hypertarget{associated-reading}{}
Bonanno, section 2.1.
\end{frame}

\begin{frame}{Utility}
\protect\hypertarget{utility}{}
A utility function (for a particular agent) is a mapping \(U\) from
situations to numbers satsifying this constraint.

\begin{itemize}
\tightlist
\item
  \(U(S_1) > U(S_2)\) iff the agent is better off in \(S_1\) than in
  \(S_2\).
\end{itemize}
\end{frame}

\begin{frame}{Welfare}
\protect\hypertarget{welfare}{}
This isn't part of the formal theory, but we usually implicitly assume
(at least in our narratives), the following principle.

\begin{quote}
The agent is better off in \(S_1\) than in \(S_2\) iff, given a choice
and assuming they are fully informed, they prefer being in \(S_1\) to
\(S_2\).
\end{quote}

That is, we'll usually speak as if a radically subjectivist view of
welfare is correct. I've been doing this already, and I'm going to keep
doing it.
\end{frame}

\begin{frame}{Ordinal Utility}
\protect\hypertarget{ordinal-utility}{}
\begin{itemize}
\tightlist
\item
  When we say that we're working with \textbf{ordinal} utility
  functions, really the only principle that applies is the one from two
  slides back.
\item
  Higher utilities are better, i.e., are preferred.
\item
  The term \textbf{ordinal} should make you think of `orders'; all an
  ordinal utility function does is provide a rank \textbf{ordering} of
  the outcomes.
\end{itemize}
\end{frame}

\begin{frame}{Two Functions}
\protect\hypertarget{two-functions}{}
So if we're working in ordinal utility, these two functions describe the
same underlying reality.

\begin{longtable}[]{@{}lcc@{}}
\toprule
& \(U_1\) & \(U_2\) \\ \addlinespace
\midrule
\endhead
\(O_1\) & 1 & 1 \\ \addlinespace
\(O_2\) & 2 & 10 \\ \addlinespace
\(O_3\) & 3 & 500 \\ \addlinespace
\(O_4\) & 4 & 7329 \\ \addlinespace
\bottomrule
\end{longtable}
\end{frame}

\begin{frame}{Cardinal Utility}
\protect\hypertarget{cardinal-utility}{}
\begin{itemize}
\tightlist
\item
  In cardinal utility theory, the differences between the numbers
  matter.
\item
  The numbers now express quantities, and the two functions from the
  previous slide do not represent the same underlying reality.
\end{itemize}
\end{frame}

\begin{frame}{Cardinal Utility (Detail)}
\protect\hypertarget{cardinal-utility-detail}{}
\begin{itemize}
\tightlist
\item
  There is a fussy point here that's worth going over.
\item
  Even cardinal utility functions don't come with a scale.
\item
  So two functions with different numbers in them can still express the
  same underlying reality.
\end{itemize}
\end{frame}

\begin{frame}{Cardinal Utility (Detail)}
\protect\hypertarget{cardinal-utility-detail-1}{}
The standard way to put this is that (cardinal) utility is defined only
up to a \textbf{positive, affine transformation}. That means that if
\(U_1\) and \(U_2\) are related by the following formula, then they
represent the same state of affairs.

\[
U_2(o) = aU_1(o) + b \text{ where } a > 0
\]
\end{frame}

\begin{frame}{Celsius and Farenheit}
\protect\hypertarget{celsius-and-farenheit}{}
\begin{itemize}
\tightlist
\item
  The main real world cases of scales that are related in this way are
  temperature scales.
\item
  To convert between Celsius and Farenheit you use the formula
  \(F = 1.8C + 32\).
\item
  But the scales are just two ways of representing the same physical
  reality.
\end{itemize}
\end{frame}

\begin{frame}{Cardinal Utility (Detail)}
\protect\hypertarget{cardinal-utility-detail-2}{}
\begin{itemize}
\tightlist
\item
  So there is no such thing as one outcome being \emph{twice as good} as
  another.
\item
  But we can say a lot of things about differences.
\end{itemize}
\end{frame}

\begin{frame}{Cardinal Utility (Detail)}
\protect\hypertarget{cardinal-utility-detail-3}{}
\begin{itemize}[<+->]
\tightlist
\item
  If the difference between \(O_1\) and \(O_2\) is the same as the
  difference between \(O_2\) and \(O_3\), that will stay the same under
  any positive affine transformation.
\item
  Indeed, for any \(k\), if the difference between \(O_1\) and \(O_2\)
  is \(k\) times the difference between \(O_2\) and \(O_3\), that will
  stay the same under any positive affine transformation.
\end{itemize}
\end{frame}

\begin{frame}{For Next Time}
\protect\hypertarget{for-next-time}{}
\begin{itemize}
\tightlist
\item
  We will start on section 2.2.
\end{itemize}
\end{frame}

\end{document}
