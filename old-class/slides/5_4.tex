% Options for packages loaded elsewhere
\PassOptionsToPackage{unicode}{hyperref}
\PassOptionsToPackage{hyphens}{url}
%
\documentclass[
  ignorenonframetext,
]{beamer}
\usepackage{pgfpages}
\setbeamertemplate{caption}[numbered]
\setbeamertemplate{caption label separator}{: }
\setbeamercolor{caption name}{fg=normal text.fg}
\beamertemplatenavigationsymbolsempty
% Prevent slide breaks in the middle of a paragraph
\widowpenalties 1 10000
\raggedbottom
\setbeamertemplate{part page}{
  \centering
  \begin{beamercolorbox}[sep=16pt,center]{part title}
    \usebeamerfont{part title}\insertpart\par
  \end{beamercolorbox}
}
\setbeamertemplate{section page}{
  \centering
  \begin{beamercolorbox}[sep=12pt,center]{part title}
    \usebeamerfont{section title}\insertsection\par
  \end{beamercolorbox}
}
\setbeamertemplate{subsection page}{
  \centering
  \begin{beamercolorbox}[sep=8pt,center]{part title}
    \usebeamerfont{subsection title}\insertsubsection\par
  \end{beamercolorbox}
}
\AtBeginPart{
  \frame{\partpage}
}
\AtBeginSection{
  \ifbibliography
  \else
    \frame{\sectionpage}
  \fi
}
\AtBeginSubsection{
  \frame{\subsectionpage}
}
\usepackage{amsmath,amssymb}
\usepackage{lmodern}
\usepackage{ifxetex,ifluatex}
\ifnum 0\ifxetex 1\fi\ifluatex 1\fi=0 % if pdftex
  \usepackage[T1]{fontenc}
  \usepackage[utf8]{inputenc}
  \usepackage{textcomp} % provide euro and other symbols
\else % if luatex or xetex
  \usepackage{unicode-math}
  \defaultfontfeatures{Scale=MatchLowercase}
  \defaultfontfeatures[\rmfamily]{Ligatures=TeX,Scale=1}
  \setmainfont[BoldFont = SF Pro Rounded Semibold]{SF Pro Rounded}
  \setmathfont[]{STIX Two Math}
\fi
\usefonttheme{serif} % use mainfont rather than sansfont for slide text
% Use upquote if available, for straight quotes in verbatim environments
\IfFileExists{upquote.sty}{\usepackage{upquote}}{}
\IfFileExists{microtype.sty}{% use microtype if available
  \usepackage[]{microtype}
  \UseMicrotypeSet[protrusion]{basicmath} % disable protrusion for tt fonts
}{}
\makeatletter
\@ifundefined{KOMAClassName}{% if non-KOMA class
  \IfFileExists{parskip.sty}{%
    \usepackage{parskip}
  }{% else
    \setlength{\parindent}{0pt}
    \setlength{\parskip}{6pt plus 2pt minus 1pt}}
}{% if KOMA class
  \KOMAoptions{parskip=half}}
\makeatother
\usepackage{xcolor}
\IfFileExists{xurl.sty}{\usepackage{xurl}}{} % add URL line breaks if available
\IfFileExists{bookmark.sty}{\usepackage{bookmark}}{\usepackage{hyperref}}
\hypersetup{
  pdftitle={444 Lecture 5.4 - Finding Mixed Strategy Equilibria},
  pdfauthor={Brian Weatherson},
  hidelinks,
  pdfcreator={LaTeX via pandoc}}
\urlstyle{same} % disable monospaced font for URLs
\newif\ifbibliography
\setlength{\emergencystretch}{3em} % prevent overfull lines
\providecommand{\tightlist}{%
  \setlength{\itemsep}{0pt}\setlength{\parskip}{0pt}}
\setcounter{secnumdepth}{-\maxdimen} % remove section numbering
\let\Tiny=\tiny

 \setbeamertemplate{navigation symbols}{} 

% \usetheme{Madrid}
 \usetheme[numbering=none, progressbar=foot]{metropolis}
 \usecolortheme{wolverine}
 \usepackage{color}
 \usepackage{MnSymbol}
% \usepackage{movie15}

\usepackage{amssymb}% http://ctan.org/pkg/amssymb
\usepackage{pifont}% http://ctan.org/pkg/pifont
\newcommand{\cmark}{\ding{51}}%
\newcommand{\xmark}{\ding{55}}%

\DeclareSymbolFont{symbolsC}{U}{txsyc}{m}{n}
\DeclareMathSymbol{\boxright}{\mathrel}{symbolsC}{128}
\DeclareMathAlphabet{\mathpzc}{OT1}{pzc}{m}{it}

\setlength{\parskip}{1ex plus 0.5ex minus 0.2ex}

\AtBeginSection[]
{
\begin{frame}
	\Huge{\color{darkblue} \insertsection}
\end{frame}
}

\renewenvironment*{quote}	
	{\list{}{\rightmargin   \leftmargin} \item } 	
	{\endlist }

\definecolor{darkgreen}{rgb}{0,0.7,0}
\definecolor{darkblue}{rgb}{0,0,0.8}

\usepackage[italic]{mathastext}
\usepackage{nicefrac}

\setbeamertemplate{caption}{\raggedright\insertcaption}

%\def\toprule{}
%\def\bottomrule{}
%\def\midrule{}
\usepackage{etoolbox}
\AfterEndEnvironment{description}{\vspace{9pt}}
\AfterEndEnvironment{oltableau}{\vspace{9pt}}
\BeforeBeginEnvironment{oltableau}{\vspace{9pt}}
\AfterEndEnvironment{center}{\vspace{9pt}}
\BeforeBeginEnvironment{tabular}{\vspace{9pt}}
\AfterEndEnvironment{longtable}{\vspace{-6pt}}
\usepackage{booktabs}
\usepackage{longtable}
\usepackage{array}
\usepackage{multirow}
\usepackage{wrapfig}
\usepackage{float}
\usepackage{colortbl}
\usepackage{pdflscape}
\usepackage{tabu}
\usepackage{threeparttable} 
\usepackage{threeparttablex} 
\usepackage[normalem]{ulem} 
\usepackage{makecell}
\usepackage{xcolor}
\usepackage{ulem}

\setlength\heavyrulewidth{0ex}
\setlength\lightrulewidth{0.08ex}

\aboverulesep=0ex
\belowrulesep=0ex
\renewcommand{\arraystretch}{1.2}
\ifluatex
  \usepackage{selnolig}  % disable illegal ligatures
\fi

\title{444 Lecture 5.4 - Finding Mixed Strategy Equilibria}
\author{Brian Weatherson}
\date{}

\begin{document}
\frame{\titlepage}

\begin{frame}{Plan}
\protect\hypertarget{plan}{}
Discuss how we can find mixed strategy equilibria.
\end{frame}

\begin{frame}{Reading}
\protect\hypertarget{reading}{}
Bonanno, Section 6.3.
\end{frame}

\begin{frame}{Basic Idea}
\protect\hypertarget{basic-idea}{}
\begin{itemize}
\tightlist
\item
  In equilibria, the other player is willing to play a mixed strategy.
\item
  That requires that they be indifferent between other strategies.
\item
  So we find the equilibria by finding the mixture that makes them
  indifferent.
\end{itemize}
\end{frame}

\begin{frame}{Example}
\protect\hypertarget{example}{}
\begin{table}[!h]
\centering
\begin{tabular}[t]{>{}r|cc}
\toprule
 & Left & Right\\
\midrule
Up & 5, 2 & 1, 3\\
Down & 2, 2 & 3, 0\\
\bottomrule
\end{tabular}
\end{table}

\begin{itemize}
\tightlist
\item
  You can see fairly quicklythat there is no pure strategy equilibria.
\item
  So the equilibria must be a mixed strategy equilibria.
\end{itemize}
\end{frame}

\begin{frame}{Theory}
\protect\hypertarget{theory}{}
What does it take for Column to play a mixed strategy in equilibria?

\begin{itemize}
\tightlist
\item
  Assume that Left has a higher expected return than Right.
\item
  The expected return of a mixed strategy is a weighted average of the
  expected returns of Left and Right.
\item
  If Left has a higher expected return than Right, that weighted average
  will be strictly between the expected returns of Left and Right.
\item
  And that means it can't be an equilibrium, since in equilibrium there
  is no alternative with a higher expected return. \pause
\item
  And the same reasoning shows Right can't have a higher expected return
  than Left.
\end{itemize}
\end{frame}

\begin{frame}{Theory}
\protect\hypertarget{theory-1}{}
So we are trying to find the mixture such that Column is indifferent
between Left and Right.

\begin{itemize}
\tightlist
\item
  The other crucial thing to remember is that probabilities add to 1.
\item
  So when working out Row's strategy, there is only one variable.
\item
  Once we set the probability of Row playing Up to \(x\), that sets all
  the probabilities, because the probablity of playing Down is \(1-x\).
\end{itemize}
\end{frame}

\begin{frame}{A Note}
\protect\hypertarget{a-note}{}
I'm only going to go over cases where the mixed strategy equilibrium
involves a mixture of two pure strategies.

\begin{itemize}
\tightlist
\item
  There are cases where the mixed strategy equilibrium involves mixtures
  of 3 or more pure strategies.
\item
  Rock, Paper, Scissors is the simplest such example.
\item
  But in general the math of calculating these is considerably fancier
  than what we'll be doing, and I'll stick to cases where the mixed
  strategy equilibrium only involves 2 pure strategies.
\end{itemize}
\end{frame}

\begin{frame}{Back to the Example}
\protect\hypertarget{back-to-the-example}{}
\begin{table}[!h]
\centering
\begin{tabular}[t]{>{}r|cc}
\toprule
 & Left & Right\\
\midrule
Up & 5, 2 & 1, 3\\
Down & 2, 2 & 3, 0\\
\bottomrule
\end{tabular}
\end{table}

\begin{itemize}
\tightlist
\item
  Assume Row plays Up with probability \(x\), and Down with probability
  \(1-x\).
\item
  Our job is to find an \(x\) such that the expected return of Left and
  Right is the same.
\end{itemize}
\end{frame}

\begin{frame}{Left and Right}
\protect\hypertarget{left-and-right}{}
\begin{table}[!h]
\centering
\begin{tabular}[t]{>{}r|cc}
\toprule
 & Left & Right\\
\midrule
Up & 5, 2 & 1, 3\\
Down & 2, 2 & 3, 0\\
\bottomrule
\end{tabular}
\end{table}

\begin{itemize}
\tightlist
\item
  The expected return of Left is \(2x + 2(1-x)\), i.e., \(2\).
\item
  The expected return of Right is \(3x + 0(1-x)\), i.e., \(3x\).
\end{itemize}
\end{frame}

\begin{frame}{Row's Strategy.}
\protect\hypertarget{rows-strategy.}{}
\begin{table}[!h]
\centering
\begin{tabular}[t]{>{}r|cc}
\toprule
 & Left & Right\\
\midrule
Up & 5, 2 & 1, 3\\
Down & 2, 2 & 3, 0\\
\bottomrule
\end{tabular}
\end{table}

\begin{itemize}
\tightlist
\item
  So \(2 = 3x\), so \(x = \nicefrac{2}{3}\).
\item
  So Row's strategy is to play Up with probability \(\nicefrac{2}{3}\),
  and hence Down with probability \(\nicefrac{1}{3}\).
\end{itemize}
\end{frame}

\begin{frame}{}
\protect\hypertarget{section}{}
\begin{itemize}
\tightlist
\item
  The expected return of Right is \(3x + 0(1-x)\), i.e., \(3x\).
\end{itemize}
\end{frame}

\begin{frame}{Onto Column}
\protect\hypertarget{onto-column}{}
\begin{table}[!h]
\centering
\begin{tabular}[t]{>{}r|cc}
\toprule
 & Left & Right\\
\midrule
Up & 5, 2 & 1, 3\\
Down & 2, 2 & 3, 0\\
\bottomrule
\end{tabular}
\end{table}

\begin{itemize}
\tightlist
\item
  Assume Column plays Left with probability \(x\), and Right with
  probability \(1-x\).
\item
  Our job is to find an \(x\) such that the expected return of Up and
  Down is the same.
\end{itemize}
\end{frame}

\begin{frame}{Left and Right}
\protect\hypertarget{left-and-right-1}{}
\begin{table}[!h]
\centering
\begin{tabular}[t]{>{}r|cc}
\toprule
 & Left & Right\\
\midrule
Up & 5, 2 & 1, 3\\
Down & 2, 2 & 3, 0\\
\bottomrule
\end{tabular}
\end{table}

\begin{itemize}
\tightlist
\item
  The expected return of Up is \(5x + 1(1-x)\), i.e., \(4x + 1\).
\item
  The expected return of Down is \(2x + 3(1-x)\), i.e., \(3 - x\).
\end{itemize}
\end{frame}

\begin{frame}{Column's Strategy.}
\protect\hypertarget{columns-strategy.}{}
\begin{align*}
4x + 1 &= 3 - x \\
5x + 1 &= 3 \\
5x    &= 2 \\
x    &= \nicefrac{2}{5}
\end{align*}

So Column's strategy is to play Left with probability
\(\nicefrac{2}{5}\), and hence Right with probability
\(\nicefrac{3}{5}\).
\end{frame}

\begin{frame}{Takeaways}
\protect\hypertarget{takeaways}{}
\begin{itemize}
\tightlist
\item
  To find a player's move probabilities in equilibria, look to the other
  player's payouts.
\item
  Try to make the other player indifferent between their choices.
\end{itemize}
\end{frame}

\begin{frame}{Extra Steps}
\protect\hypertarget{extra-steps}{}
\begin{itemize}
\tightlist
\item
  I'm not going to go over more complicated examples on the slides, but
  there is an extra step you can do (and which we can discuss in class
  if you're interested).
\item
  Sometimes you can find the mixed strategy equilibria of a game with
  more than 2 moves by first deleting \textbf{strongly} dominated
  strategies.
\item
  Bonanno works through an example like this.
\item
  I'm going to come back to it later, but for now I'll just stick to
  this example.
\end{itemize}
\end{frame}

\begin{frame}{For Next Time}
\protect\hypertarget{for-next-time}{}
I'm going to start on an idea I want to work through very slowly, and
spend a bit of time on - the idea that a mixture of strategies can
dominate another strategy.
\end{frame}

\end{document}
