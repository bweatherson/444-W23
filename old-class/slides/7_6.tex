% Options for packages loaded elsewhere
\PassOptionsToPackage{unicode}{hyperref}
\PassOptionsToPackage{hyphens}{url}
%
\documentclass[
  ignorenonframetext,
]{beamer}
\usepackage{pgfpages}
\setbeamertemplate{caption}[numbered]
\setbeamertemplate{caption label separator}{: }
\setbeamercolor{caption name}{fg=normal text.fg}
\beamertemplatenavigationsymbolsempty
% Prevent slide breaks in the middle of a paragraph
\widowpenalties 1 10000
\raggedbottom
\setbeamertemplate{part page}{
  \centering
  \begin{beamercolorbox}[sep=16pt,center]{part title}
    \usebeamerfont{part title}\insertpart\par
  \end{beamercolorbox}
}
\setbeamertemplate{section page}{
  \centering
  \begin{beamercolorbox}[sep=12pt,center]{part title}
    \usebeamerfont{section title}\insertsection\par
  \end{beamercolorbox}
}
\setbeamertemplate{subsection page}{
  \centering
  \begin{beamercolorbox}[sep=8pt,center]{part title}
    \usebeamerfont{subsection title}\insertsubsection\par
  \end{beamercolorbox}
}
\AtBeginPart{
  \frame{\partpage}
}
\AtBeginSection{
  \ifbibliography
  \else
    \frame{\sectionpage}
  \fi
}
\AtBeginSubsection{
  \frame{\subsectionpage}
}
\usepackage{amsmath,amssymb}
\usepackage{lmodern}
\usepackage{ifxetex,ifluatex}
\ifnum 0\ifxetex 1\fi\ifluatex 1\fi=0 % if pdftex
  \usepackage[T1]{fontenc}
  \usepackage[utf8]{inputenc}
  \usepackage{textcomp} % provide euro and other symbols
\else % if luatex or xetex
  \usepackage{unicode-math}
  \defaultfontfeatures{Scale=MatchLowercase}
  \defaultfontfeatures[\rmfamily]{Ligatures=TeX,Scale=1}
  \setmainfont[BoldFont = SF Pro Rounded Semibold]{SF Pro Rounded}
  \setmathfont[]{STIX Two Math}
\fi
\usefonttheme{serif} % use mainfont rather than sansfont for slide text
% Use upquote if available, for straight quotes in verbatim environments
\IfFileExists{upquote.sty}{\usepackage{upquote}}{}
\IfFileExists{microtype.sty}{% use microtype if available
  \usepackage[]{microtype}
  \UseMicrotypeSet[protrusion]{basicmath} % disable protrusion for tt fonts
}{}
\makeatletter
\@ifundefined{KOMAClassName}{% if non-KOMA class
  \IfFileExists{parskip.sty}{%
    \usepackage{parskip}
  }{% else
    \setlength{\parindent}{0pt}
    \setlength{\parskip}{6pt plus 2pt minus 1pt}}
}{% if KOMA class
  \KOMAoptions{parskip=half}}
\makeatother
\usepackage{xcolor}
\IfFileExists{xurl.sty}{\usepackage{xurl}}{} % add URL line breaks if available
\IfFileExists{bookmark.sty}{\usepackage{bookmark}}{\usepackage{hyperref}}
\hypersetup{
  pdftitle={444 Lecture 7.6 - Honest Signaling},
  pdfauthor={Brian Weatherson},
  hidelinks,
  pdfcreator={LaTeX via pandoc}}
\urlstyle{same} % disable monospaced font for URLs
\newif\ifbibliography
\setlength{\emergencystretch}{3em} % prevent overfull lines
\providecommand{\tightlist}{%
  \setlength{\itemsep}{0pt}\setlength{\parskip}{0pt}}
\setcounter{secnumdepth}{-\maxdimen} % remove section numbering
\let\Tiny=\tiny

 \setbeamertemplate{navigation symbols}{} 

% \usetheme{Madrid}
 \usetheme[numbering=none, progressbar=foot]{metropolis}
 \usecolortheme{wolverine}
 \usepackage{color}
 \usepackage{MnSymbol}
% \usepackage{movie15}

\usepackage{amssymb}% http://ctan.org/pkg/amssymb
\usepackage{pifont}% http://ctan.org/pkg/pifont
\newcommand{\cmark}{\ding{51}}%
\newcommand{\xmark}{\ding{55}}%

\DeclareSymbolFont{symbolsC}{U}{txsyc}{m}{n}
\DeclareMathSymbol{\boxright}{\mathrel}{symbolsC}{128}
\DeclareMathAlphabet{\mathpzc}{OT1}{pzc}{m}{it}

\setlength{\parskip}{1ex plus 0.5ex minus 0.2ex}

\AtBeginSection[]
{
\begin{frame}
	\Huge{\color{darkblue} \insertsection}
\end{frame}
}

\renewenvironment*{quote}	
	{\list{}{\rightmargin   \leftmargin} \item } 	
	{\endlist }

\definecolor{darkgreen}{rgb}{0,0.7,0}
\definecolor{darkblue}{rgb}{0,0,0.8}

\usepackage[italic]{mathastext}
\usepackage{nicefrac}
\usepackage{istgame}

\setbeamertemplate{caption}{\raggedright\insertcaption}

%\def\toprule{}
%\def\bottomrule{}
%\def\midrule{}
\usepackage{etoolbox}
\AfterEndEnvironment{description}{\vspace{9pt}}
\AfterEndEnvironment{oltableau}{\vspace{9pt}}
\BeforeBeginEnvironment{oltableau}{\vspace{9pt}}
\AfterEndEnvironment{center}{\vspace{9pt}}
\BeforeBeginEnvironment{tabular}{\vspace{9pt}}
\AfterEndEnvironment{longtable}{\vspace{-6pt}}
\usepackage{booktabs}
\usepackage{longtable}
\usepackage{array}
\usepackage{multirow}
\usepackage{wrapfig}
\usepackage{float}
\usepackage{colortbl}
\usepackage{pdflscape}
\usepackage{tabu}
\usepackage{threeparttable} 
\usepackage{threeparttablex} 
\usepackage[normalem]{ulem} 
\usepackage{makecell}
\usepackage{xcolor}
\usepackage{ulem}

\setlength\heavyrulewidth{0ex}
\setlength\lightrulewidth{0.08ex}

\aboverulesep=0ex
\belowrulesep=0ex
\renewcommand{\arraystretch}{1.2}
\ifluatex
  \usepackage{selnolig}  % disable illegal ligatures
\fi

\title{444 Lecture 7.6 - Honest Signaling}
\author{Brian Weatherson}
\date{}

\begin{document}
\frame{\titlepage}

\begin{frame}{Signaling by Showing}
\protect\hypertarget{signaling-by-showing}{}
\begin{itemize}
\tightlist
\item
  Change the game so that what options Sender has is a function of what
  type Sender is.
\item
  In the extreme case, one type of Sender has two options, the other has
  one.
\item
  In this case, Sender doing the thing that only their type can do is
  called \textbf{honest signaling} or \textbf{indexical signaling}.
\end{itemize}
\end{frame}

\begin{frame}{The Chase}
\protect\hypertarget{the-chase}{}
\begin{itemize}
\tightlist
\item
  Sender sees that Hearer is trying to catch them, and it will be bad if
  Hearer succeeds.
\item
  Maybe Hearer is a mugger, or maybe they are a cheetah and Sender is a
  springbok.
\item
  Sender is either Strong or Weak.
\item
  If they are Strong, they have the option of Jumping in the air before
  running away.
\item
  This will slow them down, but will display their type to Hearer.
\end{itemize}
\end{frame}

\begin{frame}{Payoffs}
\protect\hypertarget{payoffs}{}
\begin{itemize}
\tightlist
\item
  Sender loses 10 if they are chased and get caught.
\item
  Hearer gains 5 if they catch Sender; but they lose 3 if they chase and
  fail (this might be an opportunity cost).
\item
  Fast sender has a 20\% chance of being caught if they don't Jump, and
  a 30\% chance of being caught if they Jump.
\item
  Slow sender can't jump, and has a 50\% chance of being caught.
\end{itemize}
\end{frame}

\begin{frame}
\begin{center}
\begin{istgame}[scale=1.3]
   \xtdistance{20mm}{20mm}
   \istroot(0)[chance node]{$c$}
     \istb<grow=left>{0.4}[a]
     \istb<grow=right>{0.6}[a]
     \endist
   \xtdistance{10mm}{20mm}
   \istroot(1)(0-1)<180>{1}
     \istb<grow=north>{Jump}[l]
     \istb<grow=south>{Run}[l]
     \endist
   \istroot(2)(0-2)<0>{1}
     \istb<grow=south>{Run}[r]
     \endist
   \istroot'[north](a1)(1-1)
     \istb{C}[bl]{-3, -0.6}
     \istb{NC}[br]{0,0}
     \endist
   \istroot(b1)(1-2)
     \istb{C}[al]{-2,-1.4}
     \istb{NC}[ar]{0,0}
     \endist
   \istroot(a2)(2-1)
     \istb{C}[al]{-5,1}
     \istb{NC}[ar]{0,0}
     \endist
   \xtInfoset(b1)(a2){2}
   \end{istgame}
\end{center}
\end{frame}

\begin{frame}{Equilibria}
\protect\hypertarget{equilibria}{}
\begin{itemize}
\tightlist
\item
  This one really looks like it should only have one equilibrium.
\item
  If everyone does the same thing, i.e., Run, then Hearer's expected
  utility from Chasing is positive, so they will Chase everyone.
\item
  But Fast Senders don't want this; they would prefer Jump plus No Chase
  to Run plus Chase.
\item
  And if they Jump, Hearer will know it isn't worth Chasing.
\item
  So the only sensible equilibrium is that Fast Senders Jump, and Hearer
  chases all and only Senders who Run (rather than Jumping).
\end{itemize}
\end{frame}

\begin{frame}{College}
\protect\hypertarget{college}{}
\begin{itemize}
\tightlist
\item
  Could there be an honest signaling explanation of why there is a
  college wage premium?
\item
  Maybe; it seems relevant that some people aren't admitted to college
  and others could not complete it.
\item
  But I don't know what such an explanation could look like.
\end{itemize}
\end{frame}

\begin{frame}{Can't/Won't}
\protect\hypertarget{cantwont}{}
\begin{itemize}
\tightlist
\item
  In real life the boundary between a game where signaling is costly for
  one type and where it is impossible can be hard to draw.
\item
  Especially for non-human animals, what exactly does it mean to say
  they could do something but choose not to because it is too expensive,
  rather than say that they can't.
\item
  And for humans, we don't even consider some things to be viable
  options because they are prohibitively expensive.
\item
  Are these cases where something is not an option, or where it is
  rationally not chosen for expense.
\item
  It isn't clear that much could, or should, turn on this.
\end{itemize}
\end{frame}

\begin{frame}{For Next Time}
\protect\hypertarget{for-next-time}{}
\begin{itemize}
\tightlist
\item
  For next class I don't have slides, just a long-ish handout.
\item
  Next week we will look more closely at Iterated Prisoners' Dilemma.
\end{itemize}
\end{frame}

\end{document}
