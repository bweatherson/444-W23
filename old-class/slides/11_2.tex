% Options for packages loaded elsewhere
\PassOptionsToPackage{unicode}{hyperref}
\PassOptionsToPackage{hyphens}{url}
%
\documentclass[
  ignorenonframetext,
]{beamer}
\usepackage{pgfpages}
\setbeamertemplate{caption}[numbered]
\setbeamertemplate{caption label separator}{: }
\setbeamercolor{caption name}{fg=normal text.fg}
\beamertemplatenavigationsymbolsempty
% Prevent slide breaks in the middle of a paragraph
\widowpenalties 1 10000
\raggedbottom
\setbeamertemplate{part page}{
  \centering
  \begin{beamercolorbox}[sep=16pt,center]{part title}
    \usebeamerfont{part title}\insertpart\par
  \end{beamercolorbox}
}
\setbeamertemplate{section page}{
  \centering
  \begin{beamercolorbox}[sep=12pt,center]{part title}
    \usebeamerfont{section title}\insertsection\par
  \end{beamercolorbox}
}
\setbeamertemplate{subsection page}{
  \centering
  \begin{beamercolorbox}[sep=8pt,center]{part title}
    \usebeamerfont{subsection title}\insertsubsection\par
  \end{beamercolorbox}
}
\AtBeginPart{
  \frame{\partpage}
}
\AtBeginSection{
  \ifbibliography
  \else
    \frame{\sectionpage}
  \fi
}
\AtBeginSubsection{
  \frame{\subsectionpage}
}
\usepackage{amsmath,amssymb}
\usepackage{lmodern}
\usepackage{ifxetex,ifluatex}
\ifnum 0\ifxetex 1\fi\ifluatex 1\fi=0 % if pdftex
  \usepackage[T1]{fontenc}
  \usepackage[utf8]{inputenc}
  \usepackage{textcomp} % provide euro and other symbols
\else % if luatex or xetex
  \usepackage{unicode-math}
  \defaultfontfeatures{Scale=MatchLowercase}
  \defaultfontfeatures[\rmfamily]{Ligatures=TeX,Scale=1}
  \setmainfont[BoldFont = SF Pro Rounded Semibold]{SF Pro Rounded}
  \setmathfont[]{STIX Two Math}
\fi
\usefonttheme{serif} % use mainfont rather than sansfont for slide text
% Use upquote if available, for straight quotes in verbatim environments
\IfFileExists{upquote.sty}{\usepackage{upquote}}{}
\IfFileExists{microtype.sty}{% use microtype if available
  \usepackage[]{microtype}
  \UseMicrotypeSet[protrusion]{basicmath} % disable protrusion for tt fonts
}{}
\makeatletter
\@ifundefined{KOMAClassName}{% if non-KOMA class
  \IfFileExists{parskip.sty}{%
    \usepackage{parskip}
  }{% else
    \setlength{\parindent}{0pt}
    \setlength{\parskip}{6pt plus 2pt minus 1pt}}
}{% if KOMA class
  \KOMAoptions{parskip=half}}
\makeatother
\usepackage{xcolor}
\IfFileExists{xurl.sty}{\usepackage{xurl}}{} % add URL line breaks if available
\IfFileExists{bookmark.sty}{\usepackage{bookmark}}{\usepackage{hyperref}}
\hypersetup{
  pdftitle={444 Lecture 11.2 - O'Connor Chapters 7-10},
  pdfauthor={Brian Weatherson},
  hidelinks,
  pdfcreator={LaTeX via pandoc}}
\urlstyle{same} % disable monospaced font for URLs
\newif\ifbibliography
\setlength{\emergencystretch}{3em} % prevent overfull lines
\providecommand{\tightlist}{%
  \setlength{\itemsep}{0pt}\setlength{\parskip}{0pt}}
\setcounter{secnumdepth}{-\maxdimen} % remove section numbering
\let\Tiny=\tiny

 \setbeamertemplate{navigation symbols}{} 

% \usetheme{Madrid}
 \usetheme[numbering=none, progressbar=foot]{metropolis}
 \usecolortheme{wolverine}
 \usepackage{color}
 \usepackage{MnSymbol}
% \usepackage{movie15}

\usepackage{amssymb}% http://ctan.org/pkg/amssymb
\usepackage{pifont}% http://ctan.org/pkg/pifont
\newcommand{\cmark}{\ding{51}}%
\newcommand{\xmark}{\ding{55}}%

\DeclareSymbolFont{symbolsC}{U}{txsyc}{m}{n}
\DeclareMathSymbol{\boxright}{\mathrel}{symbolsC}{128}
\DeclareMathAlphabet{\mathpzc}{OT1}{pzc}{m}{it}

\setlength{\parskip}{1ex plus 0.5ex minus 0.2ex}

\AtBeginSection[]
{
\begin{frame}
	\Huge{\color{darkblue} \insertsection}
\end{frame}
}

\renewenvironment*{quote}	
	{\list{}{\rightmargin   \leftmargin} \item } 	
	{\endlist }

\definecolor{darkgreen}{rgb}{0,0.7,0}
\definecolor{darkblue}{rgb}{0,0,0.8}

\usepackage[italic]{mathastext}
\usepackage{nicefrac}
\usepackage{istgame}

\setbeamertemplate{caption}{\raggedright\insertcaption}

%\def\toprule{}
%\def\bottomrule{}
%\def\midrule{}
\usepackage{etoolbox}
\AfterEndEnvironment{description}{\vspace{9pt}}
\AfterEndEnvironment{oltableau}{\vspace{9pt}}
\BeforeBeginEnvironment{oltableau}{\vspace{9pt}}
\AfterEndEnvironment{center}{\vspace{9pt}}
\BeforeBeginEnvironment{tabular}{\vspace{9pt}}
\AfterEndEnvironment{longtable}{\vspace{-6pt}}
\usepackage{booktabs}
\usepackage{longtable}
\usepackage{array}
\usepackage{multirow}
\usepackage{wrapfig}
\usepackage{float}
\usepackage{colortbl}
\usepackage{pdflscape}
\usepackage{tabu}
\usepackage{threeparttable} 
\usepackage{threeparttablex} 
\usepackage[normalem]{ulem} 
\usepackage{makecell}
\usepackage{xcolor}
\usepackage{ulem}

\setlength\heavyrulewidth{0ex}
\setlength\lightrulewidth{0.08ex}

\aboverulesep=0ex
\belowrulesep=0ex
\renewcommand{\arraystretch}{1.2}
\AtBeginSection[]
{
    \begin{frame}
        \frametitle{Day Plan}
        \tableofcontents[currentsection]
    \end{frame}
}
\ifluatex
  \usepackage{selnolig}  % disable illegal ligatures
\fi

\title{444 Lecture 11.2 - O'Connor Chapters 7-10}
\author{Brian Weatherson}
\date{}

\begin{document}
\frame{\titlepage}

\begin{frame}{Day Plan}
\protect\hypertarget{day-plan}{}
\tableofcontents
\end{frame}

\hypertarget{network-models}{%
\section{Network Models}\label{network-models}}

\begin{frame}{Networks and Ecosystems}
\protect\hypertarget{networks-and-ecosystems}{}
\begin{itemize}
\tightlist
\item
  So far we've used models where in each round, everyone interacts with
  everyone. (Or with a randomly selected portion of everyone; it won't
  matter for modeling purposes.)
\item
  In a network model, people have `neighbors'.
\item
  In general we do not assume these are spatially arranged.
\item
  My neighbors might include my literal neighbors, but also my
  workmates, the other parents at my kid's school, people I interact
  with socially (including online), and so on.
\end{itemize}
\end{frame}

\begin{frame}{Networks}
\protect\hypertarget{networks}{}
Interactions can have two effects.

\begin{enumerate}
\tightlist
\item
  They determine our payout in a given round.
\item
  They determine what we learn from for future rounds.
\end{enumerate}
\end{frame}

\begin{frame}{Network Models}
\protect\hypertarget{network-models-1}{}
This is a very promising way, I think, for modeling gender inequality.

\begin{itemize}
\tightlist
\item
  One thing about gender is that although there is a lot of
  discrimination that persists to this day, it is very unevenly
  distributed.
\item
  In some fields, there is \emph{relatively} little.
\item
  In other fields, there is a lot.
\item
  Using network models can give us the chance to model that.
\end{itemize}
\end{frame}

\begin{frame}{Network Effects}
\protect\hypertarget{network-effects}{}
In some of these models, you get some really strange effects.

\begin{itemize}
\tightlist
\item
  Sometimes there are real benefits to cutting off certain connections.
\item
  By sticking in a smaller network for longer, sometimes you don't get
  sucked into the bad practices of the group.
\item
  Of course, sometimes you build a bubble that is bad in lots of ways.
\item
  It's hard to know in advance which will swamp.
\end{itemize}
\end{frame}

\hypertarget{bounded-rationality}{%
\section{Bounded Rationality}\label{bounded-rationality}}

\begin{frame}{Two Kinds of Models}
\protect\hypertarget{two-kinds-of-models}{}
\begin{enumerate}
\tightlist
\item
  Everyone is perfectly rational, and this is common knowledge.
\item
  Everyone has a hard-wired strategy, and they will employ it even when
  it will obviously get them killed.
\end{enumerate}
\end{frame}

\begin{frame}{Bounded Rationality}
\protect\hypertarget{bounded-rationality-1}{}
Obviously there are situations where we'd like something in between
those two situations.
\end{frame}

\begin{frame}{Two Forms}
\protect\hypertarget{two-forms}{}
\begin{enumerate}[<+->]
\tightlist
\item
  No ability to anticipate; the future will be just like the past.
\item
  Limited memory.
\end{enumerate}
\end{frame}

\begin{frame}{A Third Form}
\protect\hypertarget{a-third-form}{}
You can easily mix these two.

\begin{itemize}
\tightlist
\item
  Maybe some agents will assume the future will look like the very
  recent past.
\item
  Maybe some agents will have longer or shorter memories.
\end{itemize}
\end{frame}

\begin{frame}{Extending the Model}
\protect\hypertarget{extending-the-model}{}
The point is not that these are correct models of reality. That's not
the aim.

\begin{itemize}
\tightlist
\item
  But they are different to both the hyper-rational and the
  hyper-mechanical models.
\item
  And they do this without sending the computational complexity to
  infinity, or leaving too few constraints.
\end{itemize}
\end{frame}

\hypertarget{changing-values}{%
\section{Changing Values}\label{changing-values}}

\begin{frame}{Bicchieri}
\protect\hypertarget{bicchieri}{}
\begin{itemize}
\tightlist
\item
  I wanted to end by noting some relevant work by Penn
  philosoper(/economist/cognitive scientist) Christina Bicchieri.
\item
  I had a note to mention how this work relates to Bicchieri's, then I
  got to chapter 9 and saw that O'Connor already made that connection.
  So I won't belabor the point.
\end{itemize}
\end{frame}

\begin{frame}{Norms are Equilibria}
\protect\hypertarget{norms-are-equilibria}{}
\begin{itemize}
\tightlist
\item
  So here's one way a practice can become stable in a community.
\item
  It's an equilibrium of a game that they are playing, so no one has an
  incentive to deviate.
\item
  A very popular theory, one I think has got to be part of the true
  story, is that social norms typically arise in this way.
\end{itemize}
\end{frame}

\begin{frame}{Norms are Special Equilibria}
\protect\hypertarget{norms-are-special-equilibria}{}
Bicchieri's point is that norms, as opposed to other conventions for
dealing with regularities, do more than steer us to an equilibrium
point.

\begin{itemize}
\tightlist
\item
  They change the payoffs.
\end{itemize}
\end{frame}

\begin{frame}{Two Kinds of Equilibria}
\protect\hypertarget{two-kinds-of-equilibria}{}
\begin{enumerate}[<+->]
\tightlist
\item
  Everyone does X because it's valuable to do what everyone else does.
  Opening hours are like this.
\item
  Everyone does X because it's valuable to do what everyone else does
  \emph{and you'll get punished for doing otherwise}.
\end{enumerate}

\pause

Arguably gender is like this.
\end{frame}

\begin{frame}{Policing Gender Norms}
\protect\hypertarget{policing-gender-norms}{}
\begin{itemize}
\tightlist
\item
  Part of the story about gender is that not only do people signal it
  fairly clearly (at least most people most of the time).
\item
  But there are strong social sanctions - sometimes including violence -
  against people who do not signal in this way.
\item
  It's very hard to model this using either kinds of games that O'Connor
  describes.
\end{itemize}
\end{frame}

\begin{frame}{General Pattern}
\protect\hypertarget{general-pattern}{}
Here is a hypothesis about what is going on here.

\begin{enumerate}
\tightlist
\item
  Gender norms developed for the kind of game-theoretic reasons that
  O'Connor describes.
\item
  But over time, people internalised those norms, came to see them as
  the way things should be done, and developed a disposition to punish
  non-conformers.
\end{enumerate}

These stories are not incompatible; one is a story about generation, the
other about persistence.
\end{frame}

\begin{frame}{Why Two Stories}
\protect\hypertarget{why-two-stories}{}
\begin{itemize}
\tightlist
\item
  We need to explain relative stability of norms over time.
\item
  O'Connor complains that rational choice models can't deliver this
  explanation.
\item
  But going all the way to mechanical biological models seems like
  overkill.
\item
  A model of people internalising the norms, and thereby changing the
  payout structure, seems more promising.
\end{itemize}
\end{frame}

\begin{frame}{Where to Go Next}
\protect\hypertarget{where-to-go-next}{}
\begin{itemize}
\tightlist
\item
  And thinking about changing payouts might be a more effective way of
  moving to new solutions.
\item
  At least, it's a different way than trying to push people to different
  equilibria of the same game.
\end{itemize}
\end{frame}

\end{document}
