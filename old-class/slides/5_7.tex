% Options for packages loaded elsewhere
\PassOptionsToPackage{unicode}{hyperref}
\PassOptionsToPackage{hyphens}{url}
%
\documentclass[
  ignorenonframetext,
]{beamer}
\usepackage{pgfpages}
\setbeamertemplate{caption}[numbered]
\setbeamertemplate{caption label separator}{: }
\setbeamercolor{caption name}{fg=normal text.fg}
\beamertemplatenavigationsymbolsempty
% Prevent slide breaks in the middle of a paragraph
\widowpenalties 1 10000
\raggedbottom
\setbeamertemplate{part page}{
  \centering
  \begin{beamercolorbox}[sep=16pt,center]{part title}
    \usebeamerfont{part title}\insertpart\par
  \end{beamercolorbox}
}
\setbeamertemplate{section page}{
  \centering
  \begin{beamercolorbox}[sep=12pt,center]{part title}
    \usebeamerfont{section title}\insertsection\par
  \end{beamercolorbox}
}
\setbeamertemplate{subsection page}{
  \centering
  \begin{beamercolorbox}[sep=8pt,center]{part title}
    \usebeamerfont{subsection title}\insertsubsection\par
  \end{beamercolorbox}
}
\AtBeginPart{
  \frame{\partpage}
}
\AtBeginSection{
  \ifbibliography
  \else
    \frame{\sectionpage}
  \fi
}
\AtBeginSubsection{
  \frame{\subsectionpage}
}
\usepackage{amsmath,amssymb}
\usepackage{lmodern}
\usepackage{ifxetex,ifluatex}
\ifnum 0\ifxetex 1\fi\ifluatex 1\fi=0 % if pdftex
  \usepackage[T1]{fontenc}
  \usepackage[utf8]{inputenc}
  \usepackage{textcomp} % provide euro and other symbols
\else % if luatex or xetex
  \usepackage{unicode-math}
  \defaultfontfeatures{Scale=MatchLowercase}
  \defaultfontfeatures[\rmfamily]{Ligatures=TeX,Scale=1}
  \setmainfont[BoldFont = SF Pro Rounded Semibold]{SF Pro Rounded}
  \setmathfont[]{STIX Two Math}
\fi
\usefonttheme{serif} % use mainfont rather than sansfont for slide text
% Use upquote if available, for straight quotes in verbatim environments
\IfFileExists{upquote.sty}{\usepackage{upquote}}{}
\IfFileExists{microtype.sty}{% use microtype if available
  \usepackage[]{microtype}
  \UseMicrotypeSet[protrusion]{basicmath} % disable protrusion for tt fonts
}{}
\makeatletter
\@ifundefined{KOMAClassName}{% if non-KOMA class
  \IfFileExists{parskip.sty}{%
    \usepackage{parskip}
  }{% else
    \setlength{\parindent}{0pt}
    \setlength{\parskip}{6pt plus 2pt minus 1pt}}
}{% if KOMA class
  \KOMAoptions{parskip=half}}
\makeatother
\usepackage{xcolor}
\IfFileExists{xurl.sty}{\usepackage{xurl}}{} % add URL line breaks if available
\IfFileExists{bookmark.sty}{\usepackage{bookmark}}{\usepackage{hyperref}}
\hypersetup{
  pdftitle={444 Lecture 5.7 - Rationalizable Strategies},
  pdfauthor={Brian Weatherson},
  hidelinks,
  pdfcreator={LaTeX via pandoc}}
\urlstyle{same} % disable monospaced font for URLs
\newif\ifbibliography
\setlength{\emergencystretch}{3em} % prevent overfull lines
\providecommand{\tightlist}{%
  \setlength{\itemsep}{0pt}\setlength{\parskip}{0pt}}
\setcounter{secnumdepth}{-\maxdimen} % remove section numbering
\let\Tiny=\tiny

 \setbeamertemplate{navigation symbols}{} 

% \usetheme{Madrid}
 \usetheme[numbering=none, progressbar=foot]{metropolis}
 \usecolortheme{wolverine}
 \usepackage{color}
 \usepackage{MnSymbol}
% \usepackage{movie15}

\usepackage{amssymb}% http://ctan.org/pkg/amssymb
\usepackage{pifont}% http://ctan.org/pkg/pifont
\newcommand{\cmark}{\ding{51}}%
\newcommand{\xmark}{\ding{55}}%

\DeclareSymbolFont{symbolsC}{U}{txsyc}{m}{n}
\DeclareMathSymbol{\boxright}{\mathrel}{symbolsC}{128}
\DeclareMathAlphabet{\mathpzc}{OT1}{pzc}{m}{it}

\setlength{\parskip}{1ex plus 0.5ex minus 0.2ex}

\AtBeginSection[]
{
\begin{frame}
	\Huge{\color{darkblue} \insertsection}
\end{frame}
}

\renewenvironment*{quote}	
	{\list{}{\rightmargin   \leftmargin} \item } 	
	{\endlist }

\definecolor{darkgreen}{rgb}{0,0.7,0}
\definecolor{darkblue}{rgb}{0,0,0.8}

\usepackage[italic]{mathastext}
\usepackage{nicefrac}

\setbeamertemplate{caption}{\raggedright\insertcaption}

%\def\toprule{}
%\def\bottomrule{}
%\def\midrule{}
\usepackage{etoolbox}
\AfterEndEnvironment{description}{\vspace{9pt}}
\AfterEndEnvironment{oltableau}{\vspace{9pt}}
\BeforeBeginEnvironment{oltableau}{\vspace{9pt}}
\AfterEndEnvironment{center}{\vspace{9pt}}
\BeforeBeginEnvironment{tabular}{\vspace{9pt}}
\AfterEndEnvironment{longtable}{\vspace{-6pt}}
\usepackage{booktabs}
\usepackage{longtable}
\usepackage{array}
\usepackage{multirow}
\usepackage{wrapfig}
\usepackage{float}
\usepackage{colortbl}
\usepackage{pdflscape}
\usepackage{tabu}
\usepackage{threeparttable} 
\usepackage{threeparttablex} 
\usepackage[normalem]{ulem} 
\usepackage{makecell}
\usepackage{xcolor}
\usepackage{ulem}

\setlength\heavyrulewidth{0ex}
\setlength\lightrulewidth{0.08ex}

\aboverulesep=0ex
\belowrulesep=0ex
\renewcommand{\arraystretch}{1.2}
\ifluatex
  \usepackage{selnolig}  % disable illegal ligatures
\fi

\title{444 Lecture 5.7 - Rationalizable Strategies}
\author{Brian Weatherson}
\date{}

\begin{document}
\frame{\titlepage}

\begin{frame}{Plan}
\protect\hypertarget{plan}{}
To introduce the idea of rationalizable strategies.
\end{frame}

\begin{frame}{Reading}
\protect\hypertarget{reading}{}
Bonanno, section 6.4
\end{frame}

\begin{frame}{Playing Best Responses}
\protect\hypertarget{playing-best-responses}{}
\begin{table}[!h]
\centering
\begin{tabular}[t]{>{}r|cc}
\toprule
 & Left & Right\\
\midrule
Up & 3, 0 & 0, 1\\
Middle & 1, 1 & 1, 0\\
Down & 0, 0 & 3, 1\\
\bottomrule
\end{tabular}
\end{table}

In this game, the best responses are:

\begin{itemize}
\tightlist
\item
  Row can play Up (best response to Left) or Down (best response to
  Right);
\item
  Column can play Left (best response to Middle) or Right (best response
  to either Up or Down).
\end{itemize}
\end{frame}

\begin{frame}{Playing Best Responses}
\protect\hypertarget{playing-best-responses-1}{}
\begin{table}[!h]
\centering
\begin{tabular}[t]{>{}r|cc}
\toprule
 & Left & Right\\
\midrule
Up & 3, 0 & 0, 1\\
Middle & 1, 1 & 1, 0\\
Down & 0, 0 & 3, 1\\
\bottomrule
\end{tabular}
\end{table}

\begin{itemize}
\tightlist
\item
  But Middle is not a best response.
\item
  It is dominated by the 50/50 mixture of Left and Right.
\end{itemize}
\end{frame}

\begin{frame}{Iterated Best Responses}
\protect\hypertarget{iterated-best-responses}{}
\begin{table}[!h]
\centering
\begin{tabular}[t]{>{}r|cc}
\toprule
 & Left & Right\\
\midrule
Up & 3, 0 & 0, 1\\
Middle & 1, 1 & 1, 0\\
Down & 0, 0 & 3, 1\\
\bottomrule
\end{tabular}
\end{table}

\begin{itemize}
\tightlist
\item
  So while Left is a best response\ldots{}
\item
  It is not a best response to a best response.
\end{itemize}
\end{frame}

\begin{frame}{Iterated Best Responses}
\protect\hypertarget{iterated-best-responses-1}{}
\begin{itemize}
\tightlist
\item
  That makes it seem irrational to play Middle. \pause
\item
  I could build more complicated examples, where we had cases that are
  best responses to best responses, but not best responses to best
  responses to best responses. \pause
\item
  Actually we've already seen such a case.
\item
  In the Ice Cream game, 2 is a best response to 1, which is a best
  response to 0.
\item
  But 2 is not a best response to any best response to a best response.
\end{itemize}
\end{frame}

\begin{frame}{Iterated Best Responses}
\protect\hypertarget{iterated-best-responses-2}{}
\begin{itemize}
\tightlist
\item
  Some strategies are at the start of an infinite chain
  \(S_1, S_2, \dots\) where each strategy is a best response to the one
  that comes after it.
\item
  Call these the \textbf{rationalizable} strategies.
\end{itemize}
\end{frame}

\begin{frame}{Infinite Chains}
\protect\hypertarget{infinite-chains}{}
Here is one way to get an infinite chain like this.

\begin{itemize}
\tightlist
\item
  If the pair \(\langle S_1, S_2 \rangle\) is a Nash equilibrium,
  \ldots{}
\item
  Then \(S_1\) is a best response to \(S_2\), which is a best response
  to \(S_1\), which is a best response to \(S_2\), which \ldots{} \pause
\item
  But you don't only need to use Nash equilibria.
\item
  Think about Rock, Paper, Scissors.
\item
  Rock is a best response to Scissors, which is a best response to
  Paper, which is a best response to Rock, which is\ldots{}
\item
  But Rock is not part of a Nash equilibrium.
\end{itemize}
\end{frame}

\begin{frame}{Rationalizability}
\protect\hypertarget{rationalizability}{}
I'm not going to prove this, but the following turns out to be true.

\begin{itemize}
\tightlist
\item
  The strategies that can be at the start of these infinite chains
  \ldots{}
\item
  Are exactly those strategies that survive iterated deletion of
  strongly dominated strategies \ldots{}
\item
  Provided we include dominance by mixtures when we're doing the
  deleting.
\end{itemize}
\end{frame}

\begin{frame}{Philosophical Payoff}
\protect\hypertarget{philosophical-payoff}{}
Some economists, and a few philosophers, have argued that this is the
key philosophical notion in game theory.

\begin{itemize}
\tightlist
\item
  They say that a strategy is rational to play if and only if it is
  rationalizable in this sense. \pause
\item
  In economics, this is very much a \textbf{heterodox} view.
\item
  Note that this view is more permissive than the view that rational
  players will choose Nash equilibria.
\item
  All Nash equilibria are rationalizable, but some rationalizable
  strategies (e.g., Rock!), are not Nash equilibria. \pause
\item
  Most economists think that if there is a key notion in game theory, it
  is \textbf{less permissive} than Nash equilibrium.
\end{itemize}
\end{frame}

\begin{frame}{For Next Time}
\protect\hypertarget{for-next-time}{}
\begin{itemize}
\tightlist
\item
  We'll close this out by going back to Nash, and asking why Nash
  equilibrium is philosophically significant.
\end{itemize}
\end{frame}

\end{document}
