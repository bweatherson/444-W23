% Options for packages loaded elsewhere
\PassOptionsToPackage{unicode}{hyperref}
\PassOptionsToPackage{hyphens}{url}
%
\documentclass[
  ignorenonframetext,
]{beamer}
\usepackage{pgfpages}
\setbeamertemplate{caption}[numbered]
\setbeamertemplate{caption label separator}{: }
\setbeamercolor{caption name}{fg=normal text.fg}
\beamertemplatenavigationsymbolsempty
% Prevent slide breaks in the middle of a paragraph
\widowpenalties 1 10000
\raggedbottom
\setbeamertemplate{part page}{
  \centering
  \begin{beamercolorbox}[sep=16pt,center]{part title}
    \usebeamerfont{part title}\insertpart\par
  \end{beamercolorbox}
}
\setbeamertemplate{section page}{
  \centering
  \begin{beamercolorbox}[sep=12pt,center]{part title}
    \usebeamerfont{section title}\insertsection\par
  \end{beamercolorbox}
}
\setbeamertemplate{subsection page}{
  \centering
  \begin{beamercolorbox}[sep=8pt,center]{part title}
    \usebeamerfont{subsection title}\insertsubsection\par
  \end{beamercolorbox}
}
\AtBeginPart{
  \frame{\partpage}
}
\AtBeginSection{
  \ifbibliography
  \else
    \frame{\sectionpage}
  \fi
}
\AtBeginSubsection{
  \frame{\subsectionpage}
}
\usepackage{lmodern}
\usepackage{amssymb,amsmath}
\usepackage{ifxetex,ifluatex}
\ifnum 0\ifxetex 1\fi\ifluatex 1\fi=0 % if pdftex
  \usepackage[T1]{fontenc}
  \usepackage[utf8]{inputenc}
  \usepackage{textcomp} % provide euro and other symbols
\else % if luatex or xetex
  \usepackage{unicode-math}
  \defaultfontfeatures{Scale=MatchLowercase}
  \defaultfontfeatures[\rmfamily]{Ligatures=TeX,Scale=1}
  \setmainfont[BoldFont = SF Pro Rounded Semibold]{SF Pro Rounded}
  \setmathfont[]{STIX Two Math}
\fi
\usefonttheme{serif} % use mainfont rather than sansfont for slide text
% Use upquote if available, for straight quotes in verbatim environments
\IfFileExists{upquote.sty}{\usepackage{upquote}}{}
\IfFileExists{microtype.sty}{% use microtype if available
  \usepackage[]{microtype}
  \UseMicrotypeSet[protrusion]{basicmath} % disable protrusion for tt fonts
}{}
\makeatletter
\@ifundefined{KOMAClassName}{% if non-KOMA class
  \IfFileExists{parskip.sty}{%
    \usepackage{parskip}
  }{% else
    \setlength{\parindent}{0pt}
    \setlength{\parskip}{6pt plus 2pt minus 1pt}}
}{% if KOMA class
  \KOMAoptions{parskip=half}}
\makeatother
\usepackage{xcolor}
\IfFileExists{xurl.sty}{\usepackage{xurl}}{} % add URL line breaks if available
\IfFileExists{bookmark.sty}{\usepackage{bookmark}}{\usepackage{hyperref}}
\hypersetup{
  pdftitle={444 Lecture 2.7 - Iterated Deletion of Weakly Dominated Strategies},
  pdfauthor={Brian Weatherson},
  hidelinks,
  pdfcreator={LaTeX via pandoc}}
\urlstyle{same} % disable monospaced font for URLs
\newif\ifbibliography
\setlength{\emergencystretch}{3em} % prevent overfull lines
\providecommand{\tightlist}{%
  \setlength{\itemsep}{0pt}\setlength{\parskip}{0pt}}
\setcounter{secnumdepth}{-\maxdimen} % remove section numbering
\let\Tiny=\tiny

 \setbeamertemplate{navigation symbols}{} 

% \usetheme{Madrid}
 \usetheme[numbering=none, progressbar=foot]{metropolis}
 \usecolortheme{wolverine}
 \usepackage{color}
 \usepackage{MnSymbol}
% \usepackage{movie15}

\usepackage{amssymb}% http://ctan.org/pkg/amssymb
\usepackage{pifont}% http://ctan.org/pkg/pifont
\newcommand{\cmark}{\ding{51}}%
\newcommand{\xmark}{\ding{55}}%

\DeclareSymbolFont{symbolsC}{U}{txsyc}{m}{n}
\DeclareMathSymbol{\boxright}{\mathrel}{symbolsC}{128}
\DeclareMathAlphabet{\mathpzc}{OT1}{pzc}{m}{it}

\setlength{\parskip}{1ex plus 0.5ex minus 0.2ex}

\AtBeginSection[]
{
\begin{frame}
	\Huge{\color{darkblue} \insertsection}
\end{frame}
}

\renewenvironment*{quote}	
	{\list{}{\rightmargin   \leftmargin} \item } 	
	{\endlist }

\definecolor{darkgreen}{rgb}{0,0.7,0}
\definecolor{darkblue}{rgb}{0,0,0.8}

\usepackage[italic]{mathastext}
\usepackage{nicefrac}


%\def\toprule{}
%\def\bottomrule{}
%\def\midrule{}
\usepackage{etoolbox}
\AfterEndEnvironment{description}{\vspace{9pt}}
\AfterEndEnvironment{oltableau}{\vspace{9pt}}
\BeforeBeginEnvironment{oltableau}{\vspace{9pt}}
\AfterEndEnvironment{center}{\vspace{9pt}}
\BeforeBeginEnvironment{tabular}{\vspace{9pt}}
\AfterEndEnvironment{longtable}{\vspace{-6pt}}
\usepackage{booktabs}
\usepackage{longtable}
\usepackage{array}
\usepackage{multirow}
\usepackage{wrapfig}
\usepackage{float}
\usepackage{colortbl}
\usepackage{pdflscape}
\usepackage{tabu}
\usepackage{threeparttable} 
\usepackage{threeparttablex} 
\usepackage[normalem]{ulem} 
\usepackage{makecell}
\usepackage{xcolor}
\usepackage{ulem}

\setlength\heavyrulewidth{0ex}
\setlength\lightrulewidth{0.08ex}

\aboverulesep=0ex
\belowrulesep=0ex
\renewcommand{\arraystretch}{1.2}

\title{444 Lecture 2.7 - Iterated Deletion of Weakly Dominated Strategies}
\author{Brian Weatherson}
\date{}

\begin{document}
\frame{\titlepage}

\begin{frame}{Plan}
\protect\hypertarget{plan}{}

To look at two problems that arise when we delete strategies that are
merely weakly dominated.

\end{frame}

\begin{frame}{Reading}
\protect\hypertarget{reading}{}

Bonanno, section 2.5.2.

\end{frame}

\begin{frame}{Two Issues}
\protect\hypertarget{two-issues}{}

\begin{itemize}
\tightlist
\item
  Order effects.
\item
  Philosophical motivation.
\end{itemize}

\end{frame}

\begin{frame}{Order Effects}
\protect\hypertarget{order-effects}{}

\begin{itemize}
\tightlist
\item
  As Bonanno goes over, when deleting weakly dominated strategies, it
  matters what order you do the deletions in.
\item
  Whether a strategy weakly dominates another at a point in the process
  might depend on how you got to that point.
\item
  And the result is that different ways of applying the process lead to
  different `solutions'.
\end{itemize}

\end{frame}

\begin{frame}{Way Around This}
\protect\hypertarget{way-around-this}{}

\begin{itemize}
\tightlist
\item
  Bonanno says (as I think is standard) that you solve this by saying
  that at each stage, you delete every strategy that you possibly can.
\item
  There is still an issue I think about why that deletion process is
  justified as opposed to some other.
\item
  It does have the nice advantage of actually being a well defined
  process, so that's nice.
\end{itemize}

\end{frame}

\begin{frame}{Philosophical Justification}
\protect\hypertarget{philosophical-justification}{}

\begin{itemize}
\tightlist
\item
  The bigger issue is that it is a little hard to say why we should care
  about the result of this procedure.
\item
  Saying what's special about the result of this strategy is not
  completely obvious.
\item
  Bonanno alludes to this - let's go over it in a bit more detail.
\end{itemize}

\end{frame}

\begin{frame}{Iterated Deletion}
\protect\hypertarget{iterated-deletion}{}

\begin{table}[!h]
\centering
\begin{tabular}[t]{>{}r|ccc}
\toprule
 & Left & Center & Right\\
\midrule
Up & 1, 1 & 1, 1 & 0, 0\\
Middle & 1, 1 & 0, 0 & 1, 0\\
Down & 0, 0 & 0, 1 & 1, 1\\
\bottomrule
\end{tabular}
\end{table}

\begin{itemize}
\tightlist
\item
  Middle weakly dominates Down, and Center weakly dominates Right.
\item
  So let's delete them.
\end{itemize}

\end{frame}

\begin{frame}{Iterated Deletion}
\protect\hypertarget{iterated-deletion-1}{}

\begin{table}[!h]
\centering
\begin{tabular}[t]{>{}r|cc}
\toprule
 & Left & Center\\
\midrule
Up & 1, 1 & 1, 1\\
Middle & 1, 1 & 0, 0\\
\bottomrule
\end{tabular}
\end{table}

\begin{itemize}
\tightlist
\item
  Now Up weakly dominates Middle and Left weakly dominates Center.
\item
  So the solution is Up/Left, right?
\item
  Well, not so fast.
\end{itemize}

\end{frame}

\begin{frame}{Iterated Deletion}
\protect\hypertarget{iterated-deletion-2}{}

\begin{table}[!h]
\centering
\begin{tabular}[t]{>{}r|ccc}
\toprule
 & Left & Center & Right\\
\midrule
Up & 1, 1 & 1, 1 & 0, 0\\
Middle & 1, 1 & 0, 0 & 1, 0\\
Down & 0, 0 & 0, 1 & 1, 1\\
\bottomrule
\end{tabular}
\end{table}

\begin{itemize}
\tightlist
\item
  Think about it from Row's perspective.
\item
  We have an argument that Column will play Left.
\item
  If that argument works, Row shouldn't prefer Up - they should be
  indifferent between Up and Middle.
\item
  Why does the argument say to play Up then?
\end{itemize}

\end{frame}

\begin{frame}{Iterated Deletion}
\protect\hypertarget{iterated-deletion-3}{}

\begin{table}[!h]
\centering
\begin{tabular}[t]{>{}r|ccc}
\toprule
 & Left & Center & Right\\
\midrule
Up & 1, 1 & 1, 1 & 0, 0\\
Middle & 1, 1 & 0, 0 & 1, 0\\
Down & 0, 0 & 0, 1 & 1, 1\\
\bottomrule
\end{tabular}
\end{table}

\begin{itemize}
\tightlist
\item
  The answer is that Middle is risky.
\item
  In the game after deletion, Middle has a risk of getting 0, but Up is
  sure to get 1.
\end{itemize}

\begin{itemize}[<+->]
\tightlist
\item
  But look at the original game - Up is risky too!
\item
  I think this makes it hard to philosophically defend IDWDS
\end{itemize}

\end{frame}

\begin{frame}{For Next Time}
\protect\hypertarget{for-next-time}{}

We will look at a famous example of iterated deletion that's not in the
book.

\end{frame}

\end{document}
