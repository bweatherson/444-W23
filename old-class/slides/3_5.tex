% Options for packages loaded elsewhere
\PassOptionsToPackage{unicode}{hyperref}
\PassOptionsToPackage{hyphens}{url}
%
\documentclass[
  ignorenonframetext,
]{beamer}
\usepackage{pgfpages}
\setbeamertemplate{caption}[numbered]
\setbeamertemplate{caption label separator}{: }
\setbeamercolor{caption name}{fg=normal text.fg}
\beamertemplatenavigationsymbolsempty
% Prevent slide breaks in the middle of a paragraph
\widowpenalties 1 10000
\raggedbottom
\setbeamertemplate{part page}{
  \centering
  \begin{beamercolorbox}[sep=16pt,center]{part title}
    \usebeamerfont{part title}\insertpart\par
  \end{beamercolorbox}
}
\setbeamertemplate{section page}{
  \centering
  \begin{beamercolorbox}[sep=12pt,center]{part title}
    \usebeamerfont{section title}\insertsection\par
  \end{beamercolorbox}
}
\setbeamertemplate{subsection page}{
  \centering
  \begin{beamercolorbox}[sep=8pt,center]{part title}
    \usebeamerfont{subsection title}\insertsubsection\par
  \end{beamercolorbox}
}
\AtBeginPart{
  \frame{\partpage}
}
\AtBeginSection{
  \ifbibliography
  \else
    \frame{\sectionpage}
  \fi
}
\AtBeginSubsection{
  \frame{\subsectionpage}
}
\usepackage{amsmath,amssymb}
\usepackage{lmodern}
\usepackage{ifxetex,ifluatex}
\ifnum 0\ifxetex 1\fi\ifluatex 1\fi=0 % if pdftex
  \usepackage[T1]{fontenc}
  \usepackage[utf8]{inputenc}
  \usepackage{textcomp} % provide euro and other symbols
\else % if luatex or xetex
  \usepackage{unicode-math}
  \defaultfontfeatures{Scale=MatchLowercase}
  \defaultfontfeatures[\rmfamily]{Ligatures=TeX,Scale=1}
  \setmainfont[BoldFont = SF Pro Rounded Semibold]{SF Pro Rounded}
  \setmathfont[]{STIX Two Math}
\fi
\usefonttheme{serif} % use mainfont rather than sansfont for slide text
% Use upquote if available, for straight quotes in verbatim environments
\IfFileExists{upquote.sty}{\usepackage{upquote}}{}
\IfFileExists{microtype.sty}{% use microtype if available
  \usepackage[]{microtype}
  \UseMicrotypeSet[protrusion]{basicmath} % disable protrusion for tt fonts
}{}
\makeatletter
\@ifundefined{KOMAClassName}{% if non-KOMA class
  \IfFileExists{parskip.sty}{%
    \usepackage{parskip}
  }{% else
    \setlength{\parindent}{0pt}
    \setlength{\parskip}{6pt plus 2pt minus 1pt}}
}{% if KOMA class
  \KOMAoptions{parskip=half}}
\makeatother
\usepackage{xcolor}
\IfFileExists{xurl.sty}{\usepackage{xurl}}{} % add URL line breaks if available
\IfFileExists{bookmark.sty}{\usepackage{bookmark}}{\usepackage{hyperref}}
\hypersetup{
  pdftitle={444 Lecture 3.5 - Incredible Threats},
  pdfauthor={Brian Weatherson},
  hidelinks,
  pdfcreator={LaTeX via pandoc}}
\urlstyle{same} % disable monospaced font for URLs
\newif\ifbibliography
\setlength{\emergencystretch}{3em} % prevent overfull lines
\providecommand{\tightlist}{%
  \setlength{\itemsep}{0pt}\setlength{\parskip}{0pt}}
\setcounter{secnumdepth}{-\maxdimen} % remove section numbering
\let\Tiny=\tiny

 \setbeamertemplate{navigation symbols}{} 

% \usetheme{Madrid}
 \usetheme[numbering=none, progressbar=foot]{metropolis}
 \usecolortheme{wolverine}
 \usepackage{color}
 \usepackage{MnSymbol}
% \usepackage{movie15}

\usepackage{amssymb}% http://ctan.org/pkg/amssymb
\usepackage{pifont}% http://ctan.org/pkg/pifont
\newcommand{\cmark}{\ding{51}}%
\newcommand{\xmark}{\ding{55}}%

\DeclareSymbolFont{symbolsC}{U}{txsyc}{m}{n}
\DeclareMathSymbol{\boxright}{\mathrel}{symbolsC}{128}
\DeclareMathAlphabet{\mathpzc}{OT1}{pzc}{m}{it}

\setlength{\parskip}{1ex plus 0.5ex minus 0.2ex}

\AtBeginSection[]
{
\begin{frame}
	\Huge{\color{darkblue} \insertsection}
\end{frame}
}

\renewenvironment*{quote}	
	{\list{}{\rightmargin   \leftmargin} \item } 	
	{\endlist }

\definecolor{darkgreen}{rgb}{0,0.7,0}
\definecolor{darkblue}{rgb}{0,0,0.8}

\usepackage[italic]{mathastext}
\usepackage{nicefrac}

\setbeamertemplate{caption}{\raggedright\insertcaption}

%\def\toprule{}
%\def\bottomrule{}
%\def\midrule{}
\usepackage{etoolbox}
\AfterEndEnvironment{description}{\vspace{9pt}}
\AfterEndEnvironment{oltableau}{\vspace{9pt}}
\BeforeBeginEnvironment{oltableau}{\vspace{9pt}}
\AfterEndEnvironment{center}{\vspace{9pt}}
\BeforeBeginEnvironment{tabular}{\vspace{9pt}}
\AfterEndEnvironment{longtable}{\vspace{-6pt}}
\usepackage{booktabs}
\usepackage{longtable}
\usepackage{array}
\usepackage{multirow}
\usepackage{wrapfig}
\usepackage{float}
\usepackage{colortbl}
\usepackage{pdflscape}
\usepackage{tabu}
\usepackage{threeparttable} 
\usepackage{threeparttablex} 
\usepackage[normalem]{ulem} 
\usepackage{makecell}
\usepackage{xcolor}
\usepackage{ulem}

\setlength\heavyrulewidth{0ex}
\setlength\lightrulewidth{0.08ex}

\aboverulesep=0ex
\belowrulesep=0ex
\renewcommand{\arraystretch}{1.2}
\ifluatex
  \usepackage{selnolig}  % disable illegal ligatures
\fi

\title{444 Lecture 3.5 - Incredible Threats}
\author{Brian Weatherson}
\date{}

\begin{document}
\frame{\titlepage}

\begin{frame}{Plan}
\protect\hypertarget{plan}{}
\begin{itemize}
\tightlist
\item
  To explain why some Nash equilibria do not seem like sensible plays.
\end{itemize}
\end{frame}

\begin{frame}{Reading}
\protect\hypertarget{reading}{}
\begin{itemize}
\tightlist
\item
  Bonanno, section 3.4.
\end{itemize}
\end{frame}

\begin{frame}[fragile]{Threat Game}
\protect\hypertarget{threat-game}{}
\newcommand{\pictext}[3]{
\put(#1, #2){\makebox(0, 0)[b]{#3}}}

\begin{picture}(350, 110)
\linethickness{1pt}

\put(175, 0){\makebox(0, 0)[b]{I}}
\put(175, 12){\circle*{4}}
%\thicklines
\put(175, 12){\line(-2, 1){70}}
%\thinlines
\put(175, 12){\line(2, 1){70}}

\put(135, 20){\makebox(0, 0)[b]{$A$}}

\put(215, 20){\makebox(0, 0)[b]{$B$}}


\put(105, 35){\makebox(0, 0)[b]{II}}
\put(105, 47){\circle*{4}}
%\thicklines
\put(105, 47){\line(-1, 1){35}}

\put(70, 85){\makebox(0, 0)[b]{(4, 1)}}
%\thinlines
\put(105, 47){\line(1, 1){35}}

\put(140, 85){\makebox(0, 0)[b]{(1, 0)}}

\put(80, 55){\makebox(0, 0)[b]{$a$}}

\put(130, 55){\makebox(0, 0)[b]{$b$}}


\put(245, 35){\makebox(0, 0)[b]{II}}
\put(245, 47){\circle*{4}}
\put(245, 47){\line(-1, 1){35}}

\put(210, 85){\makebox(0, 0)[b]{(1, 1)}}
%\thicklines
\put(245, 47){\line(1, 1){35}}
%\thinlines

\put(280, 85){\makebox(0, 0)[b]{(2, 2)}}

\put(220, 55){\makebox(0, 0)[b]{$a$}}

\put(270, 55){\makebox(0, 0)[b]{$b$}}

%\multiput(105,47)(5, 0){28}{\line(1, 0){3}}

\end{picture}
\end{frame}

\begin{frame}{Strategies}
\protect\hypertarget{strategies}{}
\begin{itemize}
\tightlist
\item
  A \textbf{strategy} for a game is a set of instructions for what to do
  at each node of a game.
\item
  Even very small game trees there are a lot of possible strategies.
\item
  If there are \(k\) possible nodes a player could have a choice at, and
  \(m\) possible moves at each of these nodes, then there are \(m^k\)
  possible strategies.
\item
  Note that a strategy has to say what to do at nodes that are ruled out
  by your own prior moves.
\end{itemize}
\end{frame}

\begin{frame}{Threat Game Strategies}
\protect\hypertarget{threat-game-strategies}{}
\begin{itemize}
\tightlist
\item
  Let's work through an example of that.
\item
  For player I, there are just two strategies: \(A\) and \(B\).
\item
  For player II, there are two nodes, and two possible choices at each
  node. So there are \(2^2 = 4\) possible strategies.
\item
  We'll write \(xy\) for the strategy of doing \(x\) in response to
  \(A\), and \(y\) in response to \(B\).
\item
  And note I'm capitalising player I's moves, and using lower case for
  player II's moves, to make things clearer.
\end{itemize}
\end{frame}

\begin{frame}{Threat Game Strategies}
\protect\hypertarget{threat-game-strategies-1}{}
Here are the four strategies for player II:

\begin{enumerate}
\tightlist
\item
  \(aa\) - Do \(a\) no matter what.
\item
  \(ab\) - Do whatever player I does.
\item
  \(ba\) - Do the opposite of what player I does.
\item
  \(bb\) - Do \(b\) no matter what.
\end{enumerate}
\end{frame}

\begin{frame}{Threat Game Strategy Tables}
\protect\hypertarget{threat-game-strategy-tables}{}
The strategies for the players determine the outcome. Here is the table
for the game, given the strategies.

\begin{table}[!h]
\centering
\begin{tabular}[t]{>{}r|cccc}
\toprule
 & aa & ab & ba & bb\\
\midrule
A & 4, 1 & 4, 1 & 1, 0 & 1, 0\\
B & 1, 1 & 2, 2 & 1, 1 & 2, 2\\
\bottomrule
\end{tabular}
\end{table}
\end{frame}

\begin{frame}{Threat Game Strategy Tables}
\protect\hypertarget{threat-game-strategy-tables-1}{}
I've put boxes around the best responses.

\begin{table}[!h]
\centering
\begin{tabular}[t]{>{}r|cccc}
\toprule
 & aa & ab & ba & bb\\
\midrule
A & \fbox{4}, \fbox{1} & \fbox{4}, \fbox{1} & \fbox{1}, 0 & 1, 0\\
B & 1, 1 & \fbox{2}, \fbox{2} & \fbox{1}, 1 & \fbox{2}, \fbox{2}\\
\bottomrule
\end{tabular}
\end{table}

There are three Nash equilibria.

\begin{enumerate}
\tightlist
\item
  \(A, aa\) - with result 4, 1
\item
  \(A, ab\) - with result 4, 1
\item
  \(B, bb\) - with result 2, 2
\end{enumerate}
\end{frame}

\begin{frame}[fragile]{Threat Game with Backward Induction}
\protect\hypertarget{threat-game-with-backward-induction}{}
\newcommand{\pictext}[3]{
\put(#1, #2){\makebox(0, 0)[b]{#3}}}

\begin{picture}(350, 110)
\linethickness{1pt}

\put(175, 0){\makebox(0, 0)[b]{I}}
\put(175, 12){\circle*{4}}
\thicklines
\put(175, 12){\line(-2, 1){70}}
\thinlines
\put(175, 12){\line(2, 1){70}}

\put(135, 20){\makebox(0, 0)[b]{$A$}}

\put(215, 20){\makebox(0, 0)[b]{$B$}}


\put(105, 35){\makebox(0, 0)[b]{II}}
\put(105, 47){\circle*{4}}
\thicklines
\put(105, 47){\line(-1, 1){35}}

\put(70, 85){\makebox(0, 0)[b]{(4, 1)}}
\thinlines
\put(105, 47){\line(1, 1){35}}

\put(140, 85){\makebox(0, 0)[b]{(1, 0)}}

\put(80, 55){\makebox(0, 0)[b]{$a$}}

\put(130, 55){\makebox(0, 0)[b]{$b$}}


\put(245, 35){\makebox(0, 0)[b]{II}}
\put(245, 47){\circle*{4}}
\put(245, 47){\line(-1, 1){35}}

\put(210, 85){\makebox(0, 0)[b]{(1, 1)}}
\thicklines
\put(245, 47){\line(1, 1){35}}
\thinlines

\put(280, 85){\makebox(0, 0)[b]{(2, 2)}}

\put(220, 55){\makebox(0, 0)[b]{$a$}}

\put(270, 55){\makebox(0, 0)[b]{$b$}}

%\multiput(105,47)(5, 0){28}{\line(1, 0){3}}

\end{picture}

\begin{itemize}
\tightlist
\item
  I've bolded the best moves at each node, assuming backward induction.
\item
  The path of best moves is the (in this case unique) backward induction
  solution.
\end{itemize}
\end{frame}

\begin{frame}{Threat Game}
\protect\hypertarget{threat-game-1}{}
\begin{itemize}
\tightlist
\item
  There are three Nash equilibria of the game: strategy pairs that no
  one can improve on by unilaterally changing strategy.
\item
  There is just one backward induction solution of the game: a strategy
  pair where everyone does the best they can \textbf{at every node}
  assuming others play rationally at every node.
\end{itemize}
\end{frame}

\begin{frame}{Incredible Threats}
\protect\hypertarget{incredible-threats}{}
What makes \(\langle B, bb \rangle\) a Nash equilibrium is that Player
II can make the following speech.

\begin{quote}
``I'm going to play b whatever you do. I want that 2 payout, and I'm
going to get it. And since I'm going to play b whatever you do, you're
better off playing B. That way you'll get 2, when you'd only get 1 if
you played A. And you can tell I'm not bluffing because this strategy
makes sense for me. Since you'll play B, since I'm committed to always
playing b, it's in my best interests to stick to this strategy.''
\end{quote}
\end{frame}

\begin{frame}{Incredible Threats}
\protect\hypertarget{incredible-threats-1}{}
What makes \(\langle B, bb \rangle\) not subgame perfect, what makes it
an incredible threat, is that A can make the following reply.
\end{frame}

\begin{frame}
\begin{quote}
``That's an interesting plan. And if it was just a strategic game, I
might even believe it. But the problem for you is that you have to stick
to that bluff once you know that it's been called. To commit to always
playing b means playing b even when you know I've played A. And I don't
reckon you'll do it - it's worse for me (which doesn't matter), and it's
worse for you (which does). If we were just choosing strategies, I might
just about believe that you would adopt a disposition that's bad in some
circumstances in the hope that by adopting it, you'll guarantee that
those circumstances don't arise. But when you have to play in real time,
I don't think you can do it.''
\end{quote}
\end{frame}

\begin{frame}{Incredible Threats}
\protect\hypertarget{incredible-threats-2}{}
So I plays A, and they end up at the 4,1 outcome.
\end{frame}

\begin{frame}{For Next Time}
\protect\hypertarget{for-next-time}{}
\begin{itemize}
\tightlist
\item
  We will look, very briefly, at a notable philosophical objection to
  backwards induction reasoning.
\end{itemize}
\end{frame}

\end{document}
