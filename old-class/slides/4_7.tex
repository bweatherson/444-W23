% Options for packages loaded elsewhere
\PassOptionsToPackage{unicode}{hyperref}
\PassOptionsToPackage{hyphens}{url}
%
\documentclass[
  ignorenonframetext,
]{beamer}
\usepackage{pgfpages}
\setbeamertemplate{caption}[numbered]
\setbeamertemplate{caption label separator}{: }
\setbeamercolor{caption name}{fg=normal text.fg}
\beamertemplatenavigationsymbolsempty
% Prevent slide breaks in the middle of a paragraph
\widowpenalties 1 10000
\raggedbottom
\setbeamertemplate{part page}{
  \centering
  \begin{beamercolorbox}[sep=16pt,center]{part title}
    \usebeamerfont{part title}\insertpart\par
  \end{beamercolorbox}
}
\setbeamertemplate{section page}{
  \centering
  \begin{beamercolorbox}[sep=12pt,center]{part title}
    \usebeamerfont{section title}\insertsection\par
  \end{beamercolorbox}
}
\setbeamertemplate{subsection page}{
  \centering
  \begin{beamercolorbox}[sep=8pt,center]{part title}
    \usebeamerfont{subsection title}\insertsubsection\par
  \end{beamercolorbox}
}
\AtBeginPart{
  \frame{\partpage}
}
\AtBeginSection{
  \ifbibliography
  \else
    \frame{\sectionpage}
  \fi
}
\AtBeginSubsection{
  \frame{\subsectionpage}
}
\usepackage{amsmath,amssymb}
\usepackage{lmodern}
\usepackage{ifxetex,ifluatex}
\ifnum 0\ifxetex 1\fi\ifluatex 1\fi=0 % if pdftex
  \usepackage[T1]{fontenc}
  \usepackage[utf8]{inputenc}
  \usepackage{textcomp} % provide euro and other symbols
\else % if luatex or xetex
  \usepackage{unicode-math}
  \defaultfontfeatures{Scale=MatchLowercase}
  \defaultfontfeatures[\rmfamily]{Ligatures=TeX,Scale=1}
  \setmainfont[BoldFont = SF Pro Rounded Semibold]{SF Pro Rounded}
  \setmathfont[]{STIX Two Math}
\fi
\usefonttheme{serif} % use mainfont rather than sansfont for slide text
% Use upquote if available, for straight quotes in verbatim environments
\IfFileExists{upquote.sty}{\usepackage{upquote}}{}
\IfFileExists{microtype.sty}{% use microtype if available
  \usepackage[]{microtype}
  \UseMicrotypeSet[protrusion]{basicmath} % disable protrusion for tt fonts
}{}
\makeatletter
\@ifundefined{KOMAClassName}{% if non-KOMA class
  \IfFileExists{parskip.sty}{%
    \usepackage{parskip}
  }{% else
    \setlength{\parindent}{0pt}
    \setlength{\parskip}{6pt plus 2pt minus 1pt}}
}{% if KOMA class
  \KOMAoptions{parskip=half}}
\makeatother
\usepackage{xcolor}
\IfFileExists{xurl.sty}{\usepackage{xurl}}{} % add URL line breaks if available
\IfFileExists{bookmark.sty}{\usepackage{bookmark}}{\usepackage{hyperref}}
\hypersetup{
  pdftitle={444 Lecture 4.7 - Money and Utility},
  pdfauthor={Brian Weatherson},
  hidelinks,
  pdfcreator={LaTeX via pandoc}}
\urlstyle{same} % disable monospaced font for URLs
\newif\ifbibliography
\setlength{\emergencystretch}{3em} % prevent overfull lines
\providecommand{\tightlist}{%
  \setlength{\itemsep}{0pt}\setlength{\parskip}{0pt}}
\setcounter{secnumdepth}{-\maxdimen} % remove section numbering
\let\Tiny=\tiny

 \setbeamertemplate{navigation symbols}{} 

% \usetheme{Madrid}
 \usetheme[numbering=none, progressbar=foot]{metropolis}
 \usecolortheme{wolverine}
 \usepackage{color}
 \usepackage{MnSymbol}
% \usepackage{movie15}

\usepackage{amssymb}% http://ctan.org/pkg/amssymb
\usepackage{pifont}% http://ctan.org/pkg/pifont
\newcommand{\cmark}{\ding{51}}%
\newcommand{\xmark}{\ding{55}}%

\DeclareSymbolFont{symbolsC}{U}{txsyc}{m}{n}
\DeclareMathSymbol{\boxright}{\mathrel}{symbolsC}{128}
\DeclareMathAlphabet{\mathpzc}{OT1}{pzc}{m}{it}

\setlength{\parskip}{1ex plus 0.5ex minus 0.2ex}

\AtBeginSection[]
{
\begin{frame}
	\Huge{\color{darkblue} \insertsection}
\end{frame}
}

\renewenvironment*{quote}	
	{\list{}{\rightmargin   \leftmargin} \item } 	
	{\endlist }

\definecolor{darkgreen}{rgb}{0,0.7,0}
\definecolor{darkblue}{rgb}{0,0,0.8}

\usepackage[italic]{mathastext}
\usepackage{nicefrac}

\setbeamertemplate{caption}{\raggedright\insertcaption}

%\def\toprule{}
%\def\bottomrule{}
%\def\midrule{}
\usepackage{etoolbox}
\AfterEndEnvironment{description}{\vspace{9pt}}
\AfterEndEnvironment{oltableau}{\vspace{9pt}}
\BeforeBeginEnvironment{oltableau}{\vspace{9pt}}
\AfterEndEnvironment{center}{\vspace{9pt}}
\BeforeBeginEnvironment{tabular}{\vspace{9pt}}
\AfterEndEnvironment{longtable}{\vspace{-6pt}}
\usepackage{booktabs}
\usepackage{longtable}
\usepackage{array}
\usepackage{multirow}
\usepackage{wrapfig}
\usepackage{float}
\usepackage{colortbl}
\usepackage{pdflscape}
\usepackage{tabu}
\usepackage{threeparttable} 
\usepackage{threeparttablex} 
\usepackage[normalem]{ulem} 
\usepackage{makecell}
\usepackage{xcolor}
\usepackage{ulem}

\setlength\heavyrulewidth{0ex}
\setlength\lightrulewidth{0.08ex}

\aboverulesep=0ex
\belowrulesep=0ex
\renewcommand{\arraystretch}{1.2}
\ifluatex
  \usepackage{selnolig}  % disable illegal ligatures
\fi

\title{444 Lecture 4.7 - Money and Utility}
\author{Brian Weatherson}
\date{}

\begin{document}
\frame{\titlepage}

\begin{frame}{Plan}
\protect\hypertarget{plan}{}
\begin{itemize}
\tightlist
\item
  In this lecture we'll talk about the relationship between money and
  utility.
\end{itemize}
\end{frame}

\begin{frame}{Associated Reading}
\protect\hypertarget{associated-reading}{}
Still chapter 5, though we're really not going page by page through this
chapter.
\end{frame}

\begin{frame}{Don't Equate Dollars and Money}
\protect\hypertarget{dont-equate-dollars-and-money}{}
That's the message!
\end{frame}

\begin{frame}{Don't Equate Dollars and Money}
\protect\hypertarget{dont-equate-dollars-and-money-1}{}
\begin{enumerate}[<+->]
\tightlist
\item
  Players might care how much money other players get.
\item
  Players might not assign the same utility to each dollar.
\end{enumerate}
\end{frame}

\begin{frame}{Altruism}
\protect\hypertarget{altruism}{}
Consider an outcome where Player 1 gets \$10 and Player 2 gets \$20.

\begin{itemize}
\tightlist
\item
  Don't just say that's a utility of 10 for Player 1 and 20 for Player
  2.
\item
  Maybe Player 1 likes Player 2, and is happy they get some money - so
  this outcome has higher utility than one where Player 1 gets \$10 and
  Player 2 gets \$10.
\item
  Maybe Player 1 has a hypotrophied sense of fairness and hates other
  people getting more - so this outcome has lower utility than one where
  Player 1 gets \$10 and Player 2 gets \$10.
\end{itemize}
\end{frame}

\begin{frame}{Resolving Social Problems}
\protect\hypertarget{resolving-social-problems}{}
\begin{itemize}
\tightlist
\item
  Stepping back a bit, one of the big uses of game theory is in
  institutional design.
\item
  We look at what game we are asking people to play, and use game theory
  to predict what they'll play, and if we don't like the answer, we
  think about changing the game to get a better one.
\item
  Sometimes the most efficient way to get change is to change values -
  to get people to care about outcomes for others.
\end{itemize}
\end{frame}

\begin{frame}{Marginal Utility and Decision Making}
\protect\hypertarget{marginal-utility-and-decision-making}{}
\begin{itemize}
\tightlist
\item
  Getting \$2x is not twice as valuable as getting \$x.
\item
  That's because it's like getting \$x, then getting \$x again.
\item
  And after you get the first \$x, you're richer, and getting \$x is (in
  general) less valuable to richer people.
\end{itemize}
\end{frame}

\begin{frame}{Philosophical Point}
\protect\hypertarget{philosophical-point}{}
It's relatively uncontroversial that the following two things are true.
The philosophical claim that lies behind the theory I'm setting out in
these slides is that they are closely connected.

\begin{enumerate}
\tightlist
\item
  You're better off getting a million dollars than getting a 50/50 shot
  at two million dollars.
\item
  Getting a million dollars changes your life more than it changes the
  life of a billionaire.
\end{enumerate}

Both are grounded in the fact that the more money you have, the less
utility each extra dollar has.
\end{frame}

\begin{frame}{Utility and Money}
\protect\hypertarget{utility-and-money}{}
The graph of the relationship between utility and money should have the
following two features.

\begin{enumerate}
\tightlist
\item
  More money means more utility.
\item
  The amount of extra utility you get for each extra dollar should be
  decreasing
\end{enumerate}

Arguably utility rises with something like log of wealth.
\end{frame}

\begin{frame}{Insurance}
\protect\hypertarget{insurance}{}
Insurance is a funny business.

\begin{itemize}
\tightlist
\item
  Every insurance contract is a bet, with you and the insurance company
  on opposite sides of it.
\item
  The bet can't, as a matter of almost mathematical necessity, have a
  positive expected dollar return for both of you.
\item
  And given it involves some transaction costs, it could have a negative
  expected dollar return for both of you.
\item
  So why does the industry even exist?
\end{itemize}
\end{frame}

\begin{frame}{Declining Marginal Utility}
\protect\hypertarget{declining-marginal-utility}{}
Well let's work through an example.

\begin{itemize}
\tightlist
\item
  Assume our person has assets of \$100,000, including a car worth
  \$30,000.
\item
  They live in a risky area, so there is a 1 in 10 chance the car will
  fall in value to 0 over the next 12 months.
\item
  They are offered an insurance contract with the following terms.
\item
  They pay \$3,200.
\item
  If the risky thing happens and the car value falls to 0, the insurance
  company will reimburse them, so they will get the \$30,000 back.
\end{itemize}
\end{frame}

\begin{frame}{Should They Take the Deal}
\protect\hypertarget{should-they-take-the-deal}{}
Outcome if they take the deal

\begin{itemize}
\tightlist
\item
  A guaranteed \$96,800. \pause
\end{itemize}

Outcome if they don't take the deal.

\begin{itemize}
\tightlist
\item
  A 90\% chance of \$100,000.
\item
  A 10\% chance of \$70,000. \pause
\end{itemize}

The latter outcome has an expected dollar return of \$97,000 - that's
\(0.9 \times 100,000 + 0.1 \times 70,000\).
\end{frame}

\begin{frame}{Should They Take the Deal}
\protect\hypertarget{should-they-take-the-deal-1}{}
\begin{itemize}
\tightlist
\item
  But this doesn't settle the matter. We care about utility not dollars.
\item
  Let's re-run the question using utility.
\end{itemize}
\end{frame}

\begin{frame}{Should They Take the Deal}
\protect\hypertarget{should-they-take-the-deal-2}{}
Outcome if they take the deal

\begin{itemize}
\tightlist
\item
  A guaranteed \$96,800, which has utility roughly 4.986 \pause
\end{itemize}

Outcome if they don't take the deal.

\begin{itemize}
\tightlist
\item
  A 90\% chance of \$100,000, which has utility 5.
\item
  A 10\% chance of \$70,000, which has utility roughly 4.845 \pause
\end{itemize}

The latter outcome has an expected utility return of roughly
\(0.9 \times 5 + 0.1 \times 4.845 \approx 4.984\). Option 1 is better -
not by much, but better.
\end{frame}

\begin{frame}{Company Point of View}
\protect\hypertarget{company-point-of-view}{}
\begin{itemize}
\tightlist
\item
  Assume (for now) that they have a constant marginal utility of money.
\item
  So all that matters is that the policy has a positive dollar value.
\item
  And the expected dollar return of the deal is +\$200, so it's good for
  the company as well.
\end{itemize}
\end{frame}

\begin{frame}{Success!}
\protect\hypertarget{success}{}
\begin{itemize}
\tightlist
\item
  We found a case where both parties are rational in taking the bet,
  even though they are on opposite sides of it.
\item
  And this doesn't require fraud, or misperception of the odds for
  either party.
\end{itemize}
\end{frame}

\begin{frame}{Possibility Constraints}
\protect\hypertarget{possibility-constraints}{}
\begin{itemize}
\tightlist
\item
  This is only possible because the two sides have different utility
  curves, at least locally.
\item
  That's what makes the conflicting interests (in dollar terms) into a
  possible mutual interest.
\item
  Someone with a less steeply sloping utility curve (i.e., with more
  resources) is in a better position to absorb certain risks.
\item
  It is worth paying over the odds to them to absorb that risk.
\end{itemize}
\end{frame}

\begin{frame}{Curves (Almost) Always Slope Down}
\protect\hypertarget{curves-almost-always-slope-down}{}
\begin{itemize}
\tightlist
\item
  But eventually, the insurance company has risks it shudders at as
  well.
\item
  This only happens on enormous scale, but it happens.
\item
  And it's why insurance companies won't (happily) offer insurance
  against correlated risks, like floods or invasion.
\end{itemize}
\end{frame}

\begin{frame}{A Big Caveat}
\protect\hypertarget{a-big-caveat}{}
When you run the numbers on cases like this, three things come out.

\begin{enumerate}
\tightlist
\item
  Sometimes, insurance is good for both parties. \pause
\item
  Unless the loss is a huge portion of the customer's wealth, the
  numbers end up being really close. \pause
\item
  Even in those cases, the numbers aren't that different.
\end{enumerate}

So I end up thinking that people probably over-purchase insurance, even
though this is a model on which insurance purchase can be rational.
\end{frame}

\begin{frame}{For Next Time}
\protect\hypertarget{for-next-time}{}
\begin{itemize}
\tightlist
\item
  We will take a very quick look at the nature of probability.
\end{itemize}
\end{frame}

\end{document}
