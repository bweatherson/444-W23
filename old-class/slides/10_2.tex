% Options for packages loaded elsewhere
\PassOptionsToPackage{unicode}{hyperref}
\PassOptionsToPackage{hyphens}{url}
%
\documentclass[
  ignorenonframetext,
]{beamer}
\usepackage{pgfpages}
\setbeamertemplate{caption}[numbered]
\setbeamertemplate{caption label separator}{: }
\setbeamercolor{caption name}{fg=normal text.fg}
\beamertemplatenavigationsymbolsempty
% Prevent slide breaks in the middle of a paragraph
\widowpenalties 1 10000
\raggedbottom
\setbeamertemplate{part page}{
  \centering
  \begin{beamercolorbox}[sep=16pt,center]{part title}
    \usebeamerfont{part title}\insertpart\par
  \end{beamercolorbox}
}
\setbeamertemplate{section page}{
  \centering
  \begin{beamercolorbox}[sep=12pt,center]{part title}
    \usebeamerfont{section title}\insertsection\par
  \end{beamercolorbox}
}
\setbeamertemplate{subsection page}{
  \centering
  \begin{beamercolorbox}[sep=8pt,center]{part title}
    \usebeamerfont{subsection title}\insertsubsection\par
  \end{beamercolorbox}
}
\AtBeginPart{
  \frame{\partpage}
}
\AtBeginSection{
  \ifbibliography
  \else
    \frame{\sectionpage}
  \fi
}
\AtBeginSubsection{
  \frame{\subsectionpage}
}
\usepackage{amsmath,amssymb}
\usepackage{lmodern}
\usepackage{ifxetex,ifluatex}
\ifnum 0\ifxetex 1\fi\ifluatex 1\fi=0 % if pdftex
  \usepackage[T1]{fontenc}
  \usepackage[utf8]{inputenc}
  \usepackage{textcomp} % provide euro and other symbols
\else % if luatex or xetex
  \usepackage{unicode-math}
  \defaultfontfeatures{Scale=MatchLowercase}
  \defaultfontfeatures[\rmfamily]{Ligatures=TeX,Scale=1}
  \setmainfont[BoldFont = SF Pro Rounded Semibold]{SF Pro Rounded}
  \setmathfont[]{STIX Two Math}
\fi
\usefonttheme{serif} % use mainfont rather than sansfont for slide text
% Use upquote if available, for straight quotes in verbatim environments
\IfFileExists{upquote.sty}{\usepackage{upquote}}{}
\IfFileExists{microtype.sty}{% use microtype if available
  \usepackage[]{microtype}
  \UseMicrotypeSet[protrusion]{basicmath} % disable protrusion for tt fonts
}{}
\makeatletter
\@ifundefined{KOMAClassName}{% if non-KOMA class
  \IfFileExists{parskip.sty}{%
    \usepackage{parskip}
  }{% else
    \setlength{\parindent}{0pt}
    \setlength{\parskip}{6pt plus 2pt minus 1pt}}
}{% if KOMA class
  \KOMAoptions{parskip=half}}
\makeatother
\usepackage{xcolor}
\IfFileExists{xurl.sty}{\usepackage{xurl}}{} % add URL line breaks if available
\IfFileExists{bookmark.sty}{\usepackage{bookmark}}{\usepackage{hyperref}}
\hypersetup{
  pdftitle={444 Lecture 10.2 - O'Connor Chapter 5},
  pdfauthor={Brian Weatherson},
  hidelinks,
  pdfcreator={LaTeX via pandoc}}
\urlstyle{same} % disable monospaced font for URLs
\newif\ifbibliography
\setlength{\emergencystretch}{3em} % prevent overfull lines
\providecommand{\tightlist}{%
  \setlength{\itemsep}{0pt}\setlength{\parskip}{0pt}}
\setcounter{secnumdepth}{-\maxdimen} % remove section numbering
\let\Tiny=\tiny

 \setbeamertemplate{navigation symbols}{} 

% \usetheme{Madrid}
 \usetheme[numbering=none, progressbar=foot]{metropolis}
 \usecolortheme{wolverine}
 \usepackage{color}
 \usepackage{MnSymbol}
% \usepackage{movie15}

\usepackage{amssymb}% http://ctan.org/pkg/amssymb
\usepackage{pifont}% http://ctan.org/pkg/pifont
\newcommand{\cmark}{\ding{51}}%
\newcommand{\xmark}{\ding{55}}%

\DeclareSymbolFont{symbolsC}{U}{txsyc}{m}{n}
\DeclareMathSymbol{\boxright}{\mathrel}{symbolsC}{128}
\DeclareMathAlphabet{\mathpzc}{OT1}{pzc}{m}{it}

\setlength{\parskip}{1ex plus 0.5ex minus 0.2ex}

\AtBeginSection[]
{
\begin{frame}
	\Huge{\color{darkblue} \insertsection}
\end{frame}
}

\renewenvironment*{quote}	
	{\list{}{\rightmargin   \leftmargin} \item } 	
	{\endlist }

\definecolor{darkgreen}{rgb}{0,0.7,0}
\definecolor{darkblue}{rgb}{0,0,0.8}

\usepackage[italic]{mathastext}
\usepackage{nicefrac}
\usepackage{istgame}

\setbeamertemplate{caption}{\raggedright\insertcaption}

%\def\toprule{}
%\def\bottomrule{}
%\def\midrule{}
\usepackage{etoolbox}
\AfterEndEnvironment{description}{\vspace{9pt}}
\AfterEndEnvironment{oltableau}{\vspace{9pt}}
\BeforeBeginEnvironment{oltableau}{\vspace{9pt}}
\AfterEndEnvironment{center}{\vspace{9pt}}
\BeforeBeginEnvironment{tabular}{\vspace{9pt}}
\AfterEndEnvironment{longtable}{\vspace{-6pt}}
\usepackage{booktabs}
\usepackage{longtable}
\usepackage{array}
\usepackage{multirow}
\usepackage{wrapfig}
\usepackage{float}
\usepackage{colortbl}
\usepackage{pdflscape}
\usepackage{tabu}
\usepackage{threeparttable} 
\usepackage{threeparttablex} 
\usepackage[normalem]{ulem} 
\usepackage{makecell}
\usepackage{xcolor}
\usepackage{ulem}

\setlength\heavyrulewidth{0ex}
\setlength\lightrulewidth{0.08ex}

\aboverulesep=0ex
\belowrulesep=0ex
\renewcommand{\arraystretch}{1.2}
\AtBeginSection[]
{
    \begin{frame}
        \frametitle{Day Plan}
        \tableofcontents[currentsection]
    \end{frame}
}
\ifluatex
  \usepackage{selnolig}  % disable illegal ligatures
\fi

\title{444 Lecture 10.2 - O'Connor Chapter 5}
\author{Brian Weatherson}
\date{}

\begin{document}
\frame{\titlepage}

\begin{frame}{Day Plan}
\protect\hypertarget{day-plan}{}
\tableofcontents
\end{frame}

\hypertarget{demand-game}{%
\section{Demand Game}\label{demand-game}}

\begin{frame}{Demand Game}
\protect\hypertarget{demand-game-1}{}
First, a brief note on the structure of the games at the heart of this
chapter.

\begin{itemize}
\tightlist
\item
  These are simultaneous move games.
\item
  They are not like the ultimatum game that you may have heard about.
\item
  Nor are they like real world negotiations.
\end{itemize}
\end{frame}

\begin{frame}{Negotiations}
\protect\hypertarget{negotiations}{}
But they are a bit like negotiations.

\begin{itemize}
\tightlist
\item
  They are at least a little bit like strategies for a real world
  negotiation, espcially if it works by something like English Auction.
\item
  The numbers are something like a reserve price.
\item
  In principle you could complicate the game a bit more by adding in
  extra strategies within each round.
\item
  But this is probably the best way to think about it.
\end{itemize}
\end{frame}

\hypertarget{basins}{%
\section{Basins}\label{basins}}

\begin{frame}{Basins of Attraction}
\protect\hypertarget{basins-of-attraction}{}
I don't have much to say here, but I really wanted to draw attention to
the very surprising graph on page 114 (figure 5.3).

\begin{itemize}
\tightlist
\item
  I guess up to this point most of the models hadn't been \emph{that}
  different from what I would have guessed a priori.
\item
  But this one really was surprisingly different.
\item
  Would be kind of interested in running another version of this with
  multiple overlapping games.
\end{itemize}
\end{frame}

\hypertarget{equity}{%
\section{Equity}\label{equity}}

\begin{frame}{What is Fair}
\protect\hypertarget{what-is-fair}{}
\begin{itemize}
\tightlist
\item
  We've talked about this a bit, but it's really worth thinking about
  what counts as a `fair' distribution.
\item
  This can have effects both for the payoffs (people value fairness) and
  for dynamics (people move towards fair)
\end{itemize}
\end{frame}

\begin{frame}{Fairness and Markets}
\protect\hypertarget{fairness-and-markets}{}
\begin{itemize}
\tightlist
\item
  This matters a lot in market economies.
\item
  Often the fair outcome is the one driven by the market.
\item
  And that's true even if different forms of market infrastrcture would
  have produced different outcomes.
\end{itemize}
\end{frame}

\begin{frame}{Fairness and Deontology}
\protect\hypertarget{fairness-and-deontology}{}
\begin{itemize}
\tightlist
\item
  It also matters in the contexts of norms like ``Don't Steal''.
\item
  Sometimes whether an outcome is coded as fair depends on how it
  relates to an initial condition that we accept as fair.
\item
  And that in turn might depend on facts that, if we thought about them
  directly, we would not think of as morally significant.
\end{itemize}
\end{frame}

\hypertarget{disagreement-points}{%
\section{Disagreement Points}\label{disagreement-points}}

\begin{frame}{Disagreement Points}
\protect\hypertarget{disagreement-points-1}{}
The fact that different people are differentially able to walk away from
a game is really important. But\ldots{}

\begin{itemize}
\tightlist
\item
  I'm kinda suspicious of Figure 5.5 (page 119).
\item
  The basins of attraction of the equilibria where the type with more
  ability to walk away end up with less are surprisingly large.
\item
  Is there a real world situation that is like this?
\item
  Or is this a case where the model doesn't really reflect reality?
\end{itemize}
\end{frame}

\hypertarget{punishment}{%
\section{Punishment}\label{punishment}}

\begin{frame}{Punishment}
\protect\hypertarget{punishment-1}{}
I've been worried a bit over the course of this that we've changed what
we're talking about when we discuss punishment. In particular, do we
mean:

\begin{enumerate}
\tightlist
\item
  Changing the payoffs; or
\item
  Choosing a strategy that leads to lower payoffs for the `punished'.
\end{enumerate}

These are both important things, but I'm not sure they are the same
thing
\end{frame}

\begin{frame}{Punishment}
\protect\hypertarget{punishment-2}{}
And around page 126 we see some movement between these two notions.

\begin{itemize}
\tightlist
\item
  In the game table, punishment for a high bid is depicted as the
  payoffs being externally lowered.
\item
  But then we get discussions of non-equilibrium moves within games.
\item
  Maybe these are the same thing?
\item
  Feels like we should keep them separate.
\end{itemize}
\end{frame}

\hypertarget{varieties-of-disagreement-points}{%
\section{Varieties of Disagreement
Points}\label{varieties-of-disagreement-points}}

\begin{frame}{Population Divergence}
\protect\hypertarget{population-divergence}{}
I really liked the stuff around page 126 on what happens if there is
divergence within the population.

\begin{itemize}
\tightlist
\item
  Most of the games so far have essentially presupposed uniform
  populations, at least within types.
\item
  Here we get a nice effect of the existence of a sub-population within
  one but not the other type.
\end{itemize}
\end{frame}

\begin{frame}{Credible Signals}
\protect\hypertarget{credible-signals}{}
And this I think really does matter to the real world.

\begin{itemize}
\tightlist
\item
  Sometimes what matters in these games is not what your disagreement
  point is, but what you can credibly signal that it is.
\item
  We're getting back here to things that would come up if we modelled
  each interaction as a negotiation over time.
\item
  Anyway, sometimes it is really obvious what your type is, but there is
  no way to credibly signal willingness to walk away.
\item
  And in that case, the other player might (quite rationally!) assume
  that your disagreement point is something like the average of your
  type, whatever non-credible signal you send.
\end{itemize}
\end{frame}

\hypertarget{building-power}{%
\section{Building Power}\label{building-power}}

\begin{frame}{Building Power}
\protect\hypertarget{building-power-1}{}
I was a bit confused by the game on page 128 and after, where success
causes one to have a higher disagreement point.

\begin{itemize}
\tightlist
\item
  I don't really see the causal mechanism for this; it seemed much more
  plausible in the other direction.
\item
  Maybe it's that the returns to the game come in resources that can be
  saved?
\item
  Maybe it's that there will be other games to play - where your type by
  default gets the good side of the equilibrium?
\item
  It's a nice game, but I would like to hear more about the real world
  application.
\end{itemize}
\end{frame}

\end{document}
