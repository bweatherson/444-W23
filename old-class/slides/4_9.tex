% Options for packages loaded elsewhere
\PassOptionsToPackage{unicode}{hyperref}
\PassOptionsToPackage{hyphens}{url}
%
\documentclass[
  ignorenonframetext,
]{beamer}
\usepackage{pgfpages}
\setbeamertemplate{caption}[numbered]
\setbeamertemplate{caption label separator}{: }
\setbeamercolor{caption name}{fg=normal text.fg}
\beamertemplatenavigationsymbolsempty
% Prevent slide breaks in the middle of a paragraph
\widowpenalties 1 10000
\raggedbottom
\setbeamertemplate{part page}{
  \centering
  \begin{beamercolorbox}[sep=16pt,center]{part title}
    \usebeamerfont{part title}\insertpart\par
  \end{beamercolorbox}
}
\setbeamertemplate{section page}{
  \centering
  \begin{beamercolorbox}[sep=12pt,center]{part title}
    \usebeamerfont{section title}\insertsection\par
  \end{beamercolorbox}
}
\setbeamertemplate{subsection page}{
  \centering
  \begin{beamercolorbox}[sep=8pt,center]{part title}
    \usebeamerfont{subsection title}\insertsubsection\par
  \end{beamercolorbox}
}
\AtBeginPart{
  \frame{\partpage}
}
\AtBeginSection{
  \ifbibliography
  \else
    \frame{\sectionpage}
  \fi
}
\AtBeginSubsection{
  \frame{\subsectionpage}
}
\usepackage{amsmath,amssymb}
\usepackage{lmodern}
\usepackage{ifxetex,ifluatex}
\ifnum 0\ifxetex 1\fi\ifluatex 1\fi=0 % if pdftex
  \usepackage[T1]{fontenc}
  \usepackage[utf8]{inputenc}
  \usepackage{textcomp} % provide euro and other symbols
\else % if luatex or xetex
  \usepackage{unicode-math}
  \defaultfontfeatures{Scale=MatchLowercase}
  \defaultfontfeatures[\rmfamily]{Ligatures=TeX,Scale=1}
  \setmainfont[BoldFont = SF Pro Rounded Semibold]{SF Pro Rounded}
  \setmathfont[]{STIX Two Math}
\fi
\usefonttheme{serif} % use mainfont rather than sansfont for slide text
% Use upquote if available, for straight quotes in verbatim environments
\IfFileExists{upquote.sty}{\usepackage{upquote}}{}
\IfFileExists{microtype.sty}{% use microtype if available
  \usepackage[]{microtype}
  \UseMicrotypeSet[protrusion]{basicmath} % disable protrusion for tt fonts
}{}
\makeatletter
\@ifundefined{KOMAClassName}{% if non-KOMA class
  \IfFileExists{parskip.sty}{%
    \usepackage{parskip}
  }{% else
    \setlength{\parindent}{0pt}
    \setlength{\parskip}{6pt plus 2pt minus 1pt}}
}{% if KOMA class
  \KOMAoptions{parskip=half}}
\makeatother
\usepackage{xcolor}
\IfFileExists{xurl.sty}{\usepackage{xurl}}{} % add URL line breaks if available
\IfFileExists{bookmark.sty}{\usepackage{bookmark}}{\usepackage{hyperref}}
\hypersetup{
  pdftitle={444 Lecture 4.9 - Expected Value},
  pdfauthor={Brian Weatherson},
  hidelinks,
  pdfcreator={LaTeX via pandoc}}
\urlstyle{same} % disable monospaced font for URLs
\newif\ifbibliography
\setlength{\emergencystretch}{3em} % prevent overfull lines
\providecommand{\tightlist}{%
  \setlength{\itemsep}{0pt}\setlength{\parskip}{0pt}}
\setcounter{secnumdepth}{-\maxdimen} % remove section numbering
\let\Tiny=\tiny

 \setbeamertemplate{navigation symbols}{} 

% \usetheme{Madrid}
 \usetheme[numbering=none, progressbar=foot]{metropolis}
 \usecolortheme{wolverine}
 \usepackage{color}
 \usepackage{MnSymbol}
% \usepackage{movie15}

\usepackage{amssymb}% http://ctan.org/pkg/amssymb
\usepackage{pifont}% http://ctan.org/pkg/pifont
\newcommand{\cmark}{\ding{51}}%
\newcommand{\xmark}{\ding{55}}%

\DeclareSymbolFont{symbolsC}{U}{txsyc}{m}{n}
\DeclareMathSymbol{\boxright}{\mathrel}{symbolsC}{128}
\DeclareMathAlphabet{\mathpzc}{OT1}{pzc}{m}{it}

\setlength{\parskip}{1ex plus 0.5ex minus 0.2ex}

\AtBeginSection[]
{
\begin{frame}
	\Huge{\color{darkblue} \insertsection}
\end{frame}
}

\renewenvironment*{quote}	
	{\list{}{\rightmargin   \leftmargin} \item } 	
	{\endlist }

\definecolor{darkgreen}{rgb}{0,0.7,0}
\definecolor{darkblue}{rgb}{0,0,0.8}

\usepackage[italic]{mathastext}
\usepackage{nicefrac}

\setbeamertemplate{caption}{\raggedright\insertcaption}

%\def\toprule{}
%\def\bottomrule{}
%\def\midrule{}
\usepackage{etoolbox}
\AfterEndEnvironment{description}{\vspace{9pt}}
\AfterEndEnvironment{oltableau}{\vspace{9pt}}
\BeforeBeginEnvironment{oltableau}{\vspace{9pt}}
\AfterEndEnvironment{center}{\vspace{9pt}}
\BeforeBeginEnvironment{tabular}{\vspace{9pt}}
\AfterEndEnvironment{longtable}{\vspace{-6pt}}
\usepackage{booktabs}
\usepackage{longtable}
\usepackage{array}
\usepackage{multirow}
\usepackage{wrapfig}
\usepackage{float}
\usepackage{colortbl}
\usepackage{pdflscape}
\usepackage{tabu}
\usepackage{threeparttable} 
\usepackage{threeparttablex} 
\usepackage[normalem]{ulem} 
\usepackage{makecell}
\usepackage{xcolor}
\usepackage{ulem}

\setlength\heavyrulewidth{0ex}
\setlength\lightrulewidth{0.08ex}

\aboverulesep=0ex
\belowrulesep=0ex
\renewcommand{\arraystretch}{1.2}
\ifluatex
  \usepackage{selnolig}  % disable illegal ligatures
\fi

\title{444 Lecture 4.9 - Expected Value}
\author{Brian Weatherson}
\date{}

\begin{document}
\frame{\titlepage}

\begin{frame}{Plan}
\protect\hypertarget{plan}{}
\begin{itemize}
\tightlist
\item
  In this lecture we'll do a very quick introduction to the idea of
  expected value.
\end{itemize}
\end{frame}

\begin{frame}{Associated Reading}
\protect\hypertarget{associated-reading}{}
Still chapter 5, though we're really not going page by page through this
chapter.
\end{frame}

\begin{frame}{Random Variables}
\protect\hypertarget{random-variables}{}
\begin{itemize}
\tightlist
\item
  A \textbf{random variable} is simply a variable that takes different
  numerical values in different states.
\item
  In other words, it is a function from possibilities to numbers.
\item
  It need not be `random' in any familiar sense.
\item
  The function from possible situations to the value of 2 + 2 in that
  situation is a random variable, albeit a constant one.
\item
  It's just a slightly confusing term for any variable that takes
  different, numerical, values in different situations.
\end{itemize}
\end{frame}

\begin{frame}{Labels}
\protect\hypertarget{labels}{}
\begin{itemize}
\tightlist
\item
  Typically, random variables are denoted by capital letters.
\item
  So we might have a random variable \(X\) whose value is the age of the
  next President of the United States, and his or her inauguration.
\item
  Or we might have a random variable \(Y\) that is the number of
  children you will have in your lifetime.
\item
  Basically any mapping from possibilities to numbers can be a random
  variable.
\end{itemize}
\end{frame}

\begin{frame}{An Example}
\protect\hypertarget{an-example}{}
\begin{itemize}
\tightlist
\item
  You've asked each of your friends who will win the Lakers v Clippers
  game.
\item
  12 said the Lakers will win.
\item
  7 said the Clippers will win. \pause
\item
  Then we can let \(X\) be a random variable measuring the number of
  your friends who correctly predicted the result of the game.
\end{itemize}

\begin{equation*}
X = 
    \begin{cases}
        12,& \text{if Lakers win} ,\\ 
        7,& \text{if Clippers win} .
    \end{cases}
\end{equation*}
\end{frame}

\begin{frame}{Expected Value}
\protect\hypertarget{expected-value}{}
\begin{itemize}
\tightlist
\item
  Given a random variable \(X\) and a probability function \(Pr\), we
  can work out the \textbf{expected value} of that random variable with
  respect to that probability function.
\item
  Intuitively, the expected value of \(X\) is a weighted average of the
  possible values of \(X\), where the weights are given by the
  probability (according to \(Pr\)) of each value coming about.
\end{itemize}
\end{frame}

\begin{frame}{Calculating Expected Value}
\protect\hypertarget{calculating-expected-value}{}
\begin{itemize}
\tightlist
\item
  More formally, we work out the expected value of \(X\) this way.
\item
  For each possibility, we multiply the value of \(X\) in that case by
  the probability of the possibility obtaining.
\item
  Then we sum the numbers we've got, and the result is the expected
  value of \(X\).
\item
  We'll write the expected value of \(X\) as \(Exp(X)\).
\end{itemize}
\end{frame}

\begin{frame}{Back to the Example}
\protect\hypertarget{back-to-the-example}{}
\begin{itemize}
\tightlist
\item
  So if the probability that the Lakers win is 0.7, and the probability
  that the Clippers win is 0.3, then
\end{itemize}

\begin{align*}
Exp(X) &= 12 \times 0.7 + 7 \times 0.3 \\
 &= 8.4 + 2.1 \\
 &= 10.5
\end{align*}
\end{frame}

\begin{frame}{Notes}
\protect\hypertarget{notes}{}
\begin{enumerate}
\tightlist
\item
  The expected value of \(X\) isn't in any sense the value that we
  expect \(X\) to take. It's more like an average.
\item
  If this kind of situation recurs a lot, you would expect the long run
  average value \(X\) takes to be roundabout the expected value.
\item
  That's a better way of conceptualising what expected values are.
\end{enumerate}
\end{frame}

\begin{frame}{Summing Up}
\protect\hypertarget{summing-up}{}
The standard theory of decisions under uncertainty requires three
conceptual innovations.

\begin{enumerate}
\tightlist
\item
  Utility, understood as a measure of how well things are for the
  decider, and defined in a way such that ratios of differences are
  meaningful.
\item
  Probability, understood as measuring the likelihood of classes of
  outcomes.
\item
  Expected value, understood as something generated by multiplying
  probabilities of an outcome by the value of the random variable in
  that outcome.
\end{enumerate}
\end{frame}

\begin{frame}{For Next Time}
\protect\hypertarget{for-next-time}{}
\begin{itemize}
\tightlist
\item
  We will start looking at how to use these tools to analyse games that
  we couldn't analyse with purely ordinal utility
\end{itemize}
\end{frame}

\end{document}
