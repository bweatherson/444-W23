% Options for packages loaded elsewhere
\PassOptionsToPackage{unicode}{hyperref}
\PassOptionsToPackage{hyphens}{url}
%
\documentclass[
  ignorenonframetext,
]{beamer}
\usepackage{pgfpages}
\setbeamertemplate{caption}[numbered]
\setbeamertemplate{caption label separator}{: }
\setbeamercolor{caption name}{fg=normal text.fg}
\beamertemplatenavigationsymbolsempty
% Prevent slide breaks in the middle of a paragraph
\widowpenalties 1 10000
\raggedbottom
\setbeamertemplate{part page}{
  \centering
  \begin{beamercolorbox}[sep=16pt,center]{part title}
    \usebeamerfont{part title}\insertpart\par
  \end{beamercolorbox}
}
\setbeamertemplate{section page}{
  \centering
  \begin{beamercolorbox}[sep=12pt,center]{part title}
    \usebeamerfont{section title}\insertsection\par
  \end{beamercolorbox}
}
\setbeamertemplate{subsection page}{
  \centering
  \begin{beamercolorbox}[sep=8pt,center]{part title}
    \usebeamerfont{subsection title}\insertsubsection\par
  \end{beamercolorbox}
}
\AtBeginPart{
  \frame{\partpage}
}
\AtBeginSection{
  \ifbibliography
  \else
    \frame{\sectionpage}
  \fi
}
\AtBeginSubsection{
  \frame{\subsectionpage}
}
\usepackage{lmodern}
\usepackage{amssymb,amsmath}
\usepackage{ifxetex,ifluatex}
\ifnum 0\ifxetex 1\fi\ifluatex 1\fi=0 % if pdftex
  \usepackage[T1]{fontenc}
  \usepackage[utf8]{inputenc}
  \usepackage{textcomp} % provide euro and other symbols
\else % if luatex or xetex
  \usepackage{unicode-math}
  \defaultfontfeatures{Scale=MatchLowercase}
  \defaultfontfeatures[\rmfamily]{Ligatures=TeX,Scale=1}
  \setmainfont[BoldFont = SF Pro Rounded Semibold]{SF Pro Rounded}
  \setmathfont[]{STIX Two Math}
\fi
\usefonttheme{serif} % use mainfont rather than sansfont for slide text
% Use upquote if available, for straight quotes in verbatim environments
\IfFileExists{upquote.sty}{\usepackage{upquote}}{}
\IfFileExists{microtype.sty}{% use microtype if available
  \usepackage[]{microtype}
  \UseMicrotypeSet[protrusion]{basicmath} % disable protrusion for tt fonts
}{}
\makeatletter
\@ifundefined{KOMAClassName}{% if non-KOMA class
  \IfFileExists{parskip.sty}{%
    \usepackage{parskip}
  }{% else
    \setlength{\parindent}{0pt}
    \setlength{\parskip}{6pt plus 2pt minus 1pt}}
}{% if KOMA class
  \KOMAoptions{parskip=half}}
\makeatother
\usepackage{xcolor}
\IfFileExists{xurl.sty}{\usepackage{xurl}}{} % add URL line breaks if available
\IfFileExists{bookmark.sty}{\usepackage{bookmark}}{\usepackage{hyperref}}
\hypersetup{
  pdftitle={444 Lecture 2.4 - Dominance},
  pdfauthor={Brian Weatherson},
  hidelinks,
  pdfcreator={LaTeX via pandoc}}
\urlstyle{same} % disable monospaced font for URLs
\newif\ifbibliography
\setlength{\emergencystretch}{3em} % prevent overfull lines
\providecommand{\tightlist}{%
  \setlength{\itemsep}{0pt}\setlength{\parskip}{0pt}}
\setcounter{secnumdepth}{-\maxdimen} % remove section numbering
\let\Tiny=\tiny

 \setbeamertemplate{navigation symbols}{} 

% \usetheme{Madrid}
 \usetheme[numbering=none, progressbar=foot]{metropolis}
 \usecolortheme{wolverine}
 \usepackage{color}
 \usepackage{MnSymbol}
% \usepackage{movie15}

\usepackage{amssymb}% http://ctan.org/pkg/amssymb
\usepackage{pifont}% http://ctan.org/pkg/pifont
\newcommand{\cmark}{\ding{51}}%
\newcommand{\xmark}{\ding{55}}%

\DeclareSymbolFont{symbolsC}{U}{txsyc}{m}{n}
\DeclareMathSymbol{\boxright}{\mathrel}{symbolsC}{128}
\DeclareMathAlphabet{\mathpzc}{OT1}{pzc}{m}{it}

\setlength{\parskip}{1ex plus 0.5ex minus 0.2ex}

\AtBeginSection[]
{
\begin{frame}
	\Huge{\color{darkblue} \insertsection}
\end{frame}
}

\renewenvironment*{quote}	
	{\list{}{\rightmargin   \leftmargin} \item } 	
	{\endlist }

\definecolor{darkgreen}{rgb}{0,0.7,0}
\definecolor{darkblue}{rgb}{0,0,0.8}

\usepackage[italic]{mathastext}
\usepackage{nicefrac}


%\def\toprule{}
%\def\bottomrule{}
%\def\midrule{}
\usepackage{etoolbox}
\AfterEndEnvironment{description}{\vspace{9pt}}
\AfterEndEnvironment{oltableau}{\vspace{9pt}}
\BeforeBeginEnvironment{oltableau}{\vspace{9pt}}
\AfterEndEnvironment{center}{\vspace{9pt}}
\BeforeBeginEnvironment{tabular}{\vspace{9pt}}
\AfterEndEnvironment{longtable}{\vspace{-6pt}}
\usepackage{booktabs}
\usepackage{longtable}
\usepackage{array}
\usepackage{multirow}
\usepackage{wrapfig}
\usepackage{float}
\usepackage{colortbl}
\usepackage{pdflscape}
\usepackage{tabu}
\usepackage{threeparttable} 
\usepackage{threeparttablex} 
\usepackage[normalem]{ulem} 
\usepackage{makecell}
\usepackage{xcolor}
\usepackage{ulem}

\setlength\heavyrulewidth{0ex}
\setlength\lightrulewidth{0.08ex}

\aboverulesep=0ex
\belowrulesep=0ex
\renewcommand{\arraystretch}{1.2}

\title{444 Lecture 2.4 - Dominance}
\author{Brian Weatherson}
\date{}

\begin{document}
\frame{\titlepage}

\begin{frame}{Plan}
\protect\hypertarget{plan}{}

\begin{itemize}
\tightlist
\item
  Explain strong dominance and weak dominance.
\end{itemize}

\end{frame}

\begin{frame}{Associated Reading}
\protect\hypertarget{associated-reading}{}

Bonanno, section 2.2.

\end{frame}

\begin{frame}{A Simple Game}
\protect\hypertarget{a-simple-game}{}

\begin{table}[!h]
\centering
\begin{tabular}[t]{>{}l|ll}
\toprule
 & Left & Right\\
\midrule
Up & 4, 1 & 2, 0\\
Down & 3, 0 & 1, 1\\
\bottomrule
\end{tabular}
\end{table}

Here's how to read this table.

\begin{enumerate}[<+->]
\tightlist
\item
  Two players, call them Row and Column.
\item
  Row chooses the row, Column chooses the column - between them they
  choose a cell.
\item
  Each cell has two numbers - the first is Row's payout, the second is
  Column's payout.
\end{enumerate}

\end{frame}

\begin{frame}{Strong Dominance}
\protect\hypertarget{strong-dominance}{}

\begin{table}[!h]
\centering
\begin{tabular}[t]{>{}l|ll}
\toprule
 & Left & Right\\
\midrule
Up & 4, 1 & 2, 0\\
Down & 3, 0 & 1, 1\\
\bottomrule
\end{tabular}
\end{table}

\begin{itemize}
\tightlist
\item
  Whatever Column does, Row is better off playing Up rather than Down.
\item
  We say that Up \textbf{strongly dominates} Down.
\end{itemize}

\end{frame}

\begin{frame}{Strong Dominance}
\protect\hypertarget{strong-dominance-1}{}

\begin{table}[!h]
\centering
\begin{tabular}[t]{>{}l|ll}
\toprule
 & Left & Right\\
\midrule
Up & 4, 1 & 2, 0\\
Middle & 5, 0 & 0, 0\\
Down & 3, 0 & 1, 1\\
\bottomrule
\end{tabular}
\end{table}

\begin{itemize}
\tightlist
\item
  Adding options doesn't change things.
\item
  Up still dominates Down, even if it isn't always best.
\end{itemize}

\end{frame}

\begin{frame}{Strong Dominance}
\protect\hypertarget{strong-dominance-2}{}

\begin{table}[!h]
\centering
\begin{tabular}[t]{>{}l|ll}
\toprule
 & Left & Right\\
\midrule
Up & 3, 1 & 0, 0\\
Middle & 2, 0 & 2, 0\\
Down & 0, 0 & 3, 1\\
\bottomrule
\end{tabular}
\end{table}

\begin{itemize}
\tightlist
\item
  This is \textbf{not} a case of dominance.
\item
  Even though Middle is never the highest value, it isn't dominated by
  any one option.
\end{itemize}

\end{frame}

\begin{frame}{Strong Dominance}
\protect\hypertarget{strong-dominance-3}{}

Strategy \(S_1\) strongly dominates strategy \(S_2\) if for any strategy
\(S\) by the other player(s), if \(S\) is played, then \(S_1\) returns a
higher payoff than \(S_2\).

\end{frame}

\begin{frame}{Weak Dominance}
\protect\hypertarget{weak-dominance}{}

Strategy \(S_1\) weakly dominates strategy \(S_2\) if for any strategy
\(S\) by the other player(s), if \(S\) is played, then \(S_1\) returns a
payoff that is at least as high \(S_2\), and for some strategy by the
other player(s), \(S_1\) returns a higher payoff than \(S_2\).

\begin{itemize}
\tightlist
\item
  The difference is that weak dominance allows for \textbf{ties}.
\end{itemize}

\end{frame}

\begin{frame}{Two Dominance Notions}
\protect\hypertarget{two-dominance-notions}{}

Strong Dominance

\begin{itemize}
\tightlist
\item
  Always better.
\end{itemize}

Weak Dominance

\begin{itemize}
\tightlist
\item
  Never worse.
\item
  Sometimes better.
\end{itemize}

\end{frame}

\begin{frame}{Weak Dominance}
\protect\hypertarget{weak-dominance-1}{}

\begin{table}[!h]
\centering
\begin{tabular}[t]{>{}l|ll}
\toprule
 & Left & Right\\
\midrule
Up & 4, 1 & 2, 0\\
Down & 3, 0 & \textbf{2}, 1\\
\bottomrule
\end{tabular}
\end{table}

\begin{itemize}
\tightlist
\item
  I've changed the payoffs in the bottom right cell.
\item
  Now Up does not strongly dominate Down.
\item
  But it does weakly dominate Down.
\end{itemize}

\end{frame}

\begin{frame}{For Next Time}
\protect\hypertarget{for-next-time}{}

We'll look at some examples of famous two by two by two games.

\end{frame}

\end{document}
