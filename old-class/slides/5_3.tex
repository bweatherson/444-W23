% Options for packages loaded elsewhere
\PassOptionsToPackage{unicode}{hyperref}
\PassOptionsToPackage{hyphens}{url}
%
\documentclass[
  ignorenonframetext,
]{beamer}
\usepackage{pgfpages}
\setbeamertemplate{caption}[numbered]
\setbeamertemplate{caption label separator}{: }
\setbeamercolor{caption name}{fg=normal text.fg}
\beamertemplatenavigationsymbolsempty
% Prevent slide breaks in the middle of a paragraph
\widowpenalties 1 10000
\raggedbottom
\setbeamertemplate{part page}{
  \centering
  \begin{beamercolorbox}[sep=16pt,center]{part title}
    \usebeamerfont{part title}\insertpart\par
  \end{beamercolorbox}
}
\setbeamertemplate{section page}{
  \centering
  \begin{beamercolorbox}[sep=12pt,center]{part title}
    \usebeamerfont{section title}\insertsection\par
  \end{beamercolorbox}
}
\setbeamertemplate{subsection page}{
  \centering
  \begin{beamercolorbox}[sep=8pt,center]{part title}
    \usebeamerfont{subsection title}\insertsubsection\par
  \end{beamercolorbox}
}
\AtBeginPart{
  \frame{\partpage}
}
\AtBeginSection{
  \ifbibliography
  \else
    \frame{\sectionpage}
  \fi
}
\AtBeginSubsection{
  \frame{\subsectionpage}
}
\usepackage{amsmath,amssymb}
\usepackage{lmodern}
\usepackage{ifxetex,ifluatex}
\ifnum 0\ifxetex 1\fi\ifluatex 1\fi=0 % if pdftex
  \usepackage[T1]{fontenc}
  \usepackage[utf8]{inputenc}
  \usepackage{textcomp} % provide euro and other symbols
\else % if luatex or xetex
  \usepackage{unicode-math}
  \defaultfontfeatures{Scale=MatchLowercase}
  \defaultfontfeatures[\rmfamily]{Ligatures=TeX,Scale=1}
  \setmainfont[BoldFont = SF Pro Rounded Semibold]{SF Pro Rounded}
  \setmathfont[]{STIX Two Math}
\fi
\usefonttheme{serif} % use mainfont rather than sansfont for slide text
% Use upquote if available, for straight quotes in verbatim environments
\IfFileExists{upquote.sty}{\usepackage{upquote}}{}
\IfFileExists{microtype.sty}{% use microtype if available
  \usepackage[]{microtype}
  \UseMicrotypeSet[protrusion]{basicmath} % disable protrusion for tt fonts
}{}
\makeatletter
\@ifundefined{KOMAClassName}{% if non-KOMA class
  \IfFileExists{parskip.sty}{%
    \usepackage{parskip}
  }{% else
    \setlength{\parindent}{0pt}
    \setlength{\parskip}{6pt plus 2pt minus 1pt}}
}{% if KOMA class
  \KOMAoptions{parskip=half}}
\makeatother
\usepackage{xcolor}
\IfFileExists{xurl.sty}{\usepackage{xurl}}{} % add URL line breaks if available
\IfFileExists{bookmark.sty}{\usepackage{bookmark}}{\usepackage{hyperref}}
\hypersetup{
  pdftitle={444 Lecture 5.3 - Mixed Strategies in Equilibria},
  pdfauthor={Brian Weatherson},
  hidelinks,
  pdfcreator={LaTeX via pandoc}}
\urlstyle{same} % disable monospaced font for URLs
\newif\ifbibliography
\setlength{\emergencystretch}{3em} % prevent overfull lines
\providecommand{\tightlist}{%
  \setlength{\itemsep}{0pt}\setlength{\parskip}{0pt}}
\setcounter{secnumdepth}{-\maxdimen} % remove section numbering
\let\Tiny=\tiny

 \setbeamertemplate{navigation symbols}{} 

% \usetheme{Madrid}
 \usetheme[numbering=none, progressbar=foot]{metropolis}
 \usecolortheme{wolverine}
 \usepackage{color}
 \usepackage{MnSymbol}
% \usepackage{movie15}

\usepackage{amssymb}% http://ctan.org/pkg/amssymb
\usepackage{pifont}% http://ctan.org/pkg/pifont
\newcommand{\cmark}{\ding{51}}%
\newcommand{\xmark}{\ding{55}}%

\DeclareSymbolFont{symbolsC}{U}{txsyc}{m}{n}
\DeclareMathSymbol{\boxright}{\mathrel}{symbolsC}{128}
\DeclareMathAlphabet{\mathpzc}{OT1}{pzc}{m}{it}

\setlength{\parskip}{1ex plus 0.5ex minus 0.2ex}

\AtBeginSection[]
{
\begin{frame}
	\Huge{\color{darkblue} \insertsection}
\end{frame}
}

\renewenvironment*{quote}	
	{\list{}{\rightmargin   \leftmargin} \item } 	
	{\endlist }

\definecolor{darkgreen}{rgb}{0,0.7,0}
\definecolor{darkblue}{rgb}{0,0,0.8}

\usepackage[italic]{mathastext}
\usepackage{nicefrac}

\setbeamertemplate{caption}{\raggedright\insertcaption}

%\def\toprule{}
%\def\bottomrule{}
%\def\midrule{}
\usepackage{etoolbox}
\AfterEndEnvironment{description}{\vspace{9pt}}
\AfterEndEnvironment{oltableau}{\vspace{9pt}}
\BeforeBeginEnvironment{oltableau}{\vspace{9pt}}
\AfterEndEnvironment{center}{\vspace{9pt}}
\BeforeBeginEnvironment{tabular}{\vspace{9pt}}
\AfterEndEnvironment{longtable}{\vspace{-6pt}}
\usepackage{booktabs}
\usepackage{longtable}
\usepackage{array}
\usepackage{multirow}
\usepackage{wrapfig}
\usepackage{float}
\usepackage{colortbl}
\usepackage{pdflscape}
\usepackage{tabu}
\usepackage{threeparttable} 
\usepackage{threeparttablex} 
\usepackage[normalem]{ulem} 
\usepackage{makecell}
\usepackage{xcolor}
\usepackage{ulem}

\setlength\heavyrulewidth{0ex}
\setlength\lightrulewidth{0.08ex}

\aboverulesep=0ex
\belowrulesep=0ex
\renewcommand{\arraystretch}{1.2}
\ifluatex
  \usepackage{selnolig}  % disable illegal ligatures
\fi

\title{444 Lecture 5.3 - Mixed Strategies in Equilibria}
\author{Brian Weatherson}
\date{}

\begin{document}
\frame{\titlepage}

\begin{frame}{Plan}
\protect\hypertarget{plan}{}
Discuss the existence of mixed strategy equilibria.
\end{frame}

\begin{frame}{Reading}
\protect\hypertarget{reading}{}
Bonanno, section 6.2
\end{frame}

\begin{frame}{Key Theorem}
\protect\hypertarget{key-theorem}{}
In any finite game in which all mixtures of strategies are available,
there is at least one Nash equilibria.
\end{frame}

\begin{frame}{Example}
\protect\hypertarget{example}{}
\begin{table}[!h]
\centering
\begin{tabular}[t]{>{}r|cc}
\toprule
 & Left & Right\\
\midrule
Up & 2, 0 & 0, 3\\
Down & 0, 1 & 1, 0\\
\bottomrule
\end{tabular}
\end{table}

Let's discuss this game for a bit. Does it have an equilibrium?
\end{frame}

\begin{frame}{Example}
\protect\hypertarget{example-1}{}
\begin{table}[!h]
\centering
\begin{tabular}[t]{>{}r|cc}
\toprule
 & Left & Right\\
\midrule
Up & \fbox{2}, 0 & 0, \fbox{3}\\
Down & 0, \fbox{1} & \fbox{1}, 0\\
\bottomrule
\end{tabular}
\end{table}

No Nash equilibrium in pure strategies.
\end{frame}

\begin{frame}{A Strategy for Row}
\protect\hypertarget{a-strategy-for-row}{}
\begin{table}[!h]
\centering
\begin{tabular}[t]{>{}r|cc}
\toprule
 & Left & Right\\
\midrule
Up & 2, 0 & 0, 3\\
Down & 0, 1 & 1, 0\\
\bottomrule
\end{tabular}
\end{table}

Consider what happens if Row plays

\begin{itemize}
\tightlist
\item
  Up with probability \(\nicefrac{1}{4}\);
\item
  Down with probability \(\nicefrac{3}{4}\).
\end{itemize}
\end{frame}

\begin{frame}{Column's Expected Return}
\protect\hypertarget{columns-expected-return}{}
\begin{table}[!h]
\centering
\begin{tabular}[t]{>{}r|cc}
\toprule
 & Left & Right\\
\midrule
Up & 2, 0 & 0, 3\\
Down & 0, 1 & 1, 0\\
\bottomrule
\end{tabular}
\end{table}

Column's expected return from playing Left is

\begin{itemize}
\tightlist
\item
  0 with probability \(\nicefrac{1}{4}\) plus
\item
  1 with probability \(\nicefrac{3}{4}\), i.e.,
\item
  \(\nicefrac{3}{4}\).
\end{itemize}
\end{frame}

\begin{frame}{Column's Expected Return}
\protect\hypertarget{columns-expected-return-1}{}
\begin{table}[!h]
\centering
\begin{tabular}[t]{>{}r|cc}
\toprule
 & Left & Right\\
\midrule
Up & 2, 0 & 0, 3\\
Down & 0, 1 & 1, 0\\
\bottomrule
\end{tabular}
\end{table}

Column's expected return from playing Right is

\begin{itemize}
\tightlist
\item
  3 with probability \(\nicefrac{1}{4}\) plus
\item
  0 with probability \(\nicefrac{3}{4}\), i.e.,
\item
  \(\nicefrac{3}{4}\).
\end{itemize}
\end{frame}

\begin{frame}{Column's Expected Return}
\protect\hypertarget{columns-expected-return-2}{}
\begin{table}[!h]
\centering
\begin{tabular}[t]{>{}r|cc}
\toprule
 & Left & Right\\
\midrule
Up & 2, 0 & 0, 3\\
Down & 0, 1 & 1, 0\\
\bottomrule
\end{tabular}
\end{table}

\begin{itemize}
\tightlist
\item
  So either way, Column's expected return from playing a pure strategy
  is \(\nicefrac{3}{4}\).
\item
  And hence Column's expected return from playing any mixture of the two
  pure strategies is \(\nicefrac{3}{4}\).
\end{itemize}
\end{frame}

\begin{frame}{A Strategy for Column}
\protect\hypertarget{a-strategy-for-column}{}
\begin{table}[!h]
\centering
\begin{tabular}[t]{>{}r|cc}
\toprule
 & Left & Right\\
\midrule
Up & 2, 0 & 0, 3\\
Down & 0, 1 & 1, 0\\
\bottomrule
\end{tabular}
\end{table}

Consider what happens if Column plays

\begin{itemize}
\tightlist
\item
  Left with probability \(\nicefrac{1}{3}\);
\item
  Right with probability \(\nicefrac{2}{3}\).
\end{itemize}
\end{frame}

\begin{frame}{Row's Expected Return}
\protect\hypertarget{rows-expected-return}{}
\begin{table}[!h]
\centering
\begin{tabular}[t]{>{}r|cc}
\toprule
 & Left & Right\\
\midrule
Up & 2, 0 & 0, 3\\
Down & 0, 1 & 1, 0\\
\bottomrule
\end{tabular}
\end{table}

Row's expected return from playing Up is

\begin{itemize}
\tightlist
\item
  2 with probability \(\nicefrac{1}{3}\) plus
\item
  0 with probability \(\nicefrac{2}{3}\), i.e.,
\item
  \(\nicefrac{2}{3}\).
\end{itemize}
\end{frame}

\begin{frame}{Column's Expected Return}
\protect\hypertarget{columns-expected-return-3}{}
\begin{table}[!h]
\centering
\begin{tabular}[t]{>{}r|cc}
\toprule
 & Left & Right\\
\midrule
Up & 2, 0 & 0, 3\\
Down & 0, 1 & 1, 0\\
\bottomrule
\end{tabular}
\end{table}

Row's expected return from playing Down is

\begin{itemize}
\tightlist
\item
  0 with probability \(\nicefrac{1}{3}\) plus
\item
  1 with probability \(\nicefrac{2}{3}\), i.e.,
\item
  \(\nicefrac{2}{3}\).
\end{itemize}
\end{frame}

\begin{frame}{Column's Expected Return}
\protect\hypertarget{columns-expected-return-4}{}
\begin{table}[!h]
\centering
\begin{tabular}[t]{>{}r|cc}
\toprule
 & Left & Right\\
\midrule
Up & 2, 0 & 0, 3\\
Down & 0, 1 & 1, 0\\
\bottomrule
\end{tabular}
\end{table}

\begin{itemize}
\tightlist
\item
  So either way, Row's expected return from playing a pure strategy is
  \(\nicefrac{2}{3}\).
\item
  And hence Row's expected return from playing any mixture of the two
  pure strategies is \(\nicefrac{2}{3}\).
\end{itemize}
\end{frame}

\begin{frame}{An Equilibria}
\protect\hypertarget{an-equilibria}{}
\begin{table}[!h]
\centering
\begin{tabular}[t]{>{}r|cc}
\toprule
 & Left & Right\\
\midrule
Up & 2, 0 & 0, 3\\
Down & 0, 1 & 1, 0\\
\bottomrule
\end{tabular}
\end{table}

\begin{itemize}
\tightlist
\item
  What happens if they both play the mixed strategies we've been
  discussing? \pause
\item
  It's an equilibria.
\end{itemize}
\end{frame}

\begin{frame}{An Equilibria}
\protect\hypertarget{an-equilibria-1}{}
\begin{table}[!h]
\centering
\begin{tabular}[t]{>{}r|cc}
\toprule
 & Left & Right\\
\midrule
Up & 2, 0 & 0, 3\\
Down & 0, 1 & 1, 0\\
\bottomrule
\end{tabular}
\end{table}

\begin{itemize}
\tightlist
\item
  Whatever Column does, their expected return is \(\nicefrac{3}{4}\).
\item
  So they can't do better than play this mixed strategy.
\item
  They can't do worse either, but the definition of equilibrium just
  requires that they can't do better.
\end{itemize}
\end{frame}

\begin{frame}{An Equilibria}
\protect\hypertarget{an-equilibria-2}{}
\begin{table}[!h]
\centering
\begin{tabular}[t]{>{}r|cc}
\toprule
 & Left & Right\\
\midrule
Up & 2, 0 & 0, 3\\
Down & 0, 1 & 1, 0\\
\bottomrule
\end{tabular}
\end{table}

\begin{itemize}
\tightlist
\item
  Whatever Row does, their expected return is \(\nicefrac{2}{3}\).
\item
  So they can't do better than play this mixed strategy.
\item
  They can't do worse either, but the definition of equilibrium just
  requires that they can't do better.
\end{itemize}
\end{frame}

\begin{frame}{Two General Points}
\protect\hypertarget{two-general-points}{}
\begin{enumerate}[<+->]
\tightlist
\item
  There is always some equilibria like this (at least in finite games),
  even if it doesn't look like it at first.
\item
  Typically, the way we find equilibria is making the other player
  indifferent between a bunch of options.
\end{enumerate}
\end{frame}

\begin{frame}{For Next Time}
\protect\hypertarget{for-next-time}{}
We will use that last point to work out how to compute the Nash
equilibria for simple games.
\end{frame}

\end{document}
